\documentclass{book}
\usepackage[bidi=default,greek.polutoniko,french]{babel}
\usepackage[normalem]{ulem}
\usepackage{paracol}
\usepackage{graphicx}
\usepackage[colorlinks]{hyperref}
\usepackage{bookmark}
\usepackage{glossaries}
\usepackage{ebgaramond-maths}

\babelprovide[import]{arabic}
\babelprovide[import]{hebrew}
\babelprovide[import]{coptic}
\babelprovide{translit}

\babelfont[coptic]{rm}{ArialCoptic}
\babelfont[hebrew]{rm}{FreeSerif}
\babelfont[arabic]{rm}{Amiri}
\babelfont[translit]{rm}{HGNTransliteration}

\footnotelayout{p}

\title{Correspondances égyptologiques. Lettres d’Auguste Mariette}
\author{Thomas Lebée}
\date{Juillet 2020}

\hypersetup
{
    pdfauthor={Thomas Lebée},
    pdftitle={Correspondances égyptologies. Lettres d’Auguste Mariette},
}

\hypersetup{
    colorlinks=true,
    linkcolor=blue,
    filecolor=blue,      
    urlcolor=blue,
}
\graphicspath{ {./images/} }

\newglossary[]{place}{}{}{Lieux}{}
\newglossary[]{contemp}{}{}{Contemporains de Mariette}{}
\newglossary[]{hist}{}{}{Personnages historiques}{}
\newglossary[]{myth}{}{}{Figures mythiques et religieuses}{}
\newglossary[]{abbr}{}{}{Abréviations}{}
\newglossary[]{boat}{}{}{Bateaux}{}
\newglossary[]{entry}{}{}{Glossaire}{}
\newglossary[]{aeg}{}{}{Lexique égyptien}{}
\newglossary[]{obj}{}{}{Objets}{}
\newglossary[]{org}{}{}{Institutions}{}
\newglossary[]{bibl}{}{}{Publications}{}
\newglossary[]{keyword}{}{}{Thèmes}{}

\makenoidxglossaries
% Contemporains
\newglossaryentry{CoEg_pers_imn}{name={À identifier},type={contemp},sort={Aaa},text={\textsuperscript{!}},description={Personne non encore identifiée}}
\newglossaryentry{CoEg_pers_00000001}{name={Mariette [Pacha], Auguste},type={contemp},text={*},description={Égyptologue (1821-1881). Inventeur du Sérapéum et fondateur du service des antiquités de l’Égypte [\gls{CoEg_author_00000001}~; \href{https://catalogue.bnf.fr/ark:/12148/cb12213258k}{cat. gén. BNF}, \href{https://data.bnf.fr/ark:/12148/cb12213258k}{data.bnf}, \href{https://www.idref.fr/030787130}{IdRéf}, \href{https://viaf.org/viaf/2521520}{VIAF}, \href{https://www.wikidata.org/wiki/Q311850}{Wikidata}]}}
\newglossaryentry{CoEg_pers_00000002}{name={Nieuwerkerke (de), Émilien},type={contemp},text={*},description={Artiste et haut fonctionnaire (1811-1892). Successivement directeur général des musées, intendant des beaux-arts et surintendant des musées impériaux [\href{https://catalogue.bnf.fr/ark:/12148/cb12969884g}{cat. gén. BNF}, \href{https://data.bnf.fr/ark:/12148/cb12969884g}{data.bnf}, \href{https://www.idref.fr/035300310}{IdRéf}, \href{https://viaf.org/viaf/39508958}{VIAF}, \href{https://www.wikidata.org/wiki/Q1232403}{Wikidata}]}}
\newglossaryentry{CoEg_pers_00000003}{name={Parieu (Esquirou de), Félix},type={contemp},text={*},description={Homme d'État (1815-1893). Ministre français de l’Instruction publique en 1850 [\href{https://catalogue.bnf.fr/ark:/12148/cb13475874p}{cat. gén. BNF}, \href{https://data.bnf.fr/ark:/12148/cb13475874p}{data.bnf}, \href{https://www.idref.fr/096198931}{IdRéf}, \href{https://viaf.org/viaf/49372727}{VIAF}, \href{https://www.wikidata.org/wiki/Q535804}{Wikidata}]}}
\newglossaryentry{CoEg_pers_00000004}{name={Baroche, Jules},type={contemp},text={*},description={Homme d'État (1802-1870). Ministre français de l’Intérieur en 1850 [\href{https://catalogue.bnf.fr/ark:/12148/cb14435719k}{cat. gén. BNF}, \href{https://data.bnf.fr/ark:/12148/cb14435719k}{data.bnf}, \href{https://www.idref.fr/066962862}{IdRéf}, \href{https://viaf.org/viaf/66683795}{VIAF}, \href{https://www.wikidata.org/wiki/Q3385651}{Wikidata}]}}
\newglossaryentry{CoEg_pers_00000005}{name={Mariette, Éléonore},type={contemp},text={*},description={(1827-1865). Épouse de Mariette, née Millon\gls{CoEg_pers_00000005alias}}}
\newglossaryentry{CoEg_pers_00000005alias}{name={Millon, Éléonore},type={contemp},text={},description={voir «~Mariette, Éléonore~»\gls{CoEg_pers_00000005}}}
\newglossaryentry{CoEg_pers_00000006}{name={Mariette, Marguerite Louise},type={contemp},text={*},description={(1846-1861). Fille de Mariette}}
\newglossaryentry{CoEg_pers_00000007}{name={Mariette, Joséphine Cornélie},type={contemp},text={*},description={(1847-1873). Fille de Mariette}}
\newglossaryentry{CoEg_pers_00000008}{name={Mariette, Sophie Éléonore},type={contemp},text={*},description={(1849-1885). Fille de Mariette}}
\newglossaryentry{CoEg_pers_00000014}{name={Lafuente},type={contemp},text={*},description={Mandaté par Anastasi pour vendre sa collection}}
\newglossaryentry{CoEg_pers_00000015}{name={Anastasi, Giovanni},type={contemp},text={*},description={Marchand d'antiquités (1765-1860). Mariette utilise la formes «~D’Anastasy~»\gls{CoEg_pers_00000015alias} [\href{https://catalogue.bnf.fr/ark:/12148/cb127360717}{cat. gén. BNF}, \href{https://data.bnf.fr/ark:/12148/cb127360717}{data.bnf}, \href{http://viaf.org/viaf/32737566}{VIAF}, \href{https://www.wikidata.org/wiki/Q5563656}{Wikidata}]}}
\newglossaryentry{CoEg_pers_00000015alias}{name={D’Anastasy, Giovanni},type={contemp},sort={DAn},text={},description={Voir «~Anastasi, Giovanni~»}}
\newglossaryentry{CoEg_pers_00000016}{name={Abbas Pacha},type={contemp},text={*},description={Vice-roi d’Égypte (1813-1854). [\href{https://www.idref.fr/191464740}{IdRéf}, \href{https://viaf.org/viaf/88533334}{VIAF}, \href{https://www.wikidata.org/wiki/Q305882}{Wikidata}]}}
\newglossaryentry{CoEg_pers_00000017}{name={Méhémet Ali},type={contemp},text={*},description={Vice-roi d’Égypte (1769-1849) [\href{https://catalogue.bnf.fr/ark:/12148/cb11969320q}{cat. gén. BNF}, \href{https://data.bnf.fr/ark:/12148/cb11969320q}{data.bnf}, \href{https://www.idref.fr/161593585}{IdRéf}, \href{https://viaf.org/viaf/29512610}{VIAF}, \href{https://www.wikidata.org/wiki/Q182781}{Wikidata}]}}
\newglossaryentry{CoEg_pers_00000018}{name={Murray, Charles},type={contemp},text={*},description={Diplomate (1806-1895). Consul britannique en Égypte de 1846 à 1853 [\href{https://catalogue.bnf.fr/ark:/12148/cb12443225k}{cat. gén. BNF}, \href{https://data.bnf.fr/ark:/12148/cb12443225k}{data.bnf}, \href{https://viaf.org/viaf/22486983}{VIAF}, \href{https://www.wikidata.org/wiki/Q1063616}{Wikidata}]}}
\newglossaryentry{CoEg_pers_00000019}{name={Linant de Bellefonds [Pacha], Louis Maurice Adolphe},type={contemp},text={*},description={Ingénieur (1798-1883) [\href{https://catalogue.bnf.fr/ark:/12148/cb124342933}{cat. gén. BNF}, \href{https://data.bnf.fr/ark:/12148/cb124342933}{data.bnf}, \href{https://www.idref.fr/033481059}{IdRéf}, \href{https://viaf.org/viaf/73944857}{VIAF}, \href{https://www.wikidata.org/wiki/Q960740}{Wikidata}]}}
\newglossaryentry{CoEg_pers_00000020}{name={Lambert [Bey], Charles Joseph},type={contemp},text={*},description={Ingénieur (1804-1864) [\href{https://catalogue.bnf.fr/ark:/12148/cb15366179m}{cat. gén. BNF}, \href{https://data.bnf.fr/ark:/12148/cb15366179m}{data.bnf}, \href{https://viaf.org/viaf/29834163}{VIAF}, \href{https://www.wikidata.org/wiki/Q18744680}{Wikidata}]}}
\newglossaryentry{CoEg_pers_00000021}{name={Clot [Bey], Antoine},type={contemp},text={*},description={Médecin (1793-1868) [\href{https://catalogue.bnf.fr/ark:/12148/cb123565313}{cat. gén. BNF}, \href{https://data.bnf.fr/ark:/12148/cb123565313}{data.bnf}, \href{https://www.idref.fr/031474055}{IdRéf}, \href{https://viaf.org/viaf/81980325}{VIAF}, \href{https://www.wikidata.org/wiki/Q2372786}{Wikidata}]}}
\newglossaryentry{CoEg_pers_00000022}{name={Varin [Bey], Noël},type={contemp},text={*},description={Officier militaire (1784-1863)}}
\newglossaryentry{CoEg_pers_00000023}{name={Walker [Bey]},type={contemp},text={*},description={Boulanger britannique d'Abbas Pacha}}
\newglossaryentry{CoEg_pers_00000024}{name={Le Moyne, Arnaud},type={contemp},text={*},description={Diplomate. Consul général et agent de France en Égypte~; remplacé par Sabatier à l'été 1852}}
\newglossaryentry{CoEg_pers_00000025}{name={Benedetti, Vincent},type={contemp},text={*},description={Diplomate (1817-1900). Gendre d’Anastasi [\href{https://catalogue.bnf.fr/ark:/12148/cb13743825w}{cat. gén. BNF}, \href{https://data.bnf.fr/ark:/12148/cb13743825w}{data.bnf}, \href{https://www.idref.fr/055674976}{IdRéf}, \href{http://viaf.org/viaf/10026659}{VIAF}, \href{https://www.wikidata.org/wiki/Q1350848}{Wikidata}]}}
\newglossaryentry{CoEg_pers_00000027}{name={Batissier, Louis},type={contemp},text={*},description={ (1813-1882). Vice-consul de France à Suez entre 1848 et 1861 [\href{https://catalogue.bnf.fr/ark:/12148/cb13777288q}{cat. gén. BNF}, \href{https://data.bnf.fr/ark:/12148/cb13777288q}{data.bnf}, \href{http://viaf.org/viaf/107598696}{VIAF}, \href{http://wikidata.org/entity/Q52149686}{Wikidata}]}}
\newglossaryentry{CoEg_pers_00000028}{name={Safar-Pacha},type={contemp},text={*},description={Moudir de Giza}}
\newglossaryentry{CoEg_pers_00000029}{name={Stéphan Bey},type={contemp},text={*},description={Homme d'État. Ministre égyptien des Affaires étrangères}}
\newglossaryentry{CoEg_pers_00000032}{name={Rougé (de), Emmanuel},type={contemp},text={*},description={Égyptologue et haut fonctionnaire (1811-1872). [\gls{CoEg_author_00000032}~; \href{https://catalogue.bnf.fr/ark:/12148/cb123351571}{cat. gén. BNF}, \href{https://data.bnf.fr/ark:/12148/cb123351571}{data.bnf}, \href{https://www.idref.fr/032294190}{IdRéf}, \href{https://viaf.org/viaf/22213246}{VIAF}, \href{https://www.wikidata.org/wiki/Q2087467}{Wikidata}]}}
\newglossaryentry{CoEg_pers_00000033}{name={Longpérier (de), Adrien},type={contemp},text={*},description={Archéologue (1816-1882) [\href{https://catalogue.bnf.fr/ark:/12148/cb12568704s}{cat. gén. BNF}, \href{https://data.bnf.fr/ark:/12148/cb12568704s}{data.bnf}, \href{https://www.idref.fr/07070631X}{IdRéf}, \href{https://viaf.org/viaf/12424536}{VIAF}, \href{https://www.wikidata.org/wiki/Q127218}{Wikidata}]}}
\newglossaryentry{CoEg_pers_00000034}{name={Viel-Castel (de), Horace},type={contemp},text={*},description={Historien d'art (1802-1864) [\href{https://catalogue.bnf.fr/ark:/12148/cb12644947d}{cat. gén. BNF}, \href{https://data.bnf.fr/ark:/12148/cb12644947d}{data.bnf}, \href{https://www.idref.fr/027184501}{IdRéf}, \href{https://viaf.org/viaf/69051610}{VIAF}, \href{https://www.wikidata.org/wiki/Q3140481}{Wikidata}]}}
\newglossaryentry{CoEg_pers_00000035}{name={Villot, Frédéric},type={contemp},text={*},description={Graveur et historien d'art (1809-1875) [\href{https://catalogue.bnf.fr/ark:/12148/cb12951365p}{cat. gén. BNF}, \href{https://data.bnf.fr/ark:/12148/cb12951365p}{data.bnf}, \href{https://www.idref.fr/032393725}{IdRéf}, \href{https://viaf.org/viaf/24733526}{VIAF}, \href{https://www.wikidata.org/wiki/Q3090265}{Wikidata}]}}
\newglossaryentry{CoEg_pers_00000037}{name={Persigny (Fialin de), Victor},type={contemp},text={*},description={Homme d'État (1808-1872). Ministre français de l’Intérieur de 1852 à 1854. Les Beaux-Arts dépendaient de son portefeuille jusqu'à la fin de 1852, ainsi que la mission confiée à Mariette en Égypte en 1850 [\href{https://catalogue.bnf.fr/ark:/12148/cb10722665h}{cat. gén. BNF}, \href{https://data.bnf.fr/ark:/12148/cb10722665h}{data.bnf}, \href{https://www.idref.fr/079940234}{IdRéf}, \href{https://viaf.org/viaf/19923190}{VIAF}, \href{https://www.wikidata.org/wiki/Q3174556}{Wikidata}]}}
\newglossaryentry{CoEg_pers_00000038}{name={Boujon},type={contemp},text={*},description={Chargé des transports pour le gouvernement français
}}
\newglossaryentry{CoEg_pers_00000039}{name={Verrier},type={contemp},text={*},description={Chargé des transports pour le gouvernement français
}}
\newglossaryentry{CoEg_pers_00000040}{name={Sabatier},type={contemp},text={*},description={Diplomate. Consul général de France en Égypte~; succède à Le Moyne au cours de l'été 1852}}
\newglossaryentry{CoEg_pers_00000041}{name={Drouyn de Lhuys, Édouard},type={contemp},text={*},description={Homme d'État (1805-1881). Ministre français des Affaires étrangères entre 1852 et 1855 [\href{https://catalogue.bnf.fr/ark:/12148/cb13069509g}{cat. gén. BNF}, \href{https://data.bnf.fr/ark:/12148/cb13069509g}{data.bnf}, \href{https://www.idref.fr/093719345}{IdRéf}, \href{https://viaf.org/viaf/17525718}{VIAF}, \href{https://www.wikidata.org/wiki/Q274294}{Wikidata}]}}
\newglossaryentry{CoEg_pers_00000043}{name={Sauzay, Alexandre},type={contemp},text=*,description={Historien d'art (1803-1870). Entra au Louvre comme commis en 1836~; en 1861, il devint conservateur adjoint du musée des souverains\footnote{Archives nationales, 20150497/115, dossier 47.} [\href{https://catalogue.bnf.fr/ark:/12148/cb154854533}{cat. gén. BNF}, \href{https://data.bnf.fr/ark:/12148/cb154854533}{data.bnf}, \href{https://www.idref.fr/057144532}{IdRéf}, \href{http://viaf.org/viaf/41688136}{VIAF}, \href{https://www.wikidata.org/wiki/Q66101481}{Wikidata}]}}
\newglossaryentry{CoEg_pers_00000044}{name={Auguiot, Jean-Baptiste},type={contemp},text=*,description={Fonctionnaire. Entré au Louvre commis en 1829, il y finit sa carrière comme agent comptable en 1852\footnote{Archives nationales, 20150497/114, dossier 36.}}}
\newglossaryentry{CoEg_pers_00000046}{name={Pastré, Eugène},type={contemp},text={*},description={Homme d'affaires (1806-1868). [\href{https://www.wikidata.org/wiki/Q5408576}{Wikidata}]}}
\newglossaryentry{CoEg_pers_00000048}{name={Hékékyan [Bey], Joseph},type={contemp},text={*},description={Haut fonctionnaire égyptien (1807-1875) [\href{https://catalogue.bnf.fr/ark:/12148/cb165386095}{cat. gén. BNF}, \href{https://data.bnf.fr/ark:/12148/cb165386095}{data.bnf}, \href{http://www.idref.fr/148056822}{IdRéf}, \href{http://viaf.org/viaf/257147724}{VIAF}, \href{https://www.wikidata.org/wiki/Q3185073}{Wikidata}]}}
\newglossaryentry{CoEg_pers_00000049}{name={Fleury-Hérard},type={contemp},text={*},description={Banquier ordinaire à Paris du corps diplomatique}}
\newglossaryentry{CoEg_pers_00000050}{name={Aïdi},type={contemp},text={*},description={Négociant d'Égypte auprès de qui Mariette pouvait retirer ses fonds au début de sa première mission}}
\newglossaryentry{CoEg_pers_00000051}{name={Brunet de Presle, Wladimir},type={contemp},text={*},description={Historien  (1809-1875) [\gls{CoEg_author_00000051}~; \href{https://catalogue.bnf.fr/ark:/12148/cb124629699}{cat. gén. BNF}, \href{https://data.bnf.fr/ark:/12148/cb124629699}{data.bnf}, \href{https://www.idref.fr/033808821}{IdRéf}, \href{https://viaf.org/viaf/203410427}{VIAF}, \href{https://www.wikidata.org/wiki/Q4097433}{Wikidata}]}}
\newglossaryentry{CoEg_pers_00000052}{name={Le Moyne, Auguste},type={contemp},text={*},description={Fils d’Arnaud Le Moyne}}
\newglossaryentry{CoEg_pers_00000053}{name={Le Moyne, madame},type={contemp},text={*},description={Épouse d’Arnaud Le Moyne}}
\newglossaryentry{CoEg_pers_00000054}{name={Bray de Buyser},type={contemp},text={*},description={Membre de la Société orientale\footnote{Voir \textit{Revue de l’Orient}, 1855, p. \href{https://gallica.bnf.fr/ark:/12148/bpt6k106662h/f382.item}{372}.}. Mariette lui confia une caisse d'objets à rapporter en France}}
\newglossaryentry{CoEg_pers_00000056}{name={Nisard},type={contemp},text={*},description={Chargé de travaux au Louvre. Peut-être Charles Nisard (1808-1889) [\href{https://catalogue.bnf.fr/ark:/12148/cb11917740x}{cat. gén. BNF}, \href{https://data.bnf.fr/ark:/12148/cb11917740x}{data.bnf}, \href{http://www.idref.fr/027048497}{IdRéf}, \href{http://viaf.org/viaf/27069569}{VIAF}, \href{https://www.wikidata.org/wiki/Q1065647}{Wikidata}]}}
\newglossaryentry{CoEg_pers_00000059}{name={Soliman Pacha},type={contemp},text={*},description={Officier militaire (1788-1860). Français, né Joseph Sève\gls{CoEg_pers_00000059alias}, passé au service de l'Égypte [\href{https://catalogue.bnf.fr/ark:/12148/cb14935790t}{cat. gén. BNF}, \href{https://data.bnf.fr/ark:/12148/cb14935790t}{data.bnf}, \href{http://www.idref.fr/151246068}{IdRéf}, \href{http://viaf.org/viaf/182133269}{VIAF}, \href{https://www.wikidata.org/wiki/Q745216}{Wikidata}]}}
\newglossaryentry{CoEg_pers_00000059alias}{name={Sève, Joseph},type={contemp},text={},description={Voir «~Soliman Pacha\gls{CoEg_pers_00000059}~»}}
\newglossaryentry{CoEg_pers_00000061}{name={Lepsius, Karl Richard},type={contemp},text={*},description={Égyptologue (1810-1884) [\gls{CoEg_author_00000061}~; \href{https://catalogue.bnf.fr/ark:/12148/cb12416893t}{cat. gén. BNF}, \href{https://data.bnf.fr/ark:/12148/cb12416893t}{data.bnf}, \href{https://www.idref.fr/033287740}{IdRéf}, \href{https://viaf.org/viaf/66554760}{VIAF}, \href{https://www.wikidata.org/wiki/Q77231}{Wikidata}]}}
\newglossaryentry{CoEg_pers_00000062}{name={Fouad Effendi},type={contemp},text={*},description={}}
\newglossaryentry{CoEg_pers_00000065}{name={Rouland, Gustave},type={contemp},text={*},description={Homme d'État (1806-1878). Ministre français de l’Instruction publique de 1856 à 1863 [\href{https://catalogue.bnf.fr/ark:/12148/cb155876341}{cat. gén. BNF}, \href{https://data.bnf.fr/ark:/12148/cb155876341}{data.bnf}, \href{https://viaf.org/viaf/12625084}{VIAF}, \href{https://www.wikidata.org/wiki/Q3121315}{Wikidata}]}}
\newglossaryentry{CoEg_pers_00000067}{name={Delaporte, Pacifique-Henri},type={contemp},text={*},description={ (1816-1877). Consul de France au Caire [\href{https://catalogue.bnf.fr/ark:/12148/cb104669559}{cat. gén. BNF}, \href{https://data.bnf.fr/ark:/12148/cb104669559}{data.bnf}, \href{https://www.idref.fr/166147788}{IdRéf}, \href{https://viaf.org/viaf/24591314}{VIAF}, \href{https://www.wikidata.org/wiki/Q47461240}{Wikidata}]}}
\newglossaryentry{CoEg_pers_00000070}{name={Bonnefoy},type={contemp},text={*},description={(?-1859). Auxiliaire officieux de Mariette pendant ses premières fouilles au Sérapéum~; nommé membre du service de conservation des antiquités de l'Égypte à sa création en 1858}}
\newglossaryentry{CoEg_pers_00000071}{name={Napoléon III},type={contemp},text={*},description={Empereur des Français (1808-1873)\gls{CoEg_pers_00000071alias} [\href{https://catalogue.bnf.fr/ark:/12148/cb12462544v}{cat. gén. BNF}, \href{https://data.bnf.fr/ark:/12148/cb12462544v}{data.bnf}, \href{https://www.idref.fr/027274896}{IdRéf}, \href{https://viaf.org/viaf/88934487}{VIAF}, \href{https://www.wikidata.org/wiki/Q7721}{Wikidata}]}}
\newglossaryentry{CoEg_pers_00000071alias}{name={Bonaparte, Louis-Napoléon},type={contemp},text={},description={Voir «~Napoléon III~»}}
\newglossaryentry{CoEg_pers_00000074}{name={Napoléon (prince)},type={contemp},text={*},description={Prince français (1822-1891)\gls{CoEg_pers_00000074alias} [\href{https://catalogue.bnf.fr/ark:/12148/cb14539256j}{cat. gén. BNF}, \href{https://data.bnf.fr/ark:/12148/cb14539256j}{data.bnf}, \href{https://www.idref.fr/078526965}{IdRéf}, \href{https://viaf.org/viaf/103645462}{VIAF}, \href{https://www.wikidata.org/wiki/Q434077}{Wikidata}]}}
\newglossaryentry{CoEg_pers_00000074alias}{name={Bonaparte, Napoléon-Jérôme},type={contemp},text={},description={Voir «~Napoléon (prince)~»}}
\newglossaryentry{CoEg_pers_00000075}{name={Fould, Achille},type={contemp},text={*},description={Homme d'État (1800-1867) [\href{https://catalogue.bnf.fr/ark:/12148/cb12459740h}{cat. gén. BNF}, \href{https://data.bnf.fr/ark:/12148/cb12459740h}{data.bnf}, \href{https://www.idref.fr/029691257}{IdRéf}, \href{https://viaf.org/viaf/7485939}{VIAF}, \href{https://www.wikidata.org/wiki/Q340215}{Wikidata}]}}
\newglossaryentry{CoEg_pers_00000076}{name={Eugénie (impératrice)},type={contemp},text={*},description={Impératrice, épouse de Napoléon III (1826-1920)\gls{CoEg_pers_00000076alias} [\href{https://catalogue.bnf.fr/ark:/12148/cb13013975w}{cat. gén. BNF}, \href{https://data.bnf.fr/ark:/12148/cb13013975w}{data.bnf}, \href{https://www.idref.fr/027320448}{IdRéf}, \href{https://viaf.org/viaf/39510739}{VIAF}, \href{https://www.wikidata.org/wiki/Q157130}{Wikidata}]}}
\newglossaryentry{CoEg_pers_00000076alias}{name={Montijo (de), Eugénie},type={contemp},text={},description={Voir «~Eugénie (impératrice)~»}}
\newglossaryentry{CoEg_pers_00000077}{name={Louis-Philippe I\textsuperscript{er}},type={contemp},text={*},description={Roi des Français (1773-1850) [\href{https://catalogue.bnf.fr/ark:/12148/cb120083455}{cat. gén. BNF}, \href{https://data.bnf.fr/ark:/12148/cb120083455}{data.bnf}, \href{https://www.idref.fr/028200004}{IdRéf}, \href{https://viaf.org/viaf/55392984}{VIAF}, \href{https://www.wikidata.org/wiki/Q7771}{Wikidata}]}}
\newglossaryentry{CoEg_pers_00000078}{name={Billault, Adolphe},type={contemp},text={*},description={Homme d'État (1805-1863). Ministre français de l’Intérieur de 1854 à 1858 puis de 1859 à 1860 [\href{https://catalogue.bnf.fr/ark:/12148/cb13209940t}{cat. gén. BNF}, \href{https://data.bnf.fr/ark:/12148/cb13209940t}{data.bnf}, \href{https://www.idref.fr/147979323}{IdRéf}, \href{https://viaf.org/viaf/49367618}{VIAF}, \href{https://www.wikidata.org/wiki/Q505189}{Wikidata}]}}
\newglossaryentry{CoEg_pers_00000079}{name={Morny (de), Charles},type={contemp},text={*},description={Homme d'État (1811-1865) [\href{https://catalogue.bnf.fr/ark:/12148/cb119639291}{cat. gén. BNF}, \href{https://data.bnf.fr/ark:/12148/cb119639291}{data.bnf}, \href{https://www.idref.fr/027641082}{IdRéf}, \href{https://viaf.org/viaf/36926334}{VIAF}, \href{https://www.wikidata.org/wiki/Q965335}{Wikidata}]}}
\newglossaryentry{CoEg_pers_00000080}{name={Saïd Pacha},type={contemp},text={*},description={Vice-roi d'Égypte (1822-1863) [\href{https://www.idref.fr/170062791}{IdRéf}, \href{https://viaf.org/viaf/88532611}{VIAF}, \href{https://www.wikidata.org/wiki/Q366744}{Wikidata}]}}
\newglossaryentry{CoEg_pers_00000081}{name={Ferri-Pisani, Camille},type={contemp},text={*},description={Officier militaire (1819-1893). Aide-de-camp du prince Napoléon [\href{https://catalogue.bnf.fr/ark:/12148/cb12122728t}{cat. gén. BNF}, \href{https://data.bnf.fr/ark:/12148/cb12122728t}{data.bnf}, \href{https://www.idref.fr/09538197X}{IdRéf}, \href{https://viaf.org/viaf/49258379}{VIAF}, \href{https://www.wikidata.org/wiki/Q23054685}{Wikidata}]}}
\newglossaryentry{CoEg_pers_00000082}{name={Devéria, Théodule},type={contemp},text={*},description={Égyptologue (1831-1871) [\href{https://catalogue.bnf.fr/ark:/12148/cb129494828}{cat. gén. BNF}, \href{https://data.bnf.fr/ark:/12148/cb129494828}{data.bnf}, \href{https://www.idref.fr/032274386}{IdRéf}, \href{https://viaf.org/viaf/51821846}{VIAF}, \href{https://www.wikidata.org/wiki/Q406953}{Wikidata}]}}
\newglossaryentry{CoEg_pers_00000083}{name={Mariette, Émilie Marie},type={contemp},text={*},description={(1855-1871). Fille de Mariette}}
\newglossaryentry{CoEg_pers_00000084}{name={Mariette, Alphonse Paulin},type={contemp},text={*},description={(1856-1879). Fils de Mariette}}
\newglossaryentry{CoEg_pers_00000085}{name={Faucher, Léon},type={contemp},text={*},description={Homme d'État (1803-1854). Ministre français de l’Intérieur en 1851 (en tant que chef du gouvernement) [\href{http://catalogue.bnf.fr/ark:/12148/cb125109558}{cat. gén. BNF}, \href{http://data.bnf.fr/ark:/12148/cb125109558}{data.bnf}, \href{http://www.idref.fr/034344985}{IdRéf}, \href{http://viaf.org/viaf/41946936}{VIAF}, \href{https://www.wikidata.org/wiki/Q954880}{Wikidata}]}}
\newglossaryentry{CoEg_pers_00000086}{name={Crouseilhes (de), Marie-Jean-Pierre-Pie-Frédéric Dombidau},type={contemp},text={*},description={Homme d'État (1792-1861). Ministre français de l’Instruction publique en 1851 [\href{http://catalogue.bnf.fr/ark:/12148/cb10740894k}{cat. gén. BNF}, \href{http://data.bnf.fr/ark:/12148/cb10740894k}{data.bnf}, \href{http://www.idref.fr/152794360}{IdRéf}, \href{http://viaf.org/viaf/107023070}{VIAF}, \href{https://www.wikidata.org/wiki/Q3292595}{Wikidata}]}}
\newglossaryentry{CoEg_pers_00000087}{name={Ibrahim Pacha},type={contemp},text={*},description={Vice-roi d'Égypte (1789-1848)}}
\newglossaryentry{CoEg_pers_00000088}{name={Fernandez, Solomon},type={contemp},text={*},description={Marchand d'antiquités (?-1860)}}
\newglossaryentry{CoEg_pers_00000089}{name={Messara, Youssouf},type={contemp},text={*},description={\footnote{Un Joseph Messara était drogman auxiliaire au vice-consulat de France au Caire en 1822 (\textsc{Dardaud} G., «~Un ingénieur français au service de Mohamed Ali. Louis Alexis Jumel (1785-1823)~», \textit{Bulletin de l'Institut d'Égypte} 22, 1939-1940, p. 49-97, p. 91). Il est cité par Champollion en 1828, sous le nom de Joseph ou Joussouf Msarra, comme drogman du consulat (\textsc{Champollion le Jeune} Jean-François (\textsc{Hartleben} Hermine, éd.), \textit{Lettres et journaux de Champollion} t. 2 \textit{Lettres et journaux écrits pendant le voyage d’Égypte} (\textit{Bibliothèque égyptologique} 31), Paris, Ernest Leroux, 1909, p.~\href{https://gallica.bnf.fr/ark:/12148/bpt6k55516z/f102.image}{73} et \href{https://gallica.bnf.fr/ark:/12148/bpt6k55516z/f129.image}{98}).}. Cité en 1851 comme un Européen possédant des antiquités à Saqqarah}}

\newglossaryentry{CoEg_pers_00000090}{name={Salvandy (de), Narcisse-Achille},type={contemp},text={*},description={Homme d'État (1795-1856) [\href{https://catalogue.bnf.fr/ark:/12148/cb130079493}{cat. gén. BNF}, \href{https://data.bnf.fr/fr/13007949/narcisse-achille_de_salvandy/}{data.bnf}, \href{https://www.wikidata.org/wiki/Q1381121}{Wikidata}]}}
\newglossaryentry{CoEg_pers_00000093}{name={Camaret, Louis},type={contemp},text={*},description={Fonctionnaire français (1795-1860). Recteur de l'académie de Douai [\href{https://catalogue.bnf.fr/ark:/12148/cb102117199}{cat. gén. BNF}, \href{https://data.bnf.fr/ark:/12148/cb102117199}{data.bnf}, \href{https://www.wikidata.org/wiki/Q42904349}{Wikidata}]}}
\newglossaryentry{CoEg_pers_00000094}{name={Champollion le Jeune, Jean-François},type={contemp},text={*},description={Égyptologue (1790-1832) [\gls{CoEg_author_00000094}~; \href{https://catalogue.bnf.fr/ark:/12148/cb11895998s}{cat. gén. BNF}, \href{https://data.bnf.fr/ark:/12148/cb11895998s}{data.bnf}, \href{https://www.idref.fr/026777509}{IdRéf}, \href{https://viaf.org/viaf/34454460}{VIAF}, \href{https://www.wikidata.org/wiki/Q260}{Wikidata}]}}
\newglossaryentry{CoEg_pers_00000095}{name={Young, Thomas},type={contemp},text={*},description={Physicien (1773-1829) [\href{https://catalogue.bnf.fr/ark:/12148/cb12571287f}{cat. gén. BNF}, \href{https://data.bnf.fr/ark:/12148/cb12571287f}{data.bnf}, \href{https://www.idref.fr/035025816}{IdRéf}, \href{https://viaf.org/viaf/128851}{VIAF}, \href{https://www.wikidata.org/wiki/Q25820}{Wikidata}]}}
\newglossaryentry{CoEg_pers_00000096}{name={Åkerblad, Johan David},type={contemp},sort={Akerblad},text={*},description={Orientaliste (1763-1819) [\href{https://catalogue.bnf.fr/ark:/12148/cb12140208r}{cat. gén. BNF}, \href{https://data.bnf.fr/ark:/12148/cb12140208r}{data.bnf}, \href{https://www.idref.fr/029865840}{IdRéf}, \href{https://viaf.org/viaf/24637340}{VIAF}, \href{https://www.wikidata.org/wiki/Q692438}{Wikidata}]}}
\newglossaryentry{CoEg_pers_00000097}{name={Letronne, Jean Antoine},type={contemp},text={*},description={Antiquisant (1787-1848) [\gls{CoEg_author_00000097}~; \href{https://catalogue.bnf.fr/ark:/12148/cb12338945z}{cat. gén. BNF}, \href{https://data.bnf.fr/ark:/12148/cb12338945z}{data.bnf}, \href{https://www.idref.fr/068531230}{IdRéf}, \href{https://viaf.org/viaf/36990347}{VIAF}, \href{https://www.wikidata.org/wiki/Q574368}{Wikidata}]}}
\newglossaryentry{CoEg_pers_00000124}{name={Adam, Alexandre},type={contemp},text={*},description={ Homme politique (1790-1886). Maire de Boulogne-sur-Mer de 1830 à 1848 puis de 1855 à 1861 [\href{https://catalogue.bnf.fr/ark:/12148/cb137407027}{cat. gén. BNF}, \href{https://data.bnf.fr/ark:/12148/cb137407027}{data.bnf}, \href{https://viaf.org/viaf/74030215}{VIAF}, \href{https://www.wikidata.org/wiki/Q55979286}{Wikidata}]}}
\newglossaryentry{CoEg_pers_00000125}{name={Delessert, François},type={contemp},text={*},description={Homme politique (1780-1868). Député du Pas-de-Calais de 1838 à 1848 [\href{https://catalogue.bnf.fr/ark:/12148/cb106154492}{cat. gén. BNF}, \href{https://data.bnf.fr/ark:/12148/cb106154492}{data.bnf}, \href{https://www.idref.fr/152971629}{IdRéf}, \href{https://viaf.org/viaf/29524377}{VIAF}, \href{https://www.wikidata.org/wiki/Q3083543}{Wikidata}] }}
\newglossaryentry{CoEg_pers_00000129}{name={Huntington, Robert},type={contemp},text={*},description={Ecclésiastique orientaliste (1637-1701) [\href{https://catalogue.bnf.fr/ark:/12148/cb14481381z}{cat. gén. BNF}, \href{https://data.bnf.fr/ark:/12148/cb14481381z}{data.bnf}, \href{https://www.idref.fr/066967384}{IdRéf}, \href{https://viaf.org/viaf/34684157}{VIAF}, \href{https://www.wikidata.org/wiki/Q15440182}{Wikidata}]}}
\newglossaryentry{CoEg_pers_00000130}{name={Wansleben, Johann Michael},type={contemp},text={*},description={ Ecclésiastique orientaliste (1635-1679) [\href{https://catalogue.bnf.fr/ark:/12148/cb107038605}{cat. gén. BNF}, \href{https://data.bnf.fr/ark:/12148/cb107038605}{data.bnf}, \href{https://www.idref.fr/111888743}{IdRéf}, \href{https://viaf.org/viaf/3404021}{VIAF}, \href{https://www.wikidata.org/wiki/Q3180611}{Wikidata}]}}
\newglossaryentry{CoEg_pers_00000131}{name={Quatremère, Étienne Marc},type={contemp},text={*},description={ Orientaliste (1782-1857) [\href{https://catalogue.bnf.fr/ark:/12148/cb124433684}{cat. gén. BNF}, \href{https://data.bnf.fr/ark:/12148/cb124433684}{data.bnf}, \href{https://www.idref.fr/033581037}{IdRéf}, \href{https://viaf.org/viaf/7482667}{VIAF}, \href{https://www.wikidata.org/wiki/Q289886}{Wikidata}]}}
\newglossaryentry{CoEg_pers_00000132}{name={Prudhoe (Lord)},type={contemp},text={*},description={Homme politique et explorateur britannique (1792-1865) [\href{https://viaf.org/viaf/41730198}{VIAF}, \href{https://www.wikidata.org/wiki/Q1753204}{Wikidata}]}}
\newglossaryentry{CoEg_pers_00000133}{name={Tischendorf (von), Constantin},type={contemp},text={*},description={Helléniste bibliste (1815-1874) [\href{https://catalogue.bnf.fr/ark:/12148/cb124893014}{cat. gén. BNF}, \href{https://data.bnf.fr/ark:/12148/cb124893014}{data.bnf}, \href{https://www.idref.fr/059411937}{IdRéf}, \href{https://viaf.org/viaf/56713311}{VIAF}, \href{https://www.wikidata.org/wiki/Q76743}{Wikidata}]}}
\newglossaryentry{CoEg_pers_00000134}{name={Frédéric Auguste II},type={contemp},text={*},description={Roi de Saxe (1797-1854) [\href{https://viaf.org/viaf/40177070}{VIAF}, \href{https://www.wikidata.org/wiki/Q57986}{Wikidata}]}}
\newglossaryentry{CoEg_pers_00000135}{name={Tattam, Henry},type={contemp},text={*},description={Ecclésiastique coptisant (1788-1868) [\href{https://catalogue.bnf.fr/ark:/12148/cb102927808}{cat. gén. BNF}, \href{https://data.bnf.fr/ark:/12148/cb102927808}{data.bnf}, \href{https://www.idref.fr/114014914}{IdRéf}, \href{https://viaf.org/viaf/52471802}{VIAF}, \href{https://www.wikidata.org/wiki/Q576560}{Wikidata}]}}
\newglossaryentry{CoEg_pers_00000136}{name={Mimaut, Jean-François},type={contemp},text={*},description={Diplomate et collectionneur (1774-1837) [\href{https://catalogue.bnf.fr/ark:/12148/cb12563996m}{cat. gén. BNF}, \href{https://data.bnf.fr/fr/12563996/jean-francois_mimaut/}{data.bnf}, \href{https://www.idref.fr/034940952}{IdRéf}, \href{https://viaf.org/viaf/24718760}{VIAF}, \href{https://www.wikidata.org/wiki/Q14918072}{Wikidata}]}}
\newglossaryentry{CoEg_pers_00000137}{name={Drovetti, Bernardino},type={contemp},text={*},description={Collectionneur et consul de France en Égypte (1776-1852) [\href{https://catalogue.bnf.fr/ark:/12148/cb120689421}{cat. gén. BNF}, \href{https://data.bnf.fr/ark:/12148/cb120689421}{data.bnf}, \href{https://www.idref.fr/028962478}{IdRéf}, \href{https://viaf.org/viaf/39399994}{VIAF}, \href{https://www.wikidata.org/wiki/Q822895}{Wikidata}]}}
\newglossaryentry{CoEg_pers_00000138}{name={Wilkinson, John Gardner},type={contemp},text={*},description={ Égyptologue (1797-1875) [\href{https://catalogue.bnf.fr/ark:/12148/cb12331057s}{cat. gén. BNF}, \href{https://data.bnf.fr/ark:/12148/cb12331057s}{data.bnf}, \href{https://www.idref.fr/080734243}{IdRéf}, \href{https://viaf.org/viaf/69000129}{VIAF}, \href{https://www.wikidata.org/wiki/Q923032}{Wikidata}] }}
\newglossaryentry{CoEg_pers_00000139}{name={Lenormant, Charles},type={contemp},text={*},description={ Égyptologue (1802-1859) [\href{https://catalogue.bnf.fr/ark:/12148/cb124598503}{cat. gén. BNF}, \href{https://data.bnf.fr/ark:/12148/cb124598503}{data.bnf}, \href{https://www.idref.fr/033766398}{IdRéf}, \href{https://viaf.org/viaf/7485941}{VIAF}, \href{https://www.wikidata.org/wiki/Q547695}{Wikidata}]}}
\newglossaryentry{CoEg_pers_00000140}{name={Bunsen (von), Christian Charles Josias},text={*},type={contemp},description={Diplomate et érudit (1791-1860) [\href{https://catalogue.bnf.fr/ark:/12148/cb121845954}{cat. gén. BNF}, \href{https://data.bnf.fr/ark:/12148/cb121845954}{data.bnf}, \href{https://www.idref.fr/067070019}{IdRéf}, \href{https://viaf.org/viaf/27110558}{VIAF}, \href{https://www.wikidata.org/wiki/Q67409}{Wikidata}]}}
\newglossaryentry{CoEg_pers_00000141}{name={Jomard, Edme-François},type={contemp},text={*},description={Érudit (1777-1862) [ \href{https://catalogue.bnf.fr/ark:/12148/cb121648063}{cat. gén. BNF}, \href{https://data.bnf.fr/ark:/12148/cb121648063}{data.bnf}, \href{https://www.idref.fr/030178673}{IdRéf}, \href{https://viaf.org/viaf/2511766}{VIAF}, \href{https://www.wikidata.org/wiki/Q240788}{Wikidata}]}}
\newglossaryentry{CoEg_pers_00000143}{name={Gide},type={contemp},text={*},description={Éditeur}}
\newglossaryentry{CoEg_pers_00000144}{name={Baudry, J.},type={contemp},text={*},description={Éditeur}}
\newglossaryentry{CoEg_pers_00000146}{name={Chevalier, Michel},type={contemp},text={*},description={Haut fonctionnaire français (1806-1879) [\href{https://catalogue.bnf.fr/ark:/12148/cb12171043k}{cat. gén. BNF}, \href{https://data.bnf.fr/ark:/12148/cb12171043k}{data.bnf}, \href{https://www.idref.fr/03025518X}{IdRéf}, \href{https://viaf.org/viaf/51933}{VIAF}, \href{https://www.wikidata.org/wiki/Q1930688}{Wikidata}]}}
\newglossaryentry{CoEg_pers_00000160}{name={Frédéric-Guillaume IV de Prusse},type={contemp},text={*},description={Roi de Prusse (1795-1861) [\href{https://catalogue.bnf.fr/ark:/12148/cb12192556z}{cat. gén. BNF}, \href{https://data.bnf.fr/ark:/12148/cb12192556z}{data.bnf}, \href{https://www.idref.fr/030528593}{IdRéf}, \href{https://viaf.org/viaf/172909814}{VIAF}, \href{https://www.wikidata.org/wiki/Q57180}{Wikidata}]}}
\newglossaryentry{CoEg_pers_00000164}{name={Sharpe, Samuel},type={contemp},text={*},description={Égyptologue et bibliste (1799-1881) [\href{https://catalogue.bnf.fr/ark:/12148/cb10332256z}{cat. gén. BNF}, \href{https://data.bnf.fr/ark:/12148/cb10332256z}{data.bnf}, \href{https://www.idref.fr/066009189}{IdRéf}, \href{https://viaf.org/viaf/20462374}{VIAF}, \href{https://www.wikidata.org/wiki/Q7412601}{Wikidata}]}}
\newglossaryentry{CoEg_pers_00000174}{name={Zoega, Georg},type={hist},text={*},description={Antiquaire (1755-1809) [\gls{CoEg_author_00000174}~; \href{https://catalogue.bnf.fr/ark:/12148/cb13007843r}{cat. gén. BNF}, \href{https://data.bnf.fr/ark:/12148/cb13007843r}{data.bnf}, \href{https://www.idref.fr/139519955}{IdRéf}, \href{https://viaf.org/viaf/69064075}{VIAF}, \href{https://www.wikidata.org/wiki/Q568595}{Wikidata}]}}
\newglossaryentry{CoEg_pers_00000179}{name={Birch, Samuel},type={contemp},text={*},description={Égyptologue (1813-1885) [\gls{CoEg_author_00000179}~; \href{https://catalogue.bnf.fr/ark:/12148/cb13088768q}{cat. gén. BNF}, \href{https://data.bnf.fr/ark:/12148/cb13088768q}{data.bnf}, \href{https://www.idref.fr/066640059}{IdRéf}, \href{https://viaf.org/viaf/757450}{VIAF}, \href{https://www.wikidata.org/wiki/Q2742058}{Wikidata}]}}
\newglossaryentry{CoEg_pers_00000194}{name={Marsham, John},type={hist},text={*},description={Parlementaire et antiquaire (1602-1685) [\gls{CoEg_author_00000194}~; \href{https://www.idref.fr/103968970}{IDRéf}, \href{https://viaf.org/viaf/1067833}{VIAF}, \href{https://www.wikidata.org/wiki/Q7527919}{Wikidata}] }}
\newglossaryentry{CoEg_pers_00000196}{name={Des Vignoles, Alphonse},type={hist},text={*},description={Érudit (1649-1744) [\gls{CoEg_author_00000196}~; \href{https://catalogue.bnf.fr/ark:/12148/cb10566310v}{cat. gén. BNF}, \href{https://data.bnf.fr/ark:/12148/cb10566310v}{data.bnf}, \href{https://www.idref.fr/071420037}{IdRéf}, \href{https://viaf.org/viaf/69707326}{VIAF}, \href{https://www.wikidata.org/wiki/Q2839770}{Wikidata}]}}
\newglossaryentry{CoEg_pers_00000200}{name={Lenormant, François},type={contemp},text={*},description={Orientaliste (1837-1883) [\gls{CoEg_author_00000200}~; \href{https://catalogue.bnf.fr/ark:/12148/cb12954608d}{cat. gén. BNF}, \href{https://data.bnf.fr/ark:/12148/cb12954608d}{data.bnf}, \href{https://www.idref.fr/061234508}{IdRéf}, \href{https://viaf.org/viaf/71523229}{VIAF}, \href{https://www.wikidata.org/wiki/Q1347918}{Wikidata}]}}
\newglossaryentry{CoEg_pers_00000207}{name={Luynes (d'Albert de), Honoré Théodoric},text={*},type={contemp},description={Antiquaire (1803-1867) [\gls{CoEg_author_00000207}~; \href{https://catalogue.bnf.fr/ark:/12148/cb123479468}{cat. gén. BNF}, \href{https://data.bnf.fr/ark:/12148/cb123479468}{data.bnf}, \href{https://www.idref.fr/087189488}{IdRéf}, \href{https://viaf.org/viaf/46835180}{VIAF}, \href{https://www.wikidata.org/wiki/Q444237}{Wikidata}]}}
\newglossaryentry{CoEg_pers_00000211}{name={Peyron, Bernardino},type={contemp},text={*},description={Orientaliste et bibliothécaire (1819-1903) [\gls{CoEg_author_00000211}~; \href{https://catalogue.bnf.fr/ark:/12148/cb10481159p}{cat. gén. BNF}, \href{https://data.bnf.fr/ark:/12148/cb10481159p}{data.bnf}, \href{https://www.idref.fr/129399299}{IdRéf}, \href{https://viaf.org/viaf/39366651}{VIAF}, \href{https://www.wikidata.org/wiki/Q3638726}{Wikidata}] }}
\newglossaryentry{CoEg_pers_00000212}{name={Reuvens, Caspar},type={contemp},text={*},description={ Antiquaire (1793-1835) [\gls{CoEg_author_00000212}~; \href{https://catalogue.bnf.fr/ark:/12148/cb12789876x}{cat. gén. BNF}, \href{https://data.bnf.fr/ark:/12148/cb12789876x}{data.bnf}, \href{https://www.idref.fr/081107226}{IdRéf}, \href{https://viaf.org/viaf/90636159}{VIAF}, \href{https://www.wikidata.org/wiki/Q1863369}{Wikidata}]}}
\newglossaryentry{CoEg_pers_00000226}{name={Guigniaut, Joseph-Daniel},type={contemp},text={*},description={Helléniste (1794-1876) [\href{https://catalogue.bnf.fr/ark:/12148/cb13475757t}{cat. gén. BNF}, \href{https://data.bnf.fr/ark:/12148/cb13475757t}{data.bnf}, \href{https://www.idref.fr/03514484X}{IdRéf}, \href{https://viaf.org/viaf/5080788}{VIAF}, \href{https://www.wikidata.org/wiki/Q3184217}{Wikidata}] }}
\newglossaryentry{CoEg_pers_00000227}{name={Burnouf, Jean Louis},type={contemp},text={*},description={Philologue (1775-1844) [\href{https://catalogue.bnf.fr/ark:/12148/cb11894608x}{cat. gén. BNF}, \href{https://data.bnf.fr/ark:/12148/cb11894608x}{data.bnf}, \href{https://www.idref.fr/026759845}{IdRéf}, \href{https://viaf.org/viaf/31993809}{VIAF}, \href{https://www.wikidata.org/wiki/Q3166670}{Wikidata}]}}
\newglossaryentry{CoEg_pers_00000237}{name={Servaux, Eugène},type={contemp},text={*},description={Haut fonctionnaire (1815-1890). Chef du bureau des travaux historiques au ministère de l'Instruction publique}}
\newglossaryentry{CoEg_pers_00000238}{name={Ferry, Jules},type={contemp},text={*},description={Homme d'État (1832-1893) [\href{https://catalogue.bnf.fr/ark:/12148/cb12513455m}{cat. gén. BNF}, \href{https://data.bnf.fr/ark:/12148/cb12513455m}{data.bnf}, \href{https://www.idref.fr/027316904}{IdRéf}, \href{https://viaf.org/viaf/5038341}{VIAF}, \href{https://www.wikidata.org/wiki/Q959708}{Wikidata}]}}
\newglossaryentry{CoEg_pers_00000239}{name={Waddington, William Henry},type={contemp},text={*},description={Homme d'État (1826-1894) [\href{https://catalogue.bnf.fr/ark:/12148/cb127512741}{cat. gén. BNF}, \href{https://data.bnf.fr/ark:/12148/cb127512741}{data.bnf}, \href{https://www.idref.fr/035782951}{IdRéf}, \href{https://viaf.org/viaf/64130319}{VIAF}, \href{https://www.wikidata.org/wiki/Q548696}{Wikidata}]}}
\newglossaryentry{CoEg_pers_00000240}{name={Fortoul, Hippolyte},type={contemp},text={*},description={Homme d'État (1811-1856) [\href{https://catalogue.bnf.fr/ark:/12148/cb125108748}{cat. gén. BNF}, \href{https://data.bnf.fr/ark:/12148/cb125108748}{data.bnf}, \href{https://www.idref.fr/034344144}{IdRéf}, \href{https://viaf.org/viaf/64111595}{VIAF}, \href{https://www.wikidata.org/wiki/Q3136058}{Wikidata}]}}
\newglossaryentry{CoEg_pers_00000241}{name={Peyron, Amedeo},type={contemp},text={*},description={Philologue coptisant (1785-1870) [\href{https://catalogue.bnf.fr/ark:/12148/cb13514697g}{cat. gén. BNF}, \href{https://data.bnf.fr/ark:/12148/cb13514697g}{data.bnf}, \href{https://www.idref.fr/050486020}{IdRéf}, \href{https://viaf.org/viaf/19835436}{VIAF}, \href{https://www.wikidata.org/wiki/Q451744}{Wikidata}]}}
\newglossaryentry{CoEg_pers_00000243}{name={Maury, Alfred},type={contemp},text={*},description={Érudit (1817-1892) [\gls{CoEg_author_00000243}~; \href{https://catalogue.bnf.fr/ark:/12148/cb119152793}{cat. gén. BNF}, \href{https://data.bnf.fr/ark:/12148/cb119152793}{data.bnf}, \href{https://www.idref.fr/027019217}{IdRéf}, \href{https://viaf.org/viaf/68932704}{VIAF}, \href{https://www.wikidata.org/wiki/Q1972200}{Wikidata}]}}

%Personnages historiques

\newglossaryentry{CoEg_pers_00000009}{name={Néphéritès I\textsuperscript{er}},type={hist},text={*},description={Roi égyptien (XXIX\textsuperscript{e} dynastie) [\href{https://www.wikidata.org/wiki/Q450994}{Wikidata}]}}
\newglossaryentry{CoEg_pers_00000010}{name={Amyrtée},type={hist},text={*},description={Roi égyptien (XXVIII\textsuperscript{e} dynastie). Identifié par Mariette à Nectanébo I\textsuperscript{er}\footnote{Voir \textsc{Mariette}, Auguste, «~Lettre de M. Auguste Mariette à M. le victomte de Rougé, sur les résultats des fouilles entreprises par ordre du vice-roy d'Égypte~», \textit{Revue archéologique}, 2\textsuperscript{e} série, 1860, t.~2, p. 17-35, p. \href{https://gallica.bnf.fr/ark:/12148/bpt6k9748264q/f48.item}{34} ; \textit{Le Sérapéum de Memphis}, Gide, Paris, 1857-1866, p. \href{https://gallica.bnf.fr/ark:/12148/bpt6k5806994z/f20.image}{5}.} [\href{https://viaf.org/viaf/188145911093727060710}{VIAF}, \href{https://www.wikidata.org/wiki/Q318000}{Wikidata}]}}
\newglossaryentry{CoEg_pers_00000013}{name={Nectanébo I\textsuperscript{er}},type={hist},sort={Nectanebo1},text={*},description={Roi égyptien (XXX\textsuperscript{e} dynastie). Voir «~Amyrtée~» [\href{https://www.idref.fr/188528911}{IdRéf}, \href{http://viaf.org/viaf/838144647709934922455}{VIAF}, \href{https://www.wikidata.org/wiki/Q175775}{Wikidata}]}}
\newglossaryentry{CoEg_pers_00000026}{name={Ramsès II},type={hist},text={*},description={Roi égyptien (XIX\textsuperscript{e} dynastie) [\href{https://catalogue.bnf.fr/ark:/12148/cb11952902j}{cat. gén. BNF}, \href{https://data.bnf.fr/ark:/12148/cb11952902j}{data.bnf}, \href{http://www.idref.fr/027503658}{IdRéf}, \href{https://viaf.org/viaf/25851094/}{VIAF}, \href{https://www.wikidata.org/wiki/Q1523}{Wikidata}]}}
\newglossaryentry{CoEg_pers_00000030}{name={Cambyse II},type={hist},text={*},description={Roi perse [\href{https://www.idref.fr/123252385}{IdRéf}, \href{https://viaf.org/viaf/57990573/}{VIAF}, \href{https://www.wikidata.org/wiki/Q182483}{Wikidata}]}}
\newglossaryentry{CoEg_pers_00000031}{name={Diodore de Sicile},type={hist},text={*},description={Historien grec (I\textsuperscript{er} siècle av. J.-C.) [\gls{CoEg_author_00000031}~;  \href{https://catalogue.bnf.fr/ark:/12148/cb11900233c}{cat. gén. BNF}, \href{https://data.bnf.fr/ark:/12148/cb11900233c}{data.bnf}, \href{https://www.idref.fr/026832682}{IdRéf}, \href{https://viaf.org/viaf/56608763}{VIAF}, \href{https://www.wikidata.org/wiki/Q171241}{Wikidata}]}}
\newglossaryentry{CoEg_pers_00000045}{name={Nectanébo II},type={hist},text={*},sort={Nectanebo2},description={Roi égyptien (XXX\textsuperscript{e} dynastie). Confondu par Mariette avec Nectanébo I\textsuperscript{er} \footnote{\textsc{Lauer} Jean-Philippe, « Mariette à Sakkarah. Du Sérapéum à la direction des antiquités », dans \textit{Mélanges Mariette} (\textit{Bibliothèque d'études} 32), Le Caire, Institut français d'archéologie orientale, 1961, p. 3-55, p. \href{https://archive.org/details/BdE-32/page/n9/mode/2up}{7}, n. 2} [\href{https://viaf.org/viaf/45098268}{VIAF}, \href{https://www.wikidata.org/wiki/Q313126}{Wikidata}]}}
\newglossaryentry{CoEg_pers_00000055}{name={Artaxerxès III},type={hist},text={*},description={Roi perse. Mariette utilise le nom «~Ochus~»\gls{CoEg_pers_00000055alias} [\href{https://www.wikidata.org/wiki/Q192867}{Wikidata}]}}
\newglossaryentry{CoEg_pers_00000055alias}{name={Ochus},type={hist},text={},description={Voir  «~Artaxerxès III~»}}
\newglossaryentry{CoEg_pers_00000057}{name={Darius I\textsuperscript{er} le Grand},type={hist},text={*},description={Roi perse [\href{https://catalogue.bnf.fr/ark:/12148/cb126902644}{cat. gén. BNF}, \href{https://data.bnf.fr/ark:/12148/cb126902644}{data.bnf}, \href{https://www.idref.fr/050127276}{IdRéf}, \href{https://viaf.org/viaf/15560660}{VIAF}, \href{https://www.wikidata.org/wiki/Q44387}{Wikidata}]}}
\newglossaryentry{CoEg_pers_00000058}{name={Amasis},type={hist},text={*},description={Roi égyptien (XXVI\textsuperscript{e} dynastie)[\href{https://viaf.org/viaf/61941083}{VIAF}, \href{https://www.wikidata.org/wiki/Q312045}{Wikidata}]}}
\newglossaryentry{CoEg_pers_00000063}{name={Apriès},type={hist},text={*},description={Roi égyptien (XXVI\textsuperscript{e} dynastie). Mariette utilise la forme « Ouaphris » [\href{https://viaf.org/viaf/398519}{VIAF}, \href{https://www.wikidata.org/wiki/Q349291}{Wikidata}]}}
\newglossaryentry{CoEg_pers_00000063alias}{name={Ouaphris},type={hist},text={*},description={Voir « Apriès »}}
\newglossaryentry{CoEg_pers_00000064}{name={Chéchonq III},sort={Chéchonq4},type={hist},text={*},description={Roi égyptien (XXII\textsuperscript{e} dynastie) [\href{https://www.wikidata.org/wiki/Q878396}{Wikidata}]}}
\newglossaryentry{CoEg_pers_00000066}{name={Neshor},type={hist},text={*},description={Mariette utilise la forme «~Ensahor~»\gls{CoEg_pers_00000066alias}}}
\newglossaryentry{CoEg_pers_00000066alias}{name={Ensahor},type={hist},text={},description={Voir «~Neshor~»\gls{CoEg_pers_00000066alias}}}
\newglossaryentry{CoEg_pers_00000069}{name={Antef},type={hist},text={*},description={Mariette utilise la forme «~Entef~»\gls{CoEg_pers_00000069alias}}}
\newglossaryentry{CoEg_pers_00000069alias}{name={Entef},type={hist},text={},description={Voir «~Antef~»}}
\newglossaryentry{CoEg_pers_00000091}{name={Amenhotep III},type={hist},text=*,description={Roi égyptien (XVIII\textsuperscript{e} dynastie)\gls{CoEg_pers_00000091alias} [\href{https://catalogue.bnf.fr/ark:/12148/cb122180169}{cat. gén. BNF}, \href{https://data.bnf.fr/ark:/12148/cb122180169}{data.bnf}, \href{http://www.idref.fr/030846315}{IdRéf}, \href{https://viaf.org/viaf/262737692/}{VIAF}, \href{https://www.wikidata.org/wiki/Q42606}{Wikidata}]}}
\newglossaryentry{CoEg_pers_00000091alias}{name={Aménophis III},type={hist},text={},description={Voir «~Amenhotep III~»}}
\newglossaryentry{CoEg_pers_00000092}{name={Toutânkhamon},type={hist},text=*,description={Roi égyptien (XVIII\textsuperscript{e} dynastie) [\href{https://catalogue.bnf.fr/ark:/12148/cb12008509v}{cat. gén. BNF}, \href{https://data.bnf.fr/ark:/12148/cb12008509v}{data.bnf}, \href{https://www.idref.fr/028202171}{IdRéf}, \href{http://viaf.org/viaf/148503630}{VIAF}, \href{http://wikidata.org/entity/Q12154}{Wikidata}]}}

\newglossaryentry{CoEg_pers_00000098}{name={Critirnus ?},text={*},type={hist},description={}}
\newglossaryentry{CoEg_pers_00000103}{name={Ulfilas},text={*},type={hist},description={Religieux goth (v. 311-383). Mariette écrit « Ulphilas »[\href{https://catalogue.bnf.fr/ark:/12148/cb12005325t}{cat. gén. BNF}, \href{https://data.bnf.fr/ark:/12148/cb12005325t}{data.bnf}, \href{https://www.idref.fr/028163656}{IdRéf}, \href{https://viaf.org/viaf/66477200}{VIAF}, \href{https://www.wikidata.org/wiki/Q107317}{Wikidata}]}}
\newglossaryentry{CoEg_pers_00000104}{name={Lao-Tseu},text={*},type={hist},description={Philosophe chinois [\href{https://catalogue.bnf.fr/ark:/12148/cb11911025n}{cat. gén. BNF}, \href{https://data.bnf.fr/ark:/12148/cb11911025n}{data.bnf}, \href{https://www.idref.fr/026965402}{IdRéf}, \href{https://viaf.org/viaf/101089774}{VIAF}, \href{https://www.wikidata.org/wiki/Q9333}{Wikidata}]}}
\newglossaryentry{CoEg_pers_00000105}{name={Aristote},text={*},type={hist},description={Philosophe grec (385-322) [\href{https://catalogue.bnf.fr/ark:/12148/cb13091331s}{cat. gén. BNF}, \href{https://data.bnf.fr/ark:/12148/cb13091331s}{data.bnf}, \href{https://www.idref.fr/026690276}{IdRéf}, \href{https://viaf.org/viaf/7524651/}{VIAF}, \href{https://www.wikidata.org/wiki/Q868}{Wikidata}]}}
\newglossaryentry{CoEg_pers_00000106}{name={Platon},text={*},type={hist},description={Philosophe grec (428/427-348/347) [\href{https://catalogue.bnf.fr/ark:/12148/cb11920019p}{cat. gén. BNF}, \href{https://data.bnf.fr/ark:/12148/cb11920019p}{data.bnf}, \href{https://www.idref.fr/027076164}{IdRéf}, \href{https://viaf.org/viaf/108159964/}{VIAF}, \href{https://www.wikidata.org/wiki/Q859}{Wikidata}]}}
\newglossaryentry{CoEg_pers_00000107}{name={Polybe},text={*},type={hist},description={Historien grec (208-126) [\href{https://catalogue.bnf.fr/ark:/12148/cb11920194q}{cat. gén. BNF}, \href{https://data.bnf.fr/ark:/12148/cb11920194q}{data.bnf}, \href{https://www.idref.fr/027352188}{IdRéf}, \href{https://viaf.org/viaf/267920293/}{VIAF}, \href{https://www.wikidata.org/wiki/Q131169}{Wikidata}]}}
\newglossaryentry{CoEg_pers_00000108}{name={Hérodote d'Halicarnasse},text={*},type={hist},description={Historien grec (v. 480-v. 425) [\href{https://catalogue.bnf.fr/ark:/12148/cb119073789}{cat. gén. BNF}, \href{https://data.bnf.fr/ark:/12148/cb119073789}{data.bnf}, \href{https://www.idref.fr/085720917}{IdRéf}, \href{https://viaf.org/viaf/100225976/}{VIAF}, \href{https://www.wikidata.org/wiki/Q26825}{Wikidata}]}}
\newglossaryentry{CoEg_pers_00000109}{name={Plutarque},text={*},type={hist},description={Écrivain grec (v. 46-v. 125) [\href{https://catalogue.bnf.fr/ark:/12148/cb119200813}{cat. gén. BNF}, \href{https://data.bnf.fr/ark:/12148/cb119200813}{data.bnf}, \href{https://www.idref.fr/027076784}{IdRéf}, \href{https://viaf.org/viaf/268955446/}{VIAF}, \href{https://www.wikidata.org/wiki/Q41523}{Wikidata}]}}
\newglossaryentry{CoEg_pers_00000110}{name={Apulée},text={*},type={hist},description={Écrivain latin (v. 125-Apr. 170) [\href{https://catalogue.bnf.fr/ark:/12148/cb125686918}{cat. gén. BNF}, \href{https://data.bnf.fr/ark:/12148/cb125686918}{data.bnf}, \href{https://www.idref.fr/034995145}{IdRéf}, \href{https://viaf.org/viaf/32115433}{VIAF}, \href{https://www.wikidata.org/wiki/Q170512}{Wikidata}]}}
\newglossaryentry{CoEg_pers_00000111}{name={Tacite},text={*},type={hist},description={Historien latin (56-120) [\href{https://catalogue.bnf.fr/ark:/12148/cb11887667z}{cat. gén. BNF}, \href{https://data.bnf.fr/ark:/12148/cb11887667z}{data.bnf}, \href{https://www.idref.fr/026672316}{IdRéf}, \href{https://viaf.org/viaf/100226923}{VIAF}, \href{https://www.wikidata.org/wiki/Q2161}{Wikidata}]}}
\newglossaryentry{CoEg_pers_00000112}{name={Clément d’Alexandrie},text={*},type={hist},description={Écrivain grec (v. 150-v. 215) [\href{https://catalogue.bnf.fr/ark:/12148/cb11897014x}{cat. gén. BNF}, \href{https://data.bnf.fr/ark:/12148/cb11897014x}{data.bnf}, \href{https://www.idref.fr/026791145}{IdRéf}, \href{https://viaf.org/viaf/100185827/}{VIAF}, \href{https://www.wikidata.org/wiki/Q188883}{Wikidata}]}}
\newglossaryentry{CoEg_pers_00000113}{name={Philon d'Alexandrie},text={*},type={hist},description={Écrivain grec (v. 20-v. 45) [\href{https://catalogue.bnf.fr/ark:/12148/cb11919600w}{cat. gén. BNF}, \href{https://data.bnf.fr/ark:/12148/cb11919600w}{data.bnf}, \href{https://www.idref.fr/027070980}{IdRéf}, \href{https://viaf.org/viaf/310526579/}{VIAF}, \href{https://www.wikidata.org/wiki/Q189597}{Wikidata}]}}
\newglossaryentry{CoEg_pers_00000114}{name={Eusèbe},text={*},type={hist},description={Historien grec (v. 265-339) [\href{https://catalogue.bnf.fr/ark:/12148/cb11902007r}{cat. gén. BNF}, \href{https://data.bnf.fr/ark:/12148/cb11902007r}{data.bnf}, \href{https://www.idref.fr/088843734}{IdRéf}, \href{https://viaf.org/viaf/4929593/}{VIAF}, \href{https://www.wikidata.org/wiki/Q142999}{Wikidata}]}}
\newglossaryentry{CoEg_pers_00000115}{name={Sanchoniathon},text={*},type={hist},description={Écrivain phénicien [\href{https://www.idref.fr/083187499}{IdRéf}, \href{https://viaf.org/viaf/95303431/}{VIAF}, \href{https://www.wikidata.org/wiki/Q968582}{Wikidata}]}}
\newglossaryentry{CoEg_pers_00000116}{name={Manéthon},text={*},type={hist},description={Écrivain grec [\href{https://catalogue.bnf.fr/ark:/12148/cb123339672}{cat. gén. BNF}, \href{https://data.bnf.fr/ark:/12148/cb123339672}{data.bnf}, \href{https://www.idref.fr/032279493}{IdRéf}, \href{https://viaf.org/viaf/59155489}{VIAF}, \href{https://www.wikidata.org/wiki/Q174380}{Wikidata}]}}
\newglossaryentry{CoEg_pers_00000117}{name={Ptolémée II Philadelphe},sort={Ptolemee02},text={*},type={hist},description={Roi égyptien (309/308-246) [\href{https://viaf.org/viaf/42634080/}{VIAF}, \href{https://www.wikidata.org/wiki/Q39576}{Wikidata}]}}
\newglossaryentry{CoEg_pers_00000118}{name={Horapollon},text={*},type={hist},description={Écrivain égyptien [\href{https://catalogue.bnf.fr/ark:/12148/cb124596093}{cat. gén. BNF}, \href{https://data.bnf.fr/ark:/12148/cb124596093}{data.bnf}, \href{https://www.idref.fr/033762708}{IdRéf}, \href{https://viaf.org/viaf/250688796}{VIAF}, \href{https://www.wikidata.org/wiki/Q364379}{Wikidata}]}}
\newglossaryentry{CoEg_pers_00000119}{name={Paléphate},text={*},type={hist},description={Écrivain grec [\href{https://catalogue.bnf.fr/ark:/12148/cb12554608q}{cat. gén. BNF}, \href{https://data.bnf.fr/ark:/12148/cb12554608q}{data.bnf}, \href{https://www.idref.fr/034835563}{IdRéf}, \href{https://viaf.org/viaf/100181721}{VIAF}, \href{https://www.wikidata.org/wiki/Q580565}{Wikidata}]}}
\newglossaryentry{CoEg_pers_00000122}{name={Anacréon},text={*},type={hist},description={Poète grec (v. 550-v. 464) [\href{https://catalogue.bnf.fr/ark:/12148/cb11888717x}{cat. gén. BNF}, \href{https://data.bnf.fr/ark:/12148/cb11888717x}{data.bnf}, \href{https://www.idref.fr/032840225}{IdRéf}, \href{https://viaf.org/viaf/100165204/}{VIAF}, \href{https://www.wikidata.org/wiki/Q213484}{Wikidata}]}}
\newglossaryentry{CoEg_pers_00000123}{name={Estienne, Henri II},text={*},type={hist},description={Imprimeur et humaniste (1528/1531-1598) [\href{https://catalogue.bnf.fr/ark:/12148/cb119019504}{cat. gén. BNF}, \href{https://data.bnf.fr/ark:/12148/cb119019504}{data.bnf}, \href{https://www.idref.fr/085948276}{IdRéf}, \href{https://viaf.org/viaf/79024744}{VIAF}, \href{https://www.wikidata.org/wiki/Q664748}{Wikidata}]}}
\newglossaryentry{CoEg_pers_00000126}{name={Assemani, Giuseppe Simone},text={*},type={hist},description={Ecclésiastique orientaliste (1687-1768) [\href{https://catalogue.bnf.fr/ark:/12148/cb123705494}{cat. gén. BNF}, \href{https://data.bnf.fr/ark:/12148/cb123705494}{data.bnf}, \href{https://www.idref.fr/06706017X}{IdRéf}, \href{https://viaf.org/viaf/9927564}{VIAF}, \href{https://www.wikidata.org/wiki/Q554853}{Wikidata}]}}
\newglossaryentry{CoEg_pers_00000127}{name={Assemani, Stefano Evodio},text={*},type={hist},description={Ecclésiastique orientaliste (1711-1782) [\href{https://catalogue.bnf.fr/ark:/12148/cb12071504w}{cat. gén. BNF}, \href{https://data.bnf.fr/ark:/12148/cb12071504w}{data.bnf}, \href{https://www.idref.fr/028995597}{IdRéf}, \href{https://viaf.org/viaf/39731448}{VIAF}, \href{https://www.wikidata.org/wiki/Q711620}{Wikidata}]}}
\newglossaryentry{CoEg_pers_00000128}{name={Clément XI},text={*},type={hist},description={Pape (1649-1721) [\href{https://catalogue.bnf.fr/ark:/12148/cb118970158}{cat. gén. BNF}, \href{https://data.bnf.fr/ark:/12148/cb118970158}{data.bnf}, \href{https://www.idref.fr/069632200}{IdRéf}, \href{https://viaf.org/viaf/100251489}{VIAF}, \href{https://www.wikidata.org/wiki/Q129967}{Wikidata}]}}
\newglossaryentry{CoEg_pers_00000145}{name={Alexandre le Grand},type={hist},text={*},description={Roi macédonien et conquérant à succès (356-323) [\href{https://catalogue.bnf.fr/ark:/12148/cb11946296j}{cat. gén. BNF}, \href{https://data.bnf.fr/ark:/12148/cb11946296j}{data.bnf}, \href{https://www.idref.fr/027417077}{IdRéf}, \href{https://viaf.org/viaf/88742742/}{VIAF}, \href{https://www.wikidata.org/wiki/Q8409}{Wikidata}]}}
\newglossaryentry{CoEg_pers_00000147}{name={Pythagore},text={*},type={hist},description={Philosophe grec (v. 580-v. 495) [\href{https://catalogue.bnf.fr/ark:/12148/cb11920816v}{cat. gén. BNF}, \href{https://eye.bnf.fr/ark:/12148/cb11920816v}{data.bnf}, \href{https://www.idref.fr/027085260}{IdRéf}, \href{https://viaf.org/viaf/162237897/}{VIAF}, \href{https://www.wikidata.org/wiki/Q10261}{Wikidata}]}}
\newglossaryentry{CoEg_pers_00000148}{name={Solon},text={*},type={hist},description={Homme d'État grec (v. 640-v. 558) [\href{https://catalogue.bnf.fr/ark:/12148/cb13091860r}{cat. gén. BNF}, \href{https://data.bnf.fr/ark:/12148/cb13091860r}{data.bnf}, \href{https://www.idref.fr/028829832}{IdRéf}, \href{https://viaf.org/viaf/81144814374407675870/}{VIAF}, \href{https://www.wikidata.org/wiki/Q133337}{Wikidata}]}}
\newglossaryentry{CoEg_pers_00000149}{name={Lajard, Félix},text={*},type={hist},description={ (1783-1858) [\href{https://catalogue.bnf.fr/ark:/12148/cb10692736g}{cat. gén. BNF}, \href{https://data.bnf.fr/ark:/12148/cb10692736g}{data.bnf}, \href{https://www.idref.fr/112353053}{IdRéf}, \href{https://viaf.org/viaf/769603}{VIAF}, \href{https://www.wikidata.org/wiki/Q8313314}{Wikidata}]}}
\newglossaryentry{CoEg_pers_00000155}{name={Nebrê},text={*},type={hist},description={Roi égyptien (IIe dynastie) Appelé Καιέχως par Manéthon (francisé en « Céchoüs »\gls{CoEg_pers_00000155alias}) [\href{https://www.wikidata.org/wiki/Q152751}{Wikidata}]}}
\newglossaryentry{CoEg_pers_00000155alias}{name={Céchoüs},text={},type={hist},sort={Cechous},description={Voir « Nebrê »}}
\newglossaryentry{CoEg_pers_00000157}{name={Jules l'Africain},text={*},type={hist},description={Historien grec (v. 160-v. 240) alias Sextus Julius Africanus[\href{https://catalogue.bnf.fr/ark:/12148/cb12163447b}{cat. gén. BNF}, \href{https://data.bnf.fr/ark:/12148/cb12163447b}{data.bnf}, \href{https://www.idref.fr/030162262}{IdRéf}, \href{https://viaf.org/viaf/160788177}{VIAF}, \href{https://www.wikidata.org/wiki/Q317014}{Wikidata}]}}
\newglossaryentry{CoEg_pers_00000158}{name={Georges le Syncelle},text={*},type={hist},description={Ecclésiastique et chroniqueur byzantin [\href{https://catalogue.bnf.fr/ark:/12148/cb119914769}{cat. gén. BNF}, \href{https://data.bnf.fr/ark:/12148/cb119914769}{data.bnf}, \href{https://www.idref.fr/027988627}{IdRéf}, \href{https://viaf.org/viaf/39386582}{VIAF}, \href{https://www.wikidata.org/wiki/Q366382}{Wikidata}]}}
\newglossaryentry{CoEg_pers_00000159}{name={Strabon},type={hist},text={*},description={Géographe et historien grec (v. 60-v. 20) [\href{https://catalogue.bnf.fr/ark:/12148/cb11925629v}{cat. gén. BNF}, \href{https://data.bnf.fr/ark:/12148/cb11925629v}{data.bnf}, \href{https://www.idref.fr/027150062}{IdRéf}, \href{https://viaf.org/viaf/100219883/}{VIAF}, \href{https://www.wikidata.org/wiki/Q45936}{Wikidata}]}}
\newglossaryentry{CoEg_pers_00000161}{name={Mykérinos},text={*},type={hist},description={Roi égyptien (IV\textsuperscript{e} dynastie) [\href{https://viaf.org/viaf/16553808}{VIAF}, \href{https://www.wikidata.org/wiki/Q200986}{Wikidata}]}}
\newglossaryentry{CoEg_pers_00000162}{name={Snéfrou},text={*},type={hist},description={Roi égyptien (IV\textsuperscript{e} dynastie [\href{https://viaf.org/viaf/53177422}{VIAF}, \href{https://www.wikidata.org/wiki/Q189371}{Wikidata}]}}
\newglossaryentry{CoEg_pers_00000163}{name={Hapy},type={hist},text={*},description={}}
\newglossaryentry{CoEg_pers_00000165}{name={Pline l'Ancien},text={*},type={hist},description={Écricain latin (23-79) [\href{https://catalogue.bnf.fr/ark:/12148/cb11887294r}{cat. gén. BNF}, \href{https://data.bnf.fr/ark:/12148/cb11887294r}{data.bnf}, \href{https://www.idref.fr/026668149}{IdRéf}, \href{https://viaf.org/viaf/100219162}{VIAF}, \href{https://www.wikidata.org/wiki/Q82778}{Wikidata}]}}
\newglossaryentry{CoEg_pers_00000166}{name={Ammien Marcellin},type={hist},text={*},description={Historien latin (v. 330-v. 395) [\href{https://catalogue.bnf.fr/ark:/12148/cb118886885}{cat. gén. BNF}, \href{https://data.bnf.fr/ark:/12148/cb118886885}{data.bnf}, \href{https://www.idref.fr/026684942}{IdRéf}, \href{https://viaf.org/viaf/89594750/}{VIAF}, \href{https://www.wikidata.org/wiki/Q172198}{Wikidata}]}}
\newglossaryentry{CoEg_pers_00000167}{name={Suétone},type={hist},text={*},description={Biographe latin [\href{https://catalogue.bnf.fr/ark:/12148/cb119257294}{cat. gén. BNF}, \href{https://data.bnf.fr/ark:/12148/cb119257294}{data.bnf}, \href{https://www.idref.fr/027151387}{IdRéf}, \href{https://viaf.org/viaf/89599270/}{VIAF}, \href{https://www.wikidata.org/wiki/Q10133}{Wikidata}]}}
\newglossaryentry{CoEg_pers_00000168}{name={Germanicus},type={hist},text={*},description={Général romain (15 av. J.-C.-19) [\href{https://catalogue.bnf.fr/ark:/12148/cb11904724r}{cat. gén. BNF}, \href{https://data.bnf.fr/ark:/12148/cb11904724r}{data.bnf}, \href{https://www.idref.fr/02688903X}{IdRéf}, \href{https://viaf.org/viaf/56580498}{VIAF}, \href{https://www.wikidata.org/wiki/Q191039}{Wikidata}]}}
\newglossaryentry{CoEg_pers_00000169}{name={Spartianus},type={hist},text={*},description={Écrivain latin [\href{https://catalogue.bnf.fr/ark:/12148/cb104864618}{cat. gén. BNF}, \href{https://data.bnf.fr/ark:/12148/cb104864618}{data.bnf}, \href{https://www.idref.fr/130559571}{IdRéf}, \href{https://viaf.org/viaf/14759878}{VIAF}, \href{https://www.wikidata.org/wiki/Q1230624}{Wikidata}]}}
\newglossaryentry{CoEg_pers_00000170}{name={Titus},type={hist},text={*},description={Empereur romain (39-81) [\href{https://catalogue.bnf.fr/ark:/12148/cb11972355w}{cat. gén. BNF}, \href{https://data.bnf.fr/ark:/12148/cb11972355w}{data.bnf}, \href{https://www.idref.fr/027746887}{IdRéf}, \href{https://viaf.org/viaf/83217593/}{VIAF}, \href{https://www.wikidata.org/wiki/Q1421}{Wikidata}]}}
\newglossaryentry{CoEg_pers_00000171}{name={Hadrien},type={hist},text={*},description={Empereur romain (76-138) [\href{https://catalogue.bnf.fr/ark:/12148/cb11954875z}{cat. gén. BNF}, \href{https://data.bnf.fr/ark:/12148/cb11954875z}{data.bnf}, \href{https://www.idref.fr/027527530}{IdRéf}, \href{https://viaf.org/viaf/82440741/}{VIAF}, \href{https://www.wikidata.org/wiki/Q1427}{Wikidata}]}}
\newglossaryentry{CoEg_pers_00000172}{name={Julien l'Apostat},type={hist},text={*},description={Empereur romain (331/332-363) [\href{https://catalogue.bnf.fr/ark:/12148/cb121721673}{cat. gén. BNF}, \href{https://data.bnf.fr/ark:/12148/cb121721673}{data.bnf}, \href{https://www.idref.fr/028722620}{IdRéf}, \href{https://viaf.org/viaf/57406701/}{VIAF}, \href{https://www.wikidata.org/wiki/Q33941}{Wikidata}]}}
\newglossaryentry{CoEg_pers_00000173}{name={Théodose},type={hist},text={*},description={Empereur romain (347-379) [\href{https://catalogue.bnf.fr/ark:/12148/cb12648653g}{cat. gén. BNF}, \href{https://data.bnf.fr/ark:/12148/cb12648653g}{data.bnf}, \href{https://www.idref.fr/050559109}{IdRéf}, \href{https://viaf.org/viaf/31978444/}{VIAF}, \href{https://www.wikidata.org/wiki/Q46696}{Wikidata}]}}
\newglossaryentry{CoEg_pers_00000175}{name={Tochon, Joseph-François},text={*},type={hist},description={Dit « Tochon d'Annecy ». Homme politique et antiquaire (1772-1820) [\href{https://catalogue.bnf.fr/ark:/12148/cb16556232h}{cat. gén. BNF}, \href{https://data.bnf.fr/ark:/12148/cb16556232h}{data.bnf}, \href{https://www.idref.fr/101278586}{IdRéf}, \href{https://viaf.org/viaf/38017994}{VIAF}, \href{https://www.wikidata.org/wiki/Q3184268}{Wikidata}]}}
\newglossaryentry{CoEg_pers_00000176}{name={Solin},type={hist},text={*},description={Écrivain latin [\href{https://catalogue.bnf.fr/ark:/12148/cb12459920f}{cat. gén. BNF}, \href{https://data.bnf.fr/ark:/12148/cb12459920f}{data.bnf}, \href{https://www.idref.fr/033768986}{IdRéf}, \href{https://viaf.org/viaf/51672454/}{VIAF}, \href{https://www.wikidata.org/wiki/Q520487}{Wikidata}]}}
\newglossaryentry{CoEg_pers_00000177}{name={Porphyre de Tyr},text={*},type={hist},description={Philosophe grec et latin (234-v. 310) [\href{https://catalogue.bnf.fr/ark:/12148/cb120084710}{cat. gén. BNF}, \href{https://data.bnf.fr/ark:/12148/cb120084710}{data.bnf}, \href{https://www.idref.fr/027435873}{IdRéf}, \href{https://viaf.org/viaf/64016141/}{VIAF}, \href{https://www.wikidata.org/wiki/Q203445}{Wikidata}]}}
\newglossaryentry{CoEg_pers_00000178}{name={Pomponius Mela},text={*},type={hist},description={Géographe latin [\href{https://catalogue.bnf.fr/ark:/12148/cb121174358}{cat. gén. BNF}, \href{https://data.bnf.fr/ark:/12148/cb121174358}{data.bnf}, \href{https://www.idref.fr/029578019}{IdRéf}, \href{https://viaf.org/viaf/34489209}{VIAF}, \href{https://www.wikidata.org/wiki/Q297515}{Wikidata}]}}
\newglossaryentry{CoEg_pers_00000180}{name={Élien le Sophiste},text={*},type={hist},sort={Elien},description={Écrivain grec [\href{https://catalogue.bnf.fr/ark:/12148/cb13092007b}{cat. gén. BNF}, \href{https://data.bnf.fr/ark:/12148/cb13092007b}{data.bnf}, \href{https://www.idref.fr/030060087}{IdRéf}, \href{https://viaf.org/viaf/100219416}{VIAF}, \href{https://www.wikidata.org/wiki/Q313782}{Wikidata}]}}
\newglossaryentry{CoEg_pers_00000181}{name={Ahmès},type={hist},text={*},description={Propriétaire original d'un sarcophage\gls{CoEg_obj_imn} égyptien du musée\gls{CoEg_org_00000040} de Berlin}}
\newglossaryentry{CoEg_pers_00000182}{name={Ahmès},type={hist},text={*},description={Roi égyptien (XVIII\textsuperscript{e} dynastie) [\href{https://viaf.org/viaf/59882356/}{VIAF}, \href{https://www.wikidata.org/wiki/Q7222}{Wikidata}]}}
\newglossaryentry{CoEg_pers_00000183}{name={Pétisis},type={hist},text={*},description={}}
\newglossaryentry{CoEg_pers_00000184}{name={Jablonski, Paul Ernest},type={hist},text={*},description={Orientaliste (1693-1757) [\href{https://catalogue.bnf.fr/ark:/12148/cb12546975z}{cat. gén. BNF}, \href{https://data.bnf.fr/ark:/12148/cb12546975z}{data.bnf}, \href{https://www.idref.fr/067721117}{IdRéf}, \href{https://viaf.org/viaf/34571501}{VIAF}, \href{https://www.wikidata.org/wiki/Q110313}{Wikidata}]}}
\newglossaryentry{CoEg_pers_00000185}{name={Ahmès fils d'Abana},type={hist},text={*},description={Chef des rameurs sous Ahmès I\textsuperscript{er}\gls{CoEg_pers_00000182} (XVIII\textsuperscript{e} dynastie), enterré à El-Kab\gls{CoEg_place_00000073} [\href{https://viaf.org/viaf/24239867/}{VIAF}, \href{https://www.wikidata.org/wiki/Q402060}{Wikidata}]}}
\newglossaryentry{CoEg_pers_00000187}{name={Ptolémée VIII Évergète II},text={*},sort={Ptolemee08},type={hist},description={Roi égyptien (182-116) [\href{https://viaf.org/viaf/155419788/}{VIAF}, \href{https://www.wikidata.org/wiki/Q3350}{Wikidata}]}}
\newglossaryentry{CoEg_pers_00000193}{name={Dodwell, Henry},type={hist},text={*},description={Érudit (1641-1711) [\href{https://catalogue.bnf.fr/ark:/12148/cb14506115b}{cat. gén. BNF}, \href{https://data.bnf.fr/ark:/12148/cb14506115b}{data.bnf}, \href{https://www.idref.fr/123231558}{IdRéf}, \href{https://viaf.org/viaf/19912139}{VIAF}, \href{https://www.wikidata.org/wiki/Q3132744}{Wikidata}]}}
\newglossaryentry{CoEg_pers_00000195}{name={Osorkon II},type={hist},text={*},description={Roi égyptien (XXII\textsuperscript{e} dynastie) [\href{https://viaf.org/viaf/107787928}{VIAF}, \href{https://www.wikidata.org/wiki/Q459153}{Wikidata}]}}
\newglossaryentry{CoEg_pers_00000198}{name={Chéchonq IV},type={hist},text={*},sort={Chéchonq4},description={Roi égyptien (XXIII\textsuperscript{e} dynastie) [\href{https://www.wikidata.org/wiki/Q781765}{Wikidata}]}}
\newglossaryentry{CoEg_pers_00000201}{name={Ioufânkh},type={hist},text={*},description={Mariette emploie la forme « Aufankh »\gls{CoEg_pers_00000201alias}}}
\newglossaryentry{CoEg_pers_00000201alias}{name={Aufankh},type={hist},text={},description={Voir « Ioufânkh »}}
\newglossaryentry{CoEg_pers_00000202}{name={Macrobe},type={hist},text={*},description={[\href{https://catalogue.bnf.fr/ark:/12148/cb11999201s}{cat. gén. BNF}, \href{https://data.bnf.fr/ark:/12148/cb11999201s}{data.bnf}, \href{https://www.idref.fr/03092166X}{IdRéf}, \href{https://viaf.org/viaf/39387062}{VIAF}, \href{https://www.wikidata.org/wiki/Q313934}{Wikidata}]}}
\newglossaryentry{CoEg_pers_00000203}{name={Ptolémée XV Césarion},sort={Ptolemee15},type={hist},text={*},description={Roi égyptien (47-30)\gls{CoEg_pers_00000203alias} [\href{https://viaf.org/viaf/67910474/}{VIAF}, \href{https://www.wikidata.org/wiki/Q39589}{Wikidata}]}}
\newglossaryentry{CoEg_pers_00000203alias}{name={Césarion},type={hist},text={},description={Voir « Ptolémée XV Césarion »}}
\newglossaryentry{CoEg_pers_00000204}{name={Pausanias},type={hist},text={*},description={[\href{https://catalogue.bnf.fr/ark:/12148/cb11918834h}{cat. gén. BNF}, \href{https://data.bnf.fr/ark:/12148/cb11918834h}{data.bnf}, \href{https://www.idref.fr/028159969}{IdRéf}, \href{https://viaf.org/viaf/77132959}{VIAF}, \href{https://www.wikidata.org/wiki/Q192931}{Wikidata}]}}
\newglossaryentry{CoEg_pers_00000205}{name={César},type={hist},text={*},description={Homme d'État romain (100-44) [\href{https://catalogue.bnf.fr/ark:/12148/cb11894764p}{cat. gén. BNF}, \href{https://data.bnf.fr/ark:/12148/cb11894764p}{data.bnf}, \href{https://www.idref.fr/027303136}{IdRéf}, \href{https://viaf.org/viaf/286265178/}{VIAF}, \href{https://www.wikidata.org/wiki/Q1048}{Wikidata}]}}
\newglossaryentry{CoEg_pers_00000206}{name={Cléopâtre VII},type={hist},text={*},description={Reine égyptienne (v. 69-30) [\href{https://catalogue.bnf.fr/ark:/12148/cb11938532d}{cat. gén. BNF}, \href{https://data.bnf.fr/ark:/12148/cb11938532d}{data.bnf}, \href{https://www.idref.fr/027316564}{IdRéf}, \href{https://viaf.org/viaf/67762941/}{VIAF}, \href{https://www.wikidata.org/wiki/Q635}{Wikidata}]}}
\newglossaryentry{CoEg_pers_00000213}{name={Ptolémée I\textsuperscript{er} Sôter},type={hist},sort={Ptolemee01},text={*},description={Roi égyptien (v. 368-283) [\href{https://catalogue.bnf.fr/ark:/12148/cb12473505d}{cat. gén. BNF}, \href{https://data.bnf.fr/ark:/12148/cb12473505d}{data.bnf}, \href{https://www.idref.fr/066982855}{IdRéf}, \href{https://viaf.org/viaf/21347488/}{VIAF}, \href{https://www.wikidata.org/wiki/Q168261}{Wikidata}]}}
\newglossaryentry{CoEg_pers_00000214}{name={Timothée l’interprète},type={hist},text={*},description={}}
\newglossaryentry{CoEg_pers_00000216}{name={Denys le Périégète},type={hist},text={*},description={Écrivain grec (v. 115-v. 180) [\href{https://catalogue.bnf.fr/ark:/12148/cb121512771}{cat. gén. BNF}, \href{https://data.bnf.fr/ark:/12148/cb121512771}{data.bnf}, \href{https://www.idref.fr/03000764X}{IdRéf}, \href{https://viaf.org/viaf/100902699}{VIAF}, \href{https://www.wikidata.org/wiki/Q1226993}{Wikidata}]}}
\newglossaryentry{CoEg_pers_00000217}{name={Cyrille d’Alexandrie},type={hist},text={*},description={Ecclésiastique (v. 375-444) [\href{https://catalogue.bnf.fr/ark:/12148/cb118983578}{cat. gén. BNF}, \href{https://data.bnf.fr/ark:/12148/cb118983578}{data.bnf}, \href{https://www.idref.fr/026808048}{IdRéf}, \href{https://viaf.org/viaf/190218431/}{VIAF}, \href{https://www.wikidata.org/wiki/Q44079}{Wikidata}]}}
\newglossaryentry{CoEg_pers_00000218}{name={Auguste},type={hist},text={*},description={Empereur romain (63-14) [\href{https://catalogue.bnf.fr/ark:/12148/cb12326866z}{cat. gén. BNF}, \href{https://data.bnf.fr/ark:/12148/cb12326866z}{data.bnf}, \href{https://www.idref.fr/032190026}{IdRéf}, \href{https://viaf.org/viaf/18013086/}{VIAF}, \href{https://www.wikidata.org/wiki/Q1405}{Wikidata}]}}
\newglossaryentry{CoEg_pers_00000219}{name={Trajan},type={hist},text={*},description={Empereur romain (53-117) [\href{https://catalogue.bnf.fr/ark:/12148/cb13164032k}{cat. gén. BNF}, \href{https://data.bnf.fr/ark:/12148/cb13164032k}{data.bnf}, \href{https://www.idref.fr/081821417}{IdRéf}, \href{https://viaf.org/viaf/88067472/}{VIAF}, \href{https://www.wikidata.org/wiki/Q1425}{Wikidata}]}}
\newglossaryentry{CoEg_pers_00000220}{name={Commode},type={hist},text={*},description={Empereur romain (161-192) [\href{https://catalogue.bnf.fr/ark:/12148/cb14455354d}{cat. gén. BNF}, \href{https://data.bnf.fr/ark:/12148/cb14455354d}{data.bnf}, \href{https://www.idref.fr/061621293}{IdRéf}, \href{https://viaf.org/viaf/23502412/}{VIAF}, \href{https://www.wikidata.org/wiki/Q1434}{Wikidata}]}}
\newglossaryentry{CoEg_pers_00000221}{name={Gallien},type={hist},text={*},description={Empereur romain (v. 218-268) [\href{https://catalogue.bnf.fr/ark:/12148/cb131627817}{cat. gén. BNF}, \href{https://data.bnf.fr/ark:/12148/cb131627817}{data.bnf}, \href{https://www.idref.fr/027381331}{IdRéf}, \href{https://viaf.org/viaf/10637212/}{VIAF}, \href{https://www.wikidata.org/wiki/Q104475}{Wikidata}]}}
\newglossaryentry{CoEg_pers_00000222}{name={Ptolémée VI Philométor},type={hist},sort={Ptolemee06},text={*},description={Roi égyptien (dynastie lagide) (186-145) [\href{https://viaf.org/viaf/67258070}{VIAF}, \href{https://www.wikidata.org/wiki/Q39952}{Wikidata}]}}
\newglossaryentry{CoEg_pers_00000228}{name={Nicocréon},type={hist},text={*},description={Roi chypriote [\href{https://www.wikidata.org/wiki/Q181585}{Wikidata}]}}
\newglossaryentry{CoEg_pers_00000229}{name={Jamblique},type={hist},text={*},description={Philosophe grec (v. 250-v. 330) [\href{https://catalogue.bnf.fr/ark:/12148/cb11908535t}{cat. gén. BNF}, \href{https://data.bnf.fr/ark:/12148/cb11908535t}{data.bnf}, \href{https://www.idref.fr/026934523}{IdRéf}, \href{https://viaf.org/viaf/12310254/}{VIAF}, \href{https://www.wikidata.org/wiki/Q310916}{Wikidata}]}}
\newglossaryentry{CoEg_pers_00000231}{name={Plotin},type={hist},text={*},description={Philosophe grec (205-270) [\href{https://catalogue.bnf.fr/ark:/12148/cb11887296f}{cat. gén. BNF}, \href{https://data.bnf.fr/ark:/12148/cb11887296f}{data.bnf}, \href{https://www.idref.fr/026668165}{IdRéf}, \href{https://viaf.org/viaf/108386765/}{VIAF}, \href{https://www.wikidata.org/wiki/Q134189}{Wikidata}]}}
\newglossaryentry{CoEg_pers_00000235}{name={Jésus de Nazareth},text={*},type={hist},description={[\href{https://www.idref.fr/027325636}{IdRéf}, \href{https://viaf.org/viaf/73945424}{VIAF}, \href{https://www.wikidata.org/wiki/Q51666}{Wikidata}]}}
\newglossaryentry{CoEg_pers_00000236}{name={Nabuchodonosor II},text={*},type={hist},description={Roi babylonien [\href{https://www.idref.fr/073929395}{IdRéf}, \href{https://viaf.org/viaf/67850417}{VIAF}, \href{https://www.wikidata.org/wiki/Q12591}{Wikidata}]}}
\newglossaryentry{CoEg_pers_00000242}{name={Théophile d'Antioche},text={*},type={hist},description={Écrivain grec [\href{https://catalogue.bnf.fr/ark:/12148/cb130917149}{cat. gén. BNF}, \href{https://data.bnf.fr/ark:/12148/cb130917149}{data.bnf}, \href{https://www.idref.fr/02797863X}{IdRéf}, \href{https://viaf.org/viaf/164007222/}{VIAF}, \href{https://www.wikidata.org/wiki/Q220787}{Wikidata}]}}

%Figures mythologiques ou religieuses

\newglossaryentry{CoEg_pers_00000011}{name={Apis},type={myth},text={*},description={Taureau sacré de Ptah à Memphis [\href{https://catalogue.bnf.fr/ark:/12148/cb124694130}{cat. gén. BNF}, \href{https://data.bnf.fr/ark:/12148/cb124694130}{data.bnf}, \href{https://www.idref.fr/033882436}{IdRef}, \href{https://www.wikidata.org/wiki/Q208150}{Wikidata}]}}
\newglossaryentry{CoEg_pers_00000012}{name={Bès},type={myth},text={*},description={Dieu égyptien. Mariette utilise la désignation grecque «~Typhon\gls{CoEg_pers_00000012alias}~»\footnote{Voir \textsc{Rougé (de)}, Emmanuel, \textit{Notice sommaire des monuments égyptiens exposés dans les galeries du musée du Louvre}\gls{CoEg_bibl_00000049}, Paris, Simon Raçon et C\textsuperscript{ie}, 1855, p.~\href{https://bibliotheque-numerique.inha.fr/idviewer/12321/59}{59}.} [\href{https://catalogue.bnf.fr/ark:/12148/cb16717847z}{cat. gén. BNF}, \href{https://data.bnf.fr/ark:/12148/cb16717847z}{data.bnf}, \href{https://www.idref.fr/131833847}{IdRéf}, \href{https://viaf.org/viaf/15612609}{VIAF}, \href{https://www.wikidata.org/wiki/Q188931}{Wikidata}]}}
\newglossaryentry{CoEg_pers_00000012alias}{name={Typhon},type={myth},text={},description={Voir «~Bès~»}}
\newglossaryentry{CoEg_pers_00000036}{name={Dieu},type={myth},text={*},description={Voir «~Allah~» [\href{https://www.wikidata.org/wiki/Q190}{Wikidata}]}}
\newglossaryentry{CoEg_pers_00000042}{name={Osiris},type={myth},text={*},description={Dieu égyptien [\href{https://catalogue.bnf.fr/ark:/12148/cb11936736v}{cat. gén. BNF}, \href{https://data.bnf.fr/ark:/12148/cb11936736v}{data.bnf}, \href{https://viaf.org/viaf/61789844}{VIAF}, \href{https://www.wikidata.org/wiki/Q46491}{Wikidata}]}}
\newglossaryentry{CoEg_pers_00000047}{name={Ptah},type={myth},text={*},description={Dieu égyptien. Patron de Memphis \gls{CoEg_pers_00000047alias} [\href{https://catalogue.bnf.fr/ark:/12148/cb11980873x}{cat. gén. BNF}, \href{https://data.bnf.fr/ark:/12148/cb11980873x}{data.bnf}, \href{https://viaf.org/viaf/15571376}{VIAF}, \href{https://www.wikidata.org/wiki/Q146321}{Wikidata}]}}
\newglossaryentry{CoEg_pers_00000047alias}{name={Phtah},type={myth},text={},description={Voir «~Ptah~»}}
\newglossaryentry{CoEg_pers_00000060}{name={Sérapis},type={myth},text={*},description={Dieu égyptien [\href{https://catalogue.bnf.fr/ark:/12148/cb119726086}{cat. gén. BNF}, \href{https://data.bnf.fr/ark:/12148/cb119726086}{data.bnf}, \href{http://viaf.org/viaf/316434442}{VIAF}, \href{https://www.wikidata.org/wiki/Q214554}{Wikidata}]}}
\newglossaryentry{CoEg_pers_00000068}{name={Anubis},type={myth},text={*},description={Dieu égyptien [\href{https://catalogue.bnf.fr/ark:/12148/cb12264465p}{cat. gén. BNF}, \href{https://data.bnf.fr/ark:/12148/cb12264465p}{data.bnf}, \href{https://www.idref.fr/031428800}{IdRéf}, \href{https://viaf.org/viaf/5144782931443628507}{VIAF}, \href{https://www.wikidata.org/wiki/Q47534}{Wikidata}]}}
\newglossaryentry{CoEg_pers_00000072}{name={Allah},type={myth},text={*},description={Voir «~Dieu~» [\href{https://www.wikidata.org/wiki/Q234801}{Wikidata}]}}

\newglossaryentry{CoEg_pers_00000099}{name={Moïse},type={myth},text={*},description={Personnage biblique [\href{https://viaf.org/viaf/805492}{VIAF}, \href{https://www.wikidata.org/wiki/Q9077}{Wikidata}]}}
\newglossaryentry{CoEg_pers_00000100}{name={Abraham},type={myth},text={*},description={Personnage biblique [\href{https://catalogue.bnf.fr/ark:/12148/cb165820767}{cat. gén. BNF}, \href{https://data.bnf.fr/ark:/12148/cb165820767}{data.bnf}, \href{https://viaf.org/viaf/89660956}{VIAF}, \href{https://www.wikidata.org/wiki/Q9181}{Wikidata}]}}
\newglossaryentry{CoEg_pers_00000101}{name={Cadmus},type={myth},text={*},description={Héros grec. Fondateur de Thèbes (Boétie), à qui l'on attribue l'introduction de l'alphabet phénicien en Grèce [\href{https://catalogue.bnf.fr/ark:/12148/cb15060389h}{cat. gén. BNF}, \href{https://data.bnf.fr/ark:/12148/cb15060389h}{data.bnf}, \href{https://www.idref.fr/111293693}{IdRéf}, \href{https://viaf.org/viaf/208236507/}{VIAF}, \href{https://www.wikidata.org/wiki/Q27613}{Wikidata}]}}
\newglossaryentry{CoEg_pers_00000102}{name={Josué},type={myth},text={*},description={Personnage biblique [\href{https://catalogue.bnf.fr/ark:/12148/cb170211242}{cat. gén. BNF}, \href{https://data.bnf.fr/ark:/12148/cb170211242}{data.bnf}, \href{https://viaf.org/viaf/4580147270512135700006}{VIAF}, \href{https://www.wikidata.org/wiki/Q7734}{Wikidata}]}}
\newglossaryentry{CoEg_pers_00000120}{name={Hermès Trismégiste},type={myth},text={*},description={Personnage mythique	[\href{https://catalogue.bnf.fr/ark:/12148/cb12057248q}{cat. gén. BNF}, \href{https://data.bnf.fr/ark:/12148/cb12057248q}{data.bnf}, \href{https://www.idref.fr/028814185}{IdRéf}, \href{https://viaf.org/viaf/24571510}{VIAF}, \href{https://www.wikidata.org/wiki/Q192358}{Wikidata}]}}
\newglossaryentry{CoEg_pers_00000121}{name={Darès le Phrygien},type={myth},text={*},description={Personnage homérique [\href{https://catalogue.bnf.fr/ark:/12148/cb12215749g}{cat. gén. BNF}, \href{https://data.bnf.fr/ark:/12148/cb12215749g}{data.bnf}, \href{https://www.idref.fr/030818672}{IdRéf}, \href{https://viaf.org/viaf/55518848}{VIAF}, \href{https://www.wikidata.org/wiki/Q544378}{Wikidata}]}}
\newglossaryentry{CoEg_pers_00000142}{name={Sérapis},type={myth},text={*},description={Dieu égyptien	[\href{https://www.idref.fr/027750027}{IdRéf}, \href{https://viaf.org/viaf/805519}{VIAF}, \href{https://www.wikidata.org/wiki/Q214554}{Wikidata}]}}
\newglossaryentry{CoEg_pers_00000150}{name={Mithra},type={myth},text={*},description={Dieu oriental [\href{https://viaf.org/viaf/72187436}{VIAF}, \href{https://www.wikidata.org/wiki/Q6497135}{Wikidata}]}}
\newglossaryentry{CoEg_pers_00000151}{name={Osiris},type={myth},text={*},description={Dieu égyptien [\href{https://catalogue.bnf.fr/ark:/12148/cb11936736v}{cat. gén. BNF},	\href{https://data.bnf.fr/ark:/12148/cb11936736v}{data.bnf}, \href{https://www.idref.fr/027293807}{IdRéf}, \href{https://viaf.org/viaf/61789844}{VIAF}, \href{https://www.wikidata.org/wiki/Q46491}{Wikidata}]}}
\newglossaryentry{CoEg_pers_00000152}{name={Bacchus},type={myth},text={*},description={Dieu romain [\href{https://www.idref.fr/027394719}{IdRéf}, \href{https://viaf.org/viaf/27864934}{VIAF}, \href{https://www.wikidata.org/wiki/Q645312}{Wikidata}]}}
\newglossaryentry{CoEg_pers_00000153}{name={Jupiter},type={myth},text={*},description={Dieu romain [\href{https://www.idref.fr/027849546}{IdRéf}, \href{https://viaf.org/viaf/22933410}{VIAF}, \href{https://www.wikidata.org/wiki/Q4649}{Wikidata}]}}
\newglossaryentry{CoEg_pers_00000154}{name={Hadès},type={myth},text={*},description={Dieu grec [\href{https://catalogue.bnf.fr/ark:/12148/cb123985871}{cat. gén. BNF}, \href{https://data.bnf.fr/ark:/12148/cb123985871}{data.bnf}, \href{https://www.idref.fr/033083630}{IdRéf}, \href{https://viaf.org/viaf/47684337/}{VIAF}, \href{https://www.wikidata.org/wiki/Q41410}{Wikidata}]}}
\newglossaryentry{CoEg_pers_00000156}{name={Mnévis},type={myth},text={*},description={Taureau sacré égyptien [\href{https://www.idref.fr/241034124}{IdRéf}, \href{https://viaf.org/viaf/38144897957450201849}{VIAF}, \href{https://www.wikidata.org/wiki/Q1067292}{Wikidata}]}}
\newglossaryentry{CoEg_pers_00000186}{name={Vulcain},type={myth},text={*},description={Dieu romain. Voir aussi «~Ptah~» [\href{https://www.idref.fr/027862682}{IdRéf}, \href{https://viaf.org/viaf/42633769}{VIAF}, \href{https://www.wikidata.org/wiki/Q4640}{Wikidata}]}}
\newglossaryentry{CoEg_pers_00000188}{name={Hapy},type={myth},text={*},description={Dieu égyptien [\href{https://viaf.org/viaf/298837669}{VIAF}, \href{https://www.wikidata.org/wiki/Q32143}{Wikidata}]}}
\newglossaryentry{CoEg_pers_00000189}{name={Hathor},type={myth},text={*},description={Déesse égyptienne 	[\href{https://www.idref.fr/030147492}{IdRéf}, \href{https://viaf.org/viaf/63151776729818010780}{VIAF}, \href{https://www.wikidata.org/wiki/Q133343}{Wikidata}]}}
\newglossaryentry{CoEg_pers_00000190}{name={Amon},type={myth},text={*},description={Dieu égyptien [\href{https://www.idref.fr/027398722}{IdRéf}, \href{https://viaf.org/viaf/306390671}{VIAF}, \href{https://www.wikidata.org/wiki/Q58373}{Wikidata}]}}
\newglossaryentry{CoEg_pers_00000191}{name={Sobek},type={myth},text={*},description={Dieu égyptien [\href{https://www.idref.fr/166417173}{IdRéf}, \href{https://viaf.org/viaf/35291212}{VIAF}, \href{https://www.wikidata.org/wiki/Q146313}{Wikidata}]}}
\newglossaryentry{CoEg_pers_00000192}{name={Rê},type={myth},text={*},description={\gls{CoEg_pers_00000192alias} Dieu égyptien (aussi «~Phré~») [\href{https://www.idref.fr/027402568}{IdRéf}, \href{https://viaf.org/viaf/32792135}{VIAF}, \href{https://www.wikidata.org/wiki/Q1252904}{Wikidata}]}}
\newglossaryentry{CoEg_pers_00000192alias}{name={Phré},type={myth},text={},description={Voir «~Rê~»}}
\newglossaryentry{CoEg_pers_00000208}{name={Astarté},type={myth},text={*},description={Déesse orientale [\href{https://www.idref.fr/027454223}{IdRéf}, \href{https://viaf.org/viaf/805767}{VIAF}, \href{https://www.wikidata.org/wiki/Q130274}{Wikidata}]}}
\newglossaryentry{CoEg_pers_00000209}{name={Isis},type={myth},text={*},description={Déesse égyptienne [\href{https://www.idref.fr/027599167}{IdRéf}, \href{https://viaf.org/viaf/313298566/}{VIAF}, \href{https://www.wikidata.org/wiki/Q79876}{Wikidata}]}}
\newglossaryentry{CoEg_pers_00000210}{name={Nephthys},type={myth},text={*},description={Déesse égyptienne [\href{https://www.idref.fr/198066457}{IdRéf}, \href{https://viaf.org/viaf/359149617030003752230}{VIAF}, \href{https://www.wikidata.org/wiki/Q169040}{Wikidata}]}}
\newglossaryentry{CoEg_pers_00000215}{name={Pluton},type={myth},text={*},description={Dieu romain [\href{https://www.wikidata.org/wiki/Q152262}{Wikidata}]}}
\newglossaryentry{CoEg_pers_00000223}{name={Osorapis},type={myth},text={*},description={Forme hybride d'Osiris\gls{CoEg_pers_00000151} et Apis\gls{CoEg_pers_00000011}}}
\newglossaryentry{CoEg_pers_00000224}{name={Dionysos},type={myth},text={*},description={Dieu grec [\href{https://www.idref.fr/02739560X}{IdRéf}, \href{https://viaf.org/viaf/10152079049807110114}{VIAF}, \href{https://www.wikidata.org/wiki/Q41680}{Wikidata}]}}
\newglossaryentry{CoEg_pers_00000225}{name={Ounennéfer},type={myth},text={*},description={Épithète d'Osiris\gls{CoEg_pers_00000151} [\href{https://www.wikidata.org/wiki/Q3358332}{Wikidata}]}}
\newglossaryentry{CoEg_pers_00000230}{name={Jéovah-Elohim},type={myth},text={*},description={Dieu hébraïque [\href{https://viaf.org/viaf/27863181}{VIAF}, \href{https://www.wikidata.org/wiki/Q105173}{Wikidata}]}}
\newglossaryentry{CoEg_pers_00000232}{name={Khnoum},type={myth},text={*},description={\gls{CoEg_pers_00000232alias1} \gls{CoEg_pers_00000232alias2} \gls{CoEg_pers_00000232alias3} Dieu égyptien (aussi «~Chnouphis~», «~Chneph~» ou «~Esneph~») [\href{https://catalogue.bnf.fr/ark:/12148/cb17819771b}{cat. gén. BNF}, \href{https://data.bnf.fr/ark:/12148/cb17819771b}{data.bnf}, \href{https://www.idref.fr/24186478X}{IdRéf}, \href{https://www.wikidata.org/wiki/Q183097}{Wikidata}]}}
\newglossaryentry{CoEg_pers_00000232alias1}{name={Chnouphis},type={myth},text={},description={Voir «~Khnoum~»}}
\newglossaryentry{CoEg_pers_00000232alias2}{name={Chneph},type={myth},text={},description={Voir «~Khnoum~»}}
\newglossaryentry{CoEg_pers_00000232alias3}{name={Esneph},type={myth},text={},description={Voir «~Khnoum~»}}
\newglossaryentry{CoEg_pers_00000233}{name={Thot},type={myth},text={*},description={Dieu égyptien [\href{https://www.idref.fr/030529115}{IdRéf}, \href{https://viaf.org/viaf/54151776828318012681}{VIAF}, \href{https://www.wikidata.org/wiki/Q146921}{Wikidata}]}}
\newglossaryentry{CoEg_pers_00000234}{name={Horus},type={myth},text={*},description={Dieu égyptien [\href{https://www.idref.fr/02757508X}{IdRéf}, \href{https://viaf.org/viaf/69724039}{VIAF}, \href{https://www.wikidata.org/wiki/Q84122}{Wikidata}]}}

%Bateaux

\newglossaryentry{CoEg_pers_00000073}{name={\textit{L'Albatros}},type={boat},text={*},description={Frégate à vapeur française}}

%Institutions

\newglossaryentry{CoEg_org_00000001}{name={Musée nationaux, direction des},type={org},text={*},sort={Musees_nat},description={ [\href{https://catalogue.bnf.fr/ark:/12148/cb118639186}{cat. gén. BNF}, \href{https://data.bnf.fr/ark:/12148/cb118639186}{data.bnf}, \href{http://www.idref.fr/027551148}{IdRéf}, \href{https://viaf.org/viaf/157729592}{VIAF}]}}
\newglossaryentry{CoEg_org_00000002}{name={Musée du Louvre}, type={org},text={*},sort={Musee_Lou},description={ [\gls{CoEg_coll_00000002} ; \href{https://catalogue.bnf.fr/ark:/12148/cb11865019j}{cat. gén. BNF}, \href{https://data.bnf.fr/ark:/12148/cb11865019j}{data.bnf}, \href{https://www.idref.fr/026394677}{IdRéf}, \href{https://viaf.org/viaf/257711507}{VIAF}, \href{https://www.wikidata.org/wiki/Q19675}{Wikidata}]}}
\newglossaryentry{CoEg_org_00000003}{name={Musée du Vatican},type={org},text={*},sort={Musee_Vat},description={ [\href{https://catalogue.bnf.fr/ark:/12148/cb123124629}{cat. gén. BNF}, \href{https://data.bnf.fr/ark:/12148/cb123124629}{data.bnf}, \href{https://www.idref.fr/03201418X}{IdRéf}, \href{https://viaf.org/viaf/133520362/}{VIAF}, \href{https://www.wikidata.org/wiki/Q526381}{Wikidata}]}}
\newglossaryentry{CoEg_org_00000004}{name={Institut de France},type={org},text={*},description={ [\href{https://catalogue.bnf.fr/ark:/12148/cb11864390h}{cat. gén. BNF}, \href{https://data.bnf.fr/ark:/12148/cb11864390h}{data.bnf}, \href{https://www.idref.fr/027262065}{IdRéf}, \href{https://viaf.org/viaf/135715107}{VIAF}, \href{https://www.wikidata.org/wiki/Q377066}{Wikidata}]}}
\newglossaryentry{CoEg_org_00000005}{name={British Museum},type={org},text={*},description={Mariette emploie parfois «~Musée britannique~»\gls{CoEg_org_00000005alias}  [\gls{CoEg_coll_00000005}~; \href{https://catalogue.bnf.fr/ark:/12148/cb11871460b}{cat. gén. BNF}, \href{https://data.bnf.fr/ark:/12148/cb11871460b}{data.bnf}, \href{https://www.idref.fr/026472309}{IdRéf}, \href{https://viaf.org/viaf/134857252}{VIAF}, \href{https://www.wikidata.org/wiki/Q6373}{Wikidata}]}}
\newglossaryentry{CoEg_org_00000005alias}{name={Musée britannique},type={org},text={},sort={Musee_bri},description={Voir «~British Museum~»}}
\newglossaryentry{CoEg_org_00000006}{name={Consulat général de France à Alexandrie},type={org},text={*},description={ [\href{https://www.wikidata.org/wiki/Q16507529}{Wikidata}]}}
\newglossaryentry{CoEg_org_00000007}{name={Ministère français des Affaires étrangères},type={org},text={*},sort={Min_f_Affai},description={\gls{CoEg_org_00000007alias} [\href{https://www.idref.fr/026382490}{IdRéf}, \href{https://viaf.org/viaf/123164749}{VIAF}, \href{https://www.wikidata.org/wiki/Q789848}{Wikidata}]}}
\newglossaryentry{CoEg_org_00000007alias}{name={Affaires étrangères},type={org},text={},description={Voir «~ministère français des Affaires étrangères~»}}
\newglossaryentry{CoEg_org_00000008}{name={Égypte},type={org},text={*},sort={Egypte},description={(en tant qu'État~; voir l'index géographique pour le territoire correspondant). Gouvernement et administration de l'Égypte [\href{https://catalogue.bnf.fr/ark:/12148/cb11863530z}{cat. gén. BNF}, \href{https://data.bnf.fr/ark:/12148/cb11863530z}{data.bnf}, \href{https://www.idref.fr/026375680}{IdRéf}, \href{https://viaf.org/viaf/144801546}{VIAF}, \href{https://www.wikidata.org/wiki/Q79}{Wikidata}]}}
\newglossaryentry{CoEg_org_00000009}{name={Ministère français de l'Intérieur},type={org},text={*},sort={Min_f_Int},description={ \gls{CoEg_org_00000009alias} [\href{https://catalogue.bnf.fr/ark:/12148/cb12470548z}{cat. gén. BNF}, \href{https://data.bnf.fr/ark:/12148/cb12470548z}{data.bnf}, \href{https://www.idref.fr/03389454X}{IdRéf}, \href{https://viaf.org/viaf/148829604}{VIAF}, \href{https://www.wikidata.org/wiki/Q1518496}{Wikidata}]}}
\newglossaryentry{CoEg_org_00000009alias}{name={Intérieur},type={org},text={},description={Voir «~ministère français de l'Intérieur~»}}
\newglossaryentry{CoEg_org_00000010}{name={Ministère français de la Maison de l'empereur},type={org},text={*},sort={Min_f_Maison},description={ \gls{CoEg_org_00000010alias} [\href{https://catalogue.bnf.fr/ark:/12148/cb12132241v}{cat. gén. BNF}, \href{https://data.bnf.fr/ark:/12148/cb12132241v}{data.bnf}, \href{https://www.idref.fr/029769914}{IdRéf}, \href{https://viaf.org/viaf/168298500}{VIAF}]}}
\newglossaryentry{CoEg_org_00000010alias}{name={Maison de l'empereur},type={org},text={},description={Voir «~ministère français de la Maison de l'empereur~»}}
\newglossaryentry{CoEg_org_00000011}{name={Beaux-Arts, administration française des},type={org},text={*},description={Cette administration fut successivement une direction du ministère de l'Intérieur (jusqu'en février 1853), puis une division du ministère d'État (1853-1863), une surintendance du ministère de la Maison de l'empereur (1863-1870) et une direction du ministère de l'Instruction publique (1870-1940) [\href{https://catalogue.bnf.fr/ark:/12148/cb12238953d}{cat. gén. BNF}, \href{https://data.bnf.fr/ark:/12148/cb12238953d}{data.bnf}, \href{https://www.idref.fr/085882216}{IdRéf}, \href{https://viaf.org/viaf/135816220}{VIAF}]}}
\newglossaryentry{CoEg_org_00000012}{name={France},type={org},text={*},description={(en tant qu'État~; voir l'index géographique pour le territoire correspondant). Gouvernement et administration de la France [\href{https://catalogue.bnf.fr/ark:/12148/cb11863754h}{cat. gén. BNF}, \href{https://data.bnf.fr/ark:/12148/cb11863754h}{data.bnf}, \href{https://www.idref.fr/026378329}{IdRéf}, \href{https://viaf.org/viaf/149641640/}{VIAF}, \href{https://www.wikidata.org/wiki/Q1450662}{Wikidata}]}}
\newglossaryentry{CoEg_org_00000013}{name={Armée égyptienne},type={org},text={*},description={ [\href{https://catalogue.bnf.fr/ark:/12148/cb12049968j}{cat. gén. BNF}, \href{https://data.bnf.fr/ark:/12148/cb12049968j}{data.bnf}, \href{https://www.idref.fr/066913020}{IdRéf}, \href{https://www.wikidata.org/wiki/Q1075553}{Wikidata}]}}
\newglossaryentry{CoEg_org_00000014}{name={Royaume-Uni},type={org},text={*},description={(en tant qu'État~; voir l'index géographique pour le territoire correspondant). Gouvernement et administration du Royaume-Uni~; régulièrement appelé abusivement « Angleterre » [\href{https://catalogue.bnf.fr/fr/11872746/grande-bretagne/}{cat. gén. BNF}, \href{https://data.bnf.fr/fr/11872746/grande-bretagne/}{data.bnf}, \href{https://www.idref.fr/026488515}{IdRéf}, \href{https://viaf.org/viaf/127412899}{VIAF}, \href{https://www.wikidata.org/wiki/Q174193}{Wikidata}]}}
\newglossaryentry{CoEg_org_00000015}{name={Société géologique de Londres},type={org},text={*},description={(\textit{Geological Society of London}) [\href{https://catalogue.bnf.fr/ark:/12148/cb12324923g}{cat. gén. BNF}, \href{https://data.bnf.fr/ark:/12148/cb12324923g}{data.bnf}, \href{https://www.idref.fr/032165463}{IdRéf}, \href{https://viaf.org/viaf/130236772}{VIAF}, \href{https://www.wikidata.org/wiki/Q1230936}{Wikidata}]}}
\newglossaryentry{CoEg_org_00000016}{name={Ministère égyptien des Travaux publics},type={org},sort={Min_eg_Trav},text={*},description={ \gls{CoEg_org_00000016alias} [\href{https://catalogue.bnf.fr/ark:/12148/cb16901932w}{cat. gén. BNF}, \href{https://data.bnf.fr/ark:/12148/cb16901932w}{data.bnf}, \href{https://www.idref.fr/140165614}{IdRéf}, \href{https://viaf.org/viaf/217440582}{VIAF}]}}
\newglossaryentry{CoEg_org_00000016alias}{name={Travaux publics},type={org},text={},description={Voir «~ministère égyptien des Travaux publics~»}}
\newglossaryentry{CoEg_org_00000017}{name={Ministère égyptien de l'Instruction publique},type={org},sort={Min_eg_Inst},text={*},description={ \gls{CoEg_org_00000017alias} [\href{https://www.idref.fr/086156683}{IdRéf}, \href{https://viaf.org/viaf/208171813}{VIAF}]}}
\newglossaryentry{CoEg_org_00000017alias}{name={Instruction publique},type={org},text={},description={Voir «~ministère égyptien de l'Instruction publique~»}}
\newglossaryentry{CoEg_org_00000018}{name={Musée ethnographique},type={org},text={*},parent={CoEg_org_00000002}, description={\gls{CoEg_org_00000018alias}}}
\newglossaryentry{CoEg_org_00000018alias}{name={Musée ethnographique},type={org},sort={Musee_ethno},text={},description={Voir «~musée du Louvre. Musée ethnographique~»}}
\newglossaryentry{CoEg_org_00000019}{name={Ministère français de la Marine et des Colonies},type={org},sort={Min_f_Mar},text={*},description={(1790-1893) \gls{CoEg_org_00000019alias} [\href{https://catalogue.bnf.fr/ark:/12148/cb118761674}{cat. gén. BNF}, \href{https://data.bnf.fr/ark:/12148/cb118761674}{data.bnf}, \href{https://www.idref.fr/026531011}{IdRéf}, \href{https://viaf.org/viaf/123917644}{VIAF}]}}
\newglossaryentry{CoEg_org_00000019alias}{name={Marine et Colonies},type={org},text={},description={Voir «~ministère français de la Marine et des Colonies~»}}
\newglossaryentry{CoEg_org_00000020}{name={Département égyptien},type={org},sort={Departement_eg},text={*},parent={CoEg_org_00000002},description={[\href{https://catalogue.bnf.fr/ark:/12148/cb11877592p}{cat. gén. BNF}, \href{https://data.bnf.fr/ark:/12148/cb11877592p}{data.bnf}, \href{https://www.idref.fr/026549298}{IdRéf}]}}
\newglossaryentry{CoEg_org_00000021}{name={Musée de Turin},type={org},sort={Musee_Tur},text={*},description={ [\gls{CoEg_coll_00000021}~; \href{https://catalogue.bnf.fr/ark:/12148/cb12182616s}{cat. gén. BNF}, \href{https://data.bnf.fr/ark:/12148/cb12182616s}{data.bnf}, \href{https://www.idref.fr/030405858}{IdRéf}, \href{https://viaf.org/viaf/168181167/}{VIAF}, \href{https://www.wikidata.org/wiki/Q19877}{Wikidata}]}}
\newglossaryentry{CoEg_org_00000022}{name={Musée de Boulaq},type={org},sort={Musee_Bou},text={*},description={ \gls{CoEg_org_00000022alias} [\href{https://catalogue.bnf.fr/ark:/12148/cb17003618k}{cat. gén. BNF}, \href{https://data.bnf.fr/ark:/12148/cb17003618k}{data.bnf}, \href{https://www.idref.fr/133209660}{IdRéf}, \href{https://viaf.org/viaf/172517642}{VIAF}, \href{https://www.wikidata.org/wiki/Q2399429}{Wikidata}]}}
\newglossaryentry{CoEg_org_00000022alias}{name={Musée du Caire},type={org},sort={Musee_Cai},text={},description={Voir «~musée de Boulaq~»}}
\newglossaryentry{CoEg_org_00000023}{name={Collège de France},type={org},text={*},description={ [\href{https://catalogue.bnf.fr/ark:/12148/cb11863147k}{cat. gén. BNF}, \href{https://data.bnf.fr/ark:/12148/cb11863147k}{data.bnf}, \href{https://www.idref.fr/026370921}{IdRéf}, \href{https://viaf.org/viaf/146442710}{VIAF}, \href{https://www.wikidata.org/wiki/Q202660}{Wikidata}]}}
\newglossaryentry{CoEg_org_00000024}{name={Archives nationales},type={org},text={*},description={ [\href{https://catalogue.bnf.fr/ark:/12148/cb11862272t}{cat. gén. BNF}, \href{https://data.bnf.fr/ark:/12148/cb11862272t}{data.bnf}, \href{https://www.idref.fr/026359421}{IdRéf}, \href{https://viaf.org/viaf/131469365}{VIAF}, \href{https://www.wikidata.org/wiki/Q182542}{Wikidata}]}}
\newglossaryentry{CoEg_org_00000025}{name={Département des antiques et sculptures du musée du Louvre},type={org},sort={Departement_des_ant},text={*},parent={CoEg_org_00000002}, description={}}
\newglossaryentry{CoEg_org_00000026}{name={Bibliothèque nationale de France},type={org},text={*},description={ [\gls{CoEg_coll_00000026}~; \href{https://catalogue.bnf.fr/ark:/12148/cb12381002j}{cat. gén. BNF}, \href{https://data.bnf.fr/ark:/12148/cb12381002j}{data.bnf}, \href{https://www.idref.fr/03361122X}{IdRéf}, \href{https://viaf.org/viaf/137156173}{VIAF}]}}
\newglossaryentry{CoEg_org_00000027}{name={Assemblée nationale législative},type={org},text={*},description={Parlement de la République française (1849-1852) [\href{https://catalogue.bnf.fr/ark:/12148/cb133001967}{cat. gén. BNF}, \href{https://data.bnf.fr/ark:/12148/cb133001967}{data.bnf}, \href{https://www.idref.fr/095240918}{IdRéf}, \href{https://viaf.org/viaf/136071566}{VIAF}, \href{https://www.wikidata.org/wiki/Q2867094}{Wikidata}]}}
\newglossaryentry{CoEg_org_00000028}{name={Comité royal de l’Instruction publique},type={org},text={*},description={}}
\newglossaryentry{CoEg_org_00000029}{name={Collège communal de Boulogne},type={org},text={*},description={}}
\newglossaryentry{CoEg_org_00000030}{name={Comité local d’instruction primaire},type={org},text={*},description={}}
\newglossaryentry{CoEg_org_00000031}{name={Société d’agriculture et des sciences},type={org},text={*},description={}}
\newglossaryentry{CoEg_org_00000032}{name={Académie de Douai},type={org},text={*},description={}}
\newglossaryentry{CoEg_org_00000033}{name={Bibliothèque du Vatican},type={org},text={*},description={[\href{https://catalogue.bnf.fr/ark:/12148/cb11873553t}{cat. gén. BNF}, \href{https://data.bnf.fr/ark:/12148/cb11873553t}{data.bnf}, \href{https://www.idref.fr/027307808}{IdRéf},	\href{https://viaf.org/viaf/146085455}{VIAF},	\href{https://www.wikidata.org/wiki/Q213678}{Wikidata}]}}
\newglossaryentry{CoEg_org_00000034}{name={Bodleian Library},type={org},text={*},description={\gls{CoEg_org_00000034alias} [\href{https://catalogue.bnf.fr/ark:/12148/cb11930727f}{cat. gén. BNF},	\href{https://data.bnf.fr/ark:/12148/cb11930727f}{data.bnf},	\href{https://www.idref.fr/027217124}{IdRéf},	\href{https://viaf.org/viaf/129788129}{VIAF},	\href{https://www.wikidata.org/wiki/Q82133}{Wikidata}]}}
\newglossaryentry{CoEg_org_00000034alias}{name={Bibliothèque bodleïenne},type={org},text={},description={Voir «~Bodleian Library~»}}
\newglossaryentry{CoEg_org_00000035}{name={Couvent des Syriens},type={org},text={*},description={ [\href{https://viaf.org/viaf/193145003315161301351}{VIAF},	\href{https://www.wikidata.org/wiki/Special:EntityPage/Q1885762}{Wikidata}]}}
\newglossaryentry{CoEg_org_00000036}{name={Académie des inscriptions et belles-lettres},type={org},text={*},parent={CoEg_org_00000004},description={\gls{CoEg_org_00000036alias} [\href{https://catalogue.bnf.fr/ark:/12148/cb118621340}{cat. gén. BNF},	\href{https://data.bnf.fr/ark:/12148/cb118621340}{data.bnf},	\href{https://www.idref.fr/026357836}{IdRéf},	\href{https://viaf.org/viaf/126085610}{VIAF}, \href{https://www.wikidata.org/wiki/Q337526}{Wikidata}]}}
\newglossaryentry{CoEg_org_00000036alias}{name={Académie des inscriptions et belles-lettres},type={org},text={},description={Voir «~Institut de France. Académie des inscriptions et belles-lettres~»}}
\newglossaryentry{CoEg_org_00000037}{name={Conseil supérieur de l’Instruction publique},type={org},text={*},description={}}
\newglossaryentry{CoEg_org_00000039}{name={Commission des missions scientifiques},type={org},text={*},description={}}
\newglossaryentry{CoEg_org_00000040}{name={Musée de Berlin},type={org},text={*},sort={Musee_Ber},description={ [\href{https://catalogue.bnf.fr/ark:/12148/cb123070010}{cat. gén. BNF},	\href{https://data.bnf.fr/ark:/12148/cb123070010}{data.bnf},	\href{https://www.idref.fr/031948243}{IdRéf},	\href{https://viaf.org/viaf/124716958}{VIAF},	\href{https://www.wikidata.org/wiki/Q254156}{Wikidata}]}}
\newglossaryentry{CoEg_org_00000041}{name={Ministère français d'État},type={org},text={*},description={}}
\newglossaryentry{CoEg_org_00000042}{name={Ministère français de l'Instruction publique},type={org},text={*},description={ [\href{https://viaf.org/viaf/163583267}{VIAF}, \href{https://www.wikidata.org/wiki/Q2572330}{Wikidata}]}}
\newglossaryentry{CoEg_org_00000043}{name={Prusse},type={org},text={*},description={(en tant qu'État~; voir l'index géographique pour le territoire correspondant) [\href{https://catalogue.bnf.fr/ark:/12148/cb113529929}{cat. gén. BNF},	\href{https://data.bnf.fr/ark:/12148/cb113529929}{data.bnf},	\href{https://www.idref.fr/125686862}{IdRéf},	\href{https://viaf.org/viaf/262341823/}{VIAF}, \href{https://www.wikidata.org/wiki/Q27306}{Wikidata}]}}

% Lieux

\newglossaryentry{CoEg_place_imn}{name={À identifier},type={place},text={\textsuperscript{!}},description={}}
\newglossaryentry{CoEg_place_00000001}{name={Saqqarah},type={place},text=*,description={(\foreignlanguage{arabic}{سقّارة} [\textit{Saqqārah}]) [\href{http://www.geonames.org/349608}{GeoName}, \href{https://pleiades.stoa.org/places/796289136}{Pleiades}, \href{http://topbib.griffith.ox.ac.uk//dtb.html?topbib=309}{TopBib}, \href{https://www.wikidata.org/wiki/Q192134}{Wikidata}]}}
\newglossaryentry{CoEg_place_00000002}{name={Paris},type={place},text=*,description={[\href{http://www.geonames.org/2988507}{GeoName}, \href{https://www.trismegistos.org/place/3278}{Trismegistos}, \href{https://www.wikidata.org/wiki/Q90}{Wikidata}]}}
\newglossaryentry{CoEg_place_00000003}{name={Égypte},type={place},text=*,sort={Egypte},description={[\href{http://www.geonames.org/357994}{GeoName}, \href{https://www.trismegistos.org/place/49}{Trismegistos}, \href{https://www.wikidata.org/wiki/Q79}{Wikidata}]}}
\newglossaryentry{CoEg_place_00000005}{name={Memphis},type={place},text=*,description={Voir « Mit Rahinah » [\href{https://www.geonames.org/352547}{GeoName}, \href{https://pleiades.stoa.org/places/736963}{Pleiades}, \href{http://topbib.griffith.ox.ac.uk//dtb.html?topbib=901-101-001}{TopBib}, \href{https://www.trismegistos.org/place/1344}{Trismegistos}, \href{https://www.wikidata.org/wiki/Q5715}{Wikidata}]}}
\newglossaryentry{CoEg_place_00000004}{name={Sérapéum},type={place},text=*,parent={CoEg_place_00000005},description={\gls{CoEg_place_00000004alias} [\href{https://www.geonames.org/349484}{GeoName}, \href{https://pleiades.stoa.org/places/737044}{Pleiades}, \href{http://topbib.griffith.ox.ac.uk//dtb.html?topbib=309-030-020}{TopBib}, \href{https://www.trismegistos.org/place/10638}{Trismegistos}, \href{https://www.wikidata.org/wiki/Q287375}{Wikidata}]}}
\newglossaryentry{CoEg_place_00000004alias}{name={Sérapéum de Memphis},type={place},text={},sort={Serapeum},description={Voir « Memphis. Sérapéum »}}

\newglossaryentry{CoEg_place_00000006}{name={Alexandrie},type={place},text=*,description={(\foreignlanguage{arabic}{الإسكندريّة} [\textit{Al-Iskandarīyah}]) [\href{https://www.geonames.org/361058}{GeoName}, \href{https://pleiades.stoa.org/places/727070}{Pleiades}, \href{http://topbib.griffith.ox.ac.uk//dtb.html?topbib=401-090}{TopBib}, \href{https://www.trismegistos.org/place/100}{Trismegistos}, \href{https://www.wikidata.org/wiki/Q87}{Wikidata}]}}
\newglossaryentry{CoEg_place_00000007}{name={Gizah},type={place},text=*,description={(\foreignlanguage{arabic}{الجيزة} [\textit{Al-\v{G}īzah}]). Mariette écrit «~Gyzeh~» [\href{https://www.geonames.org/360995}{GeoName}, \href{https://pleiades.stoa.org/places/442962448}{Pleiades}, \href{http://topbib.griffith.ox.ac.uk//dtb.html?topbib=302}{TopBib}, \href{https://www.trismegistos.org/place/716}{Trismegistos}, \href{https://www.wikidata.org/wiki/Q81788}{Wikidata}]}}
\newglossaryentry{CoEg_place_00000008}{name={Abousir},type={place},text=*,description={(\foreignlanguage{arabic}{ابو صير} [\textit{Abū \d{S}īr}]). Mariette écrit «~Abousyr~» [\href{https://www.geonames.org/362069}{GeoName}, \href{https://pleiades.stoa.org/places/195443086}{Pleiades}, \href{http://topbib.griffith.ox.ac.uk//dtb.html?topbib=307}{TopBib}, \href{https://www.wikidata.org/wiki/Q336098}{Wikidata}]}}
\newglossaryentry{CoEg_place_00000009}{name={Suez},type={place},text=*,description={(\foreignlanguage{arabic}{السّويس} [\textit{As-Sūwaīs}]) [\href{https://www.geonames.org/359796}{GeoName}, \href{http://topbib.griffith.ox.ac.uk//dtb.html?topbib=402a}{TopBib}, \href{https://www.trismegistos.org/place/2794}{Trismegistos}, \href{https://www.wikidata.org/wiki/Q134514}{Wikidata}]}}
\newglossaryentry{CoEg_place_00000010}{name={Caire (Le)},type={place},text=*,description={(\foreignlanguage{arabic}{القاهرة} [\textit{Al-Qāhirah}]) [\href{https://www.geonames.org/360630}{GeoName}, \href{https://www.trismegistos.org/place/2740}{Trismegistos}, \href{https://www.wikidata.org/wiki/Q85}{Wikidata}]\gls{CoEg_place_00000010alias}}}
\newglossaryentry{CoEg_place_00000010alias}{name={Le Caire},type={place},text={},description={Voir «~Caire (Le)~»}}
\newglossaryentry{CoEg_place_00000011}{name={Londres},type={place},text=*,description={[\href{https://www.geonames.org/2643743}{GeoName}, \href{https://www.trismegistos.org/place/11298}{Trismegistos}, \href{https://www.wikidata.org/wiki/Q84}{Wikidata}]}}
\newglossaryentry{CoEg_place_00000012}{name={Livourne},type={place},text=*,description={[\href{https://www.geonames.org/3174659}{GeoName}, \href{https://www.trismegistos.org/place/45380}{Trismegistos}, \href{https://www.wikidata.org/wiki/Q6761}{Wikidata}]}}
\newglossaryentry{CoEg_place_00000013}{name={Europe},type={place},text=*,description={[\href{https://www.geonames.org/6255148}{GeoName}, \href{https://www.trismegistos.org/place/3229}{Trismegistos}, \href{https://www.wikidata.org/wiki/Q46}{Wikidata}]}}
\newglossaryentry{CoEg_place_00000014}{name={Suède},type={place},text=*,sort={Suede},description={[\href{https://www.geonames.org/2661886}{GeoName}, \href{https://www.trismegistos.org/place/20271}{Trismegistos}, \href{https://www.wikidata.org/wiki/Q34}{Wikidata}]}}
\newglossaryentry{CoEg_place_00000015}{name={Royaume-Uni},type={place},text=*,description={[\href{https://www.geonames.org/2635167}{GeoName}, \href{https://www.trismegistos.org/place/19851}{Trismegistos}, \href{https://www.wikidata.org/wiki/Q145}{Wikidata}]}}
\newglossaryentry{CoEg_place_00000016}{name={France},type={place},text=*,description={[\href{https://www.geonames.org/3017382}{GeoName}, \href{https://www.trismegistos.org/place/693}{Trismegistos}, \href{https://www.wikidata.org/wiki/Q142}{Wikidata}]}}
\newglossaryentry{CoEg_place_00000018}{name={Marseille},type={place},text=*,description={[\href{https://www.geonames.org/2995469}{GeoName}, \href{https://www.trismegistos.org/place/1318}{Trismegistos}, \href{https://www.wikidata.org/wiki/Q23482}{Wikidata}]}}
\newglossaryentry{CoEg_place_00000019}{name={Thèbes},type={place},text=*,sort={Thebes},description={[\href{https://www.geonames.org/347342}{GeoName}, \href{http://topbib.griffith.ox.ac.uk//dtb.html?topbib=901-204-005}{TopBib}, \href{https://www.trismegistos.org/place/2355}{Trismegistos}, \href{https://www.wikidata.org/wiki/Q101583}{Wikidata}]}}
\newglossaryentry{CoEg_place_00000020}{name={Haute-Égypte},type={place},parent={CoEg_place_00000003},text=*,description={\gls{CoEg_place_00000020alias}{[\href{https://www.geonames.org/359888}{GeoName}, \href{https://www.trismegistos.org/place/2766}{Trismegistos}, \href{https://www.wikidata.org/wiki/Q203751}{Wikidata}}]}}
\newglossaryentry{CoEg_place_00000020alias}{name={Haute-Égypte},type={place},text={},description={Voir «~Égypte. Haute-Égypte~»}}
\newglossaryentry{CoEg_place_00000021}{name={Nil},type={place},text=*,description={[\href{https://www.geonames.org/351036}{GeoName}, \href{https://www.trismegistos.org/place/3943}{Trismegistos}, \href{https://www.wikidata.org/wiki/Q3392}{Wikidata}]}}
\newglossaryentry{CoEg_place_00000022}{name={Algérie},type={place},text=*,description={[\href{https://www.geonames.org/2589581}{GeoName}, \href{https://www.trismegistos.org/place/10928}{Trismegistos}, \href{https://www.wikidata.org/wiki/Q262}{Wikidata}]}}
\newglossaryentry{CoEg_place_00000023}{name={Citadelle du Caire},type={place},parent={CoEg_place_00000010}, text=*,description={\gls{CoEg_place_00000023alias}{[\href{https://www.geonames.org/7913721}{GeoName}, \href{https://www.wikidata.org/wiki/Q1988240}{Wikidata}]}}}
\newglossaryentry{CoEg_place_00000023alias}{name={Citadelle du Caire},type={place},text=,description={Voir «~Caire (Le). Citadelle~»}}
\newglossaryentry{CoEg_place_00000024}{name={Héliopolis},type={place},text=*,description={[\href{https://www.geonames.org/355956}{GeoName}, \href{https://pleiades.stoa.org/places/727117}{Pleiades}, \href{http://topbib.griffith.ox.ac.uk//dtb.html?topbib=901-113-002}{TopBib}, \href{https://www.trismegistos.org/place/761}{Trismegistos}, \href{https://www.wikidata.org/wiki/Q191687}{Wikidata}]}}
\newglossaryentry{CoEg_place_00000025}{name={Mit Rahinah},type={place},text=*,description={\gls{CoEg_place_00000025alias} (\foreignlanguage{arabic}{ميت رهينة} [\textit{Mīt Rahīnah}]). Mariette écrit «~Myt Rahyneh~». Voir «~Memphis~» [\href{http://topbib.griffith.ox.ac.uk//dtb.html?topbib=311}{TopBib}, \href{https://www.wikidata.org/wiki/Q3317039}{Wikidata}]}}
\newglossaryentry{CoEg_place_00000025alias}{name={Myt Rahyneh},type={place},text={},description={Voir «~Mit Rahinah~»}}
\newglossaryentry{CoEg_place_00000026}{name={Abydos},type={place},text=*,description={[\href{https://www.geonames.org/361885}{GeoName}, \href{https://pleiades.stoa.org/places/756512}{Pleiades}, \href{http://topbib.griffith.ox.ac.uk//dtb.html?topbib=502}{TopBib}, \href{www.trismegistos.org/place/34}{Trismegistos}, \href{https://www.wikidata.org/wiki/Q192268}{Wikidata}]}}
\newglossaryentry{CoEg_place_00000027}{name={Havre (Le)},type={place},text=*,description={[\href{https://www.geonames.org/3003796}{GeoName}, \href{https://www.wikidata.org/wiki/Q42810}{Wikidata}]}}
\newglossaryentry{CoEg_place_00000028}{name={Dahchour},type={place},text=*,description={\gls{CoEg_place_00000028alias} (\foreignlanguage{arabic}{دهشور} [\textit{Dah\v{s}ūr}]). Mariette écrit «~Dashour~» [\href{https://www.geonames.org/358483}{GeoName}, \href{http://topbib.griffith.ox.ac.uk//dtb.html?topbib=313}{TopBib}, \href{https://www.trismegistos.org/place/2751}{Trismegistos}, \href{https://www.wikidata.org/wiki/Q685414}{Wikidata}]}}
\newglossaryentry{CoEg_place_00000028alias}{name={Dashour},type={place},text={},description={Voir «~Dahchour~»}}
\newglossaryentry{CoEg_place_00000029}{name={Boulaq},type={place},text=*,description={(\foreignlanguage{arabic}{بولاق} [\textit{Būlāq}]) [\href{https://www.geonames.org/358683}{GeoName}, \href{https://www.wikidata.org/wiki/Q1003523}{Wikidata}]}}
\newglossaryentry{CoEg_place_00000030}{name={El-Atf},type={place},text=*,sort={Atf},description={(\foreignlanguage{arabic}{العطف} [\textit{Al-`A\d{t}f}]) Poste de douane entre Boulaq et Alexandrie, à la jonction du canal Mahmoudiyyah et de la branche nilotique de Rosette. Mariette écrit «~Atfih~»\gls{CoEg_place_00000030alias} [\href{https://www.geonames.org/361519}{GeoName}]}}
\newglossaryentry{CoEg_place_00000030alias}{name={Atfih},type={place},text={},description={Voir «~Al-Atf~»}}
\newglossaryentry{CoEg_place_00000031}{name={Badrachin},type={place},text=*,description={(\foreignlanguage{arabic}{البدراشين} [\textit{Al-Badrā\v{s}īn}]) Village sur le Nil, au voisinnage immédiat de Saqqarah et de Mit Rahinah. Mariette écrit «~Bédréchyn~» [\href{https://www.geonames.org/361473}{GeoName}, \href{https://www.wikidata.org/wiki/Q4165238}{Wikidata}]}}
\newglossaryentry{CoEg_place_00000032}{name={Vincennes},type={place},text=*,description={[\href{https://www.geonames.org/2968054}{GeoName}, \href{https://www.wikidata.org/wiki/Q193819}{Wikidata}]}}
\newglossaryentry{CoEg_place_00000033}{name={Assiout},type={place},text=*,description={(\foreignlanguage{arabic}{أسّيوط} [\textit{Asīū\d{t}}]). Mariette écrit «~Syout~» [\href{https://www.geonames.org/359783}{GeoName}, \href{https://pleiades.stoa.org/places/756593}{Pleiades}, \href{http://topbib.griffith.ox.ac.uk//dtb.html?topbib=412-010}{TopBib}, \href{https://www.trismegistos.org/place/1271}{Trismegistos}, \href{https://www.wikidata.org/wiki/Q29962}{Wikidata}]\gls{CoEg_place_00000033alias}}}
\newglossaryentry{CoEg_place_00000033alias}{name={Syout},type={place},text={},description={Voir «~Assiout~»}}
\newglossaryentry{CoEg_place_00000034}{name={Turin},type={place},text=*,description={[\href{https://www.geonames.org/3165524}{GeoName}, \href{https://www.trismegistos.org/place/19897}{Trismegistos}, \href{https://www.wikidata.org/wiki/Q495}{Wikidata}]}}

\newglossaryentry{CoEg_place_00000035}{name={Boulogne-sur-Mer},type={place},text=*,description={[\href{https://www.geonames.org/6439391}{GeoName}, \href{https://pleiades.stoa.org/places/109008}{Pleiades}, \href{https://www.trismegistos.org/place/18965}{Trismegistos}, \href{https://www.wikidata.org/wiki/Q81997}{Wikidata}]}}
\newglossaryentry{CoEg_place_00000036}{name={Douai},type={place},text=*,description={[\href{https://www.geonames.org/3021000}{GeoName}, \href{https://www.wikidata.org/wiki/Q193826}{Wikidata}]}}
\newglossaryentry{CoEg_place_00000037}{name={Chili},type={place},text=*,description={[\href{https://www.geonames.org/3895114}{GeoName}, \href{https://www.wikidata.org/wiki/Q298}{Wikidata}]}}
\newglossaryentry{CoEg_place_00000038}{name={Pérou},type={place},text=*,description={[\href{https://www.geonames.org/3932488}{GeoName}, \href{https://www.wikidata.org/wiki/Q419}{Wikidata}]}}
\newglossaryentry{CoEg_place_00000039}{name={Chine},type={place},text=*,description={[\href{https://www.geonames.org/1814991}{GeoName}, \href{https://www.wikidata.org/wiki/Q148}{Wikidata}]}}
\newglossaryentry{CoEg_place_00000040}{name={Béni Hassan},type={place},text=*,description={(\foreignlanguage{arabic}{بني حسن} [\textit{Banī Ḥasan}])[\href{https://www.geonames.org/359243}{GeoName}, \href{https://pleiades.stoa.org/places/112509596}{Pleiades}, \href{http://topbib.griffith.ox.ac.uk//dtb.html?topbib=408-010}{TopBib}, \href{https://www.trismegistos.org/place/414}{Trismegistos}, \href{https://www.wikidata.org/wiki/Q817286}{Wikidata}]}}
\newglossaryentry{CoEg_place_00000041}{name={Éthiopie},type={place},text=*,sort={Ethiopie},description={[\href{https://www.geonames.org/337996}{GeoName}, \href{https://www.trismegistos.org/place/51}{Trismegistos}, \href{https://www.wikidata.org/wiki/Q115}{Wikidata}]}}
\newglossaryentry{CoEg_place_00000042}{name={Esna},type={place},text=*,description={(\foreignlanguage{arabic}{إسنا} [\textit{Isnā}])[\href{https://www.geonames.org/355449}{GeoName}, \href{https://pleiades.stoa.org/places/786059}{Pleiades}, \href{http://topbib.griffith.ox.ac.uk//dtb.html?topbib=504-050}{TopBib}, \href{https://www.trismegistos.org/place/1227}{Trismegistos}, \href{https://www.wikidata.org/wiki/Q626031}{Wikidata}]}}
\newglossaryentry{CoEg_place_00000043}{name={Armant},type={place},text=*,description={(\foreignlanguage{arabic}{أرمنت} [\textit{Armant}])[\href{https://www.geonames.org/360170}{GeoName}, \href{https://pleiades.stoa.org/places/786036}{Pleiades}, \href{http://topbib.griffith.ox.ac.uk//dtb.html?topbib=504-010}{TopBib}, \href{https://www.trismegistos.org/place/812}{Trismegistos}, \href{https://www.wikidata.org/wiki/Q679056}{Wikidata}]}}
\newglossaryentry{CoEg_place_00000044}{name={Éléphantine},type={place},text=*,sort={Elephantine},description={[\href{https://www.geonames.org/359790}{GeoName}, \href{https://pleiades.stoa.org/places/786021}{Pleiades}, \href{http://topbib.griffith.ox.ac.uk//dtb.html?topbib=507-020}{TopBib}, \href{https://www.trismegistos.org/place/621}{Trismegistos}, \href{https://www.wikidata.org/wiki/Q284009}{Wikidata}]}}
\newglossaryentry{CoEg_place_00000045}{name={Ouadi Natroun},type={place},text=*,description={(\foreignlanguage{arabic}{وادي النطرون} [\textit{Wādī an-Naṭrūn}])[\href{https://www.geonames.org/351277}{GeoName}, \href{http://topbib.griffith.ox.ac.uk//dtb.html?topbib=702-040}{TopBib}, \href{https://www.trismegistos.org/place/3375}{Trismegistos}, \href{https://www.wikidata.org/wiki/Q1074945}{Wikidata}]}}
\newglossaryentry{CoEg_place_00000046}{name={Saxe},type={place},text=*,description={[\href{https://www.wikidata.org/wiki/Q153015}{Wikidata}]}}
\newglossaryentry{CoEg_place_00000047}{name={Bedford},type={place},text=*,description={[\href{https://www.geonames.org/2656046}{GeoName}, \href{https://www.wikidata.org/wiki/Q208257}{Wikidata}]}}
\newglossaryentry{CoEg_place_00000048}{name={Prusse},type={place},text=*,description={[\href{https://www.wikidata.org/wiki/Q38872}{Wikidata}]}}
\newglossaryentry{CoEg_place_00000049}{name={Thébaïde},type={place},text=*,sort={Thebaide},description={[\href{https://www.trismegistos.org/place/2982}{Trismegistos}, \href{https://www.wikidata.org/wiki/Q768515}{Wikidata}]}}
\newglossaryentry{CoEg_place_00000050}{name={Hout-ched-abed},type={place},text=*,sort={Houtchedabed},description={Localité memphite (voir \textsc{Gauthier} Henri, \textit{Dictionnaire des noms géographiques contenus dans les textes hiéroglyphiques}, t. 4, Le Caire, Institut français d'archéologie orientale – Société royale de géographie d'Égypte, 1927, p. \href{https://archive.org/details/Gauthier1927/page/n69/mode/2up}{135}). Mariette écrit « Hat-schat-[avat ?] »}}
\newglossaryentry{CoEg_place_00000051}{name={Berlin},type={place},text=*,description={[\href{https://www.geonames.org/6547383}{GeoName}, \href{https://www.trismegistos.org/place/20102}{Trismegistos}, \href{https://www.wikidata.org/wiki/Q64}{Wikidata}]}}
\newglossaryentry{CoEg_place_00000052}{name={Angleterre},type={place},text=*,description={[\href{https://www.geonames.org/6269131}{GeoName}, \href{https://www.trismegistos.org/place/19851}{Trismegistos}, \href{https://www.wikidata.org/wiki/Q21}{Wikidata}]}}
\newglossaryentry{CoEg_place_00000053}{name={Sinope},type={place},text=*,description={[\href{https://www.geonames.org/739600}{GeoName}, \href{https://pleiades.stoa.org/places/857321}{Pleiades}, \href{https://www.trismegistos.org/place/2141}{Trismegistos}, \href{https://www.wikidata.org/wiki/Q599416}{Wikidata}]}}
\newglossaryentry{CoEg_place_00000054}{name={Argos},type={place},text=*,description={[\href{https://www.geonames.org/264670}{GeoName}, \href{https://www.trismegistos.org/place/301}{Trismegistos}, \href{https://www.wikidata.org/wiki/Q189901}{Wikidata}]}}
\newglossaryentry{CoEg_place_00000055}{name={Élée},type={place},text=*,sort={Elee},description={[\href{https://www.geonames.org/3164641}{GeoName}, \href{https://pleiades.stoa.org/places/452488}{Pleiades}, \href{https://www.trismegistos.org/place/2495}{Trismegistos}, \href{https://www.wikidata.org/wiki/Q272968}{Wikidata}]}}
\newglossaryentry{CoEg_place_00000056}{name={Syrie},type={place},text=*,description={[\href{https://www.geonames.org/163843}{GeoName}, \href{https://www.trismegistos.org/place/2211}{Trismegistos}, \href{https://www.wikidata.org/wiki/Q858}{Wikidata}]}}
\newglossaryentry{CoEg_place_00000057}{name={Grèce},type={place},text=*,description={[\href{https://www.geonames.org/390903}{GeoName}, \href{https://www.trismegistos.org/place/762}{Trismegistos}, \href{https://www.wikidata.org/wiki/Q41}{Wikidata}]}}
\newglossaryentry{CoEg_place_00000058}{name={Sicile},type={place},text=*,description={[\href{https://www.geonames.org/2523118}{GeoName}, \href{https://pleiades.stoa.org/places/462492}{Pleiades}, \href{https://www.trismegistos.org/place/2132}{Trismegistos}, \href{https://www.wikidata.org/wiki/Q1460}{Wikidata}]}}
\newglossaryentry{CoEg_place_00000059}{name={Italie},type={place},text=*,description={[\href{https://www.geonames.org/3175395}{GeoName}, \href{https://pleiades.stoa.org/places/1052}{Pleiades}, \href{https://www.trismegistos.org/place/932}{Trismegistos}, \href{https://www.wikidata.org/wiki/Q38}{Wikidata}]}}
\newglossaryentry{CoEg_place_00000060}{name={Gaules},type={place},text=*,description={[\href{https://www.trismegistos.org/place/693}{Trismegistos}, \href{https://www.wikidata.org/wiki/Q38060}{Wikidata}]}}
\newglossaryentry{CoEg_place_00000061}{name={Piémont},type={place},text=*,description={[\href{https://www.geonames.org/3170831}{GeoName}, \href{https://www.wikidata.org/wiki/Q1216}{Wikidata}]}}
\newglossaryentry{CoEg_place_00000062}{name={Nilopolis},type={place},text=*,description={[\href{https://www.trismegistos.org/place/2427}{Trismegistos}, \href{https://www.wikidata.org/wiki/Q3704933}{Wikidata}]}}
\newglossaryentry{CoEg_place_00000063}{name={Ombos},type={place},text=*,description={[\href{https://www.geonames.org/351356}{GeoName}, \href{https://pleiades.stoa.org/places/786079}{Pleiades}, \href{http://topbib.griffith.ox.ac.uk//dtb.html?topbib=506-131}{TopBib}, \href{https://www.trismegistos.org/place/1499}{Trismegistos}, \href{https://www.wikidata.org/wiki/Q17486769}{Wikidata}]}}
\newglossaryentry{CoEg_place_00000064}{name={Mons Claudianus},type={place},text=*,description={[\href{https://www.geonames.org/352089}{GeoName}, \href{https://pleiades.stoa.org/places/766348}{Pleiades}, \href{https://www.wikidata.org/wiki/Q1146716}{Wikidata}]}}
\newglossaryentry{CoEg_place_00000065}{name={Rome},type={place},text=*,description={[\href{https://www.geonames.org/3169070}{GeoName}, \href{https://pleiades.stoa.org/places/423025}{Pleiades}, \href{http://topbib.griffith.ox.ac.uk//dtb.html?topbib=704-020-140-070}{TopBib}, \href{https://www.trismegistos.org/place/2058}{Trismegistos}, \href{https://www.wikidata.org/wiki/Q220}{Wikidata}]}}
\newglossaryentry{CoEg_place_00000066}{name={Athènes},type={place},text=*,description={[\href{https://www.geonames.org/264371}{GeoName}, \href{https://pleiades.stoa.org/places/579885}{Pleiades}, \href{http://topbib.griffith.ox.ac.uk//dtb.html?topbib=704-020-080-040}{TopBib}, \href{https://www.trismegistos.org/place/364}{Trismegistos}, \href{https://www.wikidata.org/wiki/Q1524}{Wikidata}]}}
\newglossaryentry{CoEg_place_00000067}{name={Chypre},type={place},text=*,description={[\href{https://www.geonames.org/146669}{GeoName}, \href{https://pleiades.stoa.org/places/707498}{Pleiades}, \href{http://topbib.griffith.ox.ac.uk//dtb.html?topbib=704-020-090-010}{TopBib}, \href{https://www.trismegistos.org/place/528}{Trismegistos}, \href{https://www.wikidata.org/wiki/Q229}{Wikidata}]}}
\newglossaryentry{CoEg_place_00000069}{name={Cysis},type={place},text=*,description={}}
\newglossaryentry{CoEg_place_00000070}{name={Allemagne},type={place},text=*,description={[\href{https://www.geonames.org/2921044}{GeoName}, \href{http://topbib.griffith.ox.ac.uk//dtb.html?topbib=704-020-130}{TopBib}, \href{https://www.trismegistos.org/place/3245}{Trismegistos}, \href{https://www.wikidata.org/wiki/Q183}{Wikidata}]}}
\newglossaryentry{CoEg_place_00000071}{name={Annecy},type={place},text=*,description={[\href{https://www.geonames.org/3037543}{GeoName}, \href{https://www.trismegistos.org/place/20689}{Trismegistos}, \href{https://www.wikidata.org/wiki/Q50189}{Wikidata}]}}
\newglossaryentry{CoEg_place_00000072}{name={Basse-Égypte},type={place},text=*,parent={CoEg_place_00000003},description={\gls{CoEg_place_00000072alias} [\href{https://www.geonames.org/352260}{GeoName}, \href{https://www.trismegistos.org/place/2712}{Trismegistos}, \href{https://www.wikidata.org/wiki/Q463871}{Wikidata}]}}
\newglossaryentry{CoEg_place_00000072alias}{name={Basse-Égypte},type={place},text=,description={Voir «~Égypte. Basse-Égypte~»}}
\newglossaryentry{CoEg_place_00000073}{name={El-Kab},type={place},text=*,sort={Kab},description={\gls{CoEg_place_00000073alias} (\foreignlanguage{arabic}{الكاب} [\textit{Al-Kāb}])[\href{https://www.geonames.org/360986}{GeoName}, \href{https://pleiades.stoa.org/places/440947682}{Pleiades}, \href{http://topbib.griffith.ox.ac.uk//dtb.html?topbib=505-010}{TopBib}, \href{https://www.trismegistos.org/place/611}{Trismegistos}, \href{https://www.wikidata.org/wiki/Q287705}{Wikidata}]}}
\newglossaryentry{CoEg_place_00000073alias}{name={Éléthya},type={place},text={},sort={Elethya},description={voir «~El-Kab~»}}
\newglossaryentry{CoEg_place_00000074}{name={Orient},type={place},text=*,description={[\href{https://www.wikidata.org/wiki/Q205653}{Wikidata}]}}
\newglossaryentry{CoEg_place_00000075}{name={Vatican},type={place},text=*,description={[\href{https://www.geonames.org/3164670}{GeoName}, \href{https://www.trismegistos.org/place/30981}{Trismegistos}, \href{https://www.wikidata.org/wiki/Q237}{Wikidata}]}}
\newglossaryentry{CoEg_place_00000076}{name={Cornouailles},type={place},text=*,description={[\href{https://www.trismegistos.org/place/45017}{Trismegistos}, \href{https://www.wikidata.org/wiki/Q23148}{Wikidata}]}}
\newglossaryentry{CoEg_place_00000077}{name={Pont},type={place},text=*,description={[\href{https://pleiades.stoa.org/places/857287}{Pleiades}, \href{https://www.trismegistos.org/place/1894}{Trismegistos}, \href{https://www.wikidata.org/wiki/Q7380487}{Wikidata}]}}
\newglossaryentry{CoEg_place_00000078}{name={Antioche},type={place},text=*,description={[\href{https://pleiades.stoa.org/places/658381}{Pleiades}, \href{https://www.trismegistos.org/place/205}{Trismegistos}, \href{https://www.wikidata.org/wiki/Q200441}{Wikidata}]}}
\newglossaryentry{CoEg_place_00000079}{name={Rosette},type={place},text=*,description={(\foreignlanguage{arabic}{رشيد} [\textit{Rašīd}] [\href{https://www.geonames.org/350203}{GeoName}, \href{www.trismegistos.org/place/2059 }{Trismegistos}, \href{https://www.wikidata.org/wiki/Q243699}{Wikidata}]}}


%abréviations

\newglossaryentry{CoEg_abbr_00000005}{name={1\textsuperscript{o}},type={abbr},description={«~\textit{primo}~»}}
\newglossaryentry{CoEg_abbr_00000006}{name={2\textsuperscript{o}},type={abbr},description={«~\textit{secundo}~»}}
\newglossaryentry{CoEg_abbr_00000007}{name={3\textsuperscript{o}},type={abbr},description={«~\textit{tertio}~»}}
\newglossaryentry{CoEg_abbr_00000002}{name={Aug.},type={abbr},description={«~Auguste~»}}
\newglossaryentry{CoEg_abbr_00000012}{name={Eug.},type={abbr},description={«~Eugène~»}}
\newglossaryentry{CoEg_abbr_00000013}{name={fr.},type={abbr},description={«~francs~»}}
\newglossaryentry{CoEg_abbr_00000016}{name={LL. MM.},type={abbr},description={«~Leurs Majestés~»}}
\newglossaryentry{CoEg_abbr_00000001}{name={M\textsuperscript{\underline{r}}},type={abbr},description={«~Monsieur~»}}
\newglossaryentry{CoEg_abbr_00000009}{name={MM.},type={abbr},sort={Mr},description={«~Messieurs~»}}
\newglossaryentry{CoEg_abbr_00000014}{name={Mons.},type={abbr},sort={Mons},description={«~Monsieur~»~; forme plus rare que \gls{CoEg_abbr_00000001}}}
\newglossaryentry{CoEg_abbr_00000010}{name={n\textsuperscript{o}},type={abbr},description={«~numéro~»}}
\newglossaryentry{CoEg_abbr_00000011}{name={n\textsuperscript{os}},type={abbr},description={«~numéros~»}}
\newglossaryentry{CoEg_abbr_00000008}{name={P. S.},type={abbr},description={«~\textit{post-scriptum}~»}}
\newglossaryentry{CoEg_abbr_00000003}{name={S. A.},type={abbr},sort={SA},description={«~Son Altesse~». Prédicat notamment porté par le vice-roi d'Égypte}}
\newglossaryentry{CoEg_abbr_00000015}{name={S. A. I.},type={abbr},sort={SAI},description={«~Son Altesse Impériale~». Prédicat des princes de la famille impériale française}}
\newglossaryentry{CoEg_abbr_00000004}{name={S. E.},type={abbr},sort={SE},description={«~Son Excellence~». Prédicat des ministres ou des moudirs}}
\newglossaryentry{CoEg_abbr_00000017}{name={fr},type={abbr},description={«~francs~». Variante de l'abréviation habituelle (\gls{CoEg_abbr_00000013}) sans point abréviatif}}
\newglossaryentry{CoEg_abbr_00000018}{name={X\textsuperscript{bre}},type={abbr},description={«~décembre~»}}
\newglossaryentry{CoEg_abbr_00000019}{name={7\textsuperscript{bre}},type={abbr},description={«~septembre~»}}
\newglossaryentry{CoEg_abbr_00000020}{name={g\textsuperscript{al}},type={abbr},description={«~général~»}}
\newglossaryentry{CoEg_abbr_00000021}{name={9\textsuperscript{bre}},type={abbr},description={«~novembre~»}}
\newglossaryentry{CoEg_abbr_00000022}{name={in-8\textsuperscript{o}},type={abbr},description={}}
\newglossaryentry{CoEg_abbr_00000023}{name={in-4\textsuperscript{o}},type={abbr},description={}}
\newglossaryentry{CoEg_abbr_00000024}{name={in-fol.},type={abbr},description={}}
\newglossaryentry{CoEg_abbr_00000025}{name={S. Exc.},type={abbr},description={}}
\newglossaryentry{CoEg_abbr_00000026}{name={apud},type={abbr},description={}}
\newglossaryentry{CoEg_abbr_00000027}{name={l.},type={abbr},description={ligne/livre ?}}
\newglossaryentry{CoEg_abbr_00000028}{name={c.},type={abbr},description={Chapitre ?}}
\newglossaryentry{CoEg_abbr_00000029}{name={Abth.},type={abbr},description={Abtheilung}}
\newglossaryentry{CoEg_abbr_00000030}{name={Bl.},type={abbr},description={Blatt}}
\newglossaryentry{CoEg_abbr_00000030bis}{name={Taf.},type={abbr},description={Taffel}}
\newglossaryentry{CoEg_abbr_00000031}{name={voy.},type={abbr},description={voyez}}
\newglossaryentry{CoEg_abbr_00000032}{name={p.},type={abbr},description={p}}
\newglossaryentry{CoEg_abbr_00000033}{name={part.},type={abbr},description={partie/part}}
\newglossaryentry{CoEg_abbr_00000034}{name={t.},type={abbr},description={tome}}
\newglossaryentry{CoEg_abbr_00000035}{name={pl.},type={abbr},description={planche}}
\newglossaryentry{CoEg_abbr_00000036}{name={Annal.},type={abbr},description={Annales}}
\newglossaryentry{CoEg_abbr_00000037}{name={ch.},type={abbr},description={chapitre}}
\newglossaryentry{CoEg_abbr_00000038}{name={tab},type={abbr},description={?}}
\newglossaryentry{CoEg_abbr_00000039}{name={4\textsuperscript{to}},type={abbr},description={}}
\newglossaryentry{CoEg_abbr_00000040}{name={conf.},type={abbr},description={confer}}
\newglossaryentry{CoEg_abbr_00000041}{name={Liv.},type={abbr},description={livre}}
\newglossaryentry{CoEg_abbr_00000042}{name={loc. cit.},type={abbr},description={}}
\newglossaryentry{CoEg_abbr_00000043}{name={1\textsuperscript{er}},type={abbr},description={premier}}
\newglossaryentry{CoEg_abbr_00000044}{name={B\textsuperscript{eau}},type={abbr},description={bureau}}
\newglossaryentry{CoEg_abbr_00000045}{name={J. C.},type={abbr},description={Jésus Christ}}
\newglossaryentry{CoEg_abbr_00000046}{name={S\textsuperscript{\underline{t}}},type={abbr},description={saint}}
\newglossaryentry{CoEg_abbr_00000047}{name={S. M.},type={abbr},description={Sa Majesté}}
\newglossaryentry{CoEg_abbr_00000048}{name={M.M.},type={abbr},description={Messieurs}}
\newglossaryentry{CoEg_abbr_00000049}{name={4\textsuperscript{o}},type={abbr},description={quarto}}
\newglossaryentry{CoEg_abbr_00000050}{name={5\textsuperscript{o}},type={abbr},description={}}
\newglossaryentry{CoEg_abbr_00000051}{name={L.},type={abbr},description={liber}}
\newglossaryentry{CoEg_abbr_00000052}{name={etc.},type={abbr},description={}}
\newglossaryentry{CoEg_abbr_00000053}{name={v.},type={abbr},description={vers (unité métrique)}}
\newglossaryentry{CoEg_abbr_00000054}{name={ibid.},type={abbr},description={ibidem}}

%collections

\newglossaryentry{CoEg_coll_00000002}{name={Musée du Louvre},type={obj},sort={MuseeLouvre},text={publications},description={\gls{CoEg_org_00000002} [\href{www.trismegistos.org/collection/274}{Trismegistos}]}}
\newglossaryentry{CoEg_coll_00000021}{name={Musée de Turin},type={obj},sort={MuseeTurin},text={publications},description={\gls{CoEg_org_00000021} [\href{www.trismegistos.org/collection/334}{Trismegistos}]}}
\newglossaryentry{CoEg_coll_00000005}{name={British Museum},type={obj},sort={BritishMuseum},text={publications},description={\gls{CoEg_org_00000005} [\href{www.trismegistos.org/collection/193}{Trismegistos}]}}
\newglossaryentry{CoEg_coll_00000005bis}{name={British Library},type={obj},sort={BritishLibrary},text={publications},description={[\href{www.trismegistos.org/collection/192}{Trismegistos}]}}
\newglossaryentry{CoEg_coll_00000026}{name={Bibliothèque nationale de France},type={obj},sort={Bibliothequenationale},text={publications},description={\gls{CoEg_org_00000026} [\href{www.trismegistos.org/collection/270}{Trismegistos}]}}

%objet

\newglossaryentry{CoEg_obj_imn}{name={À identifier},type={obj},sort={AAA},text={\textsuperscript{!}},description={}}
\newglossaryentry{CoEg_obj_00000001}{name={A 26},type={obj},sort={A026},text={*},parent={CoEg_coll_00000002},description={Statue de sphinx (Basse-Époque, XXIX\textsuperscript{e} dynastie). Autre numéro d'inventaire~: «~N~26~»\gls{CoEg_obj_00000001alias} [\href{http://cartelfr.louvre.fr/cartelfr/visite?srv=car_not_frame&idNotice=31644}{Atlas}]}}
\newglossaryentry{CoEg_obj_00000001alias}{name={N 26},type={obj},sort={N0026},text={},parent={CoEg_coll_00000002},description={Voir «~A~26~»}}
\newglossaryentry{CoEg_obj_00000002}{name={N 391 A à F},type={obj},sort={N0391},text={*},parent={CoEg_coll_00000002},description={Six statues de sphinx (époque ptolémaïque~; fouilles du Sérapéum) [\href{http://cartelfr.louvre.fr/cartelfr/visite?srv=car_not_frame&idNotice=18923}{Atlas}]}}
\newglossaryentry{CoEg_obj_00000003}{name={C 318},type={obj},sort={C318},text={*},parent={CoEg_coll_00000002},description={Stèle (Basse-Époque, XXX\textsuperscript{e} dynastie~; fouilles du Sérapéum) originellement encastrée dans le socle du lion N~432~A [\href{http://cartelfr.louvre.fr/cartelfr/visite?srv=car_not_frame&idNotice=18925}{Atlas}]}}
\newglossaryentry{CoEg_obj_00000004}{name={N 432 A},type={obj},sort={N0432A},text={*},parent={CoEg_coll_00000002},description={Statue de lion (Basse-Époque, XXX\textsuperscript{e} dynastie~; fouilles du Sérapéum)~; son socle abritait la stèle C 318 [\href{http://cartelfr.louvre.fr/cartelfr/visite?srv=car_not_frame&idNotice=18924}{Atlas}]}}
\newglossaryentry{CoEg_obj_00000005}{name={N 432 B},type={obj},sort={N0432B},text={*},parent={CoEg_coll_00000002},description={Statue de lion (Basse-Époque, XXX\textsuperscript{e} dynastie~; fouilles du Sérapéum) [\href{http://cartelfr.louvre.fr/cartelfr/visite?srv=car_not_frame&idNotice=18934}{Atlas}]}}
\newglossaryentry{CoEg_obj_00000006}{name={N 432 C},type={obj},sort={N0432C},text={*},parent={CoEg_coll_00000002},description={Statue de lion (Basse-Époque, XXX\textsuperscript{e} dynastie~; fouilles du Sérapéum) [\href{http://cartelfr.louvre.fr/cartelfr/visite?srv=car_not_frame&idNotice=18930}{Atlas}]}}
\newglossaryentry{CoEg_obj_00000007}{name={N 390},type={obj},sort={N0390},text={*},parent={CoEg_coll_00000002},description={Statue d'Apis (Basse-Époque, XXX\textsuperscript{e} dynastie~; fouilles du Sérapéum) [\href{http://cartelfr.louvre.fr/cartelfr/visite?srv=car_not_frame&idNotice=20284}{Atlas}]}}
\newglossaryentry{CoEg_obj_00000008}{name={E 3023},type={obj},sort={E3023},text={*},parent={CoEg_coll_00000002},description={Statue,  dite du «~Scribe accroupi~» ([Ancien Empire, IV\textsuperscript{e}-V\textsuperscript{e} dynastie ?]~; fouilles du Sérapéum). Autre numéro d'inventaire~: N~2290 [\href{http://cartelfr.louvre.fr/cartelfr/visite?srv=car_not_frame&idNotice=14135}{Atlas}]\gls{CoEg_obj_00000008alias}}}
\newglossaryentry{CoEg_obj_00000008alias}{name={N 2290},type={obj},sort={N2290},text={},parent={CoEg_coll_00000002},description={Voir «~E~3023~»}}
\newglossaryentry{CoEg_obj_00000009}{name={N 405},type={obj},sort={N0405},text={*},parent={CoEg_coll_00000002},description={Stèle (Basse-Époque~; XXVI\textsuperscript{e} dynastie~; fouilles du Sérapéum). Reproduite dans \textsc{Mariette} Auguste, \textit{Choix de monuments et de dessins, découverts ou exécutés pendant le déblaiement du Sérapéum de Memphis}, Paris, Gide et J. Baudry, 1856, pl. \href{https://gallica.bnf.fr/ark:/12148/bpt6k3041140v/f31.image}{7} [\href{http://cartelfr.louvre.fr/cartelfr/visite?srv=car_not_frame&idNotice=20285}{Atlas}]}}
\newglossaryentry{CoEg_obj_00000010}{name={N 413},type={obj},sort={N0413},text={*},parent={CoEg_coll_00000002},description={Stèle (Troisième Période intermédiaire, XXII\textsuperscript{e} dynastie~; fouilles du Sérapéum) [\href{http://cartelfr.louvre.fr/cartelfr/visite?srv=car_not_frame&idNotice=27866}{Atlas}]}}
\newglossaryentry{CoEg_obj_00000011}{name={A 90},type={obj},sort={A090},text={*},parent={CoEg_coll_00000002},description={Statue de Neshor présentant une triade (Basse-Époque, XXVI\textsuperscript{e} dynastie). Autre numéro d'inventaire : N~91\gls{CoEg_obj_00000011alias}}}
\newglossaryentry{CoEg_obj_00000011alias}{name={N 91},type={obj},sort={N0091},text={},parent={CoEg_coll_00000002},description={Voir «~A~90~»}}
\newglossaryentry{CoEg_obj_00000012}{name={N 420},type={obj},sort={N0420},text={*},parent={CoEg_coll_00000002},description={Porte (fouilles du Sérapéum). Mariette lui a attribué le numéro 5 [\href{http://cartelfr.louvre.fr/cartelfr/visite?srv=car_not_frame&idNotice=18928}{Atlas}]}}
\newglossaryentry{CoEg_obj_00000013}{name={N 394 1 A à D},type={obj},sort={N03941A},text={*},parent={CoEg_coll_00000002},description={Vases canopes d'Apis réalisés sous Amenhotep III [\href{http://cartelfr.louvre.fr/cartelfr/visite?srv=car_not_frame&idNotice=23158}{Atlas}]}}
\newglossaryentry{CoEg_obj_00000014}{name={N 394 2 A à D}, type={obj},sort={N03942A},text={*},parent={CoEg_coll_00000002},description={Vases canopes d'Apis réalises sous Toutânkhamon [\href{http://cartelfr.louvre.fr/cartelfr/visite?srv=car_not_frame&idNotice=27861}{Atlas}]}}
\newglossaryentry{CoEg_obj_00000015}{name={N 407},type={obj},sort={N0407},text={*},parent={CoEg_coll_00000002},description={Stèle (Basse-Époque, XXVII\textsuperscript{e} dynastie~; fouilles du Sérapéum)}}
\newglossaryentry{CoEg_obj_00000016}{name={N 481},type={obj},sort={N0481},text={*},parent={CoEg_coll_00000002},description={Stèle (Troisième Période intermédiaire, XXII\textsuperscript{e} dynastie~; fouilles du Sérapéum). Autre numéro d'inventaire~: AF~123\gls{CoEg_obj_00000016alias}}}
\newglossaryentry{CoEg_obj_00000016alias}{name={AF 123},type={obj},sort={AF123},text={},parent={CoEg_coll_00000002},description={Voir «~N~481~»}}
\newglossaryentry{CoEg_obj_00000017}{name={N 488},type={obj},sort={N0488},text={*},parent={CoEg_coll_00000002},description={Stèle (Troisième Période intermédiaire, XXII\textsuperscript{e} dynastie~; fouilles du Sérapéum)}}
\newglossaryentry{CoEg_obj_00000018}{name={IM 3736},type={obj},sort={IM3736},text={*},parent={CoEg_coll_00000002},description={Stèle (Troisième Période intermédiaire, XXII\textsuperscript{e} dynastie~; fouilles du Sérapéum). Autre numéro d'inventaire~: S~1905\gls{CoEg_obj_00000018alias}}}
\newglossaryentry{CoEg_obj_00000018alias}{name={S 1905},type={obj},sort={S1905},text={},parent={CoEg_coll_00000002},description={Voir «~IM~3736~»}}
\newglossaryentry{CoEg_obj_00000019}{name={N 347},type={obj},sort={N0347},text={*},parent={CoEg_coll_00000002},description={Statue de Bès (Basse-Époque, XXX\textsuperscript{e} dynastie~; fouilles du Sérapéum). Mariette utilise le nom de Typhon [\href{http://cartelfr.louvre.fr/cartelfr/visite?srv=car_not_frame&idNotice=20282&langue=fr}{Atlas}]}}
\newglossaryentry{CoEg_obj_00000020}{name={},type={obj},sort={CoEg_org_00000021},text={*},parent={CoEg_coll_00000021},description={ms bilingue de Turin ?}}
\newglossaryentry{CoEg_obj_00000021}{name={EA 24},type={obj},sort={EA24},text={*},parent={CoEg_coll_00000005},description={Pierre de Rosette (époque ptolémaïque~; Rosette) [\href{https://www.britishmuseum.org/collection/object/Y_EA24}{Brit. Mus. coll.}~; \href{https://www.wikidata.org/wiki/Q48584}{Wikidata}]}}
\newglossaryentry{CoEg_obj_00000022}{name={1791},type={obj},sort={1791},text={*},parent={CoEg_coll_00000021},description={Livre des morts d'Ioufânkh (Basse-Époque, XXVI\textsuperscript{e} dynastie~; Thèbes ?). Il resta longtemps le livre des morts le plus complet connu [\href{http://collezioni.museoegizio.it/it-IT/material/Cat_1791}{Mus. Eg. coll.}, \href{https://papyri.museoegizio.it/!170935}{Mus. Eg. pap.}]}}
\newglossaryentry{CoEg_obj_00000023}{name={},type={obj},sort={CoEg_org_00000026},text={*},description={pap de Paris}}
\newglossaryentry{CoEg_obj_00000024}{name={Royal MS 1 D VIII},type={obj},sort={},text={*},parent={CoEg_coll_00000005bis},description={\textit{Codex alexandrinus}, manuscrit biblique. Appartenait jusqu'en 1973 aux collections du British Museum [\href{http://www.bl.uk/manuscripts/FullDisplay.aspx?ref=Royal_MS_1_D_VIII}{en ligne}]}}
\newglossaryentry{CoEg_obj_00000025}{name={7155},type={obj},sort={7155},text={*},parent={CoEg_coll_00000021},description={«~Table isiaque~», ou \textit{mensa Isiaca} (époque romaine ?) [\href{http://collezioni.museoegizio.it/it-IT/material/Cat_7155}{Mus. Eg. coll.}~; \href{https://www.wikidata.org/wiki/Special:EntityPage/Q3817829}{Wikidata}]}}
\newglossaryentry{CoEg_obj_00000026}{name={A 105},type={obj},sort={A105},text={*},parent={CoEg_coll_00000002},description={Statue de Sékhemka (Ancien Empire, V\textsuperscript{e} dynastie ; fouilles du Sérapéum\gls{CoEg_place_00000004}). Autres numéros d'inventaire~: «~E~3022~»\gls{CoEg_obj_00000026alias1} et «~N~111~»\gls{CoEg_obj_00000026alias1} [\href{http://cartelfr.louvre.fr/cartelfr/visite?srv=car_not_frame&idNotice=31644}{Atlas}]}}
\newglossaryentry{CoEg_obj_00000026alias1}{name={E 3022},type={obj},sort={E3022},text={},parent={CoEg_coll_00000002},description={Voir «~A~105~»}}
\newglossaryentry{CoEg_obj_00000026alias2}{name={N 111 (?)},type={obj},sort={N111},text={},parent={CoEg_coll_00000002},description={Voir «~A~105~»}}
\newglossaryentry{CoEg_obj_00000027}{name={2031},type={obj},sort={2031},text={*},parent={CoEg_coll_00000021},description={Papyrus érotique de Turin [\href{https://collezioni.museoegizio.it/it-IT/material/Cat_2031}{Mus. Eg. coll.}~; \href{https://papyri.museoegizio.it/!8}{Mus. Eg. pap.}, \href{https://www.wikidata.org/wiki/Q3362985}{Wikidata}]}}
\newglossaryentry{CoEg_obj_00000028}{name={A 102},type={obj},sort={A102},text={*},parent={CoEg_coll_00000002},description={Groupe familial de Sékhemka (Ancien Empire, V\textsuperscript{e} dynastie ; fouilles du Sérapéum\gls{CoEg_place_00000004}). Autres numéros d'inventaire~: «~E~3026~»\gls{CoEg_obj_00000028alias1} et «~N~116~»\gls{CoEg_obj_00000028alias2} [\href{http://cartelfr.louvre.fr/cartelfr/visite?srv=car_not_frame&idNotice=14145&langue=fr}{Atlas}]}}
\newglossaryentry{CoEg_obj_00000028alias1}{name={E 3026},type={obj},sort={E3022},text={},parent={CoEg_coll_00000002},description={Voir «~A~102~»}}
\newglossaryentry{CoEg_obj_00000028alias2}{name={N 116},type={obj},sort={N0116},text={},parent={CoEg_coll_00000002},description={Voir «~A~102~»}}
\newglossaryentry{CoEg_obj_00000029}{name={A 103},type={obj},sort={A103},text={*},parent={CoEg_coll_00000002},description={Statue de Sékhemka (Ancien Empire, V\textsuperscript{e} dynastie ; fouilles du Sérapéum\gls{CoEg_place_00000004}). Autres numéros d'inventaire~: «~E~3025~»\gls{CoEg_obj_00000029alias1} et «~N~115~»\gls{CoEg_obj_00000029alias2} [\href{http://cartelfr.louvre.fr/cartelfr/visite?srv=car_not_frame&idNotice=14146&langue=fr}{Atlas}]}}
\newglossaryentry{CoEg_obj_00000029alias1}{name={E 3025},type={obj},sort={E3025},text={},parent={CoEg_coll_00000002},description={Voir «~A~103~»}}
\newglossaryentry{CoEg_obj_00000029alias2}{name={N 115},type={obj},sort={N0115},text={},parent={CoEg_coll_00000002},description={Voir «~A~103~»}}
\newglossaryentry{CoEg_obj_00000030}{name={A 104},type={obj},sort={A104},text={*},parent={CoEg_coll_00000002},description={Statue de Sékhemka (Ancien Empire, V\textsuperscript{e} dynastie ; fouilles du Sérapéum\gls{CoEg_place_00000004}). Autres numéros d'inventaire~: «~E~3022~»\gls{CoEg_obj_00000030alias1} et «~N~111~»\gls{CoEg_obj_00000030alias2} [\href{http://cartelfr.louvre.fr/cartelfr/visite?srv=car_not_frame&idNotice=14147&langue=fr}{Atlas}]}}
\newglossaryentry{CoEg_obj_00000030alias1}{name={E 3021},type={obj},sort={E3021},text={},parent={CoEg_coll_00000002},description={Voir «~A~104~»}}
\newglossaryentry{CoEg_obj_00000030alias2}{name={N 110},type={obj},sort={N0110},text={},parent={CoEg_coll_00000002},description={Voir «~A~104~»}}
\newglossaryentry{CoEg_obj_00000031}{name={N 410},type={obj},sort={N410},text={*},parent={CoEg_coll_00000002},description={Stèle funéraire d'Apis, an LII de Ptolémée VIII Évergète II\gls{CoEg_pers_00000187} (époque ptolémaïque ; fouilles du Sérapéum\gls{CoEg_place_00000004}). Autres numéros d'inventaire~: «~IM~4246~»\gls{CoEg_obj_00000031alias}}}
\newglossaryentry{CoEg_obj_00000031alias}{name={IM 4246},type={obj},sort={IM4246},text={},parent={CoEg_coll_00000002},description={Voir «~N~410~»}}

% Auteurs

\newglossaryentry{CoEg_author_00000001}{name={},type={bibl},sort={Mariette},text={publications},description={\textsc{Mariette [Pacha]} Auguste\gls{CoEg_pers_00000001}}}
\newglossaryentry{CoEg_author_00000031}{name={},type={bibl},sort={Diodore},text={publications},description={\textsc{Diodore de Sicile}\gls{CoEg_pers_00000031}}}
\newglossaryentry{CoEg_author_00000032}{name={},type={bibl},sort={Rouge},text={publications},description={\textsc{Rougé (de)} Emmanuel\gls{CoEg_pers_00000032}}}
\newglossaryentry{CoEg_author_00000051}{name={},type={bibl},sort={Brunet},text={publications},description={\textsc{Brunet de Presle} Wladimir\gls{CoEg_pers_00000051}}}
\newglossaryentry{CoEg_author_00000061}{name={},type={bibl},sort={Lepsius},text={publications},description={\textsc{Lepsius} Karl Richard\gls{CoEg_pers_00000061}}}
\newglossaryentry{CoEg_author_00000094}{name={},type={bibl},sort={Champollion},text={publications},description={\textsc{Champollion le Jeune} Jean-François\gls{CoEg_pers_00000094}}}
\newglossaryentry{CoEg_author_00000097}{name={},type={bibl},sort={Letronne},text={publications},description={\textsc{Letronne} Jean Antoine\gls{CoEg_pers_00000097}}}
\newglossaryentry{CoEg_author_00000108}{name={},type={bibl},sort={Herodote},text={publications},description={\textsc{Hérodote}\gls{CoEg_pers_00000108}}}
\newglossaryentry{CoEg_author_00000109}{name={},type={bibl},sort={Plutarque},text={publications},description={\textsc{Plutarque}\gls{CoEg_pers_00000109}}}
\newglossaryentry{CoEg_author_00000111}{name={},type={bibl},sort={Tacite},text={publications},description={\textsc{Tacite}\gls{CoEg_pers_00000111}}}
\newglossaryentry{CoEg_author_00000112}{name={},type={bibl},sort={Clement},text={publications},description={\textsc{Clément d'Alexandrie}\gls{CoEg_pers_00000112}}}
\newglossaryentry{CoEg_author_00000114}{name={},type={bibl},sort={Eusebe},text={publications},description={\textsc{Eusèbe de Césarée}\gls{CoEg_pers_00000114}}}
\newglossaryentry{CoEg_author_00000200}{name={},type={bibl},sort={Lenormant},text={publications},description={\textsc{Lenormant} François\gls{CoEg_pers_00000200}}}
\newglossaryentry{CoEg_author_00000158}{name={},type={bibl},sort={Georges},text={publications},description={\textsc{Georges le Syncelle}\gls{CoEg_pers_00000158}}}
\newglossaryentry{CoEg_author_00000159}{name={},type={bibl},sort={Strabon},text={publications},description={\textsc{Strabon}\gls{CoEg_pers_00000159}}}
\newglossaryentry{CoEg_author_00000164}{name={},type={bibl},sort={Sharpe},text={publications},description={\textsc{Sharpe} Samuel\gls{CoEg_pers_00000164}}}
\newglossaryentry{CoEg_author_00000165}{name={},type={bibl},sort={Pline},text={publications},description={\textsc{Pline l'Ancien}\gls{CoEg_pers_00000165}}}
\newglossaryentry{CoEg_author_00000166}{name={},type={bibl},sort={Ammien},text={publications},description={\textsc{Ammien Marcellin}\gls{CoEg_pers_00000166}}}
\newglossaryentry{CoEg_author_00000167}{name={},type={bibl},sort={Suetone},text={publications},description={\textsc{Suétone}\gls{CoEg_pers_00000167}}}
\newglossaryentry{CoEg_author_00000169}{name={},type={bibl},sort={Spartianus},text={publications},description={\textsc{Spartianus}\gls{CoEg_pers_00000169}}}
\newglossaryentry{CoEg_author_00000174}{name={},type={bibl},sort={Zoega},text={publications},description={\textsc{Zoega} Georg\gls{CoEg_pers_00000174}}}
\newglossaryentry{CoEg_author_00000175}{name={},type={bibl},sort={Tochon},text={publications},description={\textsc{Tochon [d'Annecy]} Joseph-François\gls{CoEg_pers_00000175}}}
\newglossaryentry{CoEg_author_00000176}{name={},type={bibl},sort={Solin},text={publications},description={\textsc{Solin}\gls{CoEg_pers_00000176}}}
\newglossaryentry{CoEg_author_00000178}{name={},type={bibl},sort={Pomponius},text={publications},description={\textsc{Pomponius Mela}\gls{CoEg_pers_00000178}}}
\newglossaryentry{CoEg_author_00000179}{name={},type={bibl},sort={Birch},text={publications},description={\textsc{Birch} Samuel\gls{CoEg_pers_00000179}}}
\newglossaryentry{CoEg_author_00000184}{name={},type={bibl},sort={Jablonski},text={publications},description={\textsc{Jablonski} Paul Ernest\gls{CoEg_pers_00000184}}}
\newglossaryentry{CoEg_author_00000193}{name={},type={bibl},sort={Dodwell},text={publications},description={\textsc{Dodwell} Henry\gls{CoEg_pers_00000193}}}
\newglossaryentry{CoEg_author_00000194}{name={},type={bibl},sort={Marsham},text={publications},description={\textsc{Marsham} John\gls{CoEg_pers_00000194}}}
\newglossaryentry{CoEg_author_00000196}{name={},type={bibl},sort={DesVi},text={publications},description={\textsc{Des Vignoles} Alphonse\gls{CoEg_pers_00000196}}}
\newglossaryentry{CoEg_author_00000202}{name={},type={bibl},sort={Macrobe},text={publications},description={\textsc{Macrobe}\gls{CoEg_pers_00000202}}}
\newglossaryentry{CoEg_author_00000204}{name={},type={bibl},sort={Pausanias},text={publications},description={\textsc{Pausanias}\gls{CoEg_pers_00000204}}}
\newglossaryentry{CoEg_author_00000207}{name={},type={bibl},sort={Luynes},text={publications},description={\textsc{Luynes (d'Albert de)} Honoré Théodoric\gls{CoEg_pers_00000207}}}
\newglossaryentry{CoEg_author_00000211}{name={},type={bibl},sort={Peyron},text={publications},description={\textsc{Peyron} Bernardino\gls{CoEg_pers_00000211}}}
\newglossaryentry{CoEg_author_00000212}{name={},type={bibl},sort={Reuvens},text={publications},description={\textsc{Reuvens} Caspar\gls{CoEg_pers_00000212}}}
\newglossaryentry{CoEg_author_00000216}{name={},type={bibl},sort={Denys},text={publications},description={\textsc{Denys le Périégète}\gls{CoEg_pers_00000216}}}
\newglossaryentry{CoEg_author_00000217}{name={},type={bibl},sort={Cyrille},text={publications},description={\textsc{Cyrille d'Alexandrie}\gls{CoEg_pers_00000217}}}
\newglossaryentry{CoEg_author_00000229}{name={},type={bibl},sort={Jamblique},text={publications},description={\textsc{Jamblique}\gls{CoEg_pers_00000229}}}
\newglossaryentry{CoEg_author_00000242}{name={},type={bibl},sort={Theophile},text={publications},description={\textsc{Théophile d'Antioche}\gls{CoEg_pers_00000242}}}
\newglossaryentry{CoEg_author_00000243}{name={},type={bibl},sort={Maury},text={publications},description={\textsc{Maury} Alfred\gls{CoEg_pers_00000243}}}
\newglossaryentry{CoEg_author_Rochas}{name={},type={bibl},sort={Rochas},text={publications},description={\textsc{Rochas}}}
\newglossaryentry{CoEg_author_Forshall}{name={},type={bibl},sort={Forshall},text={publications},description={\textsc{Forshall} Josiah}}
\newglossaryentry{CoEg_author_Elien}{name={},type={bibl},sort={Elien},text={publications},description={\textsc{Élien le Sophiste}}}

%publications

\newglossaryentry{CoEg_bibl_00000001}{name={},type={bibl},text={*},sort={1852},description={«~Mémoire sur le Sérapéum de Memphis~», \textit{Mémoires présentés par divers savants étrangers à l’Académie} 2, 1852, p. 552-576 [\href{https://www.persee.fr/doc/mesav_0398-3587_1852_num_2_1_1014}{en ligne}]},parent={CoEg_author_00000051}}
\newglossaryentry{CoEg_bibl_00000002}{name={},type={bibl},text={*},sort={1849},description={\textit{Denkmäler aus Ägypten und Äthiopien}, Berlin, Nicolaische Buchhandlung, 1849-1859 [\href{http://edoc3.bibliothek.uni-halle.de/lepsius/start.html}{en ligne}~; \href{https://catalogue.bnf.fr/ark:/12148/cb30796243q}{cat. gén. BNF}]},parent={CoEg_author_00000061}}
\newglossaryentry{CoEg_bibl_00000003}{name={},type={bibl},text={*},sort={Moniteur},description={\textit{Le Moniteur} [\href{https://catalogue.bnf.fr/ark:/12148/cb34452336z}{cat. gén. BNF}~; \href{https://data.bnf.fr/fr/34452336/gazette_nationale_ou_le_moniteur_universel/}{data.bnf}~; \href{https://gallica.bnf.fr/ark:/12148/cb34452336z/date.item}{en ligne}~; \href{https://catalogue.bnf.fr/ark:/12148/cb34452336z}{cat. gén. BNF}]}}
\newglossaryentry{CoEg_bibl_00000004}{name={},type={bibl},sort={1853},text={*},sort={1853},description={«~Ouverture des salles égyptiennes du premier étage, au Louvre. Nouveaux monuments envoyés par M. Mariette~», \textit{Le Moniteur}\gls{CoEg_bibl_00000003}, 8 juillet 1853, p. 2 [\href{https://www.retronews.fr/journal/gazette-nationale-ou-le-moniteur-universel/8-juillet-1853/149/1538523/2}{en ligne}]},parent={CoEg_author_00000032}}
\newglossaryentry{CoEg_bibl_00000005}{name={},type={bibl},sort={1851},text={*},description={«~Moyens de conserver indéfiniment les monuments en pierre calcaire~», \textit{Comptes-rendus de l’Académie des sciences}, 1851, p. 622 [\href{https://gallica.bnf.fr/ark:/12148/bpt6k29901/f624.item}{en ligne}]},parent={CoEg_author_Rochas}}
\newglossaryentry{CoEg_bibl_00000006}{name={},type={bibl},text={*},sort={Annotateur},description={\textit{L'Annotateur}}}
\newglossaryentry{CoEg_bibl_00000007}{name={},type={bibl},text={*},description={\textit{Théophanie} [\href{https://data.bnf.fr/fr/17731536/eusebe_de_cesaree_theophanie/}{data.bnf}]},parent={CoEg_author_00000114},sort={Theo}}
\newglossaryentry{CoEg_bibl_00000008}{name={},type={bibl},text={*},description={\textit{Pédagogue} [\href{https://data.bnf.fr/fr/12012159/clement_d_alexandrie_pedagogue/}{data.bnf}]},parent={CoEg_author_00000112}}
\newglossaryentry{CoEg_bibl_00000009}{name={},type={bibl},text={*},sort={Description},description={\textit{Description de l'Égypte}, Paris, Imprimerie impériale, 1810-1829 [\href{https://catalogue.bnf.fr/ark:/12148/cb13192444t}{cat. gén. BNF}~; \href{https://data.bnf.fr/fr/13192444/description_de_l_egypte/}{data.bnf}~; \href{https://digi.ub.uni-heidelberg.de/diglit/jomard1809ga}{en ligne}]}}
\newglossaryentry{CoEg_bibl_00000010}{name={},type={bibl},text={*},description={\textit{Die Chronologie der Ägypter}, Berlin, Nicolaische Buchhandlung, 1849 [\href{https://catalogue.bnf.fr/ark:/12148/cb30796242c}{cat. gén. BNF}]},sort={1849},parent={CoEg_author_00000061}}
\newglossaryentry{CoEg_bibl_00000011}{name={},type={bibl},text={*},description={\textit{Chronographie} [\href{https://data.bnf.fr/fr/13546411/georges_le_syncelle_chronographie/}{data.bnf}]},parent={CoEg_author_00000158}}
\newglossaryentry{CoEg_bibl_00000012}{name={},type={bibl},text={*},description={\textit{Histoire naturelle} [\href{https://data.bnf.fr/fr/12011602/pline_l_ancien_histoire_naturelle/}{data.bnf}]},parent={CoEg_author_00000165}}
\newglossaryentry{CoEg_bibl_00000013}{name={},type={bibl},text={*},description={\textit{Description of the Greek Papyri in the British Museum}, 1\textsuperscript{re}, Londres, The Trustees of the British Museum, 1839},sort={1839},parent={CoEg_author_Forshall}}
\newglossaryentry{CoEg_bibl_00000014}{name={},type={bibl},text={*},description={\textit{Recueil des inscriptions grecques et latines de l'Égypte}, Paris, Imprimerie royale, 1842 [\href{https://catalogue.bnf.fr/ark:/12148/cb33996749x}{cat. gén. BNF}]},sort={1842},parent={CoEg_author_00000097}}
\newglossaryentry{CoEg_bibl_00000015}{name={},type={bibl},text={*},description={\textit{Egyptian Inscriptions from the British Museum and other Sources}, Londres, E. Moxon, 1837-1855},sort={1837},parent={CoEg_author_00000164}}
\newglossaryentry{CoEg_bibl_00000016}{name={},type={bibl},text={*},description={«~Renseignements sur les soixante-quatre Apis trouvés dans les souterrains du Sérapéum	~», \textit{Bulletin archéologique de l'Athénæum français}, 1855, p. 45, 53, 66 et 85 ; 1856, p. 58 et 74 [\href{https://digi.ub.uni-heidelberg.de/diglit/mariette1904bd1/0373}{en ligne}]},sort={1855},parent={CoEg_author_00000001}}
\newglossaryentry{CoEg_bibl_00000017}{name={},type={bibl},text={*},description={\textit{Numi Aegyptii imperatorii}, Rome, A. Fulgoni, 1787 [\href{https://catalogue.bnf.fr/ark:/12148/cb33513249g}{cat. gén. BNF}~; \href{https://digi.ub.uni-heidelberg.de/diglit/zoega1787}{en ligne}]}, sort={1787},parent={CoEg_author_00000174}}
\newglossaryentry{CoEg_bibl_00000018}{name={},type={bibl},text={*},description={\textit{Recherches historiques et géographiques sur les médailles des nomes ou préfectures de l'Égypte}, Paris, A.-A. Renouard, 1822 [\href{https://catalogue.bnf.fr/ark:/12148/cb314760452}{cat. gén. BNF}]},sort={1822},parent={CoEg_author_00000175}}
\newglossaryentry{CoEg_bibl_00000019}{name={},type={bibl},text={*},description={La Bible [\href{https://data.bnf.fr/fr/12008248/bible/}{data.bnf}]},sort={Bible}}
\newglossaryentry{CoEg_bibl_00000020}{name={},type={bibl},text={*},sort={Bible},description={\textit{Bible des Septantes} [\href{https://data.bnf.fr/fr/16659243/bible_--_versions_grecques_--_septante/}{data.bnf}]},sort={Bible},parent={CoEg_bibl_00000019}}
\newglossaryentry{CoEg_bibl_00000021}{name={},type={bibl},text={*},description={\textit{Histoires} [\href{https://data.bnf.fr/fr/12008360/herodote_histoires/}{data.bnf}]},parent={CoEg_author_00000108}}
\newglossaryentry{CoEg_bibl_00000022}{name={},type={bibl},text={*},description={\textit{Bibliothèque historique} [\href{https://data.bnf.fr/fr/12008253/diodore_de_sicile_bibliotheque_historique/}{data.bnf}]},parent={CoEg_author_00000031}}
\newglossaryentry{CoEg_bibl_00000023}{name={},type={bibl},text={*},description={\textit{Géographie} [\href{https://data.bnf.fr/fr/12194250/strabon_geographie/}{data.bnf}]},parent={CoEg_author_00000159}}
\newglossaryentry{CoEg_bibl_00000024}{name={},type={bibl},text={*},description={\textit{Histoires} [\href{https://data.bnf.fr/fr/12008409/ammien_marcellin_histoires/}{data.bnf}]},parent={CoEg_author_00000166}}
\newglossaryentry{CoEg_bibl_00000025}{name={},type={bibl},text={*},description={\textit{Annales} [\href{https://data.bnf.fr/fr/12008242/tacite_annales/}{data.bnf}]},parent={CoEg_author_00000111}}
\newglossaryentry{CoEg_bibl_00000026}{name={},type={bibl},text={*},description={\textit{Vie des douze Césars} [\href{https://catalogue.bnf.fr/ark:/12148/cb12008292d}{cat. gén. BNF}~; \href{https://data.bnf.fr/fr/12008292/suetone_vie_des_douze_cesars/}{data.bnf}]},parent={CoEg_author_00000167}}
\newglossaryentry{CoEg_bibl_00000027}{name={},type={bibl},text={*},description={\textit{Histoire auguste} [\href{https://data.bnf.fr/fr/12302233/histoire_auguste/}{data.bnf}]},parent={CoEg_author_00000169}}
\newglossaryentry{CoEg_bibl_00000028}{name={},type={bibl},text={*},description={\textit{Polyhistor} [\href{https://data.bnf.fr/fr/16590331/solin_recueil_de_faits_remarquables/}{data.bnf}~; \href{https://gallica.bnf.fr/ark:/12148/bpt6k23660m}{en ligne}]},parent={CoEg_author_00000176}}
\newglossaryentry{CoEg_bibl_00000029}{name={},type={bibl},text={*},description={\textit{De Iside et Osiride} [\href{https://data.bnf.fr/fr/12521757/plutarque_isis_et_osiris/}{data.bnf}~; \href{https://gallica.bnf.fr/ark:/12148/bpt6k23660m}{en ligne}]},parent={CoEg_author_00000109}}
\newglossaryentry{CoEg_bibl_00000030}{name={},type={bibl},text={*},description={\textit{Préparation évangélique} [\href{https://data.bnf.fr/fr/12012386/eusebe_de_cesaree_preparation_evangelique/}{data.bnf}]},parent={CoEg_author_00000114},sort={Prep}}
\newglossaryentry{CoEg_bibl_00000031}{name={},type={bibl},text={*},description={\textit{De situ orbis} [\href{https://data.bnf.fr/fr/12551639/pomponius_mela_chorographie/}{data.bnf}]},parent={CoEg_author_00000178},sort={Prep}}
\newglossaryentry{CoEg_bibl_00000032}{name={},type={bibl},text={*},description={\textit{Pantheon ægyptiorum}, Francfort-sur-l'Oder, Johann Christian Kleyb, 1750-1752 [\href{https://digi.ub.uni-heidelberg.de/diglit/jablonski1750ga}{en ligne}]}, sort={1750},parent={CoEg_author_00000184}}
\newglossaryentry{CoEg_bibl_00000033}{name={},type={bibl},text={*},description={\textit{Symposiaques} [\href{https://data.bnf.fr/fr/12285321/plutarque_propos_de_table/}{data.bnf}]},parent={CoEg_author_00000109}}
\newglossaryentry{CoEg_bibl_00000034}{name={},type={bibl},text={*},sort={1842},description={\textit{Auswahl der wichtigsten Urkunden des aegyptischen Alterthums}, Leipzig, Georg Widand, 1842 [\href{https://catalogue.bnf.fr/ark:/12148/cb307962353}{cat. gén. BNF}]},parent={CoEg_author_00000061}}
\newglossaryentry{CoEg_bibl_00000035}{name={},type={bibl},text={*},sort={1842},description={\textit{Das Todtenbuch der Ägypter}, Leipzig, Georg Widand, 1842 [\href{http://catalogue.bnf.fr/ark:/12148/cb30796269d}{cat. gén. BNF}~; \href{https://gallica.bnf.fr/ark:/12148/bpt6k91064165}{exemplaire de Mariette en ligne}]},parent={CoEg_author_00000061}}
\newglossaryentry{CoEg_bibl_00000036}{name={},type={bibl},text={*},sort={Archi},description={\textit{Archives des missions scientifiques}, Paris, Imprimerie nationale, 1850-1889 [\href{https://catalogue.bnf.fr/ark:/12148/cb32701360s}{cat. gén. BNF}]}}
\newglossaryentry{CoEg_bibl_00000038}{name={},type={bibl},text={*},description={\textit{De la nature des animaux} [\href{https://data.bnf.fr/fr/13543400/elien_le_sophiste_de_la_nature_des_animaux/}{data.bnf}]},parent={CoEg_author_Elien}}
\newglossaryentry{CoEg_bibl_00000039}{name={},type={bibl},text={*},sort={1850},description={«~Observations of a bronze figure of a bull, found in Cornwall~», \textit{Archaeological journal} 7, Londres, The Archaeological Institute of Great Britain and Ireland, 1850, p. 8-16 et 120 [\href{https://catalogue.bnf.fr/ark:/12148/cb318189595}{cat. gén. BNF}~; \href{https://archaeologydataservice.ac.uk/archiveDS/archiveDownload?t=arch-1132-1/dissemination/pdf/007/007_008-016_120.pdf}{en ligne}]},parent={CoEg_author_00000179}}
\newglossaryentry{CoEg_bibl_00000040}{name={},type={bibl},text={*},description={\textit{Histoires} [\href{https://data.bnf.fr/fr/12264318/tacite_histoires/}{data.bnf}]},parent={CoEg_author_00000111}}
\newglossaryentry{CoEg_bibl_00000041}{name={},type={bibl},text={*},description={[trad. J. L. Burnouf\gls{CoEg_pers_00000227}] \textit{Œuvres complètes}, Paris, 1828},parent={CoEg_author_00000111}}
\newglossaryentry{CoEg_bibl_00000042}{name={},type={bibl},text={*},sort={1842},description={[avec \textsc{Arundale} Francis et \textsc{Bonomi} Joseph] \textit{Gallery of Antiquities selected from the British Museum}, Londres, J. Weale, 1842 [\href{https://archive.org/details/galleryofantiqui00brit/page/n5/mode/2up}{en ligne}]},parent={CoEg_author_00000179}}
\newglossaryentry{CoEg_bibl_00000043}{name={},type={bibl},text={*},sort={1853},description={«~Mémoire sur l'inscription du tombeau d’Ahmès, chef des Nautoniers~», \textit{Mémoires présentés par divers savants étrangers à l'Académie des inscriptions et belles-lettres} 3, 1853, p. 1-196 [\href{https://www.persee.fr/doc/mesav_0398-3587_1853_num_3_1_1016}{en ligne}]},parent={CoEg_author_00000032}}
\newglossaryentry{CoEg_bibl_00000044}{name={},type={bibl},text={*},sort={1684},description={\textit{Appendix ad dissertationes cyprianicas}, Oxford, Sheldon, 1684, p. 1-196 [\href{https://catalogue.bnf.fr/ark:/12148/cb303457403}{cat. gén. BNF}]},parent={CoEg_author_00000193}}
\newglossaryentry{CoEg_bibl_00000045}{name={},type={bibl},text={*},sort={1672},description={\textit{Chronicus canon ægyptiacus, ebraicus, græcus, et disquisitiones},Londres, G. Wells et A. Scott, 1672, p. 1-196 [\href{https://catalogue.bnf.fr/ark:/12148/cb30891162p}{cat. gén. BNF}]},parent={CoEg_author_00000194}}
\newglossaryentry{CoEg_bibl_00000046}{name={},type={bibl},text={*},description={\textit{Miscellanea Berolinensia ad incrementum scientiarum}},parent={CoEg_author_00000196}}
\newglossaryentry{CoEg_bibl_00000047}{name={},type={bibl},text={*},sort={1853},description={«~Les livres ches les Égyptiens~», \textit{Le Correspondant} 40 (nouvelle série 4), Paris, Charles Douniol, 1857, p. 252-273 (Mariette se réfère à un \href{https://catalogue.bnf.fr/ark:/12148/cb30791220v}{tirage à part}) [\href{https://gallica.bnf.fr/ark:/12148/bpt6k4225632d/f258.item}{périodique complet en ligne}]},parent={CoEg_author_00000200}}
\newglossaryentry{CoEg_bibl_00000048}{name={},type={bibl},text={*},sort={1841},description={\textit{Dictionnaire égyptien en écriture hiéroglyphique}, Paris, Firmin-Didot, 1841-1843 [\href{https://catalogue.bnf.fr/ark:/12148/cb33988260b}{cat. gén. BNF}~; \href{https://gallica.bnf.fr/ark:/12148/bpt6k106209v.r=champollion.langFR}{en ligne}]},parent={CoEg_author_00000094}}
\newglossaryentry{CoEg_bibl_00000049}{name={},type={bibl},text={*},sort={1855},description={\textit{Notice sommaire des monuments égyptiens exposés dans les galeries du musée du Louvre} Paris, Simon Raçon et C\textsuperscript{ie}, 1855 [\href{https://catalogue.bnf.fr/ark:/12148/cb31253671s}{cat. gén. BNF}~; \href{https://gallica.bnf.fr/ark:/12148/bpt6k6488340d}{en ligne}]},parent={CoEg_author_00000032}}
\newglossaryentry{CoEg_bibl_00000050}{name={},type={bibl},text={*},description={\textit{Saturnales} [\href{https://data.bnf.fr/fr/13181235/macrobe_saturnales/}{data.bnf}]},parent={CoEg_author_00000202}}
\newglossaryentry{CoEg_bibl_00000051}{name={},type={bibl},text={*},description={\textit{Description de la Grèce} [\href{https://data.bnf.fr/fr/12012116/pausanias_description_de_la_grece/}{data.bnf}]},parent={CoEg_author_00000204}}
\newglossaryentry{CoEg_bibl_00000052}{name={},type={bibl},text={*},sort={1855},description={«~Inscription phénicienne sur une pierre à libation du Sérapéum de Memphis~», \textit{Bulletin archéologique de l’Athénæum Français} 1, 1855, p. 77-78 [\href{https://archive.org/details/archathenfran00pari/page/76/mode/2up}{en ligne}]},parent={CoEg_author_00000207}}
\newglossaryentry{CoEg_bibl_00000053}{name={},type={bibl},text={*},sort={1841},description={\textit{Papiri greci del Museo britannico di Londra e della bibliotheca Vaticana}, Turin, 1841},parent={CoEg_author_00000211}}
\newglossaryentry{CoEg_bibl_00000054}{name={},type={bibl},text={*},sort={1830},description={\textit{Lettres à M. Letronne sur les papyrus bilingues et grecs, et sur quelques autres monumens gréco-égyptiens du musée d'antiquités de l'université de Leide}, Leide, S. et J. Luchtmans, 1830 [\href{https://catalogue.bnf.fr/ark:/12148/cb312020230}{cat. gén. BNF}~; \href{https://gallica.bnf.fr/ark:/12148/bpt6k399020x}{en ligne}]},parent={CoEg_author_00000212}}
\newglossaryentry{CoEg_bibl_00000055}{name={},type={bibl},text={*},description={\textit{Proteptique} [\href{https://data.bnf.fr/fr/12324201/clement_d_alexandrie_protreptique/}{data.bnf}]},parent={CoEg_author_00000112}}
\newglossaryentry{CoEg_bibl_00000056}{name={},type={bibl},text={*},description={\textit{Tour du monde} [\href{https://data.bnf.fr/fr/12306745/denys_le_periegete_tour_du_monde/}{data.bnf}]},parent={CoEg_author_00000216}}
\newglossaryentry{CoEg_bibl_00000057}{name={},type={bibl},text={*},description={\textit{Trois livres à Autolycus} [\href{https://data.bnf.fr/fr/12549500/theophile_d_antioche_trois_livres_a_autolycus/}{data.bnf}]},parent={CoEg_author_00000242}}
\newglossaryentry{CoEg_bibl_00000058}{name={},type={bibl},text={*},description={\textit{Contre Julien} [\href{https://data.bnf.fr/fr/13319347/cyrille_contre_julien/}{data.bnf}]},parent={CoEg_author_00000217}}
\newglossaryentry{CoEg_bibl_00000059}{name={},type={bibl},text={*},sort={1857},description={\textit{Histoire des religions de la Grèce antique}, Paris, Ladrange, 1857-1859 [\href{https://catalogue.bnf.fr/ark:/12148/cb309128522}{cat. gén. BNF}~; \href{https://gallica.bnf.fr/services/engine/search/sru?operation=searchRetrieve&version=1.2&collapsing=disabled&query=dc.relation all "cb309128522"}{en ligne}]},parent={CoEg_author_00000243}}
\newglossaryentry{CoEg_bibl_00000060}{name={},type={bibl},text={*},description={\textit{Mystères d'Égypte} [\href{https://data.bnf.fr/fr/12381196/jamblique_mysteres_d_egypte/}{data.bnf}]},parent={CoEg_author_00000229}}
\newglossaryentry{CoEg_bibl_00000061}{name={},type={bibl},text={*},sort={Genese},description={Genèse [\href{https://data.bnf.fr/fr/12008298/bible__a_t___-_genese/}{data.bnf}]},parent={CoEg_bibl_00000019}}
\newglossaryentry{CoEg_bibl_00000062}{name={},type={bibl},sort={Evangiles},text={*},description={Évangiles [\href{https://data.bnf.fr/fr/12008275/bible__n_t___-_evangiles/}{data.bnf}]},parent={CoEg_bibl_00000019}}
\newglossaryentry{CoEg_bibl_imn}{name={titre : CoEg_bibl_imn},type={bibl},text={!},description={blabla}}

%Glossaire

\newglossaryentry{CoEg_entry_00000001}{name={simoun},type={entry},text={simoun},description={De l'arabe \foreignlanguage{arabic}{سَمُوم} [\textit{sam\={u}m}]. Vent chaud, sec et violent qui souffle sur les côtes orientales de la mer Méditerranée. Mariette utilise le terme avec une majuscule [\href{https://www.cnrtl.fr/definition/simoun}{CNRTL}, \href{https://www.wikidata.org/wiki/Q646585}{Wikidata}]}}
\newglossaryentry{CoEg_entry_00000002}{name={pacha},type={entry},text={pacha},description={Du turc ottoman \foreignlanguage{arabic}{پاشا} [\textit{p\={a}\v{s}\={a}}]. Titre honorifique ottoman. réservé aux plus hauts dignitaires et aux souverains. Porté après le nom [\href{https://www.cnrtl.fr/definition/pacha}{CNRTL}, \href{https://www.wikidata.org/wiki/Q184951}{Wikidata}]}}
\newglossaryentry{CoEg_entry_00000003}{name={bey},type={entry},text={bey},description={Du turc ottoman \foreignlanguage{arabic}{بك} [\textit{beg}] «~seigneur~». Titre honorifique ottoman. Les officiers civils et militaires le portent après leur nom. Dans ce cas, Mariette le joint par un tiret, sans majuscule [ \href{https://www.cnrtl.fr/definition/bey}{CNRTL}, \href{https://www.wikidata.org/wiki/Q181217}{Wikidata}]}}
\newglossaryentry{CoEg_entry_00000004}{name={moudir},type={entry},text={moudir},description={De l'arabe \foreignlanguage{arabic}{مُدِير} [\textit{mudīr}] «~directeur~». Gouverneur ou préfet ottoman [\href{https://www.cnrtl.fr/definition/moudir}{CNRTL}]}}
\newglossaryentry{CoEg_entry_00000005}{name={firman},type={entry},text={firman},description={Du turc ottoman \foreignlanguage{arabic}{فرمان} [\textit{ferm\={a}n}] «~ordre, décret~». Autorisation officielle quelconque [\href{https://www.cnrtl.fr/definition/firman}{CNRTL}]}}
\newglossaryentry{CoEg_entry_00000006}{name={drogman},type={entry},text={drogman},description={De l'arabe \foreignlanguage{arabic}{تُرْجُمَان} [\textit{tur\v{g}umān}] «~guide, interprète~». Agent auxiliaire des consulats ou des étrangers en voyage [\href{https://www.cnrtl.fr/definition/drogman}{CNRTL}, \href{https://www.wikidata.org/wiki/Q1136290}{Wikidata}]}}
\newglossaryentry{CoEg_entry_00000007}{name={divan},type={entry},text={divan},description={Du persan \foreignlanguage{arabic}{دیوان} [\textit{dīwān}] «~rassemblement, réunion, conseil~». Administration, gouvernement [\href{https://www.cnrtl.fr/definition/divan}{CNRTL}, \href{https://www.wikidata.org/wiki/Q830104}{Wikidata}]}}
\newglossaryentry{CoEg_entry_00000008}{name={para},type={entry},text={para},description={Pièce ottomane de petite monnaie en cuivre [\href{https://www.cnrtl.fr/definition/para}{CNRTL}, \href{https://www.wikidata.org/wiki/Q928186}{Wikidata}]}}
\newglossaryentry{CoEg_entry_00000009}{name={fellah},type={entry},text={fellah},description={De l'arabe \foreignlanguage{arabic}{فَلَّاح} [\textit{fall\={a}\d{h}}], «~paysan~» [\href{https://www.cnrtl.fr/definition/fellah}{CNRTL}, \href{https://www.wikidata.org/wiki/Q685896}{Wikidata}]}}
\newglossaryentry{CoEg_entry_00000010}{name={effendi},type={entry},text={effendi},description={Du turc ottoman \foreignlanguage{arabic}{افندی} [\textit{efendi}]. Titre de respect et de courtoisie, notamment propre aux lettrés [\href{https://www.cnrtl.fr/definition/effendi}{CNRTL}, \href{https://www.wikidata.org/wiki/Q321803}{Wikidata}]}}
\newglossaryentry{CoEg_entry_00000011}{name={proscynème},type={entry},text={proscynème},description={Du grec ancien προσκύνημα [\textit{proskynèma}] «~adoration~». Désigne les formules d’offrandres et les stèles qui les portent}}
\newglossaryentry{CoEg_entry_00000012}{name={dahabieh},type={entry},text={dahabieh},description={De l'arabe \foreignlanguage{arabic}{ذَهَبِيَّة} [\textit{\d{d}ahabiyyah}] «~dorée~». Embarcation nilotique à faible tirant d’eau et naviguant à l’aide de deux mâts à voile latine [\href{https://www.cnrtl.fr/definition/dahabieh}{CNRTL}, \href{https://www.wikidata.org/wiki/Q441975}{Wikidata}]}}
\newglossaryentry{CoEg_entry_00000013}{name={arnaoute},type={entry},text={arnaoute},description={Du turc ottoman \foreignlanguage{arabic}{آرناوود} [\textit{ārnāvut}] «~Albanais~», notamment des guerriers formant des corps mercenaires dans le monde ottoman [\href{https://www.wikidata.org/wiki/Q2863045}{Wikidata}]}}
\newglossaryentry{CoEg_entry_00000014}{name={fantasia},type={entry},text={fantasia},description={Jeu équestre [\href{https://www.cnrtl.fr/definition/fantasia}{CNRTL}, \href{https://www.wikidata.org/wiki/Q1395855}{Wikidata} ]}}
\newglossaryentry{CoEg_entry_00000015}{name={djirid},type={entry},text={djirid},description={Jeu équestre [\href{https://www.wikidata.org/wiki/Q362605}{Wikidata}]}}
\newglossaryentry{CoEg_entry_00000016}{name={sheikh el-beled},type={entry},text={sheikh el-beled},plural={sheikhs el-belled},description={De l'arabe \foreignlanguage{arabic}{شَيْخ الْبَلَد} [\textit{\v{s}aīḫ al-balad}], «~chef de village~»}}
\newglossaryentry{CoEg_entry_00000017}{name={medjidie},type={entry},text={midjidi},description={ Pièce de monnaie ottomane. Mariett semble écrire «~midjidi~»}}
\newglossaryentry{CoEg_entry_00000018}{name={ad hoc},type={entry},text={ad hoc},description={Du latin, «~à cet effet~» [\href{https://www.cnrtl.fr/definition/ad hoc}{CNRTL}, \href{https://www.wikidata.org/wiki/Q192683}{Wikidata}]}}
\newglossaryentry{CoEg_entry_00000019}{name={cawass},type={entry},text={cawass},description={De l'arabe \foreignlanguage{arabic}{قَوَّاس} [\textit{qawwās}] \footnote{\textsc{Thatcher} G. W., \textit{Arabic grammar of the written language}, Londres – Heidelberg, 1911, p.~271.}, huissier [\href{https://www.cnrtl.fr/definition/cawas}{CNRTL}]}}
\newglossaryentry{CoEg_entry_00000020}{name={reïs},type={entry},text={reïs},description={De l'arabe \foreignlanguage{arabic}{رَئِيس} [\textit{raīs}], «~chef~» (notamment les chefs d'équipes sur les chantiers de fouilles) [\href{https://www.cnrtl.fr/definition/reis}{CNRTL}]}}
\newglossaryentry{CoEg_entry_00000021}{name={moudiria},type={entry},text={moudiria},description={Province dirigée par un moudir~; siège de l'administration correspondante}}
\newglossaryentry{CoEg_entry_00000022}{name={Pasteurs},type={entry},text={Pasteurs},description={Traduction fautive donnée par Manéthon du therme « Hyksos » (qui lui est désormais préféré) pour désigner des groupes d'origine asiatique installés dans le Delta pendant la Deuxième Période intermédiaire {[\href{https://www.wikidata.org/wiki/Q192319}{Wikidata}]}}}
\newglossaryentry{CoEg_entry_00000023}{name={Tybi (mois)},type={entry},text={Tybi},description={Premier mois de la saison de la germination {[\href{https://www.wikidata.org/wiki/Q290480}{Wikidata}]}}}
\newglossaryentry{CoEg_entry_00000024}{name={Thot (mois)},type={entry},text={Thot},description={Premier mois de la saison de l'inondation {[\href{https://www.wikidata.org/wiki/Q577200}{Wikidata}]}}}
\newglossaryentry{CoEg_entry_00000025}{name={Amenti},type={entry},text={Amenti},description={« Occident » (\foreignlanguage{translit}{\gls{CoEg_aeg_00000023}}), c'est-à-dire l'au-delà}}
\newglossaryentry{CoEg_entry_00000026}{name={modius},type={entry},text={modius},description={Mesure de céréales [\href{https://www.cnrtl.fr/definition/Modius}{CNRTL}]}}
\newglossaryentry{CoEg_entry_00000027}{name={mastaba},type={entry},text={mastaba},description={De l'arabe \foreignlanguage{arabic}{مصطبة} [\textit{maṣṭabah}] « banquette » ; désigne les tombes de particuliers à l'Ancien Empire, à la superstructure massive de briques. Mariette est à l'origine de la popularité de cette expression {[\href{https://www.cnrtl.fr/definition/mastaba}{CNRTL}, \href{https://www.wikidata.org/wiki/Q180927}{Wikidata}]}}}
\newglossaryentry{CoEg_entry_00000028}{name={sérapéum},type={entry},text={sérapéum},description={Temple du dieu Sérapis {[\href{https://www.cnrtl.fr/definition/sérapéum}{CNRTL}, \href{https://www.wikidata.org/wiki/Q281132}{Wikidata}]}}}
\newglossaryentry{CoEg_entry_00000029}{name={pharaon},type={entry},text={pharaon},description={Roi d'Égypte {[\href{https://www.cnrtl.fr/definition/pharaon}{CNRTL}, \href{https://www.wikidata.org/wiki/Q37110}{Wikidata}]}}}
\newglossaryentry{CoEg_entry_00000030}{name={confer},type={entry},text={confer},description={Du latin \textit{confer} « comparez » {[\href{https://www.wikidata.org/wiki/Q1048501}{Wikidata}]}}}
\newglossaryentry{CoEg_entry_00000031}{name={desiderata},type={entry},text={desiderata},description={Du latin \textit{desiderata} « choses dont on regrette l'absence » ; lacune ou besoin {[\href{https://www.cnrtl.fr/definition/desiderata}{CNRTL}]}}}
\newglossaryentry{CoEg_entry_00000032}{name={hypogée},type={entry},text={hypogée},description={Tombe souterraine {[\href{https://www.cnrtl.fr/definition/hypogée}{CNRTL}, \href{https://www.wikidata.org/wiki/Q665247}{Wikidata}]}}}
\newglossaryentry{CoEg_entry_00000033}{name={in-octavo},type={entry},text={in-octavo},description={Format d'impression dans lequel la feuille est pliée trois fois de manière à former huit feuillets (ou seize pages) {[\href{https://www.cnrtl.fr/definition/in-octavo}{CNRTL}, \href{https://www.wikidata.org/wiki/Q1307353}{Wikidata}]}}}
\newglossaryentry{CoEg_entry_00000034}{name={vade-mecum},type={entry},text={vade-mecum},description={Du latin \textit{vade mecum} « allez avec moi » ; objet à usage personnel que l'on garde avec soi {[\href{https://www.cnrtl.fr/definition/vade-mecum}{CNRTL}]}}}
\newglossaryentry{CoEg_entry_00000035}{name={apiéum},type={entry},text={apiéum},description={Temple d'Apis}}
\newglossaryentry{CoEg_entry_00000036}{name={hiérodule},type={entry},text={hiérodule},description={Servant de temple {[\href{https://www.cnrtl.fr/definition/hiérodule}{CNRTL}]}}}
\newglossaryentry{CoEg_entry_00000037}{name={choéphore},type={entry},text={choéphore},description={Porteur d'offrande funéraire {[\href{https://www.cnrtl.fr/definition/choéphore}{CNRTL}]}}}
\newglossaryentry{CoEg_entry_00000038}{name={quipo},type={entry},text={quipo},description={Système de notation utilisé par les Incas au moyen de cordelettes nouées {[\href{https://www.cnrtl.fr/definition/quipo}{CNRTL}, \href{https://www.wikidata.org/wiki/Q185292}{Wikidata}]}}}

%Glossaire égyptien

\newglossaryentry{CoEg_aeg_00000001}{name={\foreignlanguage{translit}{ḥb}},type={aeg},text={ḥb},sort={le},description={fête {[\href{http://aaew.bbaw.de/tla/servlet/GetWcnDetails?u=guest&f=0&l=0&wn=850654&db=0}{TLA}, \href{http://awv.informatik.uni-leipzig.de/awv/links?word=850654}{AWV}]}}}
\newglossaryentry{CoEg_aeg_00000002}{name={\foreignlanguage{translit}{Ḥp}},type={aeg},text={Ḥp},sort={lf},description={Apis\gls{CoEg_pers_00000011} {[\href{http://aaew.bbaw.de/tla/servlet/GetWcnDetails?u=guest&f=0&l=0&wn=104000&db=0}{TLA}, \href{http://awv.informatik.uni-leipzig.de/awv/links?word=104000}{AWV}]}}}
\newglossaryentry{CoEg_aeg_00000003}{name={\foreignlanguage{translit}{ḥwt}},type={aeg},text={ḥwt},sort={ld1},description={temple {[\href{http://aaew.bbaw.de/tla/servlet/GetWcnDetails?u=guest&f=0&l=0&wn=99790&db=0}{TLA}, \href{http://awv.informatik.uni-leipzig.de/awv/links?word=99790}{AWV}]}}}
\newglossaryentry{CoEg_aeg_00000004}{name={\foreignlanguage{translit}{ꜥḥ}},type={aeg},text={ꜥḥ},sort={cl},description={palais {[\href{http://aaew.bbaw.de/tla/servlet/GetWcnDetails?u=guest&f=0&l=0&wn=39850&db=0}{TLA}, \href{http://awv.informatik.uni-leipzig.de/awv/links?word=39850}{AWV}]}}}
\newglossaryentry{CoEg_aeg_00000005}{name={\foreignlanguage{translit}{ḥm}},type={aeg},text={ḥm},sort={lh},description={servant, prêtre {[\href{http://aaew.bbaw.de/tla/servlet/GetWcnDetails?u=guest&f=0&l=0&wn=104680&db=0}{TLA}, \href{http://awv.informatik.uni-leipzig.de/awv/links?word=104680}{AWV}]}}}
\newglossaryentry{CoEg_aeg_00000006}{name={\foreignlanguage{translit}{m}},type={aeg},text={m},sort={h},description={dans {[\href{http://aaew.bbaw.de/tla/servlet/GetWcnDetails?u=guest&f=0&l=0&wn=64360&db=0}{TLA}, \href{http://awv.informatik.uni-leipzig.de/awv/links?word=64360}{AWV}]}}}
\newglossaryentry{CoEg_aeg_00000007}{name={\foreignlanguage{translit}{sꜣ}},type={aeg},text={sꜣ},sort={oa},description={fils {[\href{http://aaew.bbaw.de/tla/servlet/GetWcnDetails?u=guest&f=0&l=0&wn=125510&db=0}{TLA}, \href{http://awv.informatik.uni-leipzig.de/awv/links?word=125510}{AWV}]}}}
\newglossaryentry{CoEg_aeg_00000008}{name={\foreignlanguage{translit}{wḥm}},type={aeg},text={wḥm},sort={dlh},description={renouveler {[\href{http://aaew.bbaw.de/tla/servlet/GetWcnDetails?u=guest&f=0&l=0&wn=48440&db=0}{TLA}, \href{http://awv.informatik.uni-leipzig.de/awv/links?word=48440}{AWV}]}}}
\newglossaryentry{CoEg_aeg_00000009}{name={\foreignlanguage{translit}{ꜥnḫ}},type={aeg},text={ꜥnḫ},sort={cim},description={vie {[\href{http://aaew.bbaw.de/tla/servlet/GetWcnDetails?u=guest&f=0&l=0&wn=38540&db=0}{TLA}, \href{http://awv.informatik.uni-leipzig.de/awv/links?word=38540}{AWV}]}}}
\newglossaryentry{CoEg_aeg_00000010}{name={\foreignlanguage{translit}{n}},type={aeg},text={n},sort={i},description={pour, de {[\href{http://aaew.bbaw.de/tla/servlet/GetWcnDetails?u=guest&f=0&l=0&wn=78870&db=0}{TLA}, \href{http://awv.informatik.uni-leipzig.de/awv/links?word=78870}{AWV}]}}}
\newglossaryentry{CoEg_aeg_00000011}{name={\foreignlanguage{translit}{Ptḥ}},type={aeg},text={Ptḥ},sort={ftl},description={Ptah\gls{CoEg_pers_00000047} {[\href{http://aaew.bbaw.de/tla/servlet/GetWcnDetails?u=guest&f=0&l=0&wn=62980&db=0}{TLA}, \href{http://awv.informatik.uni-leipzig.de/awv/links?word=62980}{AWV}]}}}
\newglossaryentry{CoEg_aeg_00000012}{name={\foreignlanguage{translit}{ꜥḥꜥ(w)}},type={aeg},text={ꜥḥꜥ(w)},sort={clc},description={durée de vie {[\href{http://aaew.bbaw.de/tla/servlet/GetWcnDetails?u=guest&f=0&l=0&wn=40480&db=0}{TLA}, \href{http://awv.informatik.uni-leipzig.de/awv/links?word=40480}{AWV}]}}}
\newglossaryentry{CoEg_aeg_00000013}{name={\foreignlanguage{translit}{nfr}},type={aeg},text={nfr},sort={igj},description={(être) beau, bon, bien, parfait {[\href{http://aaew.bbaw.de/tla/servlet/GetWcnDetails?u=guest&f=0&l=0&wn=854519&db=0}{TLA}, \href{http://awv.informatik.uni-leipzig.de/awv/links?word=854519}{AWV}]}}}
\newglossaryentry{CoEg_aeg_00000014}{name={\foreignlanguage{translit}{nṯr}},type={aeg},text={nṯr},sort={iuj},description={dieu {[\href{http://aaew.bbaw.de/tla/servlet/GetWcnDetails?u=guest&f=0&l=0&wn=90260&db=0}{TLA}, \href{http://awv.informatik.uni-leipzig.de/awv/links?word=90260}{AWV}]}}}
\newglossaryentry{CoEg_aeg_00000015}{name={\foreignlanguage{translit}{pn}},type={aeg},text={pn},sort={fi},description={ce {[\href{http://aaew.bbaw.de/tla/servlet/GetWcnDetails?u=guest&f=0&l=0&wn=59920&db=0}{TLA}, \href{http://awv.informatik.uni-leipzig.de/awv/links?word=59920}{AWV}]}}}
\newglossaryentry{CoEg_aeg_00000016}{name={\foreignlanguage{translit}{rnpt}},type={aeg},text={rnpt},sort={jif},description={année {[\href{http://aaew.bbaw.de/tla/servlet/GetWcnDetails?u=guest&f=0&l=0&wn=94920&db=0}{TLA}, \href{http://awv.informatik.uni-leipzig.de/awv/links?word=94920}{AWV}]}}}
\newglossaryentry{CoEg_aeg_00000017}{name={\foreignlanguage{translit}{tm}},type={aeg},text={tm},sort={th},description={(être) complet, achevé, total {[\href{http://aaew.bbaw.de/tla/servlet/GetWcnDetails?u=guest&f=0&l=0&wn=172000&db=0}{TLA}, \href{http://awv.informatik.uni-leipzig.de/awv/links?word=172000}{AWV}]}}}
\newglossaryentry{CoEg_aeg_00000018}{name={\foreignlanguage{translit}{ḫprw}},type={aeg},text={ḫpr},sort={mfj},description={forme, apparence, manifestation {[\href{http://aaew.bbaw.de/tla/servlet/GetWcnDetails?u=guest&f=0&l=0&wn=116300&db=0}{TLA}, \href{http://awv.informatik.uni-leipzig.de/awv/links?word=116300}{AWV}]}}}
\newglossaryentry{CoEg_aeg_00000019}{name={\foreignlanguage{translit}{Wnn-nfr}},type={aeg},text={Wnn-nfr},sort={diiig},description={Ounennéfer\gls{CoEg_pers_00000225} {[\href{http://aaew.bbaw.de/tla/servlet/GetWcnDetails?u=guest&f=0&l=0&wn=850648&db=0}{TLA}, \href{http://awv.informatik.uni-leipzig.de/awv/links?word=850648}{AWV}]}}}
\newglossaryentry{CoEg_aeg_00000020}{name={\foreignlanguage{translit}{Ꞽw·f-ꜥnḫ}},type={aeg},text={Ꞽw·f-ꜥnḫ},sort={bdgc},description={Ioufânkh\gls{CoEg_pers_00000201}, nom masculin {[\href{http://aaew.bbaw.de/tla/servlet/GetWcnDetails?u=guest&f=0&l=0&wn=701657&db=0}{TLA}, \href{http://awv.informatik.uni-leipzig.de/awv/links?word=701657}{AWV}]}}}
\newglossaryentry{CoEg_aeg_00000021}{name={\foreignlanguage{translit}{mꜣꜥ-ḫrw}},type={aeg},text={mꜣꜥ-ḫrw},sort={hacmjd},description={juste de voix {[\href{http://aaew.bbaw.de/tla/servlet/GetWcnDetails?u=guest&f=0&l=0&wn=66750&db=0}{TLA}, \href{http://awv.informatik.uni-leipzig.de/awv/links?word=66750}{AWV}]}}}
\newglossaryentry{CoEg_aeg_00000022}{name={\foreignlanguage{translit}{gm}},type={aeg},text={gm},sort={sh},description={trouver {[\href{http://aaew.bbaw.de/tla/servlet/GetWcnDetails?u=guest&f=0&l=0&wn=167210&db=0}{TLA}, \href{http://awv.informatik.uni-leipzig.de/awv/links?word=167210}{AWV}]}}}
\newglossaryentry{CoEg_aeg_00000023}{name={\foreignlanguage{translit}{Ꞽmntt}},type={aeg},text={Ꞽmntt},sort={bhit1},description={Occident {[\href{http://aaew.bbaw.de/tla/servlet/GetWcnDetails?u=guest&f=0&l=0&wn=26180&db=0}{TLA}, \href{http://awv.informatik.uni-leipzig.de/awv/links?word=26180}{AWV}]}}}
\newglossaryentry{CoEg_aeg_00000024}{name={\foreignlanguage{translit}{Wsꞽr}},type={aeg},text={Wsꞽr},sort={dobj},description={Osiris\gls{CoEg_pers_00000151} {[\href{http://aaew.bbaw.de/tla/servlet/GetWcnDetails?u=guest&f=0&l=0&wn=49460&db=0}{TLA}, \href{http://awv.informatik.uni-leipzig.de/awv/links?word=49460}{AWV}]}}}
\newglossaryentry{CoEg_aeg_00000025}{name={\foreignlanguage{translit}{ꜥꜣ}},type={aeg},text={ꜥꜣ},sort={ca},description={grand {[\href{http://aaew.bbaw.de/tla/servlet/GetWcnDetails?u=guest&f=0&l=0&wn=450158&db=0}{TLA}, \href{http://awv.informatik.uni-leipzig.de/awv/links?word=450158}{AWV}]}}}
\newglossaryentry{CoEg_aeg_00000026}{name={\foreignlanguage{translit}{ꜣbd}},type={aeg},text={ꜣbd},sort={aev},description={mois {[\href{http://aaew.bbaw.de/tla/servlet/GetWcnDetails?u=guest&f=0&l=0&wn=93&db=0}{TLA}, \href{http://awv.informatik.uni-leipzig.de/awv/links?word=93}{AWV}]}}}
\newglossaryentry{CoEg_aeg_00000027}{name={\foreignlanguage{translit}{wꜥ}},type={aeg},text={wꜥ},sort={dc},description={unique {[\href{http://aaew.bbaw.de/tla/servlet/GetWcnDetails?u=guest&f=0&l=0&wn=44150&db=0}{TLA}, \href{http://awv.informatik.uni-leipzig.de/awv/links?word=44150}{AWV}]}}}
\newglossaryentry{CoEg_aeg_00000028}{name={\foreignlanguage{translit}{ḫpr}},type={aeg},text={ḫpr},sort={mfj},description={advenir {[\href{http://aaew.bbaw.de/tla/servlet/GetWcnDetails?u=guest&f=0&l=0&wn=854383&db=0}{TLA}, \href{http://awv.informatik.uni-leipzig.de/awv/links?word=854383}{AWV}]}}}
\newglossaryentry{CoEg_aeg_00000029}{name={\foreignlanguage{translit}{ḥꜣt}},type={aeg},text={ḥꜣt},sort={la1},description={avant {[\href{http://aaew.bbaw.de/tla/servlet/GetWcnDetails?u=guest&f=0&l=0&wn=100310&db=0}{TLA}, \href{http://awv.informatik.uni-leipzig.de/awv/links?word=100310}{AWV}]}}}
\newglossaryentry{CoEg_aeg_00000030}{name={\foreignlanguage{translit}{psḏt}},type={aeg},text={psḏt},sort={fow1},description={Ennéade {[\href{http://aaew.bbaw.de/tla/servlet/GetWcnDetails?u=guest&f=0&l=0&wn=62500&db=0}{TLA}, \href{http://awv.informatik.uni-leipzig.de/awv/links?word=62500}{AWV}]}}}
\newglossaryentry{CoEg_aeg_00000031}{name={\foreignlanguage{translit}{ḫnty}},type={aeg},text={ḫnty},sort={mitb},description={qui préside à {[\href{http://aaew.bbaw.de/tla/servlet/GetWcnDetails?u=guest&f=0&l=0&wn=119130&db=0}{TLA}, \href{http://awv.informatik.uni-leipzig.de/awv/links?word=119130}{AWV}]}}}
\newglossaryentry{CoEg_aeg_00000032}{name={\foreignlanguage{translit}{r}},type={aeg},text={r},sort={j},description={vers, contre {[\href{http://aaew.bbaw.de/tla/servlet/GetWcnDetails?u=guest&f=0&l=0&wn=91900&db=0}{TLA}, \href{http://awv.informatik.uni-leipzig.de/awv/links?word=91900}{AWV}]}}}
\newglossaryentry{CoEg_aeg_00000033}{name={\foreignlanguage{translit}{Ḥwt-šd-ꜣbd}},type={aeg},text={Ḥwt-šd-ꜣbd},sort={ld1pv},description={Hout-ched-abed\gls{CoEg_place_00000050}}}
\newglossaryentry{CoEg_aeg_00000034}{name={\foreignlanguage{translit}{m-ḫt}},type={aeg},text={m-ḫt},sort={hmt},description={après {[\href{http://aaew.bbaw.de/tla/servlet/GetWcnDetails?u=guest&f=0&l=0&wn=65300&db=0}{TLA}, \href{http://awv.informatik.uni-leipzig.de/awv/links?word=65300}{AWV}]}}}
\newglossaryentry{CoEg_aeg_00000035}{name={\foreignlanguage{translit}{pẖr}},type={aeg},text={pẖr},sort={fnj},description={parcourir {[\href{http://aaew.bbaw.de/tla/servlet/GetWcnDetails?u=guest&f=0&l=0&wn=61900&db=0}{TLA}, \href{http://awv.informatik.uni-leipzig.de/awv/links?word=61900}{AWV}]}}}
\newglossaryentry{CoEg_aeg_00000036}{name={\foreignlanguage{translit}{ꞽnt}},type={aeg},text={ꞽnt},sort={bi1},description={vallée {[\href{http://aaew.bbaw.de/tla/servlet/GetWcnDetails?u=guest&f=0&l=0&wn=26780&db=0}{TLA}, \href{http://awv.informatik.uni-leipzig.de/awv/links?word=26780}{AWV}]}}}
\newglossaryentry{CoEg_aeg_00000037}{name={\foreignlanguage{translit}{ꞽw}},type={aeg},text={ꞽw},sort={bd},description={île {[\href{http://aaew.bbaw.de/tla/servlet/GetWcnDetails?u=guest&f=0&l=0&wn=21940&db=0}{TLA}, \href{http://awv.informatik.uni-leipzig.de/awv/links?word=21940}{AWV}]}}}
\newglossaryentry{CoEg_aeg_00000038}{name={\foreignlanguage{translit}{ꞽdḥw}},type={aeg},text={ꞽdḥw},sort={bvld},description={le Delta\gls{CoEg_place_00000072} {[\href{http://aaew.bbaw.de/tla/servlet/GetWcnDetails?u=guest&f=0&l=0&wn=34240&db=0}{TLA}, \href{http://awv.informatik.uni-leipzig.de/awv/links?word=34240}{AWV}]}}}
\newglossaryentry{CoEg_aeg_00000039}{name={\foreignlanguage{translit}{tꜣ-mḥw}},type={aeg},text={tꜣ-mḥw},sort={tahld},description={le pays du nord, c.-à-d. la Basse-Égypte\gls{CoEg_place_00000072} {[\href{http://aaew.bbaw.de/tla/servlet/GetWcnDetails?u=guest&f=0&l=0&wn=169120&db=0}{TLA}, \href{http://awv.informatik.uni-leipzig.de/awv/links?word=169120}{AWV}]}}}
\newglossaryentry{CoEg_aeg_00000040}{name={\foreignlanguage{translit}{nb}},type={aeg},text={nb},sort={ie},description={tout, chacun {[\href{http://aaew.bbaw.de/tla/servlet/GetWcnDetails?u=guest&f=0&l=0&wn=81660&db=0}{TLA}, \href{http://awv.informatik.uni-leipzig.de/awv/links?word=81660}{AWV}]}}}
\newglossaryentry{CoEg_aeg_00000041}{name={\foreignlanguage{translit}{ẖꜣt}},type={aeg},text={ẖꜣ},sort={na},description={lagune {[\href{http://aaew.bbaw.de/tla/servlet/GetWcnDetails?u=guest&f=0&l=0&wn=122270&db=0}{TLA}, \href{http://awv.informatik.uni-leipzig.de/awv/links?word=122270}{AWV}]}}}
\newglossaryentry{CoEg_aeg_00000042}{name={\foreignlanguage{translit}{msꞽ}},type={aeg},text={ms},sort={ho},description={naître {[\href{http://aaew.bbaw.de/tla/servlet/GetWcnDetails?u=guest&f=0&l=0&wn=74950&db=0}{TLA}, \href{http://awv.informatik.uni-leipzig.de/awv/links?word=74950}{AWV}]}}}
\newglossaryentry{CoEg_aeg_00000043}{name={\foreignlanguage{translit}{šps}},type={aeg},text={šps},sort={pfo},description={auguste {[\href{http://aaew.bbaw.de/tla/servlet/GetWcnDetails?u=guest&f=0&l=0&wn=400546&db=0}{TLA}, \href{http://awv.informatik.uni-leipzig.de/awv/links?word=400546}{AWV}]}}}
\newglossaryentry{CoEg_aeg_00000044}{name={\foreignlanguage{translit}{Mn-nfr}},type={aeg},text={Mn-nfr},sort={hiigj},description={Memphis\gls{CoEg_place_00000005} {[\href{http://aaew.bbaw.de/tla/servlet/GetWcnDetails?u=guest&f=0&l=0&wn=70010&db=0}{TLA}, \href{http://awv.informatik.uni-leipzig.de/awv/links?word=70010}{AWV}]}}}
\newglossaryentry{CoEg_aeg_00000045}{name={\foreignlanguage{translit}{m-ẖnw}},type={aeg},text={m-ẖnw},sort={hnid},description={à l'intérieur {[\href{http://aaew.bbaw.de/tla/servlet/GetWcnDetails?u=guest&f=0&l=0&wn=65370&db=0}{TLA}, \href{http://awv.informatik.uni-leipzig.de/awv/links?word=65370}{AWV}]}}}
\newglossaryentry{CoEg_aeg_00000046}{name={\foreignlanguage{translit}{nswt-bꞽty}},type={aeg},text={nswt-bꞽty},sort={iodtebt},description={roi {[\href{http://aaew.bbaw.de/tla/servlet/GetWcnDetails?u=guest&f=0&l=0&wn=88060&db=0}{TLA}, \href{http://awv.informatik.uni-leipzig.de/awv/links?word=88060}{AWV}]}}}
\newglossaryentry{CoEg_aeg_00000047}{name={\foreignlanguage{translit}{Ꞽwnw}},type={aeg},text={Ꞽwnw},sort={bdid},description={Héliopolis\gls{CoEg_place_00000024} {[\href{http://aaew.bbaw.de/tla/servlet/GetWcnDetails?u=guest&f=0&l=0&wn=22850&db=0}{TLA}, \href{http://awv.informatik.uni-leipzig.de/awv/links?word=22850}{AWV}]}}}
\newglossaryentry{CoEg_aeg_00000048}{name={\foreignlanguage{translit}{Ḥꜥp}},type={aeg},text={Ḥꜥp},sort={lcf},description={Hâpy\gls{CoEg_pers_00000188} {[\href{http://aaew.bbaw.de/tla/servlet/GetWcnDetails?u=guest&f=0&l=0&wn=650066&db=0}{TLA}, \href{http://awv.informatik.uni-leipzig.de/awv/links?word=650066}{AWV}]}, le Nil\gls{CoEg_place_00000021} {[\href{http://aaew.bbaw.de/tla/servlet/GetWcnDetails?u=guest&f=0&l=0&wn=102190&db=0}{TLA}, \href{http://awv.informatik.uni-leipzig.de/awv/links?word=102190}{AWV}]}}}\newglossaryentry{CoEg_aeg_00000049}{name={\foreignlanguage{translit}{nty}},type={aeg},text={nty},sort={itb},description={ {[\href{http://aaew.bbaw.de/tla/servlet/GetWcnDetails?u=guest&f=0&l=0&wn=89850&db=0}{TLA}, \href{http://awv.informatik.uni-leipzig.de/awv/links?word=89850}{AWV}]}}}
\newglossaryentry{CoEg_aeg_00000050}{name={\foreignlanguage{translit}{spd}},type={aeg},text={spd},sort={ofw},description={(être) pointu, aiguisé, prêt~; fournir, munir {[\href{http://aaew.bbaw.de/tla/servlet/GetWcnDetails?u=guest&f=0&l=0&wn=133200&db=0}{TLA}, \href{http://awv.informatik.uni-leipzig.de/awv/links?word=133200}{AWV}]}}}
\newglossaryentry{CoEg_aeg_00000051}{name={\foreignlanguage{translit}{msḏr}},type={aeg},text={msḏr},sort={howj},description={ {[\href{http://aaew.bbaw.de/tla/servlet/GetWcnDetails?u=guest&f=0&l=0&wn=76230&db=0}{TLA}, \href{http://awv.informatik.uni-leipzig.de/awv/links?word=76230}{AWV}]}}}
\newglossaryentry{CoEg_aeg_00000052}{name={\foreignlanguage{translit}{hrw}},type={aeg},text={hrw},sort={kjd},description={jour {[\href{http://aaew.bbaw.de/tla/servlet/GetWcnDetails?u=guest&f=0&l=0&wn=99060&db=0}{TLA}, \href{http://awv.informatik.uni-leipzig.de/awv/links?word=99060}{AWV}]}}}
\newglossaryentry{CoEg_aeg_00000053}{name={\foreignlanguage{translit}{ꞽwꞽ}},type={aeg},text={ꞽw},sort={bdb},description={venir {[\href{http://aaew.bbaw.de/tla/servlet/GetWcnDetails?u=guest&f=0&l=0&wn=21930&db=0}{TLA}, \href{http://awv.informatik.uni-leipzig.de/awv/links?word=21930}{AWV}]}}}
\newglossaryentry{CoEg_aeg_00000054}{name={\foreignlanguage{translit}{sḫnꞽ}},type={aeg},text={sḫn},sort={omi},description={rejoindre {[\href{http://aaew.bbaw.de/tla/servlet/GetWcnDetails?u=guest&f=0&l=0&wn=142440&db=0}{TLA}, \href{http://awv.informatik.uni-leipzig.de/awv/links?word=142440}{AWV}]}}}
\newglossaryentry{CoEg_aeg_00000055}{name={\foreignlanguage{translit}{wꜥꞽ}},type={aeg},text={wꜥ},sort={dcb},description={être seul {[\href{http://aaew.bbaw.de/tla/servlet/GetWcnDetails?u=guest&f=0&l=0&wn=44350&db=0}{TLA}, \href{http://awv.informatik.uni-leipzig.de/awv/links?word=44350}{AWV}]}}}
\newglossaryentry{CoEg_aeg_00000056}{name={\foreignlanguage{translit}{pꜣwt}},type={aeg},text={pꜣwt},sort={fad},description={origine {[\href{http://aaew.bbaw.de/tla/servlet/GetWcnDetails?u=guest&f=0&l=0&wn=58830&db=0}{TLA}, \href{http://awv.informatik.uni-leipzig.de/awv/links?word=58830}{AWV}]}}}
\newglossaryentry{CoEg_aeg_00000057}{name={\foreignlanguage{translit}{ḏs}},type={aeg},text={ḏs},sort={wo},description={(avec un pronom suffixe) en personne, soi-même {[\href{http://aaew.bbaw.de/tla/servlet/GetWcnDetails?u=guest&f=0&l=0&wn=854591&db=0}{TLA}, \href{http://awv.informatik.uni-leipzig.de/awv/links?word=854591}{AWV}]}}}
\newglossaryentry{CoEg_aeg_00000058}{name={\foreignlanguage{translit}{·f}},type={aeg},text={·f},sort={g},description={pronom personnel de troisième personne masculin singulier {[\href{http://aaew.bbaw.de/tla/servlet/GetWcnDetails?u=guest&f=0&l=0&wn=10050&db=0}{TLA}, \href{http://awv.informatik.uni-leipzig.de/awv/links?word=10050}{AWV}]}}}
\newglossaryentry{CoEg_aeg_00000059}{name={\foreignlanguage{translit}{·sn}},type={aeg},text={·sn},sort={oi},description={pronom personnel de troisième personne pluriel {[\href{http://aaew.bbaw.de/tla/servlet/GetWcnDetails?u=guest&f=0&l=0&wn=10100&db=0}{TLA}, \href{http://awv.informatik.uni-leipzig.de/awv/links?word=10100}{AWV}]}}}
\newglossaryentry{CoEg_aeg_00000060}{name={\foreignlanguage{translit}{ꜣḫt}},type={aeg},text={ꜣḫt},sort={am1},description={saison de l'inondation {[\href{http://aaew.bbaw.de/tla/servlet/GetWcnDetails?u=guest&f=0&l=0&wn=216&db=0}{TLA}, \href{http://awv.informatik.uni-leipzig.de/awv/links?word=216}{AWV}]}}}
\newglossaryentry{CoEg_aeg_00000061}{name={\foreignlanguage{translit}{tw}},type={aeg},text={tw},sort={td},description={pronom indéfini {[\href{http://aaew.bbaw.de/tla/servlet/GetWcnDetails?u=guest&f=0&l=0&wn=600479&db=0}{TLA}, \href{http://awv.informatik.uni-leipzig.de/awv/links?word=600479}{AWV}]}}}
\newglossaryentry{CoEg_aeg_00000062}{name={\foreignlanguage{translit}{ꞽm}},type={aeg},text={ꞽm},sort={bh},description={là {[\href{http://aaew.bbaw.de/tla/servlet/GetWcnDetails?u=guest&f=0&l=0&wn=24640&db=0}{TLA}, \href{http://awv.informatik.uni-leipzig.de/awv/links?word=24640}{AWV}]}}}
\newglossaryentry{CoEg_aeg_00000063}{name={\foreignlanguage{translit}{prt}},type={aeg},text={prt},sort={fj1},description={saison de la germination {[\href{http://aaew.bbaw.de/tla/servlet/GetWcnDetails?u=guest&f=0&l=0&wn=60300&db=0}{TLA}, \href{http://awv.informatik.uni-leipzig.de/awv/links?word=60300}{AWV}]}}}
\newglossaryentry{CoEg_aeg_00000064}{name={\foreignlanguage{translit}{tpy}},type={aeg},text={tpy},sort={tfb},description={premier {[\href{http://aaew.bbaw.de/tla/servlet/GetWcnDetails?u=guest&f=0&l=0&wn=171460&db=0}{TLA}, \href{http://awv.informatik.uni-leipzig.de/awv/links?word=171460}{AWV}]}}}
\newglossaryentry{CoEg_aeg_00000065}{name={\foreignlanguage{translit}{snn}},type={aeg},text={snn},sort={oii},description={image {[\href{http://aaew.bbaw.de/tla/servlet/GetWcnDetails?u=guest&f=0&l=0&wn=137580&db=0}{TLA}, \href{http://awv.informatik.uni-leipzig.de/awv/links?word=137580}{AWV}]}}}
\newglossaryentry{CoEg_aeg_00000066}{name={\foreignlanguage{translit}{rdꞽ}},type={aeg},text={d},sort={jvb},description={faire, donner {[\href{http://aaew.bbaw.de/tla/servlet/GetWcnDetails?u=guest&f=0&l=0&wn=96700&db=0}{TLA}, \href{http://awv.informatik.uni-leipzig.de/awv/links?word=96700}{AWV}]}}}
\newglossaryentry{CoEg_aeg_00000067}{name={\foreignlanguage{translit}{ꜥb}},type={aeg},text={ꜥb},sort={ce},description={corne {[\href{http://aaew.bbaw.de/tla/servlet/GetWcnDetails?u=guest&f=0&l=0&wn=36250&db=0}{TLA}, \href{http://awv.informatik.uni-leipzig.de/awv/links?word=36250}{AWV}]}}}
\newglossaryentry{CoEg_aeg_00000068}{name={\foreignlanguage{translit}{ḥr}},type={aeg},text={ḥr},sort={lj},description={visage {[\href{http://aaew.bbaw.de/tla/servlet/GetWcnDetails?u=guest&f=0&l=0&wn=107510&db=0}{TLA}, \href{http://awv.informatik.uni-leipzig.de/awv/links?word=107510}{AWV}]}}}
\newglossaryentry{CoEg_aeg_00000069}{name={\foreignlanguage{translit}{sštꜣ}},type={aeg},text={sštꜣ},sort={opta},description={rendre secret {[\href{http://aaew.bbaw.de/tla/servlet/GetWcnDetails?u=guest&f=0&l=0&wn=145680&db=0}{TLA}, \href{http://awv.informatik.uni-leipzig.de/awv/links?word=145680}{AWV}]}}}


% Thèmes

\newglossaryentry{CoEg_keyword_00000001}{name={Missions scientifiques},type={keyword},text={missions scientifiques},description={Voyages d'études financés par l'État}}
\newglossaryentry{CoEg_keyword_00000002}{name={Mission de Mariette (1850-1854, Égypte)},type={keyword},sort={mission_Mariette_1850},text={mission de Mariette (1850-1854, Égypte)},parent={CoEg_keyword_00000001},description={Premier voyage de Mariette en Égypte, au cours duquel il découvrit le \Gls{CoEg_entry_00000028}\gls{CoEg_place_00000004}}}
\newglossaryentry{CoEg_keyword_00000003}{name={Mission de Mariette (1857, Italie)},type={keyword},sort={mission_Mariette_1857a},text={mission de Mariette (1857, Italie)},parent={CoEg_keyword_00000001},description={Voyage d'étude dans les musées d'Italie}}
\newglossaryentry{CoEg_keyword_00000004}{name={Mission de Mariette (1857-1858, Égypte)},type={keyword},sort={mission_Mariette_1857b}
,text={mission de Mariette (1857, Égypte)},parent={CoEg_keyword_00000001},description={Second voyage en Égypte de Mariette, sous le prétexte de préparer celui du prince Napoléon}}
\newglossaryentry{CoEg_keyword_00000005}{name={Carrière},type={keyword},text={carrière},description={Évolution de carrière, gestion de congés, etc}}
\newglossaryentry{CoEg_keyword_00000006}{name={Fouilles du Sérapéum},type={keyword},text={fouilles du Sérapéum},description={Voir aussi «~objets découverts au Sérapéum~»}}
\newglossaryentry{CoEg_keyword_00000007}{name={Objets découverts au Sérapéum},type={keyword},text={objets découverts au \Gls{CoEg_entry_00000028}\gls{CoEg_place_00000004}},description={Produit des fouilles, listes, transport. Voir aussi «~fouilles du Sérapéum~»}}
\newglossaryentry{CoEg_keyword_00000008}{name={Publications de Mariette},type={keyword},text={publications de Mariette},description={}}
\newglossaryentry{CoEg_keyword_00000009}{name={Famille de Mariette},type={keyword},text={famille de Mariette},description={}}
\newglossaryentry{CoEg_keyword_00000010}{name={Santé de Mariette},type={keyword},text={santé de Mariette},description={}}
\newglossaryentry{CoEg_keyword_00000011}{name={Collection Anastasi},type={keyword},text={collection Anastasi},description={Acquisition de la collection d'Anastasi}}
\newglossaryentry{CoEg_keyword_00000012}{name={Contexte politique et diplomatique},type={keyword},text={contexte politique et diplomatique},description={Détails sur les positions des différents acteurs politiques et diplomatiques. Comprend notamment les négociations avec le gouvernement égyptien pour obtenir la cession des objets découverts au \Gls{CoEg_entry_00000028}\gls{CoEg_place_00000004}}}
\newglossaryentry{CoEg_keyword_00000013}{name={Anecdotes},type={keyword},text={Anecdotes},description={Épisodes notables, bon mot, etc}}
\newglossaryentry{CoEg_keyword_00000014}{name={Financements},type={keyword},text={financements},description={Réclamation de fonds et considérations sur les budgets alloués aux travaux}}
\newglossaryentry{CoEg_keyword_00000015}{name={Mission de Mariette (1880, Égypte)},type={keyword},text={mission de Mariette (1880, Égypte)},parent={CoEg_keyword_00000001},sort={mission_Mariette_1880},description={Mission accordée pour financer la préparation d'un ouvrage sur les \glspl{CoEg_entry_00000027}}}
\newglossaryentry{CoEg_keyword_00000016}{name={Mission de Mariette (1855, Royaume-Uni et Prusse)},type={keyword},text={mission de Mariette (1855, Royaume-Uni et Prusse)},parent={CoEg_keyword_00000001},sort={mission_Mariette_1855},description={Voyage d'étude dans les musées de Londres et de Berlin}}


\begin{document}
\selectlanguage{french}
\begin{titlepage}
\frontmatter
\centering
\hspace{0pt}
\vfill
\large
C O R R E S P O N D A N C E S\space\space\space É G Y P T O L O G I Q U E S \vspace{3\baselineskip}

\Large L E T T R E S\space\space\space D ’ A U G.\space\space\space M A R I E T T E
\vspace{15\baselineskip}
\vfill
\hspace{0pt}

\pagebreak
\thispagestyle{empty}
\hspace{0pt}
\vfill
\includegraphics[height=12cm]{CoEg_Mariette_portrait.jpg}
\vfill
\hspace{0pt}

\pagebreak
\thispagestyle{empty}

\LARGE{C O R R E S P O N D A N C E S}
    
\Huge{É G Y P T O L O G I Q U E S}
\vspace{1\baselineskip}

\large C O N T E N A N T\space\space\space D E S
\vspace{1\baselineskip}

\Large L E T T R E S\space\space\space D ’ É G Y P T O L O G U E S
\vspace{2\baselineskip}

\large dispersées dans diverses institutions
\vspace{1\baselineskip}

et qui n’ont pas encore été rassemblées jusqu’à ce jour
\vspace{9\baselineskip}

\LARGE L E T T R E S\space\space\space D\space ’\space A U G .\space\space\space M A R I E T T E
\vspace{4\baselineskip}

\normalsize É D I T É E S\space\space\space P A R \space\space\space T H .\space\space\space L E B É E
\vspace{4\baselineskip}

Version 0,44
\vspace{1\baselineskip}

Novembre 2020
\end{titlepage}
\thispagestyle{empty}
\chapter*{Introduction}
\addcontentsline{toc}{chapter}{Introduction}
\setcounter{page}{1}

\section*{Le projet des \textit{Correspondances égyptologiques}}
\addcontentsline{toc}{section}{Le projet des Correspondances égyptologiques}

Ce fichier résulte d’un projet personnel d’édition numérique des lettres écrites par l’égyp-tologue Auguste Mariette. L’objectif de cette initiative est de rendre librement accessibles ces documents et de permettre leur exploitation scientifique.
\par Le corpus édité ici a vocation à intégrer chaque lettre repérée de Mariette. Les brouillons de lettres seront aussi incorporés, dans la mesure où il n’est pas toujours possible d’établir si une lettre a véritablement été transmise à son destinataire et que les hésitations et repentirs de la rédactions peuvent être riches d’enseignements.
\par L’édition des lettres sera progressive, afin de publier les documents régulièrement et d’en améliorer le format au moyen des suggestions qui pourront être recueillies au cours de l’entreprise. Les sources parisiennes seront dépouillées en priorité pour commencer (par pure commodité matérielle), mais bien d’autres devraient suivre.
\par Les publications successives du corpus sont disponibles sur le site \href{https://thlebee.github.io/CoEg/}{\textit{Correspondances égyp}}-\href{https://thlebee.github.io/CoEg/}{\textit{tologiques}}, à la fois au format XML-TEI et en une version PDF réalisée au moyen de Latex (que vous consultez en ce moment). Les métadonnées du corpus sont aussi disponibles. Chaque enrichissement sera signalé sur le carnet de recherche \href{https://hef.hypotheses.org/}{\textit{Histoire de l’égyptologie en formation}}.
\par Toute remarque, critique ou suggestion d’amélioration sera la bienvenue à l’adresse suivante~: \href{mailto:correspondances.egyptologiques@laposte.net}{correspondances.egyptologiques@laposte.net} (merci également d’y signaler toute utilisation qui pourra être faite de ces ressources, à titre d’information).
\par Le contenu de ce document est publié sous \href{https://creativecommons.org/licenses/by/4.0}{licence CC-BY}~: toute réutilisation en est permise, et encouragée~– sous réserve de la mention de la source (par exemple~: «~Auguste Mariette (Thomas Lebée, éd.), \textit{Correspondances égyptologiques. Lettres d’Auguste Mariette}~»).

\section*{Encodage et principes éditoriaux}
\addcontentsline{toc}{section}{Encodage et principes éditoriaux}

\par L’encodage résulte de plusieurs étapes, destinées à transcrire le document tel qu’il apparaît, puis à baliser ses composants structurels et un certain nombre de termes d’indexations.
\par Chaque lettre a été considérée comme une unité documentaire distincte, dont les réfé-rences bibliographiques et administratives sont rappelées en tête de notice, avec le cas échéant toute remarque jugée utile à sa compréhension. Les lettres peuvent dès lors être arrangées dans l’ordre chronologique pour retrouver leur continuité malgré la dispersion des fonds.
\par La ponctuation de Mariette a été conservée sans modification autant qu’elle était lisible. Pour être compréhensibles, les signes de ponctuation barrés ont parfois été remplacés par leur description entre crochets.
\par Cette édition recherche la plus grande fidélité au texte de Mariette. Les graphies variables des noms propres et l’absence d’accents sur les majuscules ont ainsi été conservées telles quelles. Les fautes d’orthographe, systématiques ou incidentes, ont également été respectées, et mar-quées par un balisage aproprié dès lors qu’elles s’éloignaient de l’orthographe et de l’usage contemporain. Toute intervention ou doute dans la lecture du texte manuscrit est signalée explicitement par le balisage ou la ponctuation.
\par Quand il existe des variantes causées par plusieurs versions d’une même lettre (par exemple un brouillon ou une copie), une des versions est choisie comme texte de base, dont les variantes sont indiquées en note, en circonscrivant les segments concernés. La notice des lettres concernées détaille alors la situation.
\par La copie numérique, comme la transcription par des caractères mécaniques, comporte cependant une part d’interprétation et de standardisation. Puisqu’il s’agissait de reproduire un texte manuscrit en caractères typographiques, les codes habituels ont été appliqués~: le texte souligné à la main a été rendu en italiques, le double soulignement par de petites capitales et les guillemets ont systématiquement été transcrits comme des guillemets typographiques (en chevrons).
\par L’écriture de Mariette n’est pas des plus régulières et les hampes de ses lettres sont parfois trompeuses. En cas de doute entre une majuscule ou une minuscule, ou même sur l’orthographe utilisée, la graphie régulière a été privilégiée en l’absence d’erreur manifeste. Les lectures hasardeuses sont signalées par le balisage, mais il est aussi à noter que les mots courts sont régulièrement de lecture délicate. Si le contexte permet d’en confirmer la plupart, certaines distinctions restent largement conjecturales (notamment la différence entre «~notre~»/ «~votre~» et «~nos~»/«~vos~»). Les ratures ont été déchiffrées dans la mesure du possible, ou juste indiquées en tant que telles.
\par Les marques postérieures à l’utilisation première des lettres (tampon de bibliothèque, foliotage, etc.) n’ont pas été reproduites. En revanche, les annotations portées sur les documents par leurs destinataires (annotation de secrétaire, indication de classement initial, etc.) sont indiquées dans la description de la lettre.

\section*{Le corpus}
\addcontentsline{toc}{section}{Le corpus}
\subsection*{Archives nationales}
\addcontentsline{toc}{subsection}{Archives nationales}

\hypertarget{CoEg_Mariette_ms_002}{}
\subsubsection*{F/17/2988/1, dossier « Mariette »}
\addcontentsline{toc}{subsubsection}{F/17/2988/1, dossier « Mariette »}
\begin{itemize}
\item (n. p.) \hyperlink{CoEg_Mariette_1846-04-13}{Le 13 avril 1846, de Boulogne-sur-Mer, au ministre de l’Instruction publique}~;
\item (n. p.) \hyperlink{CoEg_Mariette_1846-05-24}{Le 24 mai 1846, de Boulogne-sur-Mer, à Camaret, recteur de l’académie de Douai}~;
\item (n. p.) \hyperlink{CoEg_Mariette_1846-05-25}{Le 25 mai 1846, de Boulogne-sur-Mer, au ministre de l’Instruction publique}~;
\item (n. p.) \hyperlink{CoEg_Mariette_1846-09-29}{Le 29 septembre 1846, de Boulogne-sur-Mer, au ministre de l’Instruction publique}~;
\item (n. p.) \hyperlink{CoEg_Mariette_1850-05-20}{Le 20 mai 1850, de Paris, au ministre de l’Instruction publique}~;
\item (n. p.) \hyperlink{CoEg_Mariette_1850-07-06}{Le 6 juillet 1850, de Paris, au ministre de l’Instruction publique}~;
\item (n. p.) \hyperlink{CoEg_Mariette_1850-08-27}{Le 27 août 1850, de Paris, au ministre de l’Instruction publique}~;
\item (n. p.) \hyperlink{CoEg_Mariette_1851-10-14a}{Le 14 septembre 1851, de Saqqarah, à Le Moyne (copie)}~;
\item (n. p.) \hyperlink{CoEg_Mariette_1851-10-14b}{Le 14 septembre 1851, de Saqqarah, aux ministres de l'Intérieur et de l'Instruction publique (copie)}~;
\item (n. p.) \hyperlink{CoEg_Mariette_1855-01-26}{Le 26 janvier 1855, de Paris, au ministre de l’Instruction publique}~;
\item (n. p.) \hyperlink{CoEg_Mariette_1855-07-12}{Le 12 juillet 1855, de Paris, à un fonctionnaire de l’Instruction publique}~;
\item (n. p.) \hyperlink{CoEg_Mariette_1855-08-06}{Le 6 août 1855, de Paris, au ministre de l’Instruction publique}~;
\item (n. p.) \hyperlink{CoEg_Mariette_1855-12-12}{Le 12 décembre 1855, de Paris, à un fonctionnaire de l’Instruction publique}~;
\item (n. p.) \hyperlink{CoEg_Mariette_1856-02-11}{Le 11 février 1856, de Paris, au ministre de l’Instruction publique}~;
\item (n. p.) \hyperlink{CoEg_Mariette_1856-12-11}{Le 11 décembre 1856, de Paris, au ministre de l’Instruction publique}~;
\item (n. p.) \hyperlink{CoEg_Mariette_1856-12-31}{Le 31 décembre 1856, de Paris, au ministre de l’Instruction publique}~;
\item (n. p.) \hyperlink{CoEg_Mariette_1857-01-03}{Le 3 janvier 1857, de Paris, à un fonctionnaire de l’Instruction publique}~;
\item (n. p.) \hyperlink{CoEg_Mariette_1857-04-01}{Le 1\textsuperscript{er} avril 1857, de Paris, au ministre de l’Instruction publique}~;
\item (n. p.) \hyperlink{CoEg_Mariette_1857-08-26}{Le 26 août 1857, de Paris, à Servaux, chef du bureau des travaux historiques au ministère de l'Instruction publique}~;
\item (n. p.) \hyperlink{CoEg_Mariette_1857-10-04}{Le 4 octobre 1857, de Paris, au ministre de l’Instruction publique}~;
\item (n. p.) \hyperlink{CoEg_Mariette_1857-10-05}{Le 5 octobre 1857, de Paris, à un fonctionnaire de l’Instruction publique}~;
\item (n. p.) \hyperlink{CoEg_Mariette_1879-11-06}{Le 6 novembre 1879, de Paris, au président de la commission des missions scientifiques}.
\end{itemize}
\par Ces lettres ont été conservées dans le dossier qui rassemble les demandes de mission de Mariette\gls{CoEg_pers_00000001} au sein des archives du bureau des missions au ministère de l'Instruction publique\gls{CoEg_org_00000042}.
\par Les premières demandes, refusées, remontent à 1846, alors que Mariette\gls{CoEg_pers_00000001} enseignait au collège\gls{CoEg_org_00000029} de Boulogne-sur-Mer\gls{CoEg_place_00000035}. Il présenta des projets hardis qui ne convainquirent pas l'administration de lui confier une mission en Égypte\gls{CoEg_place_00000003}.
\par Il fut plus heureux en 1850, alors qu'il travaillait au Louvre\gls{CoEg_org_00000002} et avait fait connaissance avec le milieu égyptologique de la capitale\gls{CoEg_place_00000002}.
\par Les lettres qui suivent son retour en France, de 1855 à 1857, documentent ses projets de publication au sujet du \Gls{CoEg_entry_00000028}\gls{CoEg_place_00000004} de Memphis\gls{CoEg_place_00000005} et les missions qu'il entreprit dans cette optique aux musées de Londres\gls{CoEg_org_00000005}, Berlin\gls{CoEg_org_00000040} et Turin\gls{CoEg_place_00000021}.
\par En 1857, le nouveau projet qui occupa Mariette\gls{CoEg_pers_00000001} fut de retourner en Égypte\gls{CoEg_place_00000003} pour préparer le voyage (qui ne devait jamais se réaliser) du prince Napoléon\gls{CoEg_pers_00000074}. Ce fut l'occasion pour Mariette\gls{CoEg_pers_00000001} d'obtenir une mission gratuite qui, sans engager de dépense de la part du ministère\gls{CoEg_org_00000042}, plaçait son voyage préparatoire sous les auspices du gouvernement\gls{CoEg_org_00000012} et lui permettait de projeter une publication sur fonds publics à son retour. L'histoire devait en décider autrement.
\par Une dernière lettre est adressée en 1879 à la commission des missions scientifiques\gls{CoEg_org_00000039} (présidée par le ministre) pour en solliciter le financement d'une publication portant sur les \glspl{CoEg_entry_00000027}.
\par La plupart de ces lettres sont destinées au ministre de l'Instruction publique. destinataire. Quelques-unes sont plus directement destinées à un fonctionnaire du ministère\gls{CoEg_org_00000042} ; il s'agit dans un cas d'Eugène Servaux\gls{CoEg_pers_00000237}, chef du bureau des travaux historiques.
\par Les dossiers de missions scientifiques du \textsc{xix}\textsuperscript{e} siècle ont été versés par le Centre national de la recherche scientifique aux Archives nationales\gls{CoEg_org_00000024} entre 1969 et 1973.

\hypertarget{CoEg_Mariette_ms_001}{}
\subsubsection*{20150497/118, dossier 145 « Mariette, Auguste »}
\addcontentsline{toc}{subsubsection}{20150497/118, dossier 145 « Mariette, Auguste »}
\noindent Ancienne cote : Paris, Bibliothèque centrale des musées nationaux, O/30/145 (cote utilisée avant le versement aux Archives nationales en 2015).
\begin{itemize}
\item (n. p.) \hyperlink{CoEg_Mariette_1849-10-20}{Le 20 octobre 1849, de Paris, à Longpérier}~;
\item (n. p.) \hyperlink{CoEg_Mariette_1850-07-08}{Le 8 juillet 1850, de Paris, à Nieuwerkerke}~;
\item (n. p.) \hyperlink{CoEg_Mariette_1851-02-28}{Le 28 février 1851, de Saqqarah, à Nieuwerkerke}~;
\item (n. p.) \hyperlink{CoEg_Mariette_1851-08-31}{Le 31 août 1851, de Saqqarah, à Nieuwerkerke}~;
\item (n. p.) \hyperlink{CoEg_Mariette_1851-09-14a}{Le 14 septembre 1851, de Saqqarah, à Le Moyne (copie)}~;
\item (n. p.) \hyperlink{CoEg_Mariette_1851-09-14b}{Le 14 septembre 1851, de Saqqarah, aux ministres de l’Intérieur et de l’Instruction publique (copie)}~;
\item (n. p.) \hyperlink{CoEg_Mariette_1852-01-16}{Le 16 janvier 1852, d’Abousir, vraisemblablement à Nieuwerkerke}~;
\item (n. p.) \hyperlink{CoEg_Mariette_1852-08-04}{Le 4 août 1852, d’Abousir, vraisemblablement à Nieuwerkerke}~;
\item (n. p.) \hyperlink{CoEg_Mariette_1852-08-20}{Le 20 août 1852, d’Abousir, au ministre de l'Intérieur}~;
\item (n. p.) \hyperlink{CoEg_Mariette_1852-09-03}{Le 3 septembre 1852, d’Abousir, au ministre de l'Intérieur}~;
\item (n. p.) \hyperlink{CoEg_Mariette_1852-09-04}{Le 4 septembre 1852, d’Abousir, vraisemblablement à Nieuwerkerke}~;
\item (n. p.) \hyperlink{CoEg_Mariette_1852-11-12}{Le 12 novembre 1852, d’Abousir, vraisemblablement à Nieuwerkerke}~;
\item (n. p.) \hyperlink{CoEg_Mariette_1852-12-28}{Le 28 décembre 1852, d’Abousir, au ministre de l'Intérieur}~;
\item (n. p.) \hyperlink{CoEg_Mariette_1853-01-01}{Le 1\textsuperscript{er} janvier 1853, d’Abousir, vraisemblablement à Nieuwerkerke}~;
\item (n. p.) \hyperlink{CoEg_Mariette_1853-05-06}{Le 6 mai 1853, d’Abousir, à Nieuwerkerke}~;
\item (n. p.) \hyperlink{CoEg_Mariette_1853-07-30}{Le 30 juillet 1853, du Caire, à Nieuwerkerke}~;
\item (n. p.) \hyperlink{CoEg_Mariette_1853-08-10}{Le 10 août 1853, d’Abousir, vraisemblablement à Nieuwerkerke}~;
\item (n. p.) \hyperlink{CoEg_Mariette_1853-08-28}{Le 28 août 1853, d’Abousir, vraisemblablement à Nieuwerkerke}~;
\item (n. p.) \hyperlink{CoEg_Mariette_1857-02-20}{Le 20 février 1857, de Paris, à Nieuwerkerke}~;
\item (n. p.) \hyperlink{CoEg_Mariette_1857-10-26}{Le 26 octobre 1857, d’Alexandrie, à Nieuwerkerke}~;
\item (n. p.) \hyperlink{CoEg_Mariette_1857-11-29}{Le 29 novembre 1857, d’Assiout, à Nieuwerkerke}~;
\item (n. p.) \hyperlink{CoEg_Mariette_1858-01-23}{Le 23 janvier 1858, du Caire, à Nieuwerkerke}~;
\item (n. p.) \hyperlink{CoEg_Mariette_1860-12-20}{Le 20 décembre 1860, de Boulaq, à Nieuwerkerke}~;
\item (n. p.) \hyperlink{CoEg_Mariette_1867-04-13}{Le 13 avril 1867, de Paris, à Nieuwerkerke}.
\end{itemize}
\par Ces lettres ont été conservées dans le dossier personnel de Mariette\gls{CoEg_pers_00000001} au sein des archives de l’administration des musées nationaux\gls{CoEg_org_00000001}. Elles correspondent à plusieurs étapes de sa carrière. Malgré leur cordialité de ton et quelques anecdotes, il s’agit surtout d’une correspondance professionnelle, dans laquelle l’égyptologue évoque à sa hiérarchie les progrès de ses missions et ses préoccupations en ce qui concerne l’entretien matériel de sa famille.
\par La première lettre correspond à ses débuts au Louvre\gls{CoEg_org_00000002} ; il y demande l’autorisation (qui ne semble pas lui avoir été accordée) d’améliorer son traitement en accomplissant des petits travaux sur les papyrus du musée en dehors de ses heures de service.
\par Les dix-sept lettres suivantes datent de son premier voyage en Égypte\gls{CoEg_place_00000003} (1850-1853). Il y informe sa hiérarchie de la situation du terrain, réclame périodiquement des fonds et demande des directives ou explique ses initiatives. Les négociations avec le gouvernement égyptien\gls{CoEg_org_00000008}, les stratagèmes de Mariette\gls{CoEg_pers_00000001} pour interpréter très libéralement les accords conclus avec celui-ci (ou le contourner tout à fait) et la coordination de ses efforts avec ceux du ministère des Affaires étrangères\gls{CoEg_org_00000007}, par le truchement du consulat général\gls{CoEg_org_00000006} de France\gls{CoEg_org_00000012} à Alexandrie\gls{CoEg_place_00000006} sont les principaux objets de ces lettres, qui renferment également des indications précises sur l’avancée des fouilles et quelques détails de sa vie quotidienne.
\par La lettre suivante date de 1857~; Mariette\gls{CoEg_pers_00000001} y demande un congé pour accomplir une mission au musée égyptien de Turin\gls{CoEg_org_00000021}.
\par Les trois lettres qui suivent datent du second voyage de Mariette\gls{CoEg_pers_00000001} en Égypte\gls{CoEg_place_00000003} (1857-1858). Elles traitent surtout de la préparation du voyage du prince Napoléon\gls{CoEg_pers_00000074} (qui n’eut finalement pas lieu mais constituait le prétexte officiel à cette nouvelle mission)~; de l’annonce par Mariette\gls{CoEg_pers_00000001} d’acquisitions destinées au prince, mais dont il espère qu’elles rejoindront le Louvre\gls{CoEg_org_00000002}~; et enfin de la préoccupation de l’organisation de ses congés, pour lui permettre de rester éloigné du Louvre\gls{CoEg_org_00000002} sans déroger au règlement et permettre à sa famille de toucher ses appointements.
\par La lettre suivante, du 20 décembre 1860, est la réponse d’une lettre envoyée à Mariette\gls{CoEg_pers_00000001} par Nieuwerkerke\gls{CoEg_pers_00000002} le 29 novembre (conservée dans le dossier et transcrite en note) et dans laquelle il lui annonçait être contraint de nommer un conservateur adjoint à sa place, et le nommait lui-même conservateur adjoint honoraire. Mariette\gls{CoEg_pers_00000001} se trouvait alors déjà engagé au service du vice-roi\gls{CoEg_pers_00000080} d’Égypte\gls{CoEg_place_00000003} pour diriger le service des antiquités.
\par Enfin, la dernière lettre de cette série date de 1867~: alors commissaire du pavillon égyptien à l'Exposition universelle de Paris\gls{CoEg_place_00000002}, Mariette\gls{CoEg_pers_00000001} demande à Nieuwerkerke\gls{CoEg_pers_00000002} de l’excuser de n’avoir pas reçu une invitation égarée.
\par Toutes ces lettres s’adressent à la hiérarchie de Mariette\gls{CoEg_pers_00000001} à différents moments de sa car-rière~: Adrien de Longpérier\gls{CoEg_pers_00000033}\footnote{Supérieur de Mariette\gls{CoEg_pers_00000001} en 1849 en tant que conservateur du département des sculptures et des antiques\gls{CoEg_org_00000025} du musée du Louvre\gls{CoEg_org_00000002} (le département égyptien\gls{CoEg_org_00000020} venait tout juste de recevoir un conservateur propre avec la nomination de Rougé\gls{CoEg_pers_00000032} le 1\textsuperscript{er} août 1849).}~; les ministres responsables de sa première mission\footnote{Le ministre de l’Intérieur (dont dépendaient les musées nationaux\gls{CoEg_org_00000001} jusqu’en 1853) et celui de l’Instruction publique.}~; sept lettres s’adressent explicitement au directeur du musée du Louvre\gls{CoEg_org_00000002}, le comte de Nieuwerkerke\gls{CoEg_pers_00000002}. Le destinataire de neuf de ces lettres n’est pas nommé~; cependant, tout en étant distinct du vicomte de Rougé\gls{CoEg_pers_00000032}, il s’agissait manifestement d’un haut fonctionnaire parisien en relation avec les autres administrations et qui fréquentait les collègues de Mariette\gls{CoEg_pers_00000001} au Louvre\gls{CoEg_org_00000002}~: il est très probable qu’il s’agisse là aussi du comte de Nieuwerkerke\gls{CoEg_pers_00000002}.
\par Les brouillons de plusieurs de ces lettres sont conservés à la Bibliothèque nationale de France\gls{CoEg_org_00000026} sous la cote ms. NAF 20179.
\par Ces documents ont été rassemblés assez tôt au sein des archives du Louvre\gls{CoEg_org_00000002}, où la copie de douze des lettres écrites par Mariette\gls{CoEg_pers_00000001} pendant sa première mission semble avoir été réalisée. Cette copie n'est pas datée ni signée~; l’écriture est ancienne mais ne correspond ni à la main de Mariette\gls{CoEg_pers_00000001}, ni à celle de Maspero, et le copiste n’était pas familier des noms propres égyptiens. Ces copies, avec d’autres, sont aujourd’hui conservées à la bibliothèque de l’Institut de France\gls{CoEg_org_00000004} sous la cote ms. 4061 (2), f\textsuperscript{os} 11-57.
\par Les archives conservées à la bibliothèque centrale des musées nationaux ont été versées aux Archives nationales\gls{CoEg_org_00000024} en 2015.

\section*{Historique du fichier}
\addcontentsline{toc}{section}{Historique du fichier}
\begin{itemize}
\item Février 2020, v. 0,18~: essais sur un premier échantillon de lettres issues du dossier de carrière de Mariette\gls{CoEg_pers_00000001} dans l’administration des musées nationaux\gls{CoEg_org_00000001}~;
\item Juillet 2020, v.~0,24~: ajout des autres lettres du dossier échantillon, reprise de l’encodage dans le cadre d’une chaîne de traitement complète et premiers essais de publication sur \href{https://thlebee.github.io/CoEg/}{Github}~;
\item Novembre 2020, v.~0,44~: ajout des dossier de missions de Mariette dans le fonds de l'Instruction publique aux Archives nationales.
\end{itemize}

\mainmatter

\chapter*{Lettres d’Auguste Mariette}
\addcontentsline{toc}{chapter}{Lettres d’Auguste Mariette}

\hypertarget{CoEg_Mariette_1846-04-13}{}

\section*{Le 13 avril 1846, de Boulogne-sur-Mer, à Salvandy, ministre de l’Instruction publique}
\addcontentsline{toc}{section}{Le 13 avril 1846, de Boulogne-sur-Mer, à Salvandy, ministre de l’Instruction publique} \label{labCoEg_Mariette_1846-04-13}
{\footnotesize
\noindent Institution et lieu de conservation~: Archives nationales, Pierrefitte-sur-Seine.\\
Cote : \hyperlink{CoEg_Mariette_ms_002}{F/17/2988/1, dossier « Mariette »} (n. p.).\\
Support : une feuille simple de grand format.\\
Thème~: \gls{CoEg_keyword_00000001}.\\
Notes~:\begin{itemize}
\item La lettre porte un tampon : « Instruction publique\gls{CoEg_org_00000042}. 15 avril 1846 »~; un chiffre ([239.80 ?]) a été complété à la main dans le pourtour du tampon), et une annotation à l’encre au coin supérieur gauche : « [\sout{I} ?]. 2. 1 ».
\item La demande fut appuyée par le député François Delessert\gls{CoEg_pers_00000125} par une lettre du 29 mai ; le ministère répondit négativement à Mariette et à Delessert le 26 juin 1846 (Archives nationales, F/17/2988/1, dossier « Mariette »).
\end{itemize}
\begin{center} {[1\textsuperscript{re} page, r\textsuperscript{o}]}\end{center}}
\begin{flushright}Boulogne-sur-mer\gls{CoEg_place_00000035}, le 13 avril 1846\end{flushright}
\par A Monsieur le Ministre, secrétaire d’État\\
\indent au Département de l’Instruction Publique\gls{CoEg_org_00000042}\\

\hspace{1cm} Monsieur le Ministre\gls{CoEg_pers_00000090},\\
\par Je me livre, depuis long-temps déjà, à l’étude de l’histoire\\
ancienne et de l’archéologie, et surtout à l’étude de l’archéologie\\
égyptienne. C’est une spécialité à laquelle je désire me consacrer\\
pour continuer, autant qu’il me sera possible, les travaux exécutés\\
déjà par des hommes dont les noms marquent dans la science.\\
\indent J’ai l’honneur de solliciter de votre bienveillance, Monsieur le\\
Ministre, une subvention prise sur le Budget de votre Département\gls{CoEg_org_00000042},\\
qui me permette d’aller passer une année au moins en Egypte\gls{CoEg_place_00000003}. –\\
J’occuperais cette année soit à parcourir le pays, soit à décrire\\
les monuments, à en copier, à en étudier les hiéroglyphes, selon\\
que vous le désirerez.\\
\indent Je connais le français, l’anglais, le latin, le grec, et\\
un peu l’arabe que j’apprends en ce moment. – Je sais le\\
dessin assez pour l’enseigner, (je l’ai enseigné en effet pendant un an),\\
{\footnotesize \begin{center} {[1\textsuperscript{re} page, v\textsuperscript{o}]}\end{center}}
\noindent et la peinture assez pour copier la nature. – Comme écrivain,\\
j’ai fait aussi mes preuves dans l’\textit{Annotateur}\gls{CoEg_bibl_00000006}, journal conservateur,\\
dont je suis le rédacteur en chef depuis trois ans et demie. – Je ne\\
crois pas inutile d’ajouter que j’écris en ce moment un cours\\
d’histoire ancienne, dont je soumettrai bientôt la première partie\\
(histoire sainte) au Comité Royal de l’Instruction Publique\gls{CoEg_org_00000028}.\\
\indent C’est avec ces titres en main que je me présente pour\\
obtenir la faveur d’un voyage en Egypte\gls{CoEg_place_00000003}. – C’est là une\\
mission de confiance que je sollicite, confiance en échange de laquelle\\
je ne puis promettre rien autre chose que de travailler assidûment\\
aux progrès de la science.\\
\indent Je pourrais, au besoin, appuyer ma demande des protections\\
les plus hautes et les plus honorables. Mais, dans des\\
circonstances aussi graves pour moi, je ne serais content de\\
voir ma demande accueillie favorablement qu’autant que\\
j’aurais en même temps la certitude de pouvoir utilement remplir\\
ma mission : cette certitude, je la posséderai le jour où vous\\
voudrez bien m’accorder ce que je sollicite.\\
\indent Veuillez agréer l’assurance du profond respect
\begin{center}avec lequel j’ai l’honneur d’être\end{center}
\begin{center}\hspace{5cm}Votre très-humble\\
\hspace{5cm}et très-obéissant serviteur\\
\hspace{5cm}\textit{\gls{CoEg_abbr_00000002} Mariette}\\
\hspace{3,5cm}régent de septième au Collège Communal\gls{CoEg_org_00000029} de Boulogne\gls{CoEg_place_00000035},\\
\hspace{3,5cm}membre du comité local d’instruction primaire\gls{CoEg_org_00000030},\\
\hspace{3,5cm}secrétaire-rédacteur de la Société d’Agriculture et des Sciences\gls{CoEg_org_00000031}.\end{center}


\hypertarget{CoEg_Mariette_1846-05-24}{}

\section*{Le 24 mai 1846, de Boulogne-sur-Mer, à Camaret, recteur de l’académie de Douai (copie)}
\addcontentsline{toc}{section}{Le 24 mai 1846, de Boulogne-sur-Mer, à Camaret, recteur de l’académie de Douai (copie)} \label{labCoEg_Mariette_1846-05-24}
{\footnotesize
\noindent Institution et lieu de conservation~: Archives nationales, Pierrefitte-sur-Seine.\\
Cote : \hyperlink{CoEg_Mariette_ms_002}{F/17/2988/1, dossier « Mariette »} (n. p.).\\
Support : deux feuilles doubles de moyen format reliées.\\
Thème~: \gls{CoEg_keyword_00000001}.\\
Note~: Le texte de cette lettre a été copié par Mariette et joint à \hyperlink{CoEg_Mariette_1846-05-25}{celle qu'il a envoyée le lendemain au ministre de l’Instruction publique} à l’appui de sa demande.}

{\footnotesize \begin{center} {[1\textsuperscript{re} page, r\textsuperscript{o}]}\end{center}}
\begin{flushright}Boulogne\gls{CoEg_place_00000035}, le 24 Mai 1846.\end{flushright}
\par A Monsieur\\
\indent \hspace{1cm}Monsieur le Recteur de l’Académie\gls{CoEg_org_00000032} de Douai\gls{CoEg_place_00000036}.\\

\hspace{1cm} Monsieur le Recteur\gls{CoEg_pers_00000093},\\
\par Je crois obéir à un sentiment de convenance aussi bien qu’à un sentiment\\
de devoir en vous informant que je viens d’adresser à M. le Ministre\gls{CoEg_pers_00000090}\\
de l’Instruction Publique une demande tendant à obtenir une\\
subvention de son Département\gls{CoEg_org_00000042} qui me permette d’aller passer\\
une année en Egypte\gls{CoEg_place_00000003} . . . . . . . . . .\\
\indent Je me livre depuis long-temps déjà à l’étude de l’antiquité,\\
de l’histoire, de l’archéologie, et en particulier de l’antiquité\\
égyptienne. C’est une spécialité à laquelle je me suis voué par goût\\
et à laquelle je consacre ma vie . . . . J’ai toujours cru qu’il serait\\
bon et honorable pour moi de m’associer pour ma faible part\\
aux efforts des hommes remarquables qui tentent tant aujourd’hui\\
en faveur de l’histoire ancienne. Je sais que ce champ est\\
vaste, trop vaste sans aucun doute pour moi. Je ne le\\
parcourrai pas en dix ans, en vingt ans peut-être ; mais je\\
m’efforcerai toujours de faire en sorte que ma patience et mon\\
travail soient en raison directe de la difficulté de l’entreprise.\\
Telle est la cause de ma demande à M. le Ministre\gls{CoEg_pers_00000090}.\\
\indent Quant au but que je me proposerais en allant en Egypte\gls{CoEg_place_00000003},\\
si j’y allais pour mon propre compte, ce but serait triple, car\\
je partagerais mes travaux en trois branches.
{\footnotesize \begin{center} {[1\textsuperscript{re} page, v\textsuperscript{o}]}\end{center}}
\indent Il y a d’abord l’écriture égyptienne divisée en hiéroglyphique,\\
hiératique et démotique. Ce n’est pas une étude d’un jour que\\
celle-là, et pour connaître à fond Champollion\gls{CoEg_pers_00000094}, Young\gls{CoEg_pers_00000095}, Ackerblad\gls{CoEg_pers_00000096}\\
et autres, il y a bien des travaux à exécuter. Je n’oserais pas, Monsieur\\
le Recteur, toucher en quoi que ce soit à la gloire dont se sont\\
environnés ces savants ; je m’incline au contraire devant leur science.\\
Mais je ne crois pas que la clef des hiéroglyphes trouvée par\\
Champollion\gls{CoEg_pers_00000094} aidé des conseils de M. Letronne\gls{CoEg_pers_00000097}, des recherches du\\
docteur Young\gls{CoEg_pers_00000095}, soit la clef qui ouvre toutes les portes. On n’a pas\\
tout dit sur cette écriture mystérieuse qui est à la fois hiéroglyphique\\
figurative, comme les 214 [tri...?] des Chinois, hiéroglyphique\\
symbolique comme les \glspl{CoEg_entry_00000038} du Chili\gls{CoEg_place_00000037} et du Pérou\gls{CoEg_place_00000038}, ou simplement\\
alphabétique comme l’hébreu dont l’alphabet, selon [Critirnus ?]\gls{CoEg_pers_imn}\\
a été trouvé par Moïse\gls{CoEg_pers_00000099}, le Syriaque et le Chaldéen par Abraham\gls{CoEg_pers_00000100},\\
l’attique par Cadmus\gls{CoEg_pers_00000101} contemporain de Josuë\gls{CoEg_pers_00000102}, le gothique par\\
Ulphilas\gls{CoEg_pers_00000103}. Il a déjà été beaucoup publié sur cette écriture,\\
mais on n’en a pas encore trouvé la véritable clef. Selon moi\\
\sout{se le} ce problème n’est pas insoluble, s’il est vrai que la langue\\
copte moderne soit à peu près la vieille langue parlée des\\
Egyptiens, s’il est vrai que la Chine\gls{CoEg_place_00000039} ait autrefois communiqué\\
avec le monde occidental, comme le prouve, pour n’en citer\\
qu’une preuve, Lao-Tseu\gls{CoEg_pers_00000104} qui, six siècles avant \gls{CoEg_abbr_00000045}\gls{CoEg_pers_00000235}, enseignait\\
à ses compatriotes les doctrines qui ont immortalisé Aristote\gls{CoEg_pers_00000105} et\\
Platon\gls{CoEg_pers_00000106} ; – s’il est vrai enfin qu’il ait existé autrefois sur\\
les rives du Nil\gls{CoEg_place_00000021} une civilisation dont l’importance seule suffit\\
pour soutenir le courage de ceux qui cherchent à l’exhumer
{\footnotesize \begin{center} {[2\textsuperscript{e} page, r\textsuperscript{o}]}\end{center}}
\noindent des débris où elle est ensevelie depuis tant de siècles. – Je le répète,\\
Monsieur le Recteur, je ne crois pas que ce problème soit sans\\
solution possible. C’est cette solution que je désire chercher. Je\\
sais tout ce qu’elle offre de difficile, car j’ai déjà appris ce qu’a\\
enseigné Champollion le jeune\gls{CoEg_pers_00000094}. J’en suis arrivé à connaître\\
où son système peut s’appliquer, où il ne le peut pas. Conséquemment\\
je sais ce qui a été fait, et ce qui reste à faire. La tâche est\\
donc aride, mais je l’entreprendrai, Monsieur le Recteur, quelque\\
difficile que cela puisse être.\\
\indent La seconde division de mes travaux, si je voyageais à mes frais\\
en Egypte\gls{CoEg_place_00000003}, serait relative à l’archéologie proprement dite.\\
\indent Ici ce sont des fouilles à faire, des dessins à prendre,\\
des inscriptions à copier. Tout n’est pas encore terminé, quant\\
aux monuments, et il reste assez de travaux à exécuter pour que\\
le gouvernement consente à doter les sciences archéologiques de\\
nouveaux résultats de recherches multipliées. On n’a pas encore\\
ouvert le fameux puits de la grande pyramide de Gyzeh\gls{CoEg_place_00000007}, on\\
ne sait encore si de nouvelles salles n’élargissent pas les\\
grottes d’Eléthya\gls{CoEg_place_00000073}, les \glspl{CoEg_entry_00000032} de Thèbes\gls{CoEg_place_00000019} renferment des milliers de\\
momies qu’on n’a pas encore fouillées. De tous côtés, en Egypte\gls{CoEg_place_00000003},\\
il y a des monuments imparfaitement décrits, couverts d’inscriptions\\
dont les dessins n’existent pas encore et il y a mille statues, mille\\
colonnes en pierre, jusqu’à la poitrine, jusqu’au chapiteau, dans\\
le sable. Les Arabes y attachent leurs chevaux, et les savans [\textit{sic}]\\
passent sans même les regarder. Pourquoi ne pas mettre au\\
jour quelques-unes de ces ruines ? Qui sait si le hazard [\textit{sic}] ne
\begin{flushright}donnera\end{flushright}
{\footnotesize \begin{center} {[2\textsuperscript{e} page, v\textsuperscript{o}]}\end{center}}
\noindent donnera pas à l’investigateur de nouveaux manuscrits bilingues comme\\
ceux\gls{CoEg_obj_imn} de Turin\gls{CoEg_place_00000034}, une nouvelle pierre\gls{CoEg_obj_00000021} de Rosette\gls{CoEg_place_00000079}, ou quelque stèle\\
où la traduction grecque complète d’un passage hiéroglyphique\\
synoptique viendra enfin donner la clef définitive de l’écriture\\
sacrée égyptienne ? Ce sont là de grands, de sérieux travaux\\
à entreprendre. Et puis ce n’est pas seulement l’Egypte\gls{CoEg_place_00000003} qui\\
est riche en si utiles monuments ; il y a tout le pays au-delà\\
de la première cataracte. C’est là l’Ethiopie\gls{CoEg_place_00000041} dont l’histoire\\
est enveloppée d’un profond mystère, qui fit la conquête de\\
l’Egypte\gls{CoEg_place_00000003} et que Cambyse\gls{CoEg_pers_00000030} essaya vainement de subjuguer. Voilà\\
encore une civilisation à retrouver, une histoire à déchiffrer sur\\
les monuments. – Quant aux pyramides de Gyzeh\gls{CoEg_place_00000007} et de Saqqarah\gls{CoEg_place_00000001} [\textit{sic}]\footnote{Mariette utilise le plus souvent (plus tard ?) la forme « Sakkarah ».},\\
les fouilles qu’il faudrait y entreprendre sont fort importantes.\\
Il existe dans la plus grande de ces pyramides une excavation\\
profonde qui, du temps de Polybe\gls{CoEg_pers_00000107}, je crois, avait 84 coudées\\
de profondeur. Cette excavation n’est pas un puits, car elle\\
est inclinée sur la verticale, taillée en gradins, et le conduit\\
qui y mène ne se traverse qu’en rampant. Ce n’est pas non\\
plus l’escalier d’une troisième chambre mortuaire ; en certains\\
endroits ce puits n’a que dix pouces de diamètre, – une momie\\
n’aurait pu y passer. Hérodote\gls{CoEg_pers_00000108} n’en parle pas, mais il parle\\
à deux reprises des édifices souterrains que cette pyramide recouvre.\\
Je crois ce puits un conduit destiné à renouveler l’air dans ces\\
édifices, et s’il m’était permis de pousser les conjectures plus\\
loin, je ferais entrevoir le motif qui détermina les Egyptiens
{\footnotesize \begin{center} {[3\textsuperscript{e} page, r\textsuperscript{o}]}\end{center}}
\noindent à construire leurs pyramides ; et pour venger ces peuples des reproches\\
qu’on leur a fait, je représenterais ces masses énormes dont on a\\
tant blâmé la vanité, la pesanteur, les dépenses et l’inutilité,\\
comme les monumens [\textit{sic}] destinés à la conservation des sciences, des\\
arts et de toutes les connaissances égyptiennes. Ce n’est\\
pas ici le temps, Monsieur le Recteur, de discuter cette opinion.\\
Il faudrait entrer pour cela dans des détails historiques et\\
archéologiques, dans les mystères même du gouvernement et de la\\
religion des Egyptiens. Cette opinion, du reste, j’ai cherché à me la [\textit{sic}]\\
combattre à moi-même. J’ai lu les auteurs qui en font des tombeaux,\\
ceux qui en font des phares, ceux qui en font des greniers d’abondance,\\
ceux qui en font rien [\textit{sic}], ceux qui en font des masses destinées à\\
arrêter les sables – et c’est en cherchant à renverser moi-même\\
cette opinion que j’ai acquis tous les jours de plus en plus la\\
certitude de sa solidité.\\
\indent Mais ce n’est pas tout encore ce que je ferais : le reste\\
serait la 3\textsuperscript{\underline{e}} division de mes travaux. Cette 3\textsuperscript{\underline{e}} division serait\\
relative à la bibliographie ancienne. L’étude de Diodore\gls{CoEg_pers_00000031} de\\
Sicile\gls{CoEg_place_00000058}, de Plutarque\gls{CoEg_pers_00000109}, d’Apulée\gls{CoEg_pers_00000110}, de Tacite\gls{CoEg_pers_00000111}, de \gls{CoEg_abbr_00000046} Clément\gls{CoEg_pers_00000112} d’Alexandrie\gls{CoEg_place_00000006},\\
de Philon\gls{CoEg_pers_00000113}, d’Eusèbe\gls{CoEg_pers_00000114} et de quelques autres, m’a mis à même de\\
faire une liste des auteurs dont il ne reste que des fragments,\\
et une autre liste des auteurs dont il ne reste que le nom.\\
La découverte des ouvrages d’un seul de ces auteurs élargirait\\
beaucoup le cercle de l’histoire ancienne. Je ne vous apprendrai\\
rien, Monsieur le Recteur, de l’utilité d’une pareille découverte.
{\footnotesize \begin{center} {[3\textsuperscript{e} page, v\textsuperscript{o}]}\end{center}}
\noindent Si l’ouvrage complet de Sanchoniathon\gls{CoEg_pers_00000115} qui a écrit sur la Théologie\\
Phénicienne dont il ne nous reste qu’un fragment conservé par\\
Philon\gls{CoEg_pers_00000113} et Eusèbe\gls{CoEg_pers_00000114}, sur la Théologie Egyptienne qui est le but de\\
tant de recherches aujourd’hui – si l’ouvrage de Manéthon\gls{CoEg_pers_00000116}, gardien\\
des archives sacrées des Egyptiens sous Ptolémée Philadelphe\gls{CoEg_pers_00000117}, qui\\
a écrit une histoire générale d’Egypte\gls{CoEg_place_00000003} – si les 42 livres de la\\
Sagesse Egyptienne, enfermés dans le sanctuaire de chacun des\\
temples construits au bord du Nil\gls{CoEg_place_00000021}, où la médecine antique,\\
la géographie, l’histoire, la religion sont expliquées – si tout\\
cela se retrouvait, quelle révolution ne serait pas produite\\
dans l’étude de l’antiquité. Ces trois seuls exemples, Monsieur\\
le Recteur, suffisent pour vous faire voir quel intérêt s’atta-\\
-cherait à la résurrection des œuvres d’Horapollon\gls{CoEg_pers_00000118}, de Palephate\gls{CoEg_pers_00000119},\\
d’Hermès Trismégiste\gls{CoEg_pers_00000120}, de Darès le Phrygien\gls{CoEg_pers_00000121} et de tant d’autres.\\
L’histoire du monde pourrait peut-être se compléter, et nos\\
études classiques trouveraient ainsi de nouveaux aliments. –\\
Or, dans les \glspl{CoEg_entry_00000032} de Thèbes\gls{CoEg_place_00000019}, dans la partie des catacombes\\
appelée les Tombeaux des grands, il y a des milliers de momies\\
qui n’ont pu être fouillées encore. Toutes, ou presque toutes,\\
sont enfermées dans des sarcophages avec des papyrus en\\
langue égyptienne, et aussi en langue grecque. Sur mille\\
papyrus grecs, ou en trouvera peut-être un qui nous parlera\\
de l’histoire, tous nous parleront des mœurs des Egyptiens. A\\
n’en pas douter, bien des prêtres de Thèbes\gls{CoEg_place_00000019} ont écrit sur l’histoire\\
de leur pays et ont été ensevelis, selon toute probabilité,
{\footnotesize \begin{center} {[4\textsuperscript{e} page, r\textsuperscript{o}]}\end{center}}
\noindent avec leur œuvre. Il importe donc à l’histoire, à la chronologie,\\
que tout cela se retrouve. Les odes d’Anacréon\gls{CoEg_pers_00000122} ont bien été perdues\\
jusqu’en 1554, époque à laquelle H. Etienne\gls{CoEg_pers_00000123} les retrouva. –\\
Ce serait là le troisième but de mon voyage, but qui, je crois,\\
n’a encore été celui de personne jusqu’à présent. Les voyageurs qui\\
passent à Memphis\gls{CoEg_place_00000005}, à Latopolis\gls{CoEg_place_00000042}, à Hermontis\gls{CoEg_place_00000043}, à Thèbes\gls{CoEg_place_00000019}, à\\
l’île Eléphantine\gls{CoEg_place_00000044}, mesurent en effet les papyrus à leur longueur :\\
celui\gls{CoEg_obj_00000022} de Turin\gls{CoEg_place_00000034} à [\textit{sic}] 66 pieds, celui\gls{CoEg_obj_imn} de Paris\gls{CoEg_place_00000002} n’en a que 22. Pour\\
tous ceux qui parcourent maintenant l’Egypte\gls{CoEg_place_00000003}, ce serait le premier\\
le plus important. Voilà comment cherchent les voyageurs, et je\\
sais pertinemment que les habitants qui avoisinent les \glspl{CoEg_entry_00000032}\\
de Thèbes\gls{CoEg_place_00000019}, possèdent un grand nombre de petits papyrus qu’on\\
délaisse parce qu’ils n’ont pas deux pieds.\\
\indent Tel serait, Monsieur le Recteur, ce que j’entreprendrais\\
si je voyageais pour mon propre compte. – Mais, dans les\\
conditions où je me trouve, cela ne m’est pas possible, et je\\
suis forcé de me mettre tout entier à la disposition du\\
Gouvernement. J’irai donc en Egypte\gls{CoEg_place_00000003}, envoyé en mission\\
scientifique, pour y faire ce que Monsieur le Ministre de\\
l’Instruction Publique\gls{CoEg_pers_00000090} m’ordonnera d’y faire. Ce sera\\
mon premier pas sérieux dans la carrière que j’ai embrassée :\\
l’étude des hiéroglyphes, des monuments, de l’histoire d’Egypte\gls{CoEg_place_00000003}\\
enfin dans toutes ses branches. – J’espère du reste que \gls{CoEg_abbr_00000001} le\\
Ministre\gls{CoEg_pers_00000090} voudra bien seconder mes efforts en me mettant à\\
même de travailler mieux que je ne le puis faire ici où la\\
nécessité de la vie et les devoirs de ma position ne laissent
{\footnotesize \begin{center} {[4\textsuperscript{e} page, v\textsuperscript{o}]}\end{center}}
que quelques instants libres à la science . . . . .\\
\indent J’ai l’honneur d’être, etc.
\begin{center}\hspace{5cm}\gls{CoEg_abbr_00000002} Mariette\\
\hspace{5cm}Régent de septième au Collège\gls{CoEg_org_00000029},\\
\hspace{5cm}Membre du Comité local d’Inst. Prim.\gls{CoEg_org_00000030} \&\\
\hspace{5cm}Secrétaire de la Société d’Agric. et des sciences\gls{CoEg_org_00000031}\end{center}

\hypertarget{CoEg_Mariette_1846-05-25}{}

\section*{Le 25 mai 1846, de Boulogne-sur-Mer, à Salvandy, ministre de l’Instruction publique}
\addcontentsline{toc}{section}{Le 25 mai 1846, de Boulogne-sur-Mer, à Salvandy, ministre de l’Instruction publique} \label{labCoEg_Mariette_1846-05-25}
{\footnotesize
\noindent Institution et lieu de conservation~: Archives nationales, Pierrefitte-sur-Seine.\\
Cote : \hyperlink{CoEg_Mariette_ms_002}{F/17/2988/1, dossier « Mariette »} (n. p.).\\
Support : deux feuilles doubles de moyen format reliées.\\
Thème~: \gls{CoEg_keyword_00000001}.\\
Notes~:
\begin{itemize}
\item Cette lettre accompagne une copie de \hyperlink{CoEg_Mariette_1846-05-24}{celle que Mariette avait envoyée la veille} au recteur de l’académie de Douai.
\item La lettre porte un tampon « Instruction publique. 17 juin 1846 » complété par une annotation manuscrite « 281.[3 ?]0 », et une annotation à l’encre au coin supérieur gauche : « 23 ».
\end{itemize}}

\begin{flushright}Boulogne\gls{CoEg_place_00000035}, ce 25 Mai 1846.\end{flushright}
\indent A Monsieur\\
\indent \hspace{1cm} Monsieur le Ministre, Secrétaire d’État, au Département
\begin{center}de l’Instruction Publique\gls{CoEg_org_00000042}.\end{center}

\hspace{1cm} Monsieur le Ministre\gls{CoEg_pers_00000090},\\

\par La \hyperlink{CoEg_Mariette_1846-04-13_01}{pétition} que j’ai eu l’honneur de vous adresser le 13 avril\\
dernier étant restée jusqu’à ce jour sans réponse, je crois pouvoir\\
encore vous adresser aujourd’hui un extrait de la lettre que j’ai\\
écrite à M. le Recteur\gls{CoEg_pers_00000093} de l’Académie\gls{CoEg_org_00000032} de Douai\gls{CoEg_place_00000036} pour l’informer\\
de ma demande.\\
\indent Cet extrait me parait [\textit{sic}] propre à vous connaître [\textit{sic}] le but que je\\
me proposerais en allant en Egypte\gls{CoEg_place_00000003} étudier l’histoire sur les\\
lieux mêmes des événements, et à vous rendre plus faciles\\
l’examen et la solution de l’affaire qui me concerne.\\
\indent J’ai l’honneur d’être,\\
\hspace{1cm} avec le plus profond respect,
\begin{center}Monsieur le Ministre,\end{center}
\begin{center}\hspace{5cm}Votre très-humble serviteur.\\
\hspace{5cm}\gls{CoEg_abbr_00000002} Mariette\\
\hspace{5cm}régent de septième, membre de comité local\gls{CoEg_org_00000030},\\
\hspace{5cm}secrétaire de la société d’agriculture\gls{CoEg_org_00000031},\\
\hspace{5cm}rédacteur en chef de l’Annotateur\gls{CoEg_bibl_00000006}.\end{center}

\hypertarget{CoEg_Mariette_CoEg_Mariette_1846-09-29}{}

\section*{Le 29 septembre 1846, de Boulogne-sur-Mer, à Salvandy, ministre de l’Instruction publique}
\addcontentsline{toc}{section}{Le 29 septembre 1846, de Boulogne-sur-Mer, à Salvandy, ministre de l’Instruction publique} \label{labCoEg_Mariette_1846-09-29}
{\footnotesize
\noindent Institution et lieu de conservation~: Archives nationales, Pierrefitte-sur-Seine.\\
Cote : \hyperlink{CoEg_Mariette_ms_002}{F/17/2988/1, dossier « Mariette »} (n. p.).\\
Support : une feuille simple de moyen format.\\
Thème~: \gls{CoEg_keyword_00000001}.\\
Notes~:
\begin{itemize}
\item La lettre porte une annotation à l’encre au coin inférieur gauche : «~Sur la demande instante de M. le Maire\gls{CoEg_pers_00000124} de Boulogne\gls{CoEg_place_00000035}, j’ai l’honneur de recommander à Monsieur le Ministre de l’Instruction Publique\gls{CoEg_pers_00000090} la lettre de M. Mariette\gls{CoEg_pers_00000001}. D’après les renseignements qu’on m’a donnés sur lui, il me paraît digne de la bienveillante protection de Monsieur le Ministre. Paris\gls{CoEg_place_00000002}, 2 : 8\textsuperscript{bre} : 1846. Fr. Delessert\gls{CoEg_pers_00000125} député de l’arrondiss\textsuperscript{t} de Boulogne \textsuperscript{S}/mer\gls{CoEg_place_00000035}~». La lettre porte également une annotation à l’encre d’une autre main que celle de Mariette au coin supérieur droit «~23.~», et un tampon au coin supérieur gauche : «~Instruction publique\gls{CoEg_org_00000042}. 14 octobre 1843~», complété à la main à l’encre «~281.[3~?]0~».
\item Mariette reçut une nouvelle réponse négative le 10 novembre 1846~: comme il lui avait été indiqué suite à sa première demande, les crédits disponibles étaient alors épuisés, et les règlements du ministère des Finances s’opposaient à la concession de passages gratuits sur les paquebots de la Méditerranée (Archives nationales, F/17/2988/1, dossier « Mariette »).\end{itemize}}

\begin{flushright}Boulogne-sur-mer\gls{CoEg_place_00000035}, le 29 septembre 1846.\end{flushright}

\indent A Son Excellence\\
\indent \hspace{0,5cm}Monsieur le Ministre, secrétaire d’Etat, au département\\
\indent \hspace{1cm}de l’Instruction Publique\gls{CoEg_org_00000042}\\

\hspace{1cm} Monsieur le Ministre\gls{CoEg_pers_00000090},\\

\par J’ai l’honneur de vous exposer que, désirant poursuivre sur les lieux même\\
le cours des études archéologiques auxquelles je me suis consacré, j’ai\\
résolu de faire à mes frais un voyage scientifique en Egypte\gls{CoEg_place_00000003}. – Je désire\\
embrasser la carrière de voyageur archéologue, et je me préparerais ainsi\\
dans ce premier voyage, à en entreprendre d’autre plus sérieux, le\\
jour où la confiance du gouvernement m’y appellerait officiellement.\\
\indent Je viens vous demander, Monsieur le Ministre, avec le passage\\
gratuit sur un paquebot-poste de Marseille\gls{CoEg_place_00000018} à Alexandrie\gls{CoEg_place_00000006}, une\\
somme de deux mille francs. En échange je me mettrai à votre\\
disposition pour telle recherche, telle exploration qu’il vous plaira.\\
\indent Veuillez agréer l’assurance du profond respect avec lequel\\
\indent j’ai l’honneur d’être
\begin{center}Monsieur le Ministre,\end{center}
\begin{center}\hspace{5cm}Votre très-humble\\
\hspace{5cm}\& très-obéissant serviteur.\\
\hspace{5cm}\gls{CoEg_abbr_00000002} Mariette\\
\hspace{5cm}professeur au Collège\gls{CoEg_org_00000029}, membre du\\
\hspace{5cm}comité local d’instruction primaire\gls{CoEg_org_00000030}, secrétaire\\
\hspace{5cm}de la société d’agriculture et des Sciences\gls{CoEg_org_00000031}\end{center}

\hypertarget{CoEg_Mariette_1849-10-20}{}

\section*{Le 20  octobre 1849, de Paris, à Longpérier, conservateur des antiques et sculptures au Louvre}
\addcontentsline{toc}{section}{Le 20  octobre 1849, de Paris, à Longpérier, conservateur des antiques et sculptures au Louvre} \label{labCoEg_Mariette_1849-10-20}
{\footnotesize
\noindent Institution et lieu de conservation~: Archives nationales, Pierrefitte-sur-Seine.\\
Cote : \hyperlink{CoEg_Mariette_ms_001}{20150497/118, dossier 145 «~Mariette, Auguste~»} (n. p.).\\
Support : une feuille double de moyen format.\\
Thème~: \gls{CoEg_keyword_00000005}.\\
Note~: la lettre est accompagnée d’un mot à Nieuwerkerke\gls{CoEg_pers_00000002} par Longpérier\gls{CoEg_pers_00000033} du 22 octobre 1849 (tout en transmettant la demande, Longpérier\gls{CoEg_pers_00000033} formule une réserve pratique, Mariette\gls{CoEg_pers_00000001} se trouvant alors déjà rémunéré sur un fonds extraordinaire)~; toutes deux portent un tampon à l’encre rouge : «~24 octobre 1849/Ministère de l’Intérieur\gls{CoEg_org_00000009}/Musées nationaux\gls{CoEg_org_00000001}~».}
\begin{flushright}
Paris\gls{CoEg_place_00000002}, le 20 octobre 1849.
\end{flushright}
A Monsieur
\begin{center} Monsieur Adrien de Longpérier\gls{CoEg_pers_00000033}, conservateur des Antiques et\end{center}
\begin{flushright}Sculptures\gls{CoEg_org_00000025} au Musée du Louvre\gls{CoEg_org_00000002}.\end{flushright}

\hspace{1cm} Monsieur\gls{CoEg_pers_00000033},\\

\par Je ne crois pas qu’en ma qualité de simple employé du département\gls{CoEg_org_00000025} confié\\
à vos soins, je puisse écrire directement et officiellement à l’administration\\
du Musée\gls{CoEg_org_00000002} pour une demande que j’ai à lui soumettre. Permettez-moi\\
donc de m’adresser à vous, sous les ordres duquel j’ai été directement placé.\\
\indent Vous savez, Monsieur, que mes occupations du Musée\gls{CoEg_org_00000002} me laissent\\
chaque jour, en dehors d’elles-mêmes, quelques heures de liberté que je puis\\
utiliser à mon profit. Vous savez encore combien, père de famille\footnote{La famille Mariette est alors composée de son épouse Éléonore (née Millon, 1827-1865)\gls{CoEg_pers_00000005} et leurs filles Marguerite Louise\gls{CoEg_pers_00000006} (1846-1861), Joséphine Cornélie\gls{CoEg_pers_00000007} (1847-1873), Sophie Éléonore\gls{CoEg_pers_00000008} (1849-1885).}, il est\\
nécessaire que j’use de ces quelques heures pour augmenter un peu mes\\
ressources qui sont malheureusement si bornées. Je viens donc vous prier de\\
vouloir bien m’autoriser ou me faire autoriser à \textit{mettre en ordre à mes\\
heures perdues, à coller, à cataloguer quelques-uns} des papyrus égyptiens de\\
la collection du Louvre\gls{CoEg_org_00000002}, aux conditions que l’Administration a faites à\\
\textit{\gls{CoEg_abbr_00000001} Nisard\gls{CoEg_pers_00000056}} qui achève en ce moment son travail. – Je vous répète que,\\
vu les circonstances particulières dans lesquelles je me trouve en ce moment,\\
vous me rendrez un service signalé en m’accordant l’objet de la présente\\
demande.\\

\par J’ai l’honneur d’être,
\begin{center} Monsieur,\end{center}
\begin{center} \hspace{5cm}Votre très-humble serviteur\\
\hspace{5cm} \gls{CoEg_abbr_00000002} Mariette\\
\hspace{5cm} employé des Antiques et sculptures\gls{CoEg_org_00000025} du Louvre\gls{CoEg_org_00000002} \end{center}

\hypertarget{CoEg_Mariette_1850-05-20}{}

\section*{Le 20 mai 1850, de Paris, à Esquirou de Parieu, ministre de l’Instruction publique}
\addcontentsline{toc}{section}{Le 20 mai 1850, de Paris, à Esquirou de Parieu, ministre de l’Instruction publique} \label{labCoEg_Mariette_1850-05-20}
{\footnotesize
\noindent Institution et lieu de conservation~: Archives nationales, Pierrefitte-sur-Seine.\\
Cote : \hyperlink{CoEg_Mariette_ms_002}{F/17/2988/1, dossier « Mariette »} (n. p.).\\
Support : une feuille double de grand format.\\
Thème~: \gls{CoEg_keyword_00000002}.\\
Notes~: La lettre porte un tampon à l’encre noire au coin supérieur gauche : « Ministère de l’Instruction publique et des Cultes\gls{CoEg_org_00000042}. Enregistré le 23 mai 1850 » ; un tampon à l’encre rouge au coin supérieur gauche : « [... ?] enregistrement. 23 mai 1850 » complété à la main par l’annotation : « n\textsuperscript{o} 2067.[... ?] » ; au coin supérieur gauche l’annotation : « [3. 2 L ?] » ; au coin supérieure gauche l’annotation : « consulter l’Institut\gls{CoEg_org_00000004} »}

{\footnotesize \begin{center} {[1\textsuperscript{re} page, r\textsuperscript{o}]}\end{center}}
\begin{flushright}Paris\gls{CoEg_place_00000002}, le 20 Mai 1850\end{flushright}
\indent A Monsieur\\
\indent Monsieur le Ministre de l’Instruction Publique.\\

\indent Monsieur le Ministre\gls{CoEg_pers_00000003},\\

\indent L’Egypte\gls{CoEg_place_00000003} a été depuis quelques temps explorée par tant de voyageurs que\\
j’hésiterais certainement à vous faire la demande d’une allocation destinée à\\
me fournir les moyens d’y entreprendre de nouvelles recherches, si des circonstances\\
particulières, que vous voudrez bien me permettre de développer, ne donnaient\\
à ces recherche un caractère d’urgence incontestable.\\
\indent Il existe en Egypte\gls{CoEg_place_00000003} un nombre assez considérable de couvents coptes\\
qui possèdent des bibliothèques composées de manuscrits syriaques, coptes, arabes\\
et éthiopiens. Dès le XVII\textsuperscript{e} siècle, ces bibliothèques ont fixé l’attention des\\
érudits, et divers efforts, suivis presque toujours de résultats satisfaisants, furent\\
tentés pour en distraire quelques parties au profit des collections de l’Europe\gls{CoEg_place_00000013}.\\
La bibliothèque\gls{CoEg_org_00000033} du Vatican\gls{CoEg_place_00000075} doit ses plus beaux manuscrits coptes et syriaques\\
aux deux Assemani\footnote{Giuseppe Simone Assemani\gls{CoEg_pers_00000126} (1687-1768) et Stefano Evodio Assemani\gls{CoEg_pers_00000127} (1711-1782).} que le pape Clément XI\gls{CoEg_pers_00000128} avait chargés de visiter les\\
monastères de l’Egypte\gls{CoEg_place_00000003}. La bibliothèque Bodleïenne\gls{CoEg_org_00000034} a de même formé son\\
noyau principal des achats qu’Hustington\gls{CoEg_pers_00000129} et autres avaient opérés dans ces\\
même monastères, et les voyages de Vansleb\gls{CoEg_pers_00000130} ont procuré à la bibliothèque\\
Nationale\gls{CoEg_org_00000026} de Paris\gls{CoEg_place_00000002} ceux de ses manuscrits coptes qui passent encore\\
aujourd’hui pour les plus remarquables. Je n’entrerai pas, Monsieur le\\
Ministre, dans plus de détails sur ce sujet qu’a déjà traité, avec tous\\
les développements possibles, un honorable et docte académicien, \gls{CoEg_abbr_00000001} Etienne\\
Quatremère\gls{CoEg_pers_00000131}.\\
\indent Mais depuis que la découverte de Champollion\gls{CoEg_pers_00000094} a rendu plus nécessaire\\
et plus générale l’étude du copte, depuis que les langues orientales sont\\
entrées pour une plus grande part dans les préoccupations de l’Europe\gls{CoEg_place_00000013}\\
savante, les visites aux couvents de l’Egypte\gls{CoEg_place_00000003} se sont multipliées, et les
{\footnotesize \begin{center} {[1\textsuperscript{re} page, v\textsuperscript{o}]}\end{center}}
\noindent achats sont aussi devenus plus fréquents. Quatre monastères ont surtout\\
été d’une libéralité sans limites envers les voyageurs. Ce sont ceux de la\\
Vallée des Lacs de Natron\gls{CoEg_place_00000045}. L’un d’eux a fourni à Lord Prudhoe\gls{CoEg_pers_00000132} les vingt\\
manuscrits dont il a fait présent au Musée Britannique\gls{CoEg_org_00000005}. Un autre a cédé\\
à \gls{CoEg_abbr_00000001} Tischendorf\gls{CoEg_pers_00000133}, savant allemand très-connu par sa découverte dans\\
l’Asie Occidentale d’un manuscrit\gls{CoEg_obj_imn} rival du \textit{codex Alexandrinus}\gls{CoEg_obj_00000024} de Londres\gls{CoEg_place_00000011},\\
quatorze volumes en langue copte qu’à son retour en Allemagne il a\\
offerts à \gls{CoEg_abbr_00000047} le Roi\gls{CoEg_pers_00000134} de Saxe\gls{CoEg_place_00000046}. Ce même couvent, celui des Syriens\gls{CoEg_org_00000035}, a en\\
outre procuré à \gls{CoEg_abbr_00000001} Henry Tattam\gls{CoEg_pers_00000135}, de Bedford\gls{CoEg_place_00000047}, \textit{cent-vingt-cinq} manuscrits,\\
la plupart coptes, au milieu desquels s’est rencontré la fameuse Théophanie\gls{CoEg_bibl_00000007}\\
d’Eusèbe\gls{CoEg_pers_00000114}, totalement inconnue jusqu’ici. Enfin le Musée Britannique\gls{CoEg_org_00000005}\\
vient tout récemment d’y recueillir une collection unique, inestimable, de\\
manuscrits syriaques, collection sur laquelle on a eu à peine le temps de\\
jeter les yeux et qu’on a déjà pu diviser en \textit{trois-cent-soixante-dix} gros\\
volumes, contenant ensemble plus de \textit{mille} ouvrages de langues, d’histoire\\
ecclésiastique ou de liturgie, et plus de \textit{trente} versions syriaques faites sur\\
les originaux grecs de certains auteurs, sacrés et profanes, dont on croyait les\\
œuvres perdues sans retour. – Les seuls couvents de la Vallée des Lacs de\\
Natron\gls{CoEg_place_00000045} ont déjà distribué à l’Europe\gls{CoEg_place_00000013}, dans les vingt dernières années,\\
plus de cinq-cent-trente manuscrits, et je ne compte pas dans ce nombre\\
ceux de la collection encore inconnue dont la Prusse\gls{CoEg_org_00000043} s’est enrichie à la\\
suite de l’expédition du Docteur Lepsius\gls{CoEg_pers_00000061}.\\
\indent Or, Monsieur le Ministre, il est douloureux d’avoir à dire que\\
rien, dans ce partage, n’est échu à la France\gls{CoEg_org_00000012} ; que pas un Français\\
ne s’est encore donné la mission spéciale de visiter, avec les connaissances\\
suffisantes, les monastères de l’Egypte\gls{CoEg_place_00000003}, dans le but de consacrer à la\\
Bibliothèque Nationale\gls{CoEg_org_00000026} de Paris\gls{CoEg_place_00000002} le premier rang que les nations étrangères\\
ne doivent jamais lui enlever.\\
\indent Ces faits, Monsieur le Ministre, justifient la demande que j’ai\\
l’honneur de vous faire. Ils doivent vous prouver qu’une visite faite dans\\
un intérêt scientifique, non pas seulement aux couvents de la Vallée des\\
Lacs de Natron\gls{CoEg_place_00000045}, mais encore à tous les couvents de l’Egypte\gls{CoEg_place_00000003} et surtout\\
de la Thébaïde\gls{CoEg_place_00000049}, peut n’être pas sans résultat. Il ne m’appartient certes\\
pas de vous entretenir des besoins de la division des manuscrits coptes\\
et syriaques de la Bibliothèque Nationale\gls{CoEg_org_00000026} ; mais il me semble que
{\footnotesize \begin{center} {[2\textsuperscript{e} page, r\textsuperscript{o}]}\end{center}}
\noindent quelque riche que puisse être déjà cet établissement, il n’en verrait pas moins\\
avec satisfaction son fonds s’augmenter de manuscrits dont le British\\
Museum\gls{CoEg_org_00000005} a déjà une trop grande part.\\
\indent J’espère donc que vous voudrez bien m’aider à poursuivre le dessein que\\
j’ai formé de doter la Bibliothèque Nationale\gls{CoEg_org_00000026} d’une collection, aussi nombreuse\\
et aussi choisie que possible, de manuscrits orientaux. J’espère aussi que\\
vous me permettrez d’appuyer sur l’urgence du projet que j’ai l’honneur\\
de vous soumettre, car je crois savoir par une communication bienveillante\\
de \gls{CoEg_abbr_00000001} Tischendrof\gls{CoEg_pers_00000133} que le Musée Britannique\gls{CoEg_org_00000005} prépare de nouvelles\\
négociations et qu’il n’a pas perdu l’espérance de se rendre propriétaire\\
de la presque totalité des manuscrits qui restent encore aux Religieux\\
de la Vallée des Lacs de Natron\gls{CoEg_place_00000045}.\\
\indent La mission que j’ai l’honneur de solliciter pourrait d’ailleurs ne pas\\
se borner à la visite des bibliothèques chrétiennes de l’Egypte\gls{CoEg_place_00000003}. Chemin\\
faisant, je me proposerais, si vous le permettez, de répondre à bien des\\
\gls{CoEg_entry_00000031} de la science des hiéroglyphes. Je désirerais surtout \sout{dési} diriger\\
quelques recherches vers un point que les voyageurs ont jusqu’ici peu\\
exploré parce qu’il est placé à quelque distance du Nil\gls{CoEg_place_00000021}, au milieu du\\
désert ; je veux parler de l’emplacement de l’ancienne ville d’Abydos\gls{CoEg_place_00000026}.\\
Aucune recherche n’y a encore été faite sur une base véritablement\\
scientifique. L’expédition de Champollion\gls{CoEg_pers_00000094} n’y a même pas été, et\\
\gls{CoEg_abbr_00000001} Lepsius\gls{CoEg_pers_00000061} n’a pris le temps que d’y relever quelques plans. \gls{CoEg_abbr_00000048}\\
Mimaut\gls{CoEg_pers_00000136} et Drovetti\gls{CoEg_pers_00000137}, les seuls qui y aient opéré des fouilles en\\
règle, n’ont pas assisté en personne aux opérations qui ont été\\
conduites au hazard par des Arabes ignorants. Abydos\gls{CoEg_place_00000026} est pourtant,\\
avec Memphis\gls{CoEg_place_00000005}, la plus ancienne capitale de l’Egypte\gls{CoEg_place_00000003}. Les plus belles\\
stèles que le Louvre\gls{CoEg_org_00000001} possède viennent d’Abydos\gls{CoEg_place_00000026}. Au rapport de tous\\
les voyageurs et en particulier de Wilkinson\gls{CoEg_pers_00000138}, des monuments portant,\\
presque tous, les noms des souverains des anciennes dynasties se montrent\\
encore partout à fleur du sol. Enfin \gls{CoEg_abbr_00000001} Ch. Lenormant\gls{CoEg_pers_00000139}, le seul\\
des compagnons de voyage de Champollion\gls{CoEg_pers_00000094} qui ait vu Abydos\gls{CoEg_place_00000026}, y a\\
rencontré les ruines, sans doute recouvertes aujourd’hui par les sables,\\
d’un temple dédié par un des Sebekôtep de la XIII\textsuperscript{\underline{e}} dynastie. Or la\\
recherche de ces ruines vaut à elle seule un voyage en Egypte\gls{CoEg_place_00000003}. Au moment\\
où un système devenu populaire en Angleterre\gls{CoEg_place_00000052} et en Allemagne\gls{CoEg_place_00000070},\\
\sout{de} celui du savant \gls{CoEg_abbr_00000001} Bunsen\gls{CoEg_pers_00000140}, ministre de Prusse\gls{CoEg_org_00000043} à Londres\gls{CoEg_place_00000011},
{\footnotesize \begin{center} {[2\textsuperscript{e} page, v\textsuperscript{o}]}\end{center}}
\noindent change les bases de la chronologie égyptienne et fait la XIII\textsuperscript{\underline{e}} dynastie\\
contemporaine des \gls{CoEg_entry_00000022}, il est essentiel de savoir lequel des Sebekhôtep\\
connus a eu le loisir de construire un temple à Abydos\gls{CoEg_place_00000026}. Peut-être\\
même pourrait-on découvrir si les conquérants auxquels on croit devoir\\
la destruction de tous les édifices antérieurs à la XVIII\textsuperscript{\underline{e}} dynastie ont pénétré\\
jusqu’à cette ville, et vérifier le récit de Manéthon\gls{CoEg_pers_00000116} sur leurs dévastations.\\
Il y a donc, sous bien des rapports, [rature] une ample moisson à recueillir\\
au milieu des ruines d’Abydos\gls{CoEg_place_00000026}. Si vous le jugez convenable, Monsieur\\
le Ministre, j’entreprendrai cette tâche dont les résultats profiteront\\
à la collection Egyptienne du Louvre\gls{CoEg_org_00000001} qui, heureusement, est encore la\\
première de l’Europe\gls{CoEg_place_00000013}, malgré les acquisitions multipliées du British\\
Museum\gls{CoEg_org_00000005} et les agrandissements récens [\textit{sic}] dont l’expédition de \gls{CoEg_abbr_00000001} Lepsius\gls{CoEg_pers_00000061}\\
en Egypte\gls{CoEg_place_00000003} a doté le Musée\gls{CoEg_org_00000040} de Berlin\gls{CoEg_place_00000051}.\\
\indent En résumé, Monsieur le Ministre, j’ai l’honneur de solliciter de\\
vous une subvention de six mille francs, en échange de laquelle je\\
m’engage à faire tous les efforts dont je suis capable pour fournir aux\\
deux grands établissements scientifiques que j’ai nommés une collection\\
de manuscrits et de monuments qui, choisis au point de vue des\\
besoins de ces établissements, représentera pour chacun d’eux une\\
somme bien plus considérable que celle que vous aurez cru pouvoir\\
me confier.\\
\indent J’ai l’honneur d’être avec le plus profond respect,\\
\begin{center}Monsieur le Ministre,\end{center}
\begin{center}\hspace{5cm}Votre très-humble\\
\hspace{5cm}et très-obéissant serviteur.\\
\hspace{5cm}\gls{CoEg_abbr_00000002} Mariette\\
\hspace{5cm}attaché au catalogue des Antiquités\\
\hspace{5cm}Egyptiennes du Musée du Louvre\gls{CoEg_org_00000002}.\end{center}

\hypertarget{CoEg_Mariette_1850-07-06}{}

\section*{Le 6 juillet 1850, de Paris, à Esquirou de Parieu, ministre de l’Instruction publique}
\addcontentsline{toc}{section}{Le 6 juillet 1850, de Paris, à Esquirou de Parieu, ministre de l’Instruction publique} \label{labCoEg_Mariette_1850-07-06}
{\footnotesize
\noindent Institution et lieu de conservation~: Archives nationales, Pierrefitte-sur-Seine.\\
Cote : \hyperlink{CoEg_Mariette_ms_002}{F/17/2988/1, dossier « Mariette »} (n. p.).\\
Support : une feuille double de moyen format.\\
Thème~: \gls{CoEg_keyword_00000002}~; \gls{CoEg_keyword_00000014}.\\
Notes~: La lettre porte en partie supérieure gauche les annotations à l’encre  : « [3. L. ?] » et « Classer ».}

{\footnotesize \begin{center} {[1\textsuperscript{re} page, r\textsuperscript{o}]}\end{center}}
\begin{flushright}Paris\gls{CoEg_place_00000002}, le 6 Juillet 1850.\end{flushright}

\indent A Monsieur\\
\indent \hspace{1cm} Monsieur le Ministre de l’Instruction Publique.\\

\hspace{1cm} Monsieur le Ministre\gls{CoEg_pers_00000003},\\

\indent J’ai pris connaissance du rapport que l’Académie des Inscriptions et\\
Belles Lettres\gls{CoEg_org_00000036} vous a adressé sur un projet de mission scientifique que\\
j’ai eu l’honneur de vous soumettre\footnote{L'Académie avait été saisie par le ministère le 5 juin 1850 ; le 21 juin, une commission composée d’Ampère, Jomard\gls{CoEg_pers_00000141}, Lenormant\gls{CoEg_pers_00000139} et Quatremère\gls{CoEg_pers_00000131} se réunit et appuya favorablement la demande («~Nous devons croire que M. Mariette s’est bien préparé à cette mission, qu’il en a envisagé d’avance les ennuis, les lenteurs et les incertitudes~: qu’il sait l’impossibilité de réussir sans la connaissance pratique de la langue arabe, et sans une résolution ferme, soutenue par un bon tempérament et des habitudes de sobriété et de régularité, d’accepter les mœurs du pays, et d’endurer les privations auxquelles se soumettent les habitants des monastères de l’Egypte.~») dans un rapport qui parvint au ministère le 25 juin (F/17/2988/1, dossier «~Mariette~»).}.\\
\indent Une allocation de six mille francs me paraissait alors suffisante\\
pour l’exécution de ce projet, tel que je l’avais conçu et tel que\\
je l’ai développé dans ma demande.\\
\indent Mais le rapport de l’Académie\gls{CoEg_org_00000036}, en élargissant le cercle des obligations\\
qui me seraient imposées, a en même temps, par une conséquence toute\\
naturelle, appuyé sur la nécessité d’élever à un chiffre supérieur\\
l’allocation que je sollicite. Il s’agit en effet maintenant d’un\\
voyage par toute l’Egypte\gls{CoEg_place_00000003} ; – il s’agit, non plus seulement d’une\\
visite à ceux des couvents de cette contrée qui sont connus pour\\
posséder des manuscrits, mais d’une visite à tous les couvents indistinc-\\
-tement, à toutes les églises, de manière à répondre à l’un des\\
\gls{CoEg_entry_00000031} les plus urgens [\textit{sic}] de la science moderne et formant une\\
\textit{Geographia Sacra} de l’Egypte\gls{CoEg_place_00000003}, œuvre que personne jusqu’à nos jours n’a\\
tentée. Le champ de recherches à faire serait donc considérablement\\
{\footnotesize \begin{center} {[1\textsuperscript{re} page, v\textsuperscript{o}]}\end{center}}
\noindent agrandi si vous adoptez le vœu manifesté par le rapport de l’Institut\gls{CoEg_org_00000004} ;\\
mais en même temps les dépenses seraient plus fortes.\\
\indent Je crois donc, Monsieur le Ministre, ne pas vous surprendre\\
en vous demandant une augmentation sur laquelle le rapport\\
lui-même appuie, et en fixant à huit mille francs le chiffre\\
de la subvention que je vous prie de m’accorder.\\
\indent Il est bien entendu que les conditions premières de la\\
mission, c’est-à-dire l’achat de manuscrits orientaux sur\\
les fonds que vous mettrez à ma disposition, subsistent en leur\\
entier. Quant aux monuments hiéroglyphiques et aux\\
fouilles à entreprendre dans le but d’enrichir le Musée du\\
Louvre\gls{CoEg_org_00000002} de quelques-uns de ces monuments, j’avoue que je ne\\
serais pas fâché d’en être débarrassé. Ces fouilles doivent être très-\\
-coûteuses et absorber en conséquence une bonne partie de mes\\
fonds. De ces deux missions, l’une nuirait ainsi nécessaire-\\
-ment à l’autre, et dans la crainte de les voir échouer toutes\\
deux, j’aime mieux vous demander, en toute franchise, de\\
borner les instructions que vous voudrez bien me donner, à\sout{ux}\\
à la recherche des seuls faits qui intéressent l’Egypte\gls{CoEg_place_00000003} chrétienne.\\
\indent Un point a également été laissé dans le doute par\\
le rapport de l’Académie\gls{CoEg_org_00000036}, qui ne parle pas de la durée du\\
voyage que je compte entreprendre. Je pense être huit mois\\
absent.
{\footnotesize \begin{center} {[2\textsuperscript{e} page, r\textsuperscript{o}]}\end{center}}
\indent Soyez d’ailleurs persuadé, Monsieur le Ministre, que si vous\\
me faites l’honneur de ne pas repousser la demande que je vous\\
ai soumise, je me ferai un devoir de répondre à vos intentions\\
avec tout le zèle, toute la bonne foi, toute la conscience\\
que vous y mettriez vous-même. Vous avez pour garantie\\
mon amour réel de la science et le désir qui m’anime\\
de me faire, si le succès ne trompe pas mes efforts, une\\
carrière et un nom dans l’archéologie égyptienne.\\
\indent J’ai l’honneur d’être,
\begin{center}avec le plus profond respect,\end{center}
\begin{center}\hspace{5cm}Monsieur le Ministre,\\
\hspace{5cm}Votre très-humble\\
\hspace{5cm}et très-obéissant serviteur.\\
\hspace{5cm}\gls{CoEg_abbr_00000002} Mariette\end{center}

\hypertarget{CoEg_Mariette_1850-07-08}{}
\section*{Le 8 juillet 1850, de Paris, à Nieuwerkerke, directeur général des musées nationaux}
\addcontentsline{toc}{section}{Le 8 juillet 1850, de Paris, à Nieuwerkerke, directeur général des musées nationaux}
{\footnotesize
\noindent Institution et lieu de conservation~: Archives nationales, Pierrefitte-sur-Seine.\\
Cote : \hyperlink{CoEg_Mariette_ms_001}{20150497/118, dossier 145 «~Mariette, Auguste~»} (n. p.).\\
Support : une feuille double de moyen format.\\
Thèmes~: \gls{CoEg_keyword_00000005}~; \gls{CoEg_keyword_00000002}.\\
Note~: la lettre porte, d’une autre main que celle de Mariette, à l'encre et au coin supérieur gauche~: «~Accordé et l’en/prévenir officiellement/V.~» et «~Répondu [11/13 ?] Juillet /S. 1495~»~; au crayon et au coin supérieur droit~: «~Mus Egypt.\gls{CoEg_org_00000020}~»~; elle a été tamponnée à l’encre rouge «~Ministère de l’Intérieur\gls{CoEg_org_00000009}/Musées nationaux\gls{CoEg_org_00000001}/11 juillet 1850~».
\begin{center} {[1\textsuperscript{re} page, r\textsuperscript{o}]}\end{center}}
\begin{flushright}
Paris\gls{CoEg_place_00000002}, le 8 Juillet 1850.
\end{flushright}
A Monsieur
\begin{center} Monsieur le Directeur-Général des Musées Nationaux\gls{CoEg_org_00000001}.\end{center}

\hspace{1cm} Monsieur le Directeur\gls{CoEg_pers_00000002},\\

\par Je désirerais, dans l’intérêt de mes études, pouvoir disposer de\\
six mois que je compte employer à un voyage en Egypte\gls{CoEg_place_00000003}.\\
\indent En vous demandant de vouloir bien m’accorder, pour ce même\\
espace de temps, un congé qui partirait du premier septembre\\
prochain, j’ai la confiance que vous ne vous refuserez pas à\\
me rendre un service important que je regarderai comme une\\
nouvelle preuve de la protection dont vous voulez bien honorer\\
mes travaux.\\

\par J’ai l’honneur d’être,
\begin{center} Monsieur le Directeur,\end{center}
\begin{center} \hspace{5cm}Votre très-humble\\
\hspace{5cm}et très-obéissant serviteur.\\
\hspace{5cm} \gls{CoEg_abbr_00000002} Mariette\\
 \end{center}

\hypertarget{CoEg_Mariette_1850-08-27}{}

\section*{Le 27 août 1850, de Paris, à Esquirou de Parieu, ministre de l’Instruction publique}
\addcontentsline{toc}{section}{Le 27 août 1850, de Paris, à Esquirou de Parieu, ministre de l’Instruction publique} \label{labCoEg_Mariette_1850-08-27}
{\footnotesize
\noindent Institution et lieu de conservation~: Archives nationales, Pierrefitte-sur-Seine.\\
Cote : \hyperlink{CoEg_Mariette_ms_002}{F/17/2988/1, dossier « Mariette »} (n. p.).\\
Support : une feuille simple de grand format.\\
Thème~: \gls{CoEg_keyword_00000002}.\\
Notes~: La lettre porte en partie supérieure gauche les annotations à l’encre  : « [3. L. ?] » et « Classer ».}
 
\begin{flushright}Paris\gls{CoEg_place_00000002}, le 27 Août 1850.\end{flushright}

\noindent A Monsieur\\
\indent \hspace{1cm} Monsieur le Ministre de l’Instruction Publique\\

\hspace{1cm} Monsieur le Ministre,\\

\indent J’apprends par \gls{CoEg_abbr_00000001} Jomard\gls{CoEg_pers_00000141}, membre de l’Académie des Inscriptions\gls{CoEg_org_00000036}\\
et conservateur à la Bibliothèque Nationale\gls{CoEg_org_00000026}, que la plupart des voyageurs\\
qui ont été chargés avant moi de missions scientifiques en Egypte\gls{CoEg_place_00000003}, ont\\
obtenu la cession gratuite des deux parties suivantes de la \textit{Description\gls{CoEg_bibl_00000009}\\
de l’Egypte}\gls{CoEg_place_00000003}, ouvrage dont des exemplaires sont conservés à la Bibliothèque\gls{CoEg_org_00000026},\\
en assez grand nombre, dans le Département de \gls{CoEg_abbr_00000001} Jomard\gls{CoEg_pers_00000141} lui-même :\\
\indent \gls{CoEg_abbr_00000005} La grande \textit{Carte Géographique de l’Egypte}\gls{CoEg_place_00000003}, de 53 feuilles ;\\
\indent \gls{CoEg_abbr_00000006} les vingt-six volumes \gls{CoEg_entry_00000033} du texte de la \textit{Description\gls{CoEg_bibl_00000009}\\
de l’Egypte}\gls{CoEg_place_00000003}.\\
\indent Vous concevez, Monsieur le Ministre, l’empressement que je\\
mets à vous prier de vouloir bien mettre ces deux ouvrages à ma\\
disposition, quand je vous aurai dit qu’ils seront pour moi un\\
\textit{\gls{CoEg_entry_00000034}} indispensable, et que, d’un autre côté, toutes les\\
recherches que j’ai faites jusqu’ici pour me les procurer, à quelque\\
prix que ce soit, ont été infructueuses.\\
\indent J’ai l’honneur d’être,\\
\indent\hspace{5cm} Avec le plus profond respect,
\begin{center}Monsieur le Ministre,\end{center}
\begin{center}\hspace{5cm}Votre très-humble serviteur\\
\hspace{5cm}\gls{CoEg_abbr_00000002} Mariette\end{center}

\hypertarget{CoEg_Mariette_1851-02-28}{}
\section*{Le 28 février 1851, de Saqqarah, à Nieuwerkerke, directeur général des musées nationaux}
\addcontentsline{toc}{section}{Le 28 février 1851, de Saqqarah, à Nieuwerkerke, directeur général des musées nationaux}
{\footnotesize
\noindent Institution et lieu de conservation~: Archives nationales, Pierrefitte-sur-Seine.\\
Cote~: \hyperlink{CoEg_Mariette_ms_001}{20150497/118, dossier 145 «~Mariette, Auguste~»} (n. p.).\\
Support~: une feuille double.\\
Thèmes~: \gls{CoEg_keyword_00000005}~;  \gls{CoEg_keyword_00000006}~;  \gls{CoEg_keyword_00000002}~;  \gls{CoEg_keyword_00000007}.\\
Note~: la première page porte, au coin supérieur gauche et au crayon, d’une autre main que celle de Mariette et de lecture très incertaine~: «~[Donnée par/M Maspero~?]~».}
\begin{flushright}
Saqqarah\gls{CoEg_place_00000001}, le 28 février 1851.
\end{flushright}
A Monsieur
\begin{center} Monsieur le Directeur des Musées Nationaux\gls{CoEg_org_00000001}\end{center}
\begin{flushright}à Paris\gls{CoEg_place_00000002}.\end{flushright}

\hspace{1cm} Monsieur le Directeur\gls{CoEg_pers_00000002},\\

\par Au mois d’Août de l’année passée, vous avez bien voulu m’accorder\\
un congé de six mois.\\
\indent L’espoir que la mission qui m’a été confiée par \gls{CoEg_abbr_00000001} le Ministre de
\\l’Instruction Publique\gls{CoEg_pers_00000003} et \gls{CoEg_abbr_00000001} le Ministre de l’Intérieur\gls{CoEg_pers_00000004} aurait pour\\
résultat l’accroissement des Antiquités Egyptiennes du Louvre\gls{CoEg_org_00000002}, vous a\\
décidé à me faire une faveur dont je vous suis reconnaissant.\\
\indent Mais ce congé expire le 31 mars prochain, et à cette époque je\\
serai encore en Egypte\gls{CoEg_place_00000003} pour deux mois au moins.\\
\indent Vous me rendriez donc un nouveau service, Monsieur le Directeur,\\
si vous vouliez prolonger la permission d’absence que vous m’avez donnée\\
jusqu’à la fin du mois de mai, c’est-à-dire pendant deux nouveaux\\
mois.\\
\indent Je vous demanderai aussi de m’accorder pour le même temps mes\\
appointements ordinaires. S’il m’était permis de faire intervenir dans\\
cette affaire des questions toutes personnelles, je vous rappellerais\\
que je ne suis pas riche, et qu’en mon absence les deux mois\\
d’appointements que je sollicite de vous sont le seul moyen que\\
j’aie de subvenir aux besoins de ma famille que j’ai laissée à Paris\gls{CoEg_place_00000002}.\\
\indent J’attends donc de votre justice et de l’intérêt si vif que vous\\
m’avez souvent témoigné le double service que j’ai l’honneur de\\
solliciter de vous.
\begin{flushright}Je vous dirai\end{flushright}
{\footnotesize \begin{center} {[1\textsuperscript{re} page, v\textsuperscript{o}]}\end{center}}
\indent Je vous dirai d’ailleurs que si, contre toutes mes prévisions, je reste\\
en Egypte\gls{CoEg_place_00000003} plus long-temps [\textit{sic}] que je ne le pensais, chaque jour de\\
retard apporte au Louvre\gls{CoEg_org_00000002} un monument nouveau. Le hasard\\
m’a en effet réservé une des plus curieuses découvertes de l’archéologie\\
Egyptienne. Quatre mois me séparent déjà du premier jour où je\\
tentai mes premiers essais pour retrouver le Sérapéum\gls{CoEg_place_00000004} de Memphis\gls{CoEg_place_00000005},\\
et les deux autres mois que je vous prie de m’accorder ne me mèneront\\
tout au plus qu’à la moitié des travaux qu’il faudrait faire pour\\
épuiser la mine si riche en monuments de toute espèce que j’ai\\
trouvée.\\
\indent Pour vous en convaincre, Monsieur le Directeur, je vous dirai que,\\
\textit{dès maintenant}, je tiens à votre disposition \textit{comme monuments\\
principaux} :\\
\indent 1-160 = De 150 à 160 sphinx en grès, de la grandeur de ceux de\\
Néphéritès\gls{CoEg_pers_00000009} au Louvre\gls{CoEg_org_00000002}\footnote{Le musée du Louvre conserve deux sphinx tardifs dont l'un (A 26\gls{CoEg_obj_00000001}) est inscrit au nom de Néphéritès I\textsuperscript{er}\gls{CoEg_pers_00000009}.}~; j’en emporterai le nombre que vous voudrez bien\\
m’indiquer, et, en attendant, j’en ai choisi six\gls{CoEg_obj_00000002} qui vont bientôt\\
partir pour Alexandrie\gls{CoEg_place_00000006}~;\\
\indent 161 = un sphinx\gls{CoEg_obj_imn} plus grand avec les légendes d’Amyrtée\gls{CoEg_pers_00000010}~; ce roi\\
n’est pas, je crois, représenté au Louvre\gls{CoEg_org_00000002}~;\\
\indent 162-163 = deux très-beaux bas-reliefs\gls{CoEg_obj_imn} représentant Amyrtée\gls{CoEg_pers_00000010}\\
en adorateur devant Apis\gls{CoEg_pers_00000011}~;\\
\indent 164 = une base\gls{CoEg_obj_imn} en grès, commune à deux statues en basalte, avec\\
dix-neuf lignes en démotique~;\\
\indent 165 = une statue\footnote{Louvre N 347\gls{CoEg_obj_00000019} (il s'agit du dieu Bès).} de grandeur naturelle du Dieu Typhon\gls{CoEg_pers_00000012}~;\\
\indent 166 à 176 = onze statues\gls{CoEg_obj_imn} \textit{grecques} plus ou moins mutilées~; l’une\\
d’elles, d’une conservation assez remarquable, représente un personnage\\
assis, et portant sur l’épaule gauche ce qu’il m’est impossible
{\footnotesize \begin{center} {[2\textsuperscript{e} page, r\textsuperscript{o}]}\end{center}}
\noindent de ne pas prendre pour une colonne vertébrale humaine~;\\
\indent 177 = un groupe\gls{CoEg_obj_imn} colossal de style grec représentant un\\
jeune homme à cheval sur un \textit{monstre} à tête humaine, à corps de\\
chien, à pattes de lion et à griffes d’aigle~;\\
\indent 178 = 179 = deux groupes\gls{CoEg_obj_imn} représentant, chacun, un enfant\\
à cheval sur un \textit{paon}~; la queue de l’animal, développée\\
derrière lui, forme une roue qui a plus de six pieds de diamètre~;\\
\indent 180 = une stèle\gls{CoEg_obj_00000003}, trouvée encore en place à l’entrée du Sérapéum\gls{CoEg_place_00000004},\\
et représentant Nectanébo\gls{CoEg_pers_00000013} en adoration devant neuf divinités en\\
tête desquelles figure la triade thébaine~;\\
\indent 181-182 = deux \textit{magnifiques lions}\footnote{Le Louvre obtint finalement trois de ces lions, conservés sous les numéros d'inventaire N 432 A\gls{CoEg_obj_00000004} (sous lequel était encastré la stèle C 318\gls{CoEg_obj_00000003}), N 432 A\gls{CoEg_obj_00000005} et N 432 A\gls{CoEg_obj_00000006}.}, d’une conservation admirable,\\
qui sont la reproduction \textit{très-exacte} de ceux du Vatican\gls{CoEg_org_00000003} dont des\\
moulages de bronze servent de fontaines devant le Palais de l’Institut\gls{CoEg_org_00000004}\\
à Paris\gls{CoEg_place_00000002}~;\\
\indent 183 = un sarcophage\gls{CoEg_obj_imn} rectangulaire que j’ai rencontré par hasard\\
dans mes fouilles~; il reproduit à l’extérieur l’ornementation\\
du cercueil de la 3\textsuperscript{e} pyramide de Gyzeh\gls{CoEg_place_00000007}, et offre cet intérêt particulier\\
qu’il n’a jamais été achevé~; d’un côté les sculptures sont parfaites,\\
de l’autre elles ne sont qu’ébauchées à grands traits~; quelques\\
figures sont simplement dessinées à l’ocre rouge~; la plupart\\
des légendes sont aussi en [rature] ocre rouge~; on y remarque des\\
corrections, des additions tracées en surcharge avec de l’encre noire.\\
\indent Ces monuments, Monsieur le Directeur, ne sont que les\\
principaux de ceux que j’ai trouvés. Je vous les cite parce que je\\
les ai tous vus et dessinés. D’un autre côté mes fouilles ne sont\\
pas encore à leur première moitié, puisque je suis à peine entré\\
dans le Sérapéum\gls{CoEg_place_00000004}. Il y a une huitaine de jours, des fouilles
\begin{flushright}partielles\end{flushright}
{\footnotesize \begin{center} [2\textsuperscript{e} page, v\textsuperscript{o}]\end{center}}
\noindent partielles m’ont révélé la place de huit autres groupes\gls{CoEg_obj_imn} de style\\
grec (l’un d’entre eux représente un enfant à cheval sur un coq),\\
et de onze stèles\gls{CoEg_obj_imn} en place, dont trois, m’ont assuré mes Arabes,\\
sont en basalte. Je n’ai pas introduit ces monuments dans la\\
liste qui précède, parce que je n’ai pas pu les bien voir. Un\\
accident trop fréquent dans les sables du désert de Saqqarah\gls{CoEg_place_00000001} a\\
en effet bouleversé tout le Sérapéum\gls{CoEg_place_00000004}~; pendant trois jours le\\
\Gls{CoEg_entry_00000001} a soufflé avec une telle violence que toutes mes excavations\\
ont été bouchées, mes tentes enlevées dans les airs, et que depuis\\
cinq jours, je n’ai pu encore réparer les désastres de cette tempête.\\
\indent Mais quoi qu’il en soit, ce que j’ai déjà et dont je vous ai donné\\
une liste très-sommaire, vous fait assez voir qu’en vous demandant\\
de m’accorder mes appointements pendant deux nouveaux mois,\\
je vous offre en retour des compensations plus que suffisantes.\\
\indent Permettez-moi donc d’espérer, Monsieur le Directeur, que vous\\
ne vous refuserez pas à faciliter, autant que vous le pouvez,\\
des recherches que je poursuis moi-même avec toute la persévérance\\
dont je suis capable et que je n’abandonnerai que lorsque les\\
chaleurs rendront impossibles le travail des sables du désert.
\par J’ai l’honneur d’être,
\begin{center} Monsieur le Directeur,\end{center}
\begin{center} \hspace{5cm}Votre très-humble\\
\hspace{5cm}et très-obéissant serviteur\\
\hspace{5cm} \gls{CoEg_abbr_00000002} Mariette\end{center}

\hypertarget{CoEg_Mariette_1851-08-31}{}
\section*{Le 31 août 1851, de Saqqarah, à Nieuwerkerke, directeur général des musées nationaux}
\addcontentsline{toc}{section}{Le 31 août 1851, de Saqqarah, à Nieuwerkerke, directeur général des musées nationaux}
{\footnotesize
\noindent Institution et lieu de conservation : Archives nationales, Pierrefitte-sur-Seine\\
Cote : \hyperlink{CoEg_Mariette_ms_001}{20150497/118, dossier 145 «~Mariette, Auguste~»} (n. p.).\\
Support : une feuille double.\\
Thèmes~: \gls{CoEg_keyword_00000002}~; \gls{CoEg_keyword_00000011}~; \gls{CoEg_keyword_00000012}~; \gls{CoEg_keyword_00000006}~; \gls{CoEg_keyword_00000010}.\\
Note : une copie non datée de cette lettre se trouve dans les papiers Mariette conservés au sein du fonds Maspero à la bibliothèque de l’Institut de France (ms. 4061 (2), f\textsuperscript{os} 11-13 pour cette lettre). Ces copies ne sont pas de la main de Mariette ni de Maspero, mais correspondent à une écriture ancienne (parmi elles, la lettre copiée la plus récente est de 1869). Elles mentionnent parfois que l’original se trouvait aux archives du Louvre. Cette copie n’est pas toujours très fiable, notamment pour les noms propres.
\begin{center} {[1\textsuperscript{re} page, r\textsuperscript{o}]}\end{center}}
\begin{flushright}Saqqarah\gls{CoEg_place_00000001}, le 31 août 1851.\end{flushright}
A Monsieur
\begin{center}Monsieur le Directeur des Musées Nationaux\gls{CoEg_org_00000001}\end{center}
\begin{flushright}à Paris\gls{CoEg_place_00000002}.\end{flushright}

\indent \hspace{1cm}Monsieur le Directeur\gls{CoEg_pers_00000002},\\

J’ai reçu en son temps votre lettre du 17 avril. Mais atteint alors\\
d’une ophthalmie [\textit{sic}] qui me privait de l’usage de mes yeux, je n’ai pu\\
prendre connaissance de cette lettre que le 4 Juin suivant.\\
\indent Le 6 Juin j’envoyai au Caire\gls{CoEg_place_00000010} un exprès chargé – ou de rencontrer\\
\gls{CoEg_abbr_00000001} Lafuente\gls{CoEg_pers_00000014} et de lui remettre un mot de moi – ou de chercher à savoir\\
où il se trouvait.\\
\indent Malheureusement \gls{CoEg_abbr_00000001} Lafuente\gls{CoEg_pers_00000014} était alors à Londres\gls{CoEg_place_00000011}, et ce n’est\\
qu’au commencement de ce mois que j’appris son retour à Alexandrie\gls{CoEg_place_00000006}, sa\\
résidence ordinaire.\\
\indent Je lui écrivis immédiatement dans le sens de vos instructions. Je lui\\
demandai :\\
\indent \gls{CoEg_abbr_00000005} le prix de \gls{CoEg_abbr_00000001} d’Anastasy\gls{CoEg_pers_00000015} pour la partie de la collection égyptienne\\
de Livourne\gls{CoEg_place_00000012}, qui comprend les stèles~;\\
\indent \gls{CoEg_abbr_00000006} le prix de la seconde partie qui comprend les papyrus~;\\
\indent \gls{CoEg_abbr_00000007} enfin le prix des deux sections réunies.\\
\indent J’ai reçu il y a peu de jours la réponse de \gls{CoEg_abbr_00000001} Lafuente\gls{CoEg_pers_00000014} – \gls{CoEg_abbr_00000001}\\
d’Anastasy\gls{CoEg_pers_00000015} consent à couper sa collection, non pas en trois, mais
{\footnotesize \begin{center} [1\textsuperscript{re} page, v\textsuperscript{o}]\end{center}}
\noindent en deux~; il distrait du tout les \textit{bijoux} et les \textit{scarabées}, et demande\\
du reste 80,000 francs.\\
\indent J’ai l’honneur, Monsieur le Directeur, de vous soumettre les\\
propositions de \gls{CoEg_abbr_00000001} d’Anastasy\gls{CoEg_pers_00000015}, et dans le cas où vous auriez\\
de nouvelles instructions à me donner, je suis naturellement\\
à vos ordres.\\
\indent Je\footnote{Mariette\gls{CoEg_pers_00000001} a d'abord écrit «~J'~» puis a barré l'apostrophe.} dois ajouter que j’avais profité de mes bonnes relations avec\\
\gls{CoEg_abbr_00000001} Lafuente\gls{CoEg_pers_00000014} pour le prier officieusement d’intervenir dans cette\\
affaire, en usant de son influence sur \gls{CoEg_abbr_00000001} d’Anastasy\gls{CoEg_pers_00000015} pour\\
engager celui-ci – soit à vous offrir un prix plus raisonnable\\
de la collection – soit à choisir le Louvre\gls{CoEg_org_00000002}, dans le cas où il\\
se déciderait définitivement à faire don de cette même collection\\
à l’un des Musées de l’Europe\gls{CoEg_place_00000013}.\\
\indent Sur la première de ces deux questions, \gls{CoEg_abbr_00000001} Lafuente\gls{CoEg_pers_00000014} me fait\\
savoir que les 80,000 francs ne représentent pas le prix définitif de\\
la collection, mais qu’il semble à \gls{CoEg_abbr_00000001} d’Anastasy\gls{CoEg_pers_00000015} que c’est sur\\
cette première base que peuvent commencer les pourparlers.\\
\indent Sur le second point, \gls{CoEg_abbr_00000001} Lafuente\gls{CoEg_pers_00000014} ne se prononce aucunement.\\
Je n’aurai donc rien à ajouter à ce que je vous ai déjà dit à\\
ce sujet, puisque je ne sais pas mieux qu’avant si \gls{CoEg_abbr_00000001}\\
d’Anastasy\gls{CoEg_pers_00000015} veut réellement doter l’un des établissements scientifiques\\
de l’Europe\gls{CoEg_place_00000013} des richesses archéologiques qu’il a réunies à\\
Livourne\gls{CoEg_place_00000012}, ou si, en parlant à tout le monde du plaisir\\
qu’il aurait à attacher son nom à une belle collection, il ne\\
veut pas se donner à lui-même l’honneur d’une intention\\
généreuse. Cependant, Monsieur le Directeur, si vous voulez\\
bien me permettre de vous exprimer mon opinion personnelle
{\footnotesize \begin{center} [2\textsuperscript{e} page, r\textsuperscript{o}]\end{center}}
\noindent je vous dirai que, pour le [moment ?], toutes les distinctions honorifiques\\
dont vous pouvez disposer ne tenteront pas \gls{CoEg_abbr_00000001} d’Anastasy\gls{CoEg_pers_00000015}.\\
\indent \gls{CoEg_abbr_00000001} d’Anastasy\gls{CoEg_pers_00000015} n’est en effet consul-général de Suède\gls{CoEg_place_00000014} que\\
pour l’honneur de ce titre. Négociant et banquier de Son Altesse\\
le Vice-Roi\gls{CoEg_pers_00000016}, il est ce qu’on appelle un homme d’argent, et\\
par conséquent de ceux que n’éblouissent pas les distinctions\\
honorifiques. En [rature] général, \gls{CoEg_abbr_00000001} d’Anastasy\gls{CoEg_pers_00000015} ne donnerait donc\\
la collection de Livourne\gls{CoEg_place_00000012}, que s’il lui devient bien prouvé qu’il ne\\
peut la vendre.\\
\indent Je dirai de plus que, dans les circonstances actuelles, \gls{CoEg_abbr_00000001}\\
d’Anastasy\gls{CoEg_pers_00000015} est moins porté que jamais à céder à un mouvement\\
de générosité. Permettez-moi, pour être clair, de vous parler en\\
insistant le langage familier du Caire\gls{CoEg_place_00000010}. En ce moment, les choses\\
{[s’arrangent ?]} ainsi en Egypte\gls{CoEg_place_00000003} que, de quelque nation que l’on\\
soit, on n’est jamais qu’\textit{anglais} ou \textit{français}. Ces [discriminations ?],\\
pour ceux qui voient de près les affaires publiques de ce pays,\\
indiquent de la manière la plus expressive les deux extrêmes qui\\
sont en présence. Méhémet-Ali\gls{CoEg_pers_00000017} était \textit{français}~; Abbas-\Gls{CoEg_entry_00000002}\gls{CoEg_pers_00000016}\\
est \textit{anglais}. Le premier faisait de la France\gls{CoEg_org_00000012} son alliée~; il\\
appelait des français au gouvernement de l’Egypte\gls{CoEg_place_00000003}~; Abbas-\Gls{CoEg_entry_00000002}\gls{CoEg_pers_00000016}\\
les congédie, un à un et systématiquement. C’est ainsi que\\
Linant-\gls{CoEg_entry_00000003}\gls{CoEg_pers_00000019}, Lambert-\gls{CoEg_entry_00000003}\gls{CoEg_pers_00000020}, Clot-\gls{CoEg_entry_00000003}\gls{CoEg_pers_00000021}, Varin-\gls{CoEg_entry_00000003}\gls{CoEg_pers_00000022} sont en disgrâce,\\
tandis que le Vice-Roi\gls{CoEg_pers_00000016} actuel élève aux hautes fonctions des\\
sujets anglais. Il est vrai qu’il n’a encore fait qu’un \gls{CoEg_entry_00000003}\\
anglais, et que ce \gls{CoEg_entry_00000003} est son \textit{boulanger}. Il s’appelle\\
Walker-\gls{CoEg_entry_00000003}\gls{CoEg_pers_00000023}.\\
\indent Quoi qu’il en soit, les deux systèmes sont aujourd’hui\\
parfaitement définis et il ne faut pas être venu deux fois\\
au Caire\gls{CoEg_place_00000010} pour s’apercevoir que rien n’est plus exact que les\\
deux grandes divisions qui partagent les colonies européennes
de l’Egypte\gls{CoEg_place_00000003}.
{\footnotesize \begin{center} [2\textsuperscript{e} page, v\textsuperscript{o}]\end{center}}
\indent Or \gls{CoEg_abbr_00000001} d’Anastasy\gls{CoEg_pers_00000015} est Anglais. Et il l’est d’autant plus en\\
ce moment que, banquier de \gls{CoEg_abbr_00000003}\gls{CoEg_pers_00000016}, il va être pour beaucoup dans\\
la grande entreprise de Chemin de fer d’Alexandrie\gls{CoEg_place_00000006} au Caire\gls{CoEg_place_00000010} qui\\
vient d’être concédé à une compagnie anglaise sur la demande\\
expresse de \gls{CoEg_abbr_00000001} Murray\gls{CoEg_pers_00000018}, consul-général d’Angleterre\gls{CoEg_org_00000014}.\\
\indent Dans les circonstances présentes, il me semble donc que vous n’avez\\
guère à espérer de \gls{CoEg_abbr_00000001} d’Anastasy\gls{CoEg_pers_00000015} le don, à titre gratuit, de\\
sa magnifique collection de Livourne\gls{CoEg_place_00000012}. J’ai la conviction que,\\
s’il la donnait à quelqu’un, ce serait au Musée Britannique\gls{CoEg_org_00000005}.\\
\indent Mais je crois qu’il y aurait peut-être, plus tard, un moyen\\
d’obtenir ce cadeau~; ce serait celui d’\textit{attendre}. On parle en effet\\
du remplacement de \gls{CoEg_abbr_00000001} Lemoyne\gls{CoEg_pers_00000024}, notre consul-général, par \gls{CoEg_abbr_00000001}\\
Benedetti\gls{CoEg_pers_00000025} – Or \gls{CoEg_abbr_00000001} Benedetti\gls{CoEg_pers_00000025} est le gendre de \gls{CoEg_abbr_00000001} d’Anastasy\gls{CoEg_pers_00000015}.\\
\indent Je vous transmets, Monsieur le Directeur, ces renseignements pour\\
vous éclairer dans la décision que \textsuperscript{vous} voudrez bien prendre. Je n’ai plus\\
maintenant qu’à attendre vos ordres.\\
\indent J’ajouterai que, connaissant le caractère et la situation présente\\
de \gls{CoEg_abbr_00000001} d’Anastasy\gls{CoEg_pers_00000015}, j’aurai peut-être dû m’abstenir d’entamer les\\
négociations dont vous m’avez chargé~; pour obtenir un cadeau de\\
\gls{CoEg_abbr_00000001} d’Anastasy\gls{CoEg_pers_00000015}, il ne faut pas en effet commencer par lui laisser\\
voir qu’on est disposé à acheter. Mais j’ai cru devoir parler haut\\
de l’argent du Louvre\gls{CoEg_org_00000002}, et je pense que traîner les pourparlers en\\
longueur est le seul moyen que nous ayons d’empêcher \gls{CoEg_abbr_00000001}\\
d’Anastasy\gls{CoEg_pers_00000015} de céder aux obsessions de quelques personnes et\\
d’honorer de sa générosité un autre établissement que le Louvre\gls{CoEg_org_00000002}.\\
Je vous répète en effet que tant que \gls{CoEg_abbr_00000001} d’Anastasy\gls{CoEg_pers_00000015} croira\\
que le Louvre\gls{CoEg_org_00000002} veut acheter, il ne donnera à personne, pas même\\
au Musée Britannique\gls{CoEg_org_00000005}.\\
\indent J’ai l’honneur d’être,
\begin{center}Monsieur le Directeur,\end{center}
\begin{center}\hspace{5cm}Votre très-humble serviteur.\\
\hspace{5cm}\gls{CoEg_abbr_00000002} Mariette\end{center}
\noindent \gls{CoEg_abbr_00000008} Je continue à être satisfait de\\
mes fouilles. Le Sérapéum\gls{CoEg_place_00000004} de Memphis\gls{CoEg_place_00000005}\\
a été décidément construit par Ramsès II\gls{CoEg_pers_00000026}.\\
Quelques parties \textit{grecques} sont du temps de Nectanébo\footnote{Nectanébo I\textsuperscript{er}\gls{CoEg_pers_00000013} ou Nectanébo II\gls{CoEg_pers_00000045}~?}.

\hypertarget{CoEg_Mariette_1851-09-14a}{}
\section*{Le 14 septembre 1851, de Saqqarah, à Le Moyne, consul général de France en Égypte (copie)}
\addcontentsline{toc}{section}{Le 14 septembre 1851, de Saqqarah, à Le Moyne, consul général de France en Égypte (copie)}
{\footnotesize
\noindent Institution et lieu de conservation : Archives nationales, Pierrefitte-sur-Seine\\
Cote : \hyperlink{CoEg_Mariette_ms_002}{F/17/2988/1, dossier « Mariette »} (n. p.) ; \hyperlink{CoEg_Mariette_ms_001}{20150497/118, dossier 145 «~Mariette, Auguste~»} (n. p.).\\
Support : deux feuilles doubles.\\
Thèmes~: \gls{CoEg_keyword_00000012}~; \gls{CoEg_keyword_00000006}~; \gls{CoEg_keyword_00000002}~;  \gls{CoEg_keyword_00000007}.\\
Notes : Nous n’avons pas localisé pour l’instant l’original de cette lettre~; il en existe encore cependant au moins trois versions versions. 
\begin{itemize} \item Celle qui nous sert de texte de base est une copie réalisée en double exemplaire par l’administration (une copie de la lettre par Mariette -~pas encore repérée non plus~- lui était parvenue en même temps que \hyperlink{CoEg_Mariette_1851-09-14b}{la lettre du même jour adressée aux ministres de l’Intérieur et de l’Instruction publique}). Elle témoigne du texte final qu’ont reçu les destinataires. La lecture des noms propres de la copie est hasardeuse (avec par exemple «~Saggarah~» pour «~Saqqarah~» ou encore «~moudir d’Egypte~» pour «~moudir de Gyzeh~») et ceux-ci ont été rétablis d'après la forme habituelle sous la plume de Mariette. Puisqu’il s’agit d’une copie à la fiabilité relative, le texte donné ici ne reprend pas le découpage en lignes ni les variations de ponctuation ou d’orthographes insignifiantes.
\item Le brouillon de cette même lettre, de la main de Mariette, est conservée à la Bibliothèque nationale de France\gls{CoEg_org_00000026} sous la cote NAF 20179 (f\textsuperscript{os} 66-69). Les hésitations et les modestes divergences dont il témoigne sont indiquées comme variantes en notes~;
\item Une autre copie de cette lettre, non datée mais postérieure à la première (et peut-être réalisée à partir de celle-ci), se trouve dans les papiers Mariette conservés au sein du fonds Maspero à la bibliothèque de l’Institut de France (ms.~4061 (2), f\textsuperscript{os}~14-18 pour cette lettre). Ces copies ne sont pas de la main de Mariette ni de Maspero, mais correspondent à une écriture ancienne (parmi elles, la lettre copiée la plus récente est de 1869). Elles mentionnent parfois que l’original se trouvait aux archives du Louvre. Cette copie n’est pas toujours très fiable, notamment pour les noms propres. \end{itemize}            }

\begin{flushright}Saqqarah, le 14 \gls{CoEg_abbr_00000019} 1851\end{flushright}
\indent A Monsieur\\
\indent \indent Monsieur l’Agent et Consul Général\\
\indent \indent de France\gls{CoEg_org_00000012} en Egypte\gls{CoEg_place_00000003} à Alexandrie\gls{CoEg_place_00000006}\\

\indent \hspace{1 cm} M. l’agent et consul Général\gls{CoEg_pers_00000024}\\
                  		
\indent J’ai l’honneur de sous informer que le 11 du mois courant, son Excellence Stéphan-\Gls{CoEg_entry_00000003}\gls{CoEg_pers_00000029}, ministre des affaires Etrangères de son Altesse le vice-roi\gls{CoEg_pers_00000016}, m’invita à me rendre au Caire\gls{CoEg_place_00000010}\footnote{Brouillon~: «~\sout{m’appela au Caire\gls{CoEg_place_00000010}} m’invita à me rendre au Caire\gls{CoEg_place_00000010}~».}, et me fit la communication suivante que je vais vous répéter aussi textuellement que ma mémoire a pu la conserver~:\\
\indent «~Son Altesse\gls{CoEg_pers_00000016}, informée que les monuments que vous trouviez à Saqqarah\gls{CoEg_place_00000001} étaient, les uns volés, les autres détruits ou mutilés, a pris la résolution de faire transporter ceux de ces monuments qui peuvent l’être au Ministère de l’Instruction publique\gls{CoEg_org_00000017}, à la citadelle du Caire\gls{CoEg_place_00000023}. Des ordres ont été donnés à M. le \Gls{CoEg_entry_00000004}\gls{CoEg_pers_00000028} de Gyzeh\gls{CoEg_place_00000007} et deux officiers d’État major mis à la disposition du \Gls{CoEg_entry_00000004} pour l’exécution de ces ordres. Quant aux monuments qui ne peuvent pas être transportés, ils resteront sur le sable à la place où vous les avez trouvées et les deux mêmes officiers veilleront à leur conservation. Du reste les uns et les autres objets seront\footnote{Brouillon~: «~\sout{resteront} \textsuperscript{seront}~».} la propriété de \gls{CoEg_abbr_00000003}\gls{CoEg_pers_00000016} qui en disposera selon son bon plaisir (textuel)~; peut-être, plus tard, pourra-t-elle en donner quelques-uns à la France\gls{CoEg_org_00000012} \footnote{Brouillon~: «~\textsuperscript{en} donner \sout{à la France\gls{CoEg_org_00000012}} quelques-uns \sout{d’entre eux} \textsuperscript{à la France\gls{CoEg_org_00000012}}~».}. (textuel)~»\\
\indent Cette communication me fut faite en français et ne m’a ainsi rien présenté d’ambigu.\\
\indent J’ai répondu à son Excellence\gls{CoEg_pers_00000029}~:\\
\indent «~Que je ne méconnaissais aucunement l’autorité de son Altesse\gls{CoEg_pers_00000016}, que mon intention n’était pas du tout de faire de l’opposition à l’exécution de ses décrets~; mais que je suppliais son Ex. Stephan-\gls{CoEg_entry_00000003}\gls{CoEg_pers_00000029} de se rappeler que je ne suis dans tout cela qu’un infiniment petit~; qu’en m’appelant au Caire\gls{CoEg_place_00000010} pour me donner connaissance d’une résolution si importante, son Excellence\gls{CoEg_pers_00000029} m’a fait un honneur inaccoutumé, qu’en un mot c’est aux autorités reconnues de mon pays que M. le Ministre\gls{CoEg_pers_00000029} doit s’adresser et que c’est à ces mêmes autorités que moi-même, \footnote{Brouillon~: «~\sout{que Son Altesse\gls{CoEg_pers_00000016}} qu’en m’appelant au Caire\gls{CoEg_place_00000010} pour me \sout{faire une communication} \textsuperscript{donner connaissance d’une résolution} si importante, Son Excellence\gls{CoEg_pers_00000029} me \sout{rend} fait un honneur  inaccoutumée, [rature] qu’en un mot c’est aux autorités reconnues de mon pays que \gls{CoEg_abbr_00000001} le Ministre\gls{CoEg_pers_00000029} \textsuperscript{devrait s’adresser} et que c’est à ces mêmes autorités que moi-même, \sout{que}~».}malgré tout mon respect pour le gouvernement\gls{CoEg_org_00000008} de Son Altesse\gls{CoEg_pers_00000016}, je dois obéir~; que le jour où le gouvernement français\gls{CoEg_org_00000012} m’ordonnera\footnote{Brouillon~: «~m’ordonnerait~».} de livrer mes monuments, je le ferai~; mais que, jusque là, je n’osais pas prendre sur moi seul le poids d’une si grande responsabilité.~»\\
\indent L’honorable M. Delaporte\gls{CoEg_pers_00000067}, Consul français du Caire\gls{CoEg_place_00000010}, était présent. Il ajouta qu’il avait\footnote{Brouillon~: «~\gls{CoEg_abbr_00000001} le consul\gls{CoEg_pers_00000067} du Caire\gls{CoEg_place_00000010} était présent \sout{à cette entrevue}. Il ajouta qu’il av.~».} déjà écrit à M. le Consul Général\gls{CoEg_pers_00000024} de son coté [\textit{sic}], que j’allais écrire du mien, et qu’il priait son Excellence\gls{CoEg_pers_00000029}, avant de parler de nouveau de cette affaire au Vice-Roi\gls{CoEg_pers_00000016}, d’attendre une réponse officieuse.\\
\indent Son Excellence\gls{CoEg_pers_00000029} voulut bien consentir.\\
\indent Maintenant, M. le consul, je remplis un devoir en vous informant de la communication qui m’a été faite de la part de son Altesse\gls{CoEg_pers_00000016} par M. le Ministre des affaires Etrangères\gls{CoEg_pers_00000029}\footnote{Brouillon~: «~\sout{J’[?] inform[?]e également,} \sout{de ce résultat} \sout{le gouvernement français} \sout{les} \sout{M} [rature] à Paris, Messieurs les Ministres de l’Intérieur\gls{CoEg_pers_00000085} et de l’Instruction Publique\gls{CoEg_pers_00000086} auxquels, selon mon instruction écrite, je fois rendre compte directement de ma mission. Veuillez, je vous prie, \sout{en} prendre \sout{note}, autant que vous le jugerez bon, cette affaire en main. Vous êtes le défenseur naturel aussi zélé de tous les droits de la France\gls{CoEg_org_00000012} en Egypte\gls{CoEg_place_00000003} et je ne doute pas~» (ce passage, barré, est absent de la lettre finale telle qu’elle a été copiée par l’administration).}. Je n’ai rien à ajouter parce que, cette affaire une fois mise entre vos mains, je n’ai à m’en occuper que pour l’exécution des ordres qui me seront donnés.\\
\indent Cependant, Monsieur le Consul, je vous crois aussi devoir vous faire connaître les faits qui ont précédé la communication que je viens d’avoir l’honneur de vous transmettre.\\
\indent Le 6 septembre dernier je vis arriver chez moi, à Saqqarah\gls{CoEg_place_00000001}, un \gls{CoEg_entry_00000019} (sorte de domestique) de son excellence le \Gls{CoEg_entry_00000004}\gls{CoEg_pers_00000028} de Gyzeh\gls{CoEg_place_00000007}. Le \gls{CoEg_entry_00000019}  me pria de la part de M. le \Gls{CoEg_entry_00000004} (Safar-\Gls{CoEg_entry_00000002}\gls{CoEg_pers_00000028}) de laisser aller à la \Gls{CoEg_entry_00000021} les deux chefs de mes travaux et en même temps de désigner ceux de ces chefs que j’avais pu employer autrefois et que j’avais renvoyés.\\
\indent Depuis que je travaille à Saqqarah\gls{CoEg_place_00000001} je n’ai employé que trois \gls{CoEg_entry_00000020} et j’en fis la déclaration au \gls{CoEg_entry_00000019} qui prit ces trois \gls{CoEg_entry_00000020} avec lui et les emmena effectivement à Gyzeh\gls{CoEg_place_00000007}.\footnote{Brouillon~: «~J’avoue que je fus inquiet. Lorsque, le 4 juin dernier, le gouvernement égyptien\gls{CoEg_org_00000008} fit suspendre mes fouilles et qu’il fallut obtenir un \gls{CoEg_entry_00000005}, vous-même, Monsieur le consul, comme moi-même de mon côté, nous fîmes la [promesse ?] de ne pas enlever un seul des monuments du Sérapéum\gls{CoEg_place_00000004}. Doutait-on, non pas de votre parole, [mais ?] de la mienne ? \sout{Voulait-on interroger les arabes pour} avait-on fait contre moi, à \gls{CoEg_abbr_00000001} le \gls{CoEg_entry_00000004}\gls{CoEg_pers_00000028}, la millième de ces dénonciations fausses dont j’ai été l’objet ? voulait-on interroger mes gens et savoir d’eux quand et comment j’avais enlevé des monuments ?\\
Heureusement cette inquiétude était sans fondements.~» (ce passage, barré, est absent de la lettre finale telle qu’elle a été copiée par l’administration).}\\
\indent Là ces gens apprirent de la bouche même de son Excellence\gls{CoEg_pers_00000028} que mes monuments allaient être transportés en France\gls{CoEg_place_00000016}, et comme M. le \Gls{CoEg_entry_00000004}\gls{CoEg_pers_00000028} les priait, (dans le but, disait-il, de faciliter les opérations de douane qu’allait nécessiter ce transport), d’indiquer le nombre et la nature de ces objets, ils ne crurent pas devoir refuser ce que, d’ailleurs, on avait le droit d’exiger d’eux. Ils dictèrent donc la liste de mes monuments à l’un des \glspl{CoEg_entry_00000010} présents à la communication.\footnote{Brouillon~:
\begin{itemize}
\item «~\sout{[rature] \gls{CoEg_abbr_00000001} le \Gls{CoEg_entry_00000004}\gls{CoEg_pers_00000028} [rature] fit part avec [?] à mes reïs de tout l’intérêt qu’il portait à mes travaux~; puis il leur dis que, pour faciliter toutes les opérations de douane qu’allait nécessiter le transport de ces monuments en France\gls{CoEg_place_00000016}, on désirait dès à présent, savoir combien j’avais de ces monuments~; enfin il ajouta qu’il leur enjoignait d’en dicter, \textsuperscript{la liste,} sur le champ, à l’un des \glspl{CoEg_entry_00000010} présents à la communication.}~»~;
\item (Ce second essai est écrit entre les premières lignes de la précédent version) «~\sout{Là ces gens apprirent, de la bouche de \gls{CoEg_abbr_00000001} le \Gls{CoEg_entry_00000004}\gls{CoEg_pers_00000028} lui-même tout l’intérêt que \gls{CoEg_abbr_00000004}\gls{CoEg_pers_00000028} daignait porter à mes travaux~; ils y apprirent encore que mes monuments allaient être transportés en France\gls{CoEg_place_00000016}, et}~»~;
\item (Cette ultime version est condensée en bouts de lignes entre les paragraphes raturés, en trois blocs qui ne se succèdent pas dans l’ordre.)
\begin{itemize}
\item «~La [\textit{sic}] mes gens apprirent, de la [bouche~?] de \gls{CoEg_abbr_00000004}\gls{CoEg_pers_00000028}, que mes monuments allaient être transportés en France\gls{CoEg_place_00000016}, et comme +~»~;
\item «~+ \gls{CoEg_abbr_00000001} le \Gls{CoEg_entry_00000004}\gls{CoEg_pers_00000028} l[...es priait~?] dans le but, disait-il, de faciliter les opérations de douane qu’allaient~»~;
\item  «~+ nécessiter le transport, d’indiquer le nombre \sout{de mes} et la nature de ces objets, il ne crurent pas devoir refuser ce que, d’ailleurs, on avait le droit d’exiger d’eux. Ils dictèrent donc la liste de mes monuments à l’un des \glspl{CoEg_entry_00000010} qui~».
\end{itemize}
\end{itemize}
}\\
\indent Les trois \gls{CoEg_entry_00000020} revinrent\footnote{Brouillon~: «~Mes gens rentrèrent~».} à Saqqarah\gls{CoEg_place_00000001}, me parlèrent de douane et d’Alexandrie\gls{CoEg_place_00000006} et je ne pu m’empêcher de manifester ma joie.\\
\indent C’était le 9 \gls{CoEg_abbr_00000019}.\\
\indent Mais le même jour arriva à Saqqarah\gls{CoEg_place_00000001} l’\gls{CoEg_entry_00000010} qui avait écrit sous la dictée de mes \gls{CoEg_entry_00000020}\footnote{Accent aigu sur réïs}.\\
\indent Il eut l’air d’accomplir un devoir de politesse en venant me rendre visite.\footnote{Brouillon~: «~Son Excellence Safar-\Gls{CoEg_entry_00000002}\gls{CoEg_pers_00000028} ne l’avait pas envoyé et j’avoue que je [p/f...~?]~» (ce passage, barré, est absent de la lettre finale telle qu’elle a été copiée par l’administration).}\\
\indent Ce n’était pas pour moi qu’il venait à Saqqarah\gls{CoEg_place_00000001}, mais pour estimer les écuries que le gouvernement possède aux environs de ce village, écuries bâties dans le temps par Ibrahim-\Gls{CoEg_entry_00000002}\gls{CoEg_pers_00000028}. Et il m’annonça qu’il profitait de l’occasion pour faire l’inventaire des antiquités déposées à Saqqarah\gls{CoEg_place_00000001} et appartenant soit à M. Fernandez\gls{CoEg_pers_00000088}, soit à M. [Youssouf Messara~?]\gls{CoEg_pers_00000089} soit à tout autre Européen. «~Le but de cette mesure, a-t-il dit, est de ne pas confondre ces objets avec les vôtres~; les vôtres auront la permission de sortir~; les autres, au contraire, continueront à être prohibés.~»\\
\indent On avait eu\footnote{Brouillon~: «~\sout{fait} \textsuperscript{eu}~».}, la veille, la liste de mes monuments par mes \gls{CoEg_entry_00000020}~; on venait prendre aujourd’hui celle des objets qui sont, comme les miens, le produit des fouilles faites à Saqqarah\gls{CoEg_place_00000001}. Je trouvai donc la mission de l’\Gls{CoEg_entry_00000010} parfaitement justifiée.\\
\indent Mais l’\Gls{CoEg_entry_00000010} ajouta ceci~:\\
\indent «~Son Excellence\gls{CoEg_pers_00000028} me charge de vous dire que vous n’avez pas à croire qu’elle veuille vous tourmenter, vous inquiéter en m’envoyant vous demander la liste de vos monuments. Au contraire, la permission de transporter ces objets en France\gls{CoEg_place_00000016} va être donnée, et pour hâter les formalités de douane à Alexandrie\gls{CoEg_place_00000006}, on voudrait, dès à présent en connaître le nombre.~»\\
\indent J’avoue, Monsieur le Consul, que je ne pus m’empêcher d’être un peu étonné. On avait déjà une liste dictée par un \gls{CoEg_entry_00000020}, et on venait me prier moi-même de dicter encore cette même liste. Mes anciens soupçons revinrent~; en voyant que j’enlevais les monuments à mesure que je les découvrais, on doutait ainsi de la promesse que nous avons faite de ne rien enlever, on doutait de notre bonne foi\footnote{Le brouillon passe directement de «~revinrent~;~» à «~on doutait de notre bonne foi~».} et on voulait l’éprouver, car en confrontant les deux listes, le \textit{menteur} serait celui qui aurait dicté la liste la plus courte. Autrement pourquoi commencer par prendre la liste de mes \gls{CoEg_entry_00000020}~? si on avait complètement foi en ma parole, il me semble que ma seule liste devait passer aux yeux du \Gls{CoEg_entry_00000004}\gls{CoEg_pers_00000028} pour l’expression de la vérité.\\
\indent Je crus donc nécessaire de me tenir, à partir de ce moment, dans une plus grande réserve, et je me fis un scrupule d’indiquer à l’\Gls{CoEg_entry_00000010} jusqu’au dernier et au plus insignifiant de mes objets.\\
\indent L’\Gls{CoEg_entry_00000010} emporta sa liste et partit pour Gyzeh\gls{CoEg_place_00000007}. Quant aux écuries d’Ibrahim \Gls{CoEg_entry_00000002}\gls{CoEg_pers_00000028} – quant aux antiquités de MM. Fernandez\gls{CoEg_pers_00000088} et Messara\gls{CoEg_pers_00000089}, il ne s’en occupa nullement\footnote{Brouillon~: «~pendant tout le temps de son séjour à Saqqarah\gls{CoEg_place_00000001}~».}. La possession de ma liste était évidemment le but de sa mission. Or c’est le lendemain même que je fus appelé au Caire\gls{CoEg_place_00000010} par \gls{CoEg_abbr_00000004} Stéphan-\Gls{CoEg_entry_00000003}\gls{CoEg_pers_00000029}.\\
\indent J’étais donc tombé dans un \textit{piège} à Saqqarah\gls{CoEg_place_00000001} et Safar-\Gls{CoEg_entry_00000002}\gls{CoEg_pers_00000028} m’y avait fait tomber (et ici, Monsieur le Consul, je regrette d’être obligé d’employer une expression un peu dure) m’y avait fait tomber à l’aide d’un mensonge\footnote{Brouillon~: «~j’y étais tombé à l’aide d’un \textit{mensonge}~».}. Mes monuments n’allaient pas être, en effet, transportés à Alexandrie\gls{CoEg_place_00000006}~; ils allaient être \textit{confisqués}. Et pour que vous et moi-même nous ne trompions pas le gouvernement égyptien\gls{CoEg_org_00000008} lorsqu’il s’agirait de faire la remise des objets, on avait eu le soin de se munir d’avance d’une liste de mes objets dictée par moi-même.\\
\indent Voilà, Monsieur le Consul, les faits qui ont précédé la communication qui m’a été faite le 11 \gls{CoEg_abbr_00000019}.\\
\indent J’espère qu’en raison de la difficulté de ma position, vous approuverez la grande réserve\footnote{Brouillon~: «~difficulté de la position \sout{qui m’a été} dans laquelle je me trouvais en présence de Stéphan-\gls{CoEg_entry_00000003}\gls{CoEg_pers_00000029}, vous approuverez la \sout{réso} grande \textsuperscript{réserve}~».} que je me suis imposée dans ma réponse.\\
\indent Vous êtes, Monsieur le Consul, naturellement trop bien instruit des choses de ce pays pour que j’aie à faire ressortir la gravité de l’affaire que je prends la liberté de vous recommander. J’ajouterai, en terminant, un fait que j’oubliais~: c’est que le surlendemain même du jour où arriva au Caire\gls{CoEg_place_00000010} la nouvelle du vote par lequel l’Assemblée Nationale de France\gls{CoEg_org_00000027} mettait une somme de 30,000 francs à ma disposition pour le déblaiement du Sérapéum\gls{CoEg_place_00000004}, \gls{CoEg_abbr_00000004} Safar-\Gls{CoEg_entry_00000002}\gls{CoEg_pers_00000028} daigna venir de sa personne au désert que j’habite~; il visita mes travaux, se fit montrer la place où les statues reposent sous le sable, voulut voir une ou deux de ces fameuses inscriptions que les arabes savent que je recherche avec tant d’avidité, et partit en me félicitant, avec toute l’apparence de la sincérité, du succès inattendu de mon entreprise. Je crus alors que la visite de son Excellence\gls{CoEg_pers_00000028} était un acte de courtoisie envers un envoyé du gouvernement français\gls{CoEg_org_00000012}\footnote{Brouillon~: «~\sout{que \textsuperscript{[rature]} \gls{CoEg_abbr_00000004}} envers un envoyé du gouvernement français. \sout{Je crus aussi qu’après les [sévices ?] violences dont j’avais été l’objet le 4 juin, lorsque Safar-\Gls{CoEg_entry_00000002}\gls{CoEg_pers_00000028} fit \xout{interrompre} suspendre mes travaux, cette même visite était une sorte de réconcili}~».}~; je m’aperçois aujourd’hui que, dès ce jour là, la confiscation du Sérapéum\gls{CoEg_place_00000004} était résolue dans les conseils de son Altesse\gls{CoEg_pers_00000016}.\\
\indent J’ai l’honneur … \\
\begin{flushright}
Signé \gls{CoEg_abbr_00000002} Mariette\gls{CoEg_pers_00000001}
\end{flushright}

\hypertarget{CoEg_Mariette_1851-09-14b}{}
\section*{Le 14 septembre 1851, de Saqqarah, à Faucher, ministre de l’Intérieur, et Crouseilhes, ministre de l’Instruction publique (copie)} \addcontentsline{toc}{section}{Le 14 septembre 1851, de Saqqarah, à Faucher, ministre de l’Intérieur, et Crouseilhes, ministre de l’Instruction publique (copie)}
{\footnotesize
\noindent Institution et lieu de conservation : Archives nationales, Pierrefitte-sur-Seine\\
Cote : \hyperlink{CoEg_Mariette_ms_002}{F/17/2988/1, dossier « Mariette »} (n. p.) ; \hyperlink{CoEg_Mariette_ms_001}{20150497/118, dossier 145 «~Mariette, Auguste~»} (n. p.).\\
Support : une feuille double.\\
Thèmes~: \gls{CoEg_keyword_00000012}~; \gls{CoEg_keyword_00000002}.\\
Notes~: \begin{itemize} \item Comme le texte l’indique, cette lettre accompagnait une copie de \hyperlink{CoEg_Mariette_1851-09-14a}{la lettre du même jour adressée à Le Moyne}.
\item Nous n’avons pas localisé pour l’instant l’original de cette lettre~; il existe encore cependant au moins trois versions. Celle qui nous sert de texte de base pour cette lettre-ci est une copie réalisée en double exemplaire par l’administration, sur papier à en-tête de la direction générale des musées impériaux\gls{CoEg_org_00000001} au ministère de l’Intérieur\gls{CoEg_org_00000009}. Elle témoigne du texte final qu’ont reçu les destinataires. La lecture des noms propres de la copie est hasardeuse (avec par exemple «~Saggarah~» pour «~Saqqarah~»). Puisqu’il s’agit d’une copie à la fiabilité relative, le texte donné ici ne reprend pas le découpage en lignes, la pagination ni les variations de ponctuation ou d’orthographes insignifiantes.
\par Le brouillon de cette même lettre, de la main de Mariette\gls{CoEg_pers_00000001}, est conservée à la Bibliothèque nationale de France\gls{CoEg_org_00000026} (Paris) sous la cote NAF 20179 (f\textsuperscript{o} 75, une page). Les hésitations et les modestes divergences dont il témoigne sont indiquées comme variantes en note~;
\item Une autre copie de cette lettre, non datée mais postérieure à la première (et peut-être réalisée à partir de celle-ci), se trouve dans les papiers Mariette conservés au sein du fonds Maspero à la bibliothèque de l’Institut de France (ms.~4061 (2), f\textsuperscript{o}~14 pour cette lettre). Ces copies ne sont pas de la main de Mariette ni de Maspero, mais correspondent à une écriture ancienne (parmi elles, la lettre copiée la plus récente est de 1869). Elles mentionnent parfois que l’original se trouvait aux archives du Louvre. Cette copie n’est pas toujours très fiable, notamment pour les noms propres. \end{itemize}}
\begin{flushright}Saqqarah\gls{CoEg_place_00000001} le 14 \gls{CoEg_abbr_00000019} 1851\footnote{Brouillon~:«~20 Sept 1851~» Cette divergence est surprenante~: la lecture de «~20~» sur le brouillon de Mariette semble fiable~; il serait cependant étonnant qu’il ait laissé passer une semaine avant d’écrire aux ministres, dans la précipitation qu’il décrit. Peut-être s’agit-il d’une erreur de lecture au moment de la copie de la lettre originale par l’administration~?}\end{flushright}
\indent \hspace{1cm} Messieurs les Ministres de l’Intérieur\gls{CoEg_pers_00000085} et de l’Instruction publique\gls{CoEg_pers_00000086}\footnote{Brouillon~:\\
\hspace{1cm}«~A Messieurs\\
\begin{center}Messieurs les Ministres de l’Intérieur\\
\indent et de l’Instruction Publique\end{center}
\begin{flushright}à Paris\gls{CoEg_place_00000002}.~»\end{flushright}}\\

\indent Malgré le temps qui me presse, et qui, par la force des choses, va me manquer dans quelques minutes, je ne crois pas devoir laisser passer ce courrier sans porter à votre connaissance la résolution inattendue que vient de prendre son Altesse Abbas-Pacha\gls{CoEg_pers_00000016}, relativement aux monuments du Sérapéum\gls{CoEg_place_00000004} de Memphis\gls{CoEg_place_00000005} \footnote{Brouillon~: «~du Sérapéum\gls{CoEg_place_00000004}. \\
\indent \sout{Si les}~».}\\
\indent \gls{CoEg_abbr_00000003} Abbas-Pacha\gls{CoEg_pers_00000016}, par une communication qu’elle m’a faite officiellement, déclare \footnote{Brouillon~: «~\gls{CoEg_abbr_00000003} Abbas-Pacha\gls{CoEg_pers_00000016} déclare, par une communication qu’elle m’a faite \textsuperscript{officiellement},~».} que ces monuments sont sa propriété et qu’elle entend en disposer selon son bon plaisir. En d’autres termes le gouvernement Egyptien\gls{CoEg_org_00000008} confisque le Sérapéum\gls{CoEg_place_00000004}.\\
\indent Si les circonstances dont j’aurais à vous rendre compte, ne se présentaient de telle façon que j’ai à peine quelques minutes\footnote{Brouillon~: «~[rature] quelques instants~».} pour vous écrire, j’aurais porté directement et officiellement à votre connaissance l’annonce de la nouvelle que j’ai à vous transmettre.\\
\indent Mais le temps m’échappe, et je vous supplie de vouloir bien vous contenter de la copie de la
lettre que j’adresse à \gls{CoEg_abbr_00000001} le Consul \gls{CoEg_abbr_00000020}\footnote{Brouillon~: «~-Général~».}\gls{CoEg_pers_00000024} de France\gls{CoEg_org_00000012} à Alexandrie\gls{CoEg_place_00000006}.\\
\footnote{Brouillon~: «~\sout{Vous y trouverez des\\
Excusez, je vous en supplie, Messieurs\\
les Ministres,\\
renseignements assez \textit{détaillés}}\\
\indent \sout{Je vous renouvelle, Messieurs les\\
Ministres, l’expression de tous mes\\
regrets, \xout{et je vous prie de croire que}\\
Mais, en conscience, je [comp ?]}~»}\\
\indent J’espère toutefois que les renseignements que contient cette lettre vous paraîtront\\
suffisants. Dans tous les cas, Messieurs les Ministres, je suis à mon poste et j’attends vos ordres.\footnote{Le brouillon s’achève ici.}\\
\indent J’ai l’honneur d’être avec le plus profond respect,
\begin{center}Messieurs les Ministres,\end{center}
\begin{center}\hspace{5cm}Votre très-humble\\
\hspace{5cm}et très-obéissant serviteur\\
\hspace{5cm} Signé \gls{CoEg_abbr_00000002} Mariette\end{center}

\hypertarget{CoEg_Mariette_1852-01-16}{}
\section*{Le 16 janvier 1852, d’Abousir, vraisemblablement à Nieuwerkerke, directeur général des musées nationaux}
\addcontentsline{toc}{section}{Le 16 janvier 1852, d’Abousir, vraisemblablement à Nieuwerkerke, directeur général des musées nationaux}
{\footnotesize
\noindent Institution et lieu de conservation~: Archives nationales, Pierrefitte-sur-Seine\\
Cote~: \hyperlink{CoEg_Mariette_ms_001}{20150497/118, dossier 145 «~Mariette, Auguste~»} (n. p.).\\
Support~: une feuille double.\\
Thème~: \gls{CoEg_keyword_00000012}~; \gls{CoEg_keyword_00000007}~; \gls{CoEg_keyword_00000002}~; \gls{CoEg_keyword_00000006}.\\
Note~: une copie non datée de cette lettre se trouve dans les papiers Mariette conservés au sein du fonds Maspero à la bibliothèque de l’Institut de France (ms.~4061 (2), f\textsuperscript{os}~20-23 pour cette lettre). Ces copies ne sont pas de la main de Mariette ni de Maspero, mais correspondent à une écriture ancienne (parmi elles, la lettre copiée la plus récente est de 1869). Elles mentionnent parfois que l’original se trouvait aux archives du Louvre. Cette copie n’est pas toujours très fiable, notamment pour les noms propres.
\begin{center} {[1\textsuperscript{re} page, r\textsuperscript{o}]}\end{center}}
\begin{flushright}Du désert d’Abousir\gls{CoEg_place_00000008}, le 16 Janvier 1852\end{flushright}

\hspace{1cm}Monsieur\gls{CoEg_pers_00000002},\\

\indent Permettez-moi de vous entretenir d’une affaire dont j’attends de vous la\\
solution comme un véritable service.\\
\indent Je me hâte d’abord de vous rassurer. Il ne s’agit pas de moi, mais de\\
l’excellent \gls{CoEg_abbr_00000001} Batissier\gls{CoEg_pers_00000027} auquel, je crois, vous devez vous intéresser à cause des\\
services très-importants qu’il nous a rendus dans l’affaire de la confiscation des\\
monuments du Sérapéum\gls{CoEg_place_00000004}.\\
\indent Voici ce qui arrive :\\
\indent \gls{CoEg_abbr_00000001} Batissier\gls{CoEg_pers_00000027}, comme vous le savez, est Vice-Consul de France\gls{CoEg_org_00000012} à Suez\gls{CoEg_place_00000009},\\
et en cette qualité est tenu de faire sa résidence dans cette dernière ville.\\
\indent Mais comme il y est absolument inutile et comme, d’un autre côté,\\
son intelligence des affaires lui permet d’aider \gls{CoEg_abbr_00000001} Le Moyne\gls{CoEg_pers_00000024} pendant le temps\\
de la résidence de celui-ci au Caire\gls{CoEg_place_00000010}, il s’est décidé, non pas à\\
venir résider définitivement avec \gls{CoEg_abbr_00000001} Le Moyne\gls{CoEg_pers_00000024}, mais à venir passer ici\\
une partie de l’hiver. Il travaille alors dans les bureaux du Consulat-Général\gls{CoEg_org_00000006},\\
et je sais, par \gls{CoEg_abbr_00000001} Le Moyne\gls{CoEg_pers_00000024} lui-même, que \gls{CoEg_abbr_00000001} Batissier\gls{CoEg_pers_00000027} lui est de\\
la plus grande utilité.\\
\indent Tout ceci, bien entendu, se passe à l’insu du Ministère des Affaires\\
Etrangères\gls{CoEg_org_00000007} qui ne veut pas permettre que ses agents se fixent dans d’autres\\
localités que [rature] celles qui leur sont assignées.\\
\indent Malheureusement \gls{CoEg_abbr_00000001} Batissier\gls{CoEg_pers_00000027} vient d’être dénoncé à Paris\gls{CoEg_place_00000002}\\
comme résidant habituellement au Caire\gls{CoEg_place_00000010}, et il m’écrit aujourd’hui qu’il se\\
trouve placé entre une destitution et un séjour forcé à Suez\gls{CoEg_place_00000009}.\\
\indent Mon premier mouvement, Monsieur, est de m’adresser à vous pour\\
vous prier d’intervenir. Je vous dirai que, sans faire de tout ceci une\\
affaire personnelle, vous rendez un grand service au Louvre\gls{CoEg_org_00000002} en obtenant,\\
non pas que le Ministère\gls{CoEg_org_00000007} autorise \gls{CoEg_abbr_00000001} Batissier\gls{CoEg_pers_00000027} à résider au Caire\gls{CoEg_place_00000010},\\
mais qu’il ferme simplement les yeux pendant quelques temps encore.\\
\gls{CoEg_abbr_00000001} Batissier\gls{CoEg_pers_00000027} a été en effet l’homme le plus utile au Sérapéum\gls{CoEg_place_00000004}. Si\\
j’avais voulu vous ennuyer de réclamations et de plaintes, vous auriez su\\
de combien d’avanies j’ai été poursuivi par [rature] Safar-\Gls{CoEg_entry_00000002}\gls{CoEg_pers_00000028}, \gls{CoEg_entry_00000004} de\\
Gyzeh\gls{CoEg_place_00000007}, et Stéphan-\gls{CoEg_entry_00000003}\gls{CoEg_pers_00000029}, Ministre des affaires Etrangères, tous deux des\\
dévoués de \gls{CoEg_abbr_00000001} le Consul-Général Anglais\gls{CoEg_pers_00000018}. Or sans \gls{CoEg_abbr_00000001} Batissier\gls{CoEg_pers_00000027}, je\\
ne serais jamais sorti de là. \gls{CoEg_abbr_00000001} Le Moyne\gls{CoEg_pers_00000024} lui-même vous dira de\\
quel secours il lui a été dans toutes les affaires très-délicates que nous\\
avons eu à traiter avec le gouvernement égyptien\gls{CoEg_org_00000008}. Je vous répète donc
{\footnotesize \begin{center} [1\textsuperscript{re} page, v\textsuperscript{o}]\end{center}}
\noindent qu’en laissant même de côté la question de faire plaisir à \gls{CoEg_abbr_00000001} Batissier\gls{CoEg_pers_00000027},\\
vous avez intérêt à conserver celui-ci au Caire\gls{CoEg_place_00000010}. D’ailleurs, l’avenir nous\\
réserve peut-être encore bien des négociations difficiles à entamer, et je\\
ne vois pas que vous puissiez les faire aboutir aisément si \gls{CoEg_abbr_00000001} Batissier\gls{CoEg_pers_00000027}\\
n’est pas là pour profiter de sa position particulière auprès de \gls{CoEg_abbr_00000001} Le Moyne\gls{CoEg_pers_00000024}\\
et lui expliquer l’état réel des choses à mesure que je lui fais connaître.\\
\indent Ayez donc la bonté, Monsieur, de prendre cette affaire en main.\\
Je \textsuperscript{vous} la recommande d’une manière toute particulière en vous priant\\
d’agir en faveur d’un excellent homme qui mérite à tous les égards\\
votre protection. \gls{CoEg_abbr_00000001} Batissier\gls{CoEg_pers_00000027}, qui ne sait pas d’ailleurs que je\\
vous écris, ne demande pas, je pense, à être autorisé à fixer\\
son séjour au Caire\gls{CoEg_place_00000010}~; il demande seulement que, quand il y\\
vient, on ferme les yeux. Voyez, s’il-vous-plaît, les Bureaux\\
des affaires Etrangères\gls{CoEg_org_00000007} et tâchez d’arranger cette affaire à\\
l’amiable.\\
\indent Je vais profiter de l’occasion pour vous donner quelques détails\\
sur la position de notre affaire du Sérapéum\gls{CoEg_place_00000004}.\\
\indent Les travaux sont toujours suspendus et quoique vivant au\\
{[désert ?]} je n’ai personne autour de moi, que quelques gardiens sur\\
lesquels je puis à peu près compter. Mais les négociations de\\
\gls{CoEg_abbr_00000001} Le Moyne\gls{CoEg_pers_00000024} avec Son Altesse\gls{CoEg_pers_00000016} sont en très-bon chemin. Si \gls{CoEg_abbr_00000001}\\
Le Moyne\gls{CoEg_pers_00000024} voulait, le \gls{CoEg_entry_00000005} nécessaire pour reprendre les travaux\\
serait même déjà entre mes mains. Malheureusement l’Intérieur\gls{CoEg_org_00000009}\\
ne m’a pas encore envoyé d’argent et \gls{CoEg_abbr_00000001} Le Moyne\gls{CoEg_pers_00000024} le regrette\\
beaucoup. L’affaire des négociations a été en effet très-chaude~;\\
\gls{CoEg_abbr_00000001} Le Moyne\gls{CoEg_pers_00000024} s’est presque fâché avec Son Altesse\gls{CoEg_pers_00000016}. Maintenant\\
que dirait le gouvernement égyptien\gls{CoEg_org_00000008} si, la permission obtenue\\
après tant d’efforts, nous ne pouvions reprendre les fouilles faute\\
d’argent. \gls{CoEg_abbr_00000001} Le Moyne\gls{CoEg_pers_00000024} ne veut pas vous donner ce ridicule,\\
et il attend que j’aie reçu mon argent pour voir une dernière fois\\
le Vice-Roi\gls{CoEg_pers_00000016} et en finir définitivement.\\
\indent Par suite des mêmes circonstances, l’affaire de l’emballage\\
des monuments donnés n’est pas encore terminée. Vous vous rappelez\\
que \gls{CoEg_abbr_00000001} Le Moyne\gls{CoEg_pers_00000024} n’a pas voulu accepter les 515 monuments\\
dont je vous ai envoyé la liste et depuis ce temps cet incident\\
n’a pas fait un pas. Les monuments sont donc encore la
{\footnotesize \begin{center} [2\textsuperscript{e} page, r\textsuperscript{o}]\end{center}}
\noindent propriété du gouvernement égyptien\gls{CoEg_org_00000008}, et comme celui-ci les regarde\\
encore comme tels, je n’ai pas, jusqu’à un certain point, le droit\\
d’y toucher. Néanmoins d’accord avec \gls{CoEg_abbr_00000001} Le Moyne\gls{CoEg_pers_00000024}, j’ai forcé\\
quelque peu la consigne, et j’ai réussi à confectionner sans bruit\\
72 caisses de toutes grandeurs, contenant ensemble 1471 monuments,\\
lesquelles partiront pour Alexandrie\gls{CoEg_place_00000006} le jour même où l’affaire\\
sera réglée avec Son Altesse\gls{CoEg_pers_00000016}.\\
\indent Malheureusement ces caisses ne contiennent pas ceux des grands\\
monuments auxquels vous tenez peut-être le plus. L’emballage\\
de ces objets exige, d’abord des machines qu’on ne trouve pas ici\\
et qu’il me faudrait faire faire à grands frais, et ensuite des\\
hommes que le \Gls{CoEg_entry_00000004} me refuserait parfaitement. Je suis donc\\
obligé de les laisser encore sous le sable et de les réserver pour des\\
temps meilleurs.\\
\indent Néanmoins j’attache une grande importance à vous les expédier.\\
J’ai un Cerbère, un Lion et une Lionne, de proportions très-\\
-grandes, et ces monuments me paraissent tout-à-fait dignes du\\
Louvre\gls{CoEg_org_00000002}. Ils feraient avec la statue\gls{CoEg_obj_00000007} d’Apis\gls{CoEg_pers_00000011}, les \textit{trois} beaux \sout{de}\\
lions\footnote{N 432 A\gls{CoEg_obj_00000004}, N 432 B\gls{CoEg_obj_00000005} et N 432 C\gls{CoEg_obj_00000006}.} de Nectanébo\gls{CoEg_pers_00000013} et quelques autres figures de marbre, une\\
très-bonne salle que les stèles et les bronzes compléteraient\\
admirablement.\\
\indent Je suis aussi en négociation avec \gls{CoEg_abbr_00000001} Le Moyne\gls{CoEg_pers_00000024} pour obtenir\\
que \gls{CoEg_abbr_00000003}\gls{CoEg_pers_00000016} ajoute 16 sphinx à sa liste. Quatre nous sont\\
déjà donnés, ce qui porterait le nombre de ces monuments à 20\footnote{D'après la  \hyperref{CoEg_Mariette_1851-02-28}{lettre du 28 février 1851}, Mariette\gls{CoEg_pers_00000001} avait déjà envoyé six de ces sphinx au Louvre\gls{CoEg_org_00000002} - qui n'en obtint pas d'autres -, où ils furent enregistrés collectivement sous le numéro d'inventaire N 391\gls{CoEg_obj_00000002}.}.\\
\indent Voilà, Monsieur, où nous en sommes. Si le courrier\\
anglais, qui arriver demain, nous apporte de l’argent, je\\
ne doute que, dans quatre ou cinq jours, nous n’ayons recom-\\
-mencé nos travaux.\\
\indent Depuis ma dernière lettre, j’ai fait de nombreuses visites nocturnes\\
aux souterrains d’Apis\gls{CoEg_pers_00000011}. Je les avais jugés, à première vue,\\
Ptolémaïques : ils sont au contraire Pharaoniques et tous\\
antérieurs à Cambyse\gls{CoEg_pers_00000030}. Les souterrains Ptolémaïques sont [rature]\\
par conséquent encore à trouver et c’est de ces souterrains que\\
Diodore de Sicile\gls{CoEg_pers_00000031} veut parler quand il blâme l’extravagance
{\footnotesize \begin{center} [2\textsuperscript{e} page, v\textsuperscript{o}]\end{center}}
\noindent des prêtres qui dépensaient plus d’un demi-million pour chacun\\
des dieux qu’ils y introduisaient. Je connais l’emplacement de ces\\
souterrains, et à la reprise des travaux, je ne les manquerai pas.\\
\indent Je me suis aussi aperçu avec satisfaction d’un fait assez\\
singulier. On arrivait à la porte de la sépulture d’Apis\gls{CoEg_pers_00000011} par\\
un plan incliné qui servait en même temps à introduire les énormes\\
sarcophages dont je vous ai parlé. Voici à peu près le \sout{plan} dessin\\
de ce chemin en pente :\\
\begin{center}
\includegraphics{CoEg_Mariette_1852-01-16_fig-1.png}
\end{center}
\indent Le plan incliné commence en A = B, C, D, E sont des portes qui\\
communiquent dans l’intérieur des souterrains à l’est par la porte B\\
que j’ai pénétrée le 12 novembre. F est une 5\textsuperscript{e} porte qui conduit\\
à des galeries inconnues, car elles sont ensablées jusqu’aux voutes [\textit{sic}].\\
{[rature]} Le plan incliné tout entier est, bien entendu, taillé dans\\
le roc. Or à hauteur d’appui sur chacune de ses parois, se\\
voient encore une quantité incroyable de stèles votives en\\
hiéroglyphes ou en démotiques. Le même fait se répète dans un\\
grand nombre de chambres de l’intérieur = Ce fait singulier\\
mérite, je crois, une grande attention et mon premier soin, à\\
la reprise des travaux, sera d’enlever toutes celles de ces stèles\\
que je pourrai rencontrer.\\
\indent J’ai encore bien des choses à vous dire. Mais, vous le voyez,\\
la place me manque. Ayez la complaisance de présenter mes hommages\\
à \gls{CoEg_abbr_00000001} de Rougé\gls{CoEg_pers_00000032}, à \gls{CoEg_abbr_00000001} de Longpérier\gls{CoEg_pers_00000033}, à \gls{CoEg_abbr_00000001} de Viel-Castel\gls{CoEg_pers_00000034}\\
et à \gls{CoEg_abbr_00000001} Villot\gls{CoEg_pers_00000035}. Si Dieu\gls{CoEg_pers_00000036} me conserve l’excellente santé dont je\\
jouis, je compte avoir encore ici du travail pour une année.\\
\indent Mais que de choses à faire.
\begin{center} \hspace{5cm} Votre tout dévoué serviteur :\\
\hspace{5cm} \gls{CoEg_abbr_00000002} Mariette\end{center}

\hypertarget{CoEg_Mariette_1852-08-04}{}
\section*{Le 4 août 1852, d’Abousir, vraisemblablement à Nieuwerkerke, directeur général des musées nationaux}
\addcontentsline{toc}{section}{Le 4 août 1852, d’Abousir, vraisemblablement à Nieuwerkerke, directeur général des musées nationaux} 
{\footnotesize \noindent Institution et lieu de conservation~: Archives nationales, Pierrefitte-sur-Seine\\
Cote~: \hyperlink{CoEg_Mariette_ms_001}{20150497/118, dossier 145 «~Mariette, Auguste~»} (n. p.).\\
Support~: une feuille double.\\
Thèmes~: \gls{CoEg_keyword_00000011}~; \gls{CoEg_keyword_00000006}~;\gls{CoEg_keyword_00000002} ~; \gls{CoEg_keyword_00000007}.\\
Note~: une copie non datée de cette lettre se trouve dans les papiers Mariette conservés au sein du fonds Maspero à la bibliothèque de l’Institut de France (ms.~4061 (2), f\textsuperscript{os}~24-27 pour cette lettre). Ces copies ne sont pas de la main de Mariette ni de Maspero, mais correspondent à une écriture ancienne (parmi elles, la lettre copiée la plus récente est de 1869). Elles mentionnent parfois que l’original se trouvait aux archives du Louvre. Cette copie n’est pas toujours très fiable, notamment pour les noms propres.
\begin{center} {[1\textsuperscript{re} page, r\textsuperscript{o}]}\end{center}}
\begin{flushright} Du désert d’Abousyr\gls{CoEg_place_00000008}, le 4 août 1852.\end{flushright}

\hspace{1cm}Monsieur\gls{CoEg_pers_00000002},\\

\indent J’ai écrit avant-hier à \gls{CoEg_abbr_00000001} le Ministre de l’Intérieur\gls{CoEg_pers_00000037} pour l’avertir\\
du départ très-prochain d’Alexandrie\gls{CoEg_place_00000006} de trois de mes caisses. Ces\\
caisses seront vers le 15 août à Marseille\gls{CoEg_place_00000018}, et si le commissionnaire\footnote{La fin du mot est écrite par-dessus un autre mot illisible.}\\
de roulage de l’Intérieur\gls{CoEg_org_00000009} veut bien se hâter, vous les recevrez quelques\\
jours après.\\
\indent J’ai joint à ma lettre à \gls{CoEg_abbr_00000001} le Ministre\gls{CoEg_pers_00000037} [rature] une autre\\
lettre pour \gls{CoEg_abbr_00000009} B{[oujon ?]}\gls{CoEg_pers_00000038} et Verrier\gls{CoEg_pers_00000039}, 75, rue de Rambuteau, aujourd’hui\\
chargés des transports de votre Ministère\gls{CoEg_org_00000009}. Ayez la bonté, Monsieur,\\
de faire dire à ces Messieurs l’intérêt que vous avez à posséder ces\\
caisses, et recommandez-leur surtout de ne les manier qu’avec\\
précautions, car les objets qu’ils contiennent, tout en pierre qu’ils\\
sont, sont des plus fragiles.\\
\indent Je prie aussi \gls{CoEg_abbr_00000001} le Ministre de l’Intérieur\gls{CoEg_pers_00000037} de vous faire passer\\
une copie de l’extrait de mon catalogue que je lui ai envoyé. Cet\\
extrait concerne les monuments renfermés dans les trois colis. Je\\
vous serait très-obligé si vous vouliez bien réclamer cette copie\\
aux Beaux-Arts\gls{CoEg_org_00000011}.\\
\indent J’aurais voulu joindre à cet envoi quelque monument qui, pour\\
son exécution artistique, vous intéressât plus particulièrement. Mais\\
les caisses sont trop lourdes, ou bien elles sont encore ici et vont\\
faire partie d’une seconde expédition pour Alexandrie\gls{CoEg_place_00000006}. Je tâcherai\\
néanmoins de vous faire passer un de ces jours mon \textit{écrivain\gls{CoEg_obj_00000008}}. Ce\\
monument est au moins de la IV\textsuperscript{e} dynastie et il surpasse, pour le\\
modelé des chairs et l’expression générale du personnage, tout ce\\
que vous avez vu jusqu’ici, même de ce qu’on appelle la bonne époque.\\
La photographie que je vous en ai envoyée a mal rendu ces formes si\\
naturelles, et vous ne devez pas la regarder comme une copie exacte du modèle.
{\footnotesize \begin{center} {[1\textsuperscript{re} page, v\textsuperscript{o}]}\end{center}}
\indent J’ai jusqu’ici livré au gouvernement égyptien\gls{CoEg_org_00000008} 656 monuments, et je\\
m’arrange de manière à passer pour n’en garder aucun par devers moi,\\
ce qui, entre nous, est tout de la contraire de la vérité. Son Altesse\gls{CoEg_pers_00000016} sera\\
enchantée quand elle apprendra mon empressement à obéir à ses\\
ordres et elle n’en sera que plus disposée à nous faire plus tard\\
un second cadeau. Mais pour cela je pense qu’il faudrait, dès-\\
-à-présent, que le nouveau consul-général\gls{CoEg_pers_00000040} d’Egypte\gls{CoEg_place_00000003} (de qui tout\\
dépend) fût instruit par le Ministre des Affaires Etrangères\gls{CoEg_pers_00000041} de l’importance\\
que le gouvernement français\gls{CoEg_org_00000012} attache aux fouilles du Sérapéum\gls{CoEg_place_00000004}, afin\\
qu’il ne soit plus, comme \gls{CoEg_abbr_00000001} Le Moyne\gls{CoEg_pers_00000024}, qu’on a laissé un an\\
sans instruction, exposé à pêcher {[\textit{sic}]} par ignorance. Causez-en avec\\
\gls{CoEg_abbr_00000001} Batissier\gls{CoEg_pers_00000027}, et celui-ci vous dira que si le nouveau Consul-\\
-général\gls{CoEg_pers_00000040} le veut bien, il peut obtenir de Son Altesse\gls{CoEg_pers_00000016} même le droit\\
de fouiller dans l’Egypte\gls{CoEg_place_00000003} entière, ce que je désire bien vivement,\\
Monsieur, car il m’en coûterait beaucoup de retourner en France\\
sans avoir visité Thèbes\gls{CoEg_place_00000019} et la Haute-Egypte\gls{CoEg_place_00000020}.\\
\indent \gls{CoEg_abbr_00000001} D’Anastasy\gls{CoEg_pers_00000015} est mort il y a quelques jours\footnote{Il s'agissait d'une fausse rumeur (voir la \hyperref{CoEg_Mariette_1852-09-04}{lettre du 4 septembre 1852})~; Anastasi\gls{CoEg_pers_00000015} mourrut en 1860.} et peut-être\\
ses héritiers n’auront-ils pas la même prétention quant à la\\
collection de Livourne\gls{CoEg_place_00000012}. J’ai déjà écrit à Alexandrie\gls{CoEg_place_00000006} pour qu’on\\
sonde le terrain à ce sujet et je vous ferai part de toutes les\\
informations que je pourrai recueillir. De votre côté, dites-moi\\
si, avec une réduction considérable de prix, vous seriez disposé à\\
terminer cette affaire.\\
\indent Rien de nouveau ici. J’attends avec impatience le moment de\\
reprendre les travaux et les souterrains grecs m’empêchent de\\
dormir. Du reste, si on m’accorde des fonds, je pousserai les fouilles\\
avec la plus grande activité, car j’ai hâte d’en finir. En six mois\\
j’espère que tout sera fait.
{\footnotesize \begin{center} {[2\textsuperscript{e} page, r\textsuperscript{o}]}\end{center}}
\indent Mais le plus difficile sera d’emballer les grands lions\gls{CoEg_obj_imn} grecs\\
et les autres statues de même style. Ces objets ont été taillés dans\\
une pierre très-friable qui s’écaille et je ne vois pas de moyen de\\
les ramener sans les briser. Aussi, Monsieur, je m’adresse à vous\\
et je vous prie de me faire savoir si vous ne connaissez pas\\
quelque composition chimique qui rende à la pierre sa dureté\\
primitive.\footnote{En juillet 1851, Rochas publia dans les comptes rendus de l'Académie des sciences une lettre sur le procédé de silicatisation~; il mentionnait un voyage en Orient au cours duquel il avait observé les monuments du Sérapéum et échangé avec Mariette à ce sujet (\textsc{Rochas}, «~Moyens de conserver indéfiniment les monuments en pierre calcaire~»\gls{CoEg_bibl_00000005}, \textit{Comptes-rendus de l’Académie des sciences}, 1851, p. 622~: «~Qu’il me soit permis, en terminant cette Lettre, d’appeler l’attention de l’Académie sur les monuments découverts récemment par M. Mariette, dans les fouilles qu’il exécute dans le temple de Sérapis, à Memphis. Au commencement de cette année, lors de mon voyage en Orient, j’eus occasion de visiter sur les lieux les statues, les sphinx, etc., qui étaient à découvert à cette époque. Ces monuments sont la plupart en calcaire tendre de la chaîne arabique, qui offre naturellement peu de cohésion. Je reconnus, qu’étant resté enfoui pendant tant de siècles, ce calcaire était, pour ainsi dire, totalement privé de solidité~; en effet, peu de temps après que ces statues eurent été exposées à l’air, après leur exhumation, elles se sont écaillées et détériorées si promptement, que l’on a jugé indispensable de les faire recouvrir de sable.\\
\indent M. Mariette me fit par des inquiétudes qu’il éprouvait pour la conservation et le transport en France de ces statues~; je lui fis remarquer alors qu’il était possible de leur donner sur place, en les silicatisant, la solidité nécessaire pour le transport, et je lui offris de me charger de cette opération.~»)~; le département égyptien du Louvre constitua d'ailleurs en 1853 un dossier à ce sujet - conservé sous la cote 20144775/24 aux Archives nationales. Rochas obtint l'autorisation de faire des essais de son procédé sur des statues égyptiennes du Louvre (voir aussi l'article 20144793/33 des Archives nationales où se trouvent des courriers archivés par le département des sculptures).} Dans ce cas, veuillez me la faire connaître, afin que\\
je l’applique ici, car les monuments dont je vous entretiens, sans\\
être très-précieux au point de vue de l’art, le sont beaucoup\\
pour les archéologues, et dans tous les cas feront toujours au\\
Louvre\gls{CoEg_org_00000002} un excellent fond de salle. En attendant que vous veuillez\\
bien me répondre, ces monuments sont sous le sable à l’abri de\\
toute cause de destruction.\\
\indent Je ne compte pas vous envoyer toutes les statues\gls{CoEg_obj_imn} grecques de\\
l’hémicycle de l’Apéum. Elles sont trop mauvaises. J’en ferai\\
un choix d’une ou deux. Mais je vous demanderai à mouler\\
les autres à cause des inscriptions grecques qu’on y lit.\\
\indent Vous aurez remarqué sans doute dans mon plan général\\
de la tombe d’Apis\gls{CoEg_pers_00000011} et d’Osiris\gls{CoEg_pers_00000151} l’indication, dans la tombe\\
d’Osiris\gls{CoEg_pers_00000151}, de quelques salles éboulées. J’ai oublié de noter, dans\\
mon programme des travaux qui restent à faire, le déblaiement\\
de ces salles. Je les ai bien nettoyées jusqu’à un mètre du sol,\\
mais pas assez pour être sûr qu’ils n’y reste rien. Il existe là\\
en effet d’énorme rochers qui recouvrent peut-être des monuments\\
précieux et que j’ai craint de faire sauter. Je crois bien que\footnote{Mariette\gls{CoEg_pers_00000001} avait écrit «~qu'~», mais a biffé l'apostrophe et complété en «~que~».} des\\
fouilles plus attentives dans cette partie du Sérapéum\gls{CoEg_place_00000004} pourront\\
ne pas être improductives.
{\footnotesize \begin{center} {[2\textsuperscript{e} page, v\textsuperscript{o}]}\end{center}}
\indent J’ai à vous remercier beaucoup, Monsieur, à vous remercier du\\
fond de mon cœur de ce que vous avez bien voulu [pour ma femme\gls{CoEg_pers_00000005} ?]\footnote{Si «~ma~» est assez clair, le premier mot pourrait se lire «~fait~».}.\\
Vous savez bien que mon dévouement et celui de toute ma famille\\
vous est acquis et je n’ai pas besoin de vous exprimer par de\\
plus longues phrases un sentiment que vous savez sincère. Je suis\\
tout entier à vos ordres et prêt pour vous à aller, si vous le voulez,\\
au bout du monde.\\
\indent Hier j’ai fait cuire des œufs sous le sable. Le soleil nous dévore\\
et le sable est si chaud qu’on ne peut littéralement en tenir une\\
poignée dans la main. Heureusement nous touchons au terme\\
de ces chaleurs accablantes. Le Nil\gls{CoEg_place_00000021} monte et couvre déjà les\\
campagnes~; la fraîcheur vient avec lui. Quel beau pays que\\
l’Egypte\gls{CoEg_place_00000003} et comme le temps des\footnote{Le mot a été inscrit sur d'autres lettres.} Ramsès reviendrait pour lui\\
s’il était à la France\gls{CoEg_org_00000012}. En attendant les Anglais le convoitent\\
bien et ne tarderont pas à en faire leur Algérie\gls{CoEg_place_00000022}. Adieu alors\\
les antiquités pour le Louvre\gls{CoEg_org_00000002}, adieu le Sérapéum\gls{CoEg_place_00000004} que le sable\\
recouvre encore.\\
\indent Présentez, s’il vous plaît, mes civilités à \gls{CoEg_abbr_00000001} de Viel-Castel\gls{CoEg_pers_00000034},\\
à \gls{CoEg_abbr_00000001} de Longpérier\gls{CoEg_pers_00000033}, à \gls{CoEg_abbr_00000001} Villot\gls{CoEg_pers_00000035}, à \gls{CoEg_abbr_00000001} Auguiot\gls{CoEg_pers_00000044}, à \gls{CoEg_abbr_00000001} Sauzay\gls{CoEg_pers_00000043},\\
et à bien d’autres que j’oublie sans doute, car depuis bientôt\\
deux ans j’ai eu le temps de laisser ma pauvre mémoire s’envoler\\
avec le vent du désert. Quant à vous, Monsieur, je n’ai pas\\
besoin de vous renouveler l’assurance de tous mes sentiments\\
de respect. Vous savez que je suis tout à vous
\begin{center} \hspace{5cm}\gls{CoEg_abbr_00000002} Mariette\end{center}
Je vous fais mes excuses pour une bien mauvaise petite boîte qui s’est\\
glissée dans le colis qui vous a été apportée par Batissier\gls{CoEg_pers_00000027}. Cette petite\\
boîte ne contenait que du rebut, et elle a été envoyée par erreur au\\
Caire\gls{CoEg_place_00000010}.\\
\indent Faites-moi le plaisir de bien remercier pour moi Batissier\gls{CoEg_pers_00000027} de tous\\
les services qu’il m’a rendu au Caire\gls{CoEg_place_00000010}. Dieu\gls{CoEg_pers_00000036} veuille que je revoie bientôt cet excellent
\begin{flushright}ami.\end{flushright}

\hypertarget{CoEg_Mariette_1852-08-20}{}
\section*{Le 20 août 1852, d’Abousir, à Persigny, ministre de l'Intérieur}
\addcontentsline{toc}{section}{Le 20 août 1852, d’Abousir, à Persigny, ministre de l'Intérieur} 
{\footnotesize \noindent Institution et lieu de conservation~: Archives nationales, Pierrefitte-sur-Seine\\
Cote~: \hyperlink{CoEg_Mariette_ms_001}{20150497/118, dossier 145 «~Mariette, Auguste~»} (n. p.).\\
Support~: une feuille double.\\
Thèmes~: \gls{CoEg_keyword_00000002}~; \gls{CoEg_keyword_00000007}.\\
Note~: une copie non datée de cette lettre se trouve dans les papiers Mariette conservés au sein du fonds Maspero à la bibliothèque de l’Institut de France (ms.~4061 (2), f\textsuperscript{os}~28-29 pour cette lettre). Ces copies ne sont pas de la main de Mariette ni de Maspero, mais correspondent à une écriture ancienne (parmi elles, la lettre copiée la plus récente est de 1869). Elles mentionnent parfois que l’original se trouvait aux archives du Louvre. Cette copie n’est pas toujours très fiable, notamment pour les noms propres.
\begin{center} [1\textsuperscript{re} page, r\textsuperscript{o}]
\end{center}}
\begin{flushright}
Du désert d’Abousyr\gls{CoEg_place_00000008}, le 20 août 1852.
\end{flushright}
\indent A Monsieur\\
\begin{center}Monsieur le Ministre Secrétaire d’État au département\end{center}
\begin{flushright}de l’Intérieur\gls{CoEg_org_00000009}, à Paris\gls{CoEg_place_00000002}.\end{flushright}

\hspace{1cm} Monsieur le Ministre\gls{CoEg_pers_00000037},\\

\indent Par ma lettre en date du 1\textsuperscript{er} Août dernier, j’ai eu l’honneur\\
de vous faire savoir que je venais de m’entendre avec \gls{CoEg_abbr_00000001} le Consul-\\
Général\gls{CoEg_pers_00000024} de France\gls{CoEg_org_00000012} à Alexandrie\gls{CoEg_place_00000006} à l’effet d’expédier, à destination\\
de Marseille\gls{CoEg_place_00000018}, trois colis d’antiquités provenant du Sérapéum\gls{CoEg_place_00000004}\\
de Memphis\gls{CoEg_place_00000005}. – J’avais alors entre les mains une lettre de \gls{CoEg_abbr_00000001} le\\
second \gls{CoEg_entry_00000006}\gls{CoEg_pers_imn} du Consulat-Général\gls{CoEg_org_00000006} qui m’autorisait à vous\\
faire cette déclaration, et d’un autre côté je savais officieusement\\
notre honorable consul-général\gls{CoEg_pers_00000024} tout disposé à seconder mes intentions\\
à l’égard du transport de ces mêmes colis.\\
\indent Mais à l’époque où nous décidions ensemble cette mesure,\\
le vapeur qui devait être chargé du transport n’était pas encore\\
à Alexandrie\gls{CoEg_place_00000006} et nous ne devions pas supposer qu’un empêchement\\
quelconque pût se présenter. C’est pourtant ce qui advint et il\\
résulte de la copie de la lettre de \gls{CoEg_abbr_00000001} Le Moyne\gls{CoEg_pers_00000024} jointe ici\footnote{La lettre en question est recopiée par Mariette\gls{CoEg_pers_00000001} à la main sur la deuxième page de la feuille, en-tête compris.} qu’à\\
son arrivée à Alexandrie\gls{CoEg_place_00000006} le capitaine du bâtiment, consulté\\
à ce sujet, déclara ne pouvoir se charger de l’embarquement
\begin{flushright}de trois\end{flushright}
{\footnotesize \begin{center} {[1\textsuperscript{re} page, v\textsuperscript{o}]}\end{center}}
de trois caisses. J’ai donc à vous prier aujourd’hui de regarder comme\\
non avenue ma lettre du 1\textsuperscript{er} Août~; les antiquités que j’eusse\\
désiré expédier en France\gls{CoEg_place_00000016} le plus promptement possible attendront\\
avec les autres dans les magasins du Consulat-Général\gls{CoEg_org_00000006} le\\
navire de guerre que je vous supplie de nouveau de vouloir\\
bien nous faire envoyer.\\
\indent D’ailleurs, Monsieur le Ministre, vous voudrez bien\\
considérer que la fausse démarche que j’ai faite le 1\textsuperscript{er} août\\
était inévitable, tant par la nécessité où je me trouvais de\\
vous informer de la résolution prise, que par la distance qui\\
me sépare d’Alexandrie\gls{CoEg_place_00000006} et l’arrivée tardive du bateau-poste\\
dans le port de cette ville. La lettre de \gls{CoEg_abbr_00000001} le Consul-Général\gls{CoEg_pers_00000024}\\
porte en effet la date du 4 août~; elle m’est ainsi arrivée\\
le 7, c’est-à-dire le jour même du départ du paquebot qui\\
qui emportait ma lettre d’avis. Je ne crois donc pas qu’il y ait\\
de ma faute si la nouvelle que je me suis hâté de porter à\\
votre connaissance a pu exposer vos bureaux à des démarches\\
inutiles.\\
\indent J’ai l’honneur d’être avec le plus profond respect,
\begin{center} Monsieur le Ministre,\end{center}
\begin{center}\hspace{5cm} Votre très-humble\\
\hspace{5cm} et très-obéissant serviteur.\\
\hspace{5cm} \gls{CoEg_abbr_00000002} Mariette\end{center}
{\footnotesize \begin{center} {[2\textsuperscript{e} page, r\textsuperscript{o}]}\end{center}}
Copie.\\
Agence et Consulat Général\gls{CoEg_org_00000006}\\
\hspace{3cm} de France\gls{CoEg_org_00000012}\\
\hspace{3cm} en Egypte\gls{CoEg_place_00000003}.
\begin{flushright}Alexandrie\gls{CoEg_place_00000006}, le 4 avril 1852.\end{flushright}
\begin{flushright}Monsieur \gls{CoEg_abbr_00000002} Mariette\gls{CoEg_pers_00000001}, à Abousyr\gls{CoEg_place_00000008}.\end{flushright}

\hspace{1cm} Monsieur,\\

\indent D’après la lettre que vous m’avez fait l’honneur de m’écrire le 29\\
du mois dernier, j’ai prié \gls{CoEg_abbr_00000001} le Commandant du paquebot français qui\\
se trouve actuellement dans le port d’Alexandrie\gls{CoEg_place_00000006} de venir voir les trois\\
caisses que vous désirez faire parvenir aussi promptement que possible\\
en France\gls{CoEg_place_00000016}~; mais ce commandant, après les avoir examinées, m’a\\
dit qu’il n’avait pas à son bord d’appareil assez fort pour soulever\\
et embarquer notamment la caisse \gls{CoEg_abbr_00000010} 40, en un mot, qu’il ne\\
pouvait pas se charger de la prendre à cause de son poids et de\\
sa grandeur~; dans cet état de choses, j’ai pensé qu’il y avait d’autant\\
moins d’inconvénients à suspendre l’envoi des deux autres caisses\\
\gls{CoEg_abbr_00000011} 4 et 7 que, sans doute, un bâtiment de l’État\gls{CoEg_org_00000012} ne devra plus\\
beaucoup tarder maintenant à venir chercher tous vos monuments.\\
Du reste lorsqu’il s’agira de leur départ, je me chargerai volontiers\\
de les adresser à \gls{CoEg_abbr_00000001} l’Agent du Ministère des Affaires Etrangères\gls{CoEg_org_00000007}\\
à Marseille\gls{CoEg_place_00000018} pour les consigner à \gls{CoEg_abbr_00000001} \gls{CoEg_abbr_00000012} Pastré\gls{CoEg_pers_00000046} ....\\
\indent Agréez, Monsieur – etc.
\begin{center} \hspace{5cm} Signé A. Le Moyne \gls{CoEg_pers_00000024}.\end{center}

\hypertarget{CoEg_Mariette_1852-09-03}{}
\section*{Le 3 septembre 1852, d’Abousir, à Persigny, ministre de l'Intérieur}
\addcontentsline{toc}{section}{Le 3 septembre 1852, d’Abousir, à Persigny, ministre de l'Intérieur}
{\footnotesize \noindent Institution et lieu de conservation~: Archives nationales, Pierrefitte-sur-Seine\\
Cote~: \hyperlink{CoEg_Mariette_ms_001}{20150497/118, dossier 145 «~Mariette, Auguste~»} (n. p.).\\
Support~: une feuille double.\\
Thèmes~: \gls{CoEg_keyword_00000012}~; \gls{CoEg_keyword_00000002}.\\
Note~: \begin{itemize} \item La lettre porte, d’une autre main que celle de Mariette, au crayon rouge et au coin supérieur gauche~: «~lettres de \gls{CoEg_abbr_00000001}/Mariette~»~; et au crayon gris~: «~A classer~»~;
\item Une copie non datée de cette lettre se trouve dans les papiers Mariette conservés au sein du fonds Maspero à la bibliothèque de l’Institut de France (ms.~4061 (2), f\textsuperscript{os}~30-33 pour cette lettre). Ces copies ne sont pas de la main de Mariette ni de Maspero, mais correspondent à une écriture ancienne (parmi elles, la lettre copiée la plus récente est de 1869). Elles mentionnent parfois que l’original se trouvait aux archives du Louvre. Cette copie n’est pas toujours très fiable, notamment pour les noms propres. \end{itemize}
\begin{center} {[1\textsuperscript{re} page, r\textsuperscript{o}]}\end{center}}

\begin{flushright} Du désert d’Abousyr\gls{CoEg_place_00000008}, le 3 septembre 1852.\end{flushright}
\indent A Monsieur\\
\begin{center}Monsieur le Ministre Secrétaire d’Etat au Département\\
de l’Intérieur\gls{CoEg_org_00000009}\end{center}
\begin{flushright}à Paris\gls{CoEg_place_00000002}.\end{flushright}

\hspace{1cm}Monsieur le Ministre\gls{CoEg_pers_00000037},\\

\indent J’ai déjà eu souvent l’occasion de vous entretenir de la position\\
difficile qui résulte pour moi des conventions arrêtées au mois de février\\
dernier entre le \gls{CoEg_entry_00000002}\gls{CoEg_pers_00000016} d’Egypte\gls{CoEg_place_00000003} et le gouvernement français\gls{CoEg_org_00000012}. En vertu de\\
ces conventions, mon droit de fouiller ne s’étend pas au-delà du Sérapéum\gls{CoEg_place_00000004}\\
de Memphis\gls{CoEg_place_00000005} et chacun des objets découverts appartient de droit au gouvernement\\
égyptien\gls{CoEg_org_00000008} qui s'en empare aussitôt trouvés et les fait transporter à la Citadelle\gls{CoEg_place_00000023}\\
du Caire\gls{CoEg_place_00000010}. Deux officiers d’état-major de l’armée égyptienne\gls{CoEg_org_00000013} stationnent\\
continuellement sur les lieux, enregistrent jour par jour les résultats obtenus\\
et veillent à ce que rien ne soit détourné. C’est ainsi que, depuis le\\
mois de février jusqu’au mois de juin, j’ai été forcé de livrer à ces agents\\
656 objets antiques.\\
\indent Je viens de vous dire que ces conventions me faisaient une position\\
très-difficile. En effet, d’une part, je ne crois pas devoir vous cacher mon\\
désir d’aller visiter, après l’achèvement des travaux du Sérapéum\gls{CoEg_place_00000004}, les ruines\\
de la Haute-Egypte\gls{CoEg_place_00000020} que je n’ai jamais vues et que, pour moi qui fais\\
profession d’égyptologie, il serait trop dur de ne jamais voir après les avoir\\
approchées de si près~; or un voyage de cette sorte, entrepris en érudit plutôt\\
qu’en touriste, exige toujours quelques petites déblaiements, puisque la plupart\\
des inscriptions de l’Egypte\gls{CoEg_place_00000003} ne peuvent être copiées et étudiées qu’à condition\\
d’écarter le sable qui les couvre, ce qui, depuis près d’une année, est formellement\\
interdit à tous les voyageurs. D’autre part je suis obligé de vous rappeler\\
que les circonstances me forcent à violer ces mêmes conventions arrêtées entre\\
\begin{flushright}les deux\end{flushright}
 {\footnotesize \begin{center} {[1\textsuperscript{re} page, v\textsuperscript{o}]}\end{center}}
les deux gouvernements et que loin de livrer au \gls{CoEg_entry_00000002}\gls{CoEg_pers_00000016} les monuments découverts\\
je lui laisse ceux de ces objets qui me semblent n’avoir aucune valeur, et que\\
j’organise pour les autre un système de contrebande qu’à cause même de sa\\
hardiesse je crains toujours de voir s’écrouler. C’est là, Monsieur le Ministre,\\
ce qui me fait la situation dont je me plains, situation sur laquelle\\
j’appelle toute votre attention, parce qu’elle est très-délicate et en même\\
temps très-périlleuse.\\
\indent Je viens donc vous prier de vouloir bien, dans le cas où vous\\
adopteriez ces vues, vous entendre avec \gls{CoEg_abbr_00000001} le Ministre des Affaires Etrangères\gls{CoEg_pers_00000041} et\\
faire donner au nouveau Consul-Général\gls{CoEg_pers_00000040} de France\gls{CoEg_org_00000012} en Egypte\gls{CoEg_place_00000003} des instructions\\
au nom desquelles cet agent pourrait travailler à faire obtenir, en ce qui\\
me concerne, des conditions un peu plus libérales. Je crois devoir vous faire\\
observer à ce sujet que ce que j’ai l’honneur de vous proposer me paraît d’autant\\
moins dangereux à solliciter du Vice-Roi\gls{CoEg_pers_00000016} que le gouvernement français\gls{CoEg_org_00000012}, en\\
m’envoyant l’ordre exprès de livrer les objets découverts, a reconnu par là\\
même le droit de \gls{CoEg_abbr_00000003}\gls{CoEg_pers_00000016} et a donné en même temps la preuve de son désir\\
d’entretenir avec elle des relations amicales. Les 656 objets que j’ai livrés me\\
paraissent ainsi un argument en notre faveur. – D’un autre côté, peut-être\\
les conditions dans lesquelles nous nous trouvons aujourd’hui ne sont-elles\\
plus les mêmes qu’au mois de février dernier. Mes travaux, vous vous le\\
rappelez, étaient suspendus depuis le 21 novembre, et le 12 septembre auparavant\\
l’ordre m’avait été donné, de la part du Vice-Roi\gls{CoEg_pers_00000016}, de livrer tous les\\
monuments que j’avais en magasin. Mais le Vice-Roi\gls{CoEg_pers_00000016} n’était, en quelque\\
sorte, pour rien dans cette affaire~; il était poussé aux mesures un peu\\
violentes dont je fus alors l’objet par son conseiller ordinaire, \gls{CoEg_abbr_00000001} le\\
Consul-Général anglais\gls{CoEg_pers_00000018}. C’est n’est pas en effet que le \gls{CoEg_entry_00000002}\gls{CoEg_pers_00000016} attache un\\
grand prix aux antiquités qui couvrent son royaume et qu’il ait regardé\\
mes découvertes comme une spoliation de son propre bien : vous savez au\\
contraire avec quelle désolante persévérance ses agents détruisent un à un\\
les vénérables témoins de la grandeur des Pharaons. Ce n’est pas non plus\\
qu’il eût eu sérieusement l’idée, ou de s’approprier mes monuments, ou de\\
m’empêcher de continuer mes travaux~; je crois que si nous avions résolument\\
cédé devant des exigences, en réservant notre recours à l’opinion publique,\\
{\footnotesize \begin{center} {[2\textsuperscript{e} page, r\textsuperscript{o}]}\end{center}}
\noindent nous eussions été moins embarrassés de notre défaite que \gls{CoEg_abbr_00000001} Murray\gls{CoEg_pers_00000018} et lui\\
d’une victoire qu’ils ne cherchaient pas, qu’ils ne désiraient pas, parce que\\
{\small le droit seul \textsuperscript{qu’ils invoquaient} ne suffisait pas pour prendre violemment possession des monuments}\\
acquis avec l’argent de la France\gls{CoEg_org_00000012} et l’autorisation régulière du \gls{CoEg_entry_00000002}\gls{CoEg_pers_00000016}\\
lui-même. Ce qu’on voulait au contraire, c’était que par nos fautes nous\\
créassions \sout{[un~?] droit} nous-mêmes un droit nouveau à \gls{CoEg_abbr_00000003}\gls{CoEg_pers_00000016}, et pour cela\\
on a affecté de traiter directement avec moi sans passer par l’intermédiaire\\
obligé du Consul-Général\gls{CoEg_pers_00000024}, afin de profiter de mon inexpérience et de faire\\
naître par ma propre incapacité une raison légitime de garder les monuments\\
confisqués et de m’interdire l’accès du Sérapéum\gls{CoEg_place_00000004}. Deux mois après, les\\
Anglais se fussent installés sur les ruines que, selon eux, nous n’eussions\\
pas su garder et les 515 monuments confisqués eussent bientôt après pris\\
incognito le chemin de Londres\gls{CoEg_place_00000011} avec ceux que la continuation des fouilles\\
eût fait découvrir. Je vous répète donc, Monsieur le Ministre, que tout cela\\
a été le résultat d’une intrigue anglaise~; mais j’ajoute que peut-être\\
aujourd’hui les réclamations de notre consul-général\gls{CoEg_pers_00000040} ne trouveraient pas\\
\gls{CoEg_abbr_00000003}\gls{CoEg_pers_00000016} dans les mêmes dispositions.\\
\indent En tout cas, \gls{CoEg_abbr_00000001} Sabatier\gls{CoEg_pers_00000040} pourra sans doute à son arrivée\\
sonder le terrain et je pense, Monsieur le Ministre, que si le moment venait\\
où ce fonctionnaire croirait pouvoir risquer la demande que j’ai l’honneur\\
de vous soumettre, il devrait d’autant mieux saisir l’occasion que le\\
changement tout récent de \Gls{CoEg_entry_00000004} de la province de Gyzeh\gls{CoEg_place_00000007} va amener\\
un mouvement dans le personnel de mes officier et que je ne sais pas s’il\\
me sera toujours possible d’échapper à la surveillance de ces gens et de\\
sauver au profit du Louvre\gls{CoEg_org_00000002} les monuments nouveaux que la reprise des\\
travaux pourra me faire découvrir.\\
\indent J’ai l’honneur d’être avec le plus profond respect,
\begin{center}Monsieur le Ministre,\end{center}
\begin{center}\hspace{5cm}Votre très-humble\\
\hspace{5cm}et très-obéissant serviteur.\\
\hspace{5cm}\gls{CoEg_abbr_00000002} Mariette\end{center}
{\footnotesize\begin{center} {[2\textsuperscript{e} page, v\textsuperscript{o}]}\end{center}
\noindent \gls{CoEg_abbr_00000008} Après avoir rappelé au commencement de cette lettre les conditions qui nous sont imposées par le\\
gouvernement\gls{CoEg_org_00000008} du \gls{CoEg_entry_00000002}\gls{CoEg_pers_00000016}, je crois devoir vous faire connaître celles que, dans les mêmes circonstances,\\
le Vice-Roi\gls{CoEg_pers_00000016} a consenties en faveur du gouvernement anglais\gls{CoEg_org_00000014}. Il y a un an environ, la Société Géologique\gls{CoEg_org_00000015}\\
de Londres\gls{CoEg_place_00000011} manifesta le désir de faire quelques excavations sur le sol des anciennes capitales de\\
l’Egypte\gls{CoEg_place_00000003}. L’enceinte d’Héliopolis\gls{CoEg_place_00000024} fut explorée l’été passé, et la saison actuelle a été occupée par\\
de grandes fouilles sur l’emplacement de Memphis\gls{CoEg_place_00000005}. Mais, ainsi que j’ai pu m’en assurer par\\
des visites presque quotidiennes, la géologie n’est, à Memphis\gls{CoEg_place_00000005} du moins, que l’accessoire de\\
l’archéologie, et c’est le Musée Britannique\gls{CoEg_org_00000005} qui, surtout, profitera de ces travaux. En effet de\\
longues tranchées ont été ouvertes autour du colosse de Ramsès II\gls{CoEg_pers_00000026} à Myt-Rahyneh\gls{CoEg_place_00000025} et poussées\\
dans toutes les directions à travers les buttes de décombres qui recouvrent Memphis\gls{CoEg_place_00000005}. Chacune\\
de ces buttes a été ouverte, et en ce moment même les travailleurs de la Société\gls{CoEg_org_00000015}, chassés des\\
terres cultivées par l’inondation, viennent s’installer au milieu des sables de la nécropole\\
avec lesquels la géologie ne peut avoir rien à faire. Ces recherches, poursuivies avec\\
persévérance depuis cinq mois, n’ont pas été vaines~; l’emplacement et les limites du temple\\
de Ptah\gls{CoEg_pers_00000047} sont reconnus, les restes d’un nombre incroyable de colosses en granit sont retrouvés,\\
et le British Muséum\gls{CoEg_org_00000005} va s’enrichir d’une cinquantaine de statuettes de toute matière,\\
débris de l’ancienne splendeur du fameux temple de Vulcain\gls{CoEg_pers_00000047}. – Or ces recherches\\
se font toutes exclusivement aux frais du gouvernement égyptien\gls{CoEg_org_00000008}. Aussitôt que l’intention\\
de la Société Géologique\gls{CoEg_org_00000015} a été connue, \gls{CoEg_abbr_00000003}\gls{CoEg_pers_00000016} s’est empressée de mettre à la disposition de\\
\gls{CoEg_abbr_00000001} Murray\gls{CoEg_pers_00000018}, outre \gls{CoEg_abbr_00000004} Hékékyan-\gls{CoEg_entry_00000003}\gls{CoEg_pers_00000048} comme directeur, un capitaine d’état-major comme\\
surveillant-général, trois ingénieurs détachés pour ce service du \gls{CoEg_entry_00000007} des Travaux Publics\gls{CoEg_org_00000016},\\
et des ouvriers en aussi grand nombre qu’il pourrait en désirer. Le traitement de ces\\
agents et des hommes à leurs ordres constitue, avec les frais d’approvisionnement, de\\
campement, de machines, d’outils etc. – une dépense de près de 6 000 \gls{CoEg_abbr_00000013} par mois que le\\
\gls{CoEg_entry_00000002}\gls{CoEg_pers_00000016} supporte en faveur de l’Angleterre\gls{CoEg_org_00000014}. Ajoutez que, loin de contester à \gls{CoEg_abbr_00000001} Murray\gls{CoEg_pers_00000018}\\
le droit de posséder les antiquités provenant de ces fouilles, \gls{CoEg_abbr_00000003}\gls{CoEg_pers_00000016} fait les frais de leur transport\\
jusqu’à Alexandrie\gls{CoEg_place_00000006}. Enfin Hékékyan-\gls{CoEg_entry_00000003}\gls{CoEg_pers_00000048} devant incessamment porter ses recherches sur\\
Abydos\gls{CoEg_place_00000026} et Thèbes\gls{CoEg_place_00000019}, le gouvernement égyptien\gls{CoEg_org_00000008} met à sa disposition un bâteau {[\textit{sic}]} à vapeur.\\
– Tels sont, Monsieur le Ministre, les avantages faits en cette circonstance à l’Angleterre\gls{CoEg_org_00000014}.\\
Je n’établis pas ce parallèle parce que je désire jouir des mêmes facilités que Hékékyan-\\
-\gls{CoEg_entry_00000003}\gls{CoEg_pers_00000048}, et je ne crois pas non plus que la France\gls{CoEg_org_00000012} se soucie beaucoup de la collaboration\\
d’Abbas-\Gls{CoEg_entry_00000002}\gls{CoEg_pers_00000016}. Ce que je demande, c’est que le gouvernement égyptien\gls{CoEg_org_00000008} ne mette\\
pas d’empêchement à mes travaux~; c’est aussi que – maintenant que nous avons\\
suffisamment reconnu le droit de \gls{CoEg_abbr_00000003}\gls{CoEg_pers_00000016} en lui livrant 656 objets – Le Vice-Roi\gls{CoEg_pers_00000016}\\
veuille bien, en étendant mon \gls{CoEg_entry_00000005} à toute l’Egypte\gls{CoEg_place_00000003}, me permettre de disposer\\
des objets que j’aurai découverts. –}

\hypertarget{CoEg_Mariette_1852-09-04}{}
\section*{Le 4 septembre 1852, d’Abousir, vraisemblablement à Nieuwerkerke, directeur général des musées nationaux}
\addcontentsline{toc}{section}{Le 4 septembre 1852, d’Abousir, vraisemblablement à Nieuwerkerke, directeur général des musées nationaux} 
{\footnotesize \noindent Institution et lieu de conservation~: Archives nationales, Pierrefitte-sur-Seine\\
Cote~: \hyperlink{CoEg_Mariette_ms_001}{20150497/118, dossier 145 «~Mariette, Auguste~»} (n. p.).\\
Support~: une feuille double de petit format.\\
Thèmes~: \gls{CoEg_keyword_00000014}~; \gls{CoEg_keyword_00000002}~; \gls{CoEg_keyword_00000007}.\\
Note~: une copie non datée de cette lettre se trouve dans les papiers Mariette conservés au sein du fonds Maspero à la bibliothèque de l’Institut de France (ms.~4061 (2), f\textsuperscript{os}~34-35 pour cette lettre). Ces copies ne sont pas de la main de Mariette ni de Maspero, mais correspondent à une écriture ancienne (parmi elles, la lettre copiée la plus récente est de 1869). Elles mentionnent parfois que l’original se trouvait aux archives du Louvre. Cette copie n’est pas toujours très fiable, notamment pour les noms propres.
\begin{center} {[1\textsuperscript{re} page, r\textsuperscript{o}]}\end{center}}
\begin{flushright}Abousyr\gls{CoEg_place_00000008}, le 4 septembre 1852.\end{flushright}

\hspace{1cm} Monsieur\gls{CoEg_pers_00000002},\\

\indent Ayez la bonté de faire remettre à la\\
Direction des Beaux-Arts\gls{CoEg_org_00000011} les deux plis\\
ci-joints. Comme je désire que leur contenu\\
ne soit pas ignoré de vous, je devrais, ou vous\\
en envoyer un duplicata, ou les rédiger pour\\
vous-mêmes à votre propre adresse. Mais à\\
force d’attendre le courrier de France\gls{CoEg_place_00000016} qui\\
est pourtant arrivé à Alexandrie\gls{CoEg_place_00000006} le 31 du\\
mois dernier, je me trouve acculé à la\\
dernière heure du courrier qui part, et\\
le temps me manque. Veuillez donc prendre\\
connaissance de ces deux lettres, les cacheter,\\
et les envoyer au Ministre\gls{CoEg_pers_00000037} [rature] – . Je serais\\
très-aise, dans le cas où vous approuveriez\\
la demande qui fait l’objet de l’une de\\
ces lettres, que vous voulussiez bien l’appuyer\\
de votre influence.\\
\indent Comme je viens de vous le dire, le\\
courrier ne m’a rien apporté, et il me faut
{\footnotesize \begin{center} {[1\textsuperscript{re} page, v\textsuperscript{o}]}\end{center}}
\noindent remettre à 10 jours le plaisir d’avoir de\\
vos nouvelles. Il me tarde pourtant bien\\
de reprendre les travaux. Heureusement cela\\
ne peut plus tarder et permettez-moi de\\
vous dire que je compte surtout sur vous.\\
\indent Dans le cas où le Ministère\gls{CoEg_org_00000009} aurait\\
de l’argent à m’envoyer, priez \gls{CoEg_abbr_00000001} Fleury\\
Hérard\gls{CoEg_pers_00000049} de me permettre de tirer à vue\\
sur lui, au lieu de me remettre des lettres\\
de crédit sur \gls{CoEg_abbr_00000001} Aïdi\gls{CoEg_pers_00000050}. Quoique celui-ci\\
me fasse ses paiements en pièces de 5 \glspl{CoEg_entry_00000008},\\
qui sont la monnaie principale du\\
pays, il \sout{veut} s’obstine à convertir\\
toujours les \glspl{CoEg_entry_00000008} en piastres et à me\\
payer ces piastres en pièces de cinq\\
francs. Il en résulte un tripotage\\
auquel je n’entends rien. D’un autre\\
côté un négociant du Caire\gls{CoEg_place_00000010}, qui m’est\\
recommandé spécialement par \gls{CoEg_abbr_00000001} Le Moyne\gls{CoEg_pers_00000024},\\
m’offre de me solder en francs, comme\\
si nous étions à Paris\gls{CoEg_place_00000002}. J’aime mille\\
fois mieux cette offre vraisemblable qui\\
me permet de voir clair dans mes
{\footnotesize \begin{center} {[2\textsuperscript{e} page, v\textsuperscript{o}]}\end{center}}
\noindent comptes, et je voudrais pouvoir l’accepter.\\
J’écrirais à \gls{CoEg_abbr_00000001} Fleury Hérard\gls{CoEg_pers_00000049}, si peut-être\\
il n’était déjà trop tard. Dans tous les cas,\\
si vous veniez à le rencontrer, ayez la bonté\\
de l’entretenir de cette affaire sur laquelle\\
d’ailleurs Batissier\gls{CoEg_pers_00000027} vous donnera tous\\
les renseignements désirables.\\
\indent Je clos à la hâte ce billet dont je vous\\
prie d’excuser le désordre. Il se fait\\
tard et le courrier n’attend pas. Veuillez\\
présenter mes civilités à ces Messieurs\\
et en particulier à \gls{CoEg_abbr_00000001} de Rougé\gls{CoEg_pers_00000032}, et\\
croyez-moi
\begin{center} Votre bien dévoué\end{center}
\begin{center}\hspace{1cm}\gls{CoEg_abbr_00000002} Mariette\end{center}
Ayez la bonté de dire à Batissier\gls{CoEg_pers_00000027} que\\
j’attends toujours de ses nouvelles et\\
que je n’ai pas reçu la brochure\footnote{Sans doute \textsc{Brunet de Presle}, Wladimir, «~Mémoire sur le Sérapéum de Memphis\gls{CoEg_bibl_00000001}~», \textit{Mémoires présentés par divers savants à l'Académie des inscriptions et belles-lettres de l'Institut de France. 1\textsuperscript{re} série Sujets divers d'érudition} 2, 1852, p. 552-576~; l'auteur, helléniste, y détaille les mentions du Sérapéum qu'il a trouvé dans les papyrus du Louvre\gls{CoEg_org_00000002} («~Je serais heureux si quelques-uns des textes que je vais citer pouvaient guider M. Mariette\gls{CoEg_pers_00000001} dans ses recherches, comme ils recevront certainement de ses découvertes le plus utile commentaire~»).} de\\
\gls{CoEg_abbr_00000001} Brunet de Presle\gls{CoEg_pers_00000051}. Le fils de\\
\gls{CoEg_abbr_00000001} Le Moyne\gls{CoEg_pers_00000024} (Auguste\gls{CoEg_pers_00000052}) a été en danger\\
de mort~; il va heureusement mieux.\\
Ceci me remet en mémoire ce pauvre\\
\gls{CoEg_abbr_00000001} D’Anastasy\gls{CoEg_pers_00000015} qui se porte mieux
{\footnotesize \begin{center} {[2\textsuperscript{e} page, v\textsuperscript{o}]}\end{center}}
\noindent que jamais et que les bruits du Caire\gls{CoEg_place_00000010}\\
avaient enterré fort mal-à-propos.\\
\indent Les 23 nouveaux colis sont prêts. Si\\
j’avais de l’argent, ils seraient dans huit\\
jours à Alexandrie\gls{CoEg_place_00000006}. Pressez néanmoins\\
l’envoi d’un navire de guerre. Je\\
crois que j’expédierai le tout au Hâvre\gls{CoEg_place_00000027} {[\textit{sic}]}.\\
Avec les 23 colis s’en vont tous les\\
objets que j’ai trouvés jusqu’ici. Il\\
ne reste que les grosses pièces encore\\
sous le sable. Mais vous savez pour\\
quels motifs je les réserve. Demandez\\
à \gls{CoEg_abbr_00000001} de Rougé\gls{CoEg_pers_00000032} s’il veut d’une\\
grande stèle\gls{CoEg_obj_imn} avec le cartouche de\\
Se{[son ?]}-en-ra\footnote{«~Setep-en-Rê~» (\foreignlanguage{translit}{stp-n-Rꜥ}) était un composant fréquent dans le nom royaux, mais la graphie ne semble pas correspondent à «~Setep~»~; il ne suffirait de toute façon pas à identifier le personnage en question.}.

\hypertarget{CoEg_Mariette_1852-11-12}{}
\section*{Le 12 novembre 1852, d’Abousir, vraisemblablement à Nieuwerkerke, directeur général des musées impériaux}
\addcontentsline{toc}{section}{Le 12 novembre 1852, d’Abousir, vraisemblablement à Nieuwerkerke, directeur général des musées impériaux}
{\footnotesize
\noindent Institution et lieu de conservation~: Archives nationales, Pierrefitte-sur-Seine\\
Cote~: \hyperlink{CoEg_Mariette_ms_001}{20150497/118, dossier 145 «~Mariette, Auguste~»} (n. p.).\\
Support~: deux feuilles doubles.\\
Thèmes~: \gls{CoEg_keyword_00000014}~; \gls{CoEg_keyword_00000006}~; \gls{CoEg_keyword_00000002}~;  \gls{CoEg_keyword_00000007}.\\
Note~: la lettre porte, d’une autre main que celle de Mariette, à l’encre et au coin supérieur gauche~: «~Vu~».
\begin{center} {[1\textsuperscript{er} feuillet, 1\textsuperscript{re} page, r\textsuperscript{o}]}\end{center}}
\begin{flushright}Du désert d’Abousyr\gls{CoEg_place_00000008}, le 12 Novembre 1852.\end{flushright}

\hspace{1cm}Monsieur\gls{CoEg_pers_00000002},\\

Je savais par les journaux et les nouvelles de Batissier\gls{CoEg_pers_00000027} votre\\
absence de Paris\gls{CoEg_place_00000002}. Je n’apprends pas plus tôt votre retour que je\\
m’empresse de vous écrire. Non pas que j’aie grand’chose à vous apprendre.\\
Mais je sais qu’en un temps mon long silence vous a paru de\\
l’indifférence, et je tiens par dessus tout à ce que vous ne me jugiez\\
pas tel. Tout au contraire je suis et je reste toujours votre dévoué\\
serviteur et je saisis toutes les occasions de vous le prouver.\\
\indent Il semble que la fatalité poursuit ma malheureuse mission.\\
Les fonds me manquent de nouveau et voici, pour la dixième fois,\\
mes travaux interrompus. Je vous supplie de considérer que\\
l’inaction ici me coûte très-cher, que je suis obligé de vivre dans le\\
désert, d’avoir des gardiens, de faire venir de bien loin mes moyens de\\
subsistance, et que quand vous m’envoyez des fonds, ces fonds me\\
suffisent à peine à payer les dettes que j’ai faites pendant que, faute\\
d’argent, j’ai passé quelques mois à vivre à rien faire dans le désert.\\
C’est ce qui vient d’arriver avec les 3000 \gls{CoEg_abbr_00000013} que \gls{CoEg_abbr_00000001} Fleury Hérard\gls{CoEg_pers_00000049}\\
a mis à ma disposition il y a deux mois. Depuis le mois de mai\\
j’étais sans un liard et du mois de mai au mois de septembre j’ai\\
passé mon temps à emprunter de droite et de gauche sans subvenir\\
aux frais de séjour qui, même dans l’inaction, sont énormes. Les\\
3,000 \gls{CoEg_abbr_00000013} arrivés, il m’a fallu rembourser les sommes empruntées et\\
je me suis trouvé presque sans rien pour reprendre les fouilles. Voilà\\
pourquoi, comme je vous l’annonçais tout-à-l’heure, mes travaux\\
sont de nouveaux interrompus.
{\footnotesize\begin{center} {[1\textsuperscript{er} feuillet, 1\textsuperscript{re} page, v\textsuperscript{o}]}\end{center}}
\indent Du reste, Monsieur, si réellement vous avez l’intention de compléter\\
notre œuvre et de consacrer encore 50 000 \gls{CoEg_abbr_00000013} au Sérapéum\gls{CoEg_place_00000004}, faites, je\\
vous en supplie, que cette affaire se termine le plus tôt possible. Je\\
vous le demande pour moi-même d’abord : un été passé pour la 3\textsuperscript{e} fois\\
dans le désert me serait mortel et je vous assure que je ne me sens\\
plus le courage d’affronter pendant cinq mois 48 degrés Réaumur\\
et un soleil dévorant contre lequel mes chameaux eux-mêmes ne\\
luttent pas impunément. Je vous le demande ensuite pour le succès\\
même de l’entreprise. Le Nil\gls{CoEg_place_00000021} est encore haut, mais l’inondation\\
baisse et dans un mois tous les \glspl{CoEg_entry_00000009} seront occupés à l’ensemencement\\
des terres et c’est avec beaucoup de peine que je réussirai à réunir\\
quelques ouvriers. Les travaux ne pourront donc être repris qu’avec\\
lenteur, sans résultats, et c’est vous-même alors qui m’en\\
gronderez. Je vous renouvelle donc ma prière : ne me laissez pas\\
plus long-temps [\textit{sic}] dans cette position épineuse~; avec des charges\\
inévitables, auxquelles il m’est impossible d’échapper, je me trouve\\
absolument sans ressources et dans ma position ici, alors que tant\\
de regards sont fixés sur moi, j’en suis très souvent honteux.\\
Permettez-moi, Monsieur, de compter sur vous.\\
\indent Je vous prie aussi de faire en sorte que le fameux navire\\
arrive enfin à Alexandrie\gls{CoEg_place_00000006}. Mes colis vous attendent depuis\\
six mois et je donnerais tout au monde pour les voir au\\
Louvre\gls{CoEg_org_00000002}.\\
\indent Voici la note générale de ce que vous avez dû recevoir jusqu’ici :\\
\hspace*{1cm} colis \gls{CoEg_abbr_00000010} 50 – envoyé comme dépêche diplomatique\\
\hspace*{1cm} colis \gls{CoEg_abbr_00000010} 49 – confié à \gls{CoEg_abbr_00000001} Batissier\gls{CoEg_pers_00000027}.\\
\hspace*{1cm} colis \gls{CoEg_abbr_00000010} 4 – confié à Madame Le Moyne\gls{CoEg_pers_00000053}.\\
\hspace*{1cm} colis \gls{CoEg_abbr_00000010} 7 – \hspace*{1cm} 	― idem ―\\
\hspace*{1cm}\begin{tabular}{ r l }
  colis \gls{CoEg_abbr_00000010} & 51 \\
   & 55 \\
   & 51 bis \\
   & 55 bis
\end{tabular} \Bigg\}
\hspace*{1cm} confiés à \gls{CoEg_abbr_00000014} Le Moyne\gls{CoEg_pers_00000024}\\
\hspace*{1cm}Plus une petite caisse confiée à \gls{CoEg_abbr_00000001} {Bray de Buyser\gls{CoEg_pers_00000054}}.
{\footnotesize \begin{center} {[1\textsuperscript{er} feuillet, 2\textsuperscript{e} page, v\textsuperscript{o}]}\end{center}}
\indent Veuillez m’accuser réception de tout ceci. De mon côté je vais\\
vous envoyer les bordereaux du contenu de chaque caisse avec\\
la description sommaire de chaque monument et l’indication\\
de l’endroit où il a été trouvé. Je vous serais très-obligé de\\
garder les bordereaux dans vos archives. A mesure que les\\
caisses partiront, je vous en enverrai [rature] pour chacune d’elles.\\
De cette façon, quand tous les colis seront parvenus à\\
destination, vous aurez mon catalogue complet, tel que je\\
l’ai rédigé sur les lieux.\\
\indent Les découvertes nouvelles que j’ai faites pendant les travaux que\\
je viens d’interrompre me mettent dans un embarras cruel. Je\\
ne sais plus où j’en suis. Jusqu’ici j’avais toujours cru que mes\\
souterrains étaient purement pharaoniques et que la série des\\
tombeaux et des stèles, commençant à Ramsès II\gls{CoEg_pers_00000026}, s’arrêtait\\
à Nectanébo\gls{CoEg_pers_00000045}, c’est-à-dire à la seconde invasion des Perses. Et\\
en effet sur 1000 stèles je n’avais pas trouvé un seul nom\\
ptolémaïque et pas un mot de grec au milieu des innombrables\\
inscriptions dont les murs sont couverts. D’un autre côté, comme\\
chacun des sarcophages sont {[\textit{sic}]} tous beaucoup plus larges que\\
les portes d’entrée de la tombe, j’en devais conclure que les portes\\
sont toutes postérieures à l’introduction des sarcophages. Or\\
ces portes sont aussi couvertes d’inscriptions, et dans ces inscriptions\\
pas un seul nom de Ptolémée. Il me semble donc que je\\
devais avoir raison en soutenant que ma {[porte/série ?]} s’arrêtait\\
aux Perses, que les Perses avaient, sous {[Ochus ?]}\gls{CoEg_pers_00000055}, démoli la\\
tombe d’Apis\gls{CoEg_pers_00000011} et que les Ptolémées en avaient creusé une autre\\
autre part pour leur dieu favori. – Mais voilà l’autre jour\\
qu’en déblayant les souterrains pour la visite de Soliman-\Gls{CoEg_entry_00000002}\gls{CoEg_pers_00000059}\\
et de \gls{CoEg_abbr_00000001} Sabatier\gls{CoEg_pers_00000040}, je trouve deux stèles\gls{CoEg_obj_imn} dédicatoires hérissées de\\
Ptolémées, de Cléopâtres, et d’Arsinoë. – C’étaient les deux
{\footnotesize\begin{center} {[1\textsuperscript{er} feuillet, 2\textsuperscript{e} page, v\textsuperscript{o}]}\end{center}}
\noindent premières stèles ptolémaïques que j’y eusses jamais trouvées. D’où\\
viennent-elles ? ont-elles été apportées par hazard {[\textit{sic}]} du dehors ? Mes\\
souterrains ne commenceraient-ils pas à Ramsès II\gls{CoEg_pers_00000026} pour finir\\
sous les Romains et n’y aurait-il pas eu sous les Grecs \textsuperscript{seulement} une loi\\
qui en interdisais l’entrée aux profanes ? Mais alors si les\\
sarcophages introduits sous les Grecs sont plus grands que les portes\\
qu’on a dû [rature] bâtir après leur introduction, pourquoi ces portes ne\\
portent-elles que des noms de pharaons ? Vous voyez là, Monsieur,\\
tous mes embarras, car, à part la question scientifique, il\\
s’agit là d’une dizaine de 1000 \gls{CoEg_abbr_00000013} de plus ou de moins, puisque\\
si mes souterrains sont ptolémaïques je n’ai plus besoin de\\
dépenser de l’argent pour les chercher autre part. Veuillez\\
donc, je vous prie, demander pour \sout{qu{[?]}} \textsuperscript{moi} à \gls{CoEg_abbr_00000001} de Rougé\gls{CoEg_pers_00000032} qu’il aie la\\
complaisance de me dire, le plus tôt possible, de quelles dates sont\\
les stèles\gls{CoEg_obj_imn} enfermées dans le colis \gls{CoEg_abbr_00000010} 7 que vous devez avoir : les\\
stèles sont démotiques et, outre que je lis à peine un cartouche\\
dans le démotique, je n’ai pas eu le temps de les étudier, pressé comme\\
je le suis de faire disparaître tout à mesure que je le trouve.\\
Je voudrais donc bien que je \gls{CoEg_abbr_00000001} de Rougé\gls{CoEg_pers_00000032} me rendît le service\\
de me dire s’il n’y a pas là des dates et des noms propres ptolémaïques.\\
La question sera alors tranchée pour moi. Les sarcophages auraient\\
été introduits, tous ensemble, sous Ramsès II\gls{CoEg_pers_00000026}, je suppose, et auraient\\
servi au fur et à mesure de la mort d’un Apis\gls{CoEg_pers_00000011}. Quan\sout{d}t\footnote{Le t a été écrit par-dessus le d.} à la\\
destruction de la tombe, elle serait contemporaine de l’abolition\\
même du culte de Sérapis\gls{CoEg_pers_00000060}. Du reste tout ce que je viens de\\
vous dire est un peu, comme on dit, en l’air, et il me faudrait\\
plus d’explications que je n’en puis donner ici pour vous\\
prouver que si j’ai des doutes ils sont réellement fondés.\\
\indent J’ai encore trouvé une salle comme celle des bijoux que\\
vous avez, et inviolée. Malheureusement le roi inconnu qui l’a\\
fait creuser dans la montagne y a mis une économie désespérante
{\footnotesize\begin{center} {[2\textsuperscript{e} feuillet, 1\textsuperscript{re} page, r\textsuperscript{o}]}\end{center}}
\noindent et si j’y ai recueilli des renseignements scientifiques très-importants,\\
le Louvre\gls{CoEg_org_00000002} n’y gagnera rien du tout, que quatre beaux canopes\\
à têtes humaines de près d’un mètre de hauteur et ornés de\\
beaux hiéroglyphes\footnote{Peut-être les canopes N 394 1 A à D\gls{CoEg_obj_00000013} (du règne d'Amenhotep III\gls{CoEg_pers_00000091}) ou N 394 2 A à D\gls{CoEg_obj_00000014} (du règne de Toutânkhamon\gls{CoEg_pers_00000092})~?}.\\
\indent J’attends avec impatience de nouveaux ordres pour les\\
travaux. L’ennui me tue. Je me recommande vivement à vous.\\
Entouré comme je le suis de visiteurs de tous les pays, préoccupé\\
du soin de mettre en ordre mon catalogue, je n’ai pas réussi\\
à écrire ni à \gls{CoEg_abbr_00000001} de Rougé\gls{CoEg_pers_00000032}, ni à \gls{CoEg_abbr_00000001} de Viel-Castel\gls{CoEg_pers_00000034}. Veuillez,\\
s’il-vous-plaît, présenter tous mes respects à ces Messieurs.\\
Comment \gls{CoEg_abbr_00000001} de Rougé\gls{CoEg_pers_00000032} a-t-il trouvé la stèle\gls{CoEg_obj_imn} du colis \gls{CoEg_abbr_00000010} 4 ?\\
comment avez-vous trouvé mes deux statues rouges\footnote{Vraisemblablement le «~Scribe accroupi~»\gls{CoEg_obj_00000008} et une des statues de Sékhemka (A~102\gls{CoEg_obj_00000028} ou A~103\gls{CoEg_obj_00000029} ?).} ? Que de\\
choses, Monsieur, se cachent encore sous [nos ?] sables, et si\\
j’avais de l’argent et la permission comme je vous ferais bien\\
vite le plus beau Musée du monde !\\
\indent Permettez-moi, en terminant, de vous serrer la main dans\\
toute l’affection de mon cœur.
\begin{center}\hspace{5cm} Votre bien dévoué\\
\hspace{5cm} \gls{CoEg_abbr_00000002} Mariette\end{center}
\gls{CoEg_abbr_00000008} Pour la visite dont je vous ai parlé, j’ai fait nettoyer\\
en entier le grand sarcophage\gls{CoEg_obj_imn} d’Amasis\gls{CoEg_pers_00000058}, en granit rose. Il\\
est vraiment magnifique. \gls{CoEg_abbr_00000001} Linant\gls{CoEg_pers_00000019} a eu la complaisance\\
de le cuber et estime son poids à environ cent mille kilos –\\
le tiers de l’obélisque. Il a en hauteur totale presque 13 pieds.\\
Une bande de beaux hiéroglyphes rehaussés de vert court autour\\
de la cuve. Je ne crois pas qu’il existe au monde un sarcophage\\
plus grand et d’aspect plus saisissant. Aussi viens-je vous
{\footnotesize\begin{center} {[2\textsuperscript{e} feuillet, 1\textsuperscript{re} page, v\textsuperscript{o}]}\end{center}}
\noindent annoncer que je vous en demanderai un jour officiellement le\\
transport, car si vous ne le prenez pas les Anglais le prendront.\\
De même aussi, je vous demanderai à sortir l’autre sarcophage\\
{[décrit ?]}, celui dont vous avez les inscriptions. Il me semble que\\
ces deux colosses, uniques au monde, méritent les honneurs du\\
Louvre\gls{CoEg_org_00000002} et pour ma part je regretterais beaucoup qu’ils n’y\\
arrivassent pas. – Malheureusement vous savez qu’ils ne sont pas\\
à nous et il m’est absolument impossible de vous les faire\\
passer en contrebande ou de les adjoindre à la donation officielle\\
du Vice-Roi\gls{CoEg_pers_00000016}. Je reviens donc sur la demande que je vous ai\\
communiquée il y a deux mois et que j’ai adressée par votre\\
intermédiaire à l’Intérieur\gls{CoEg_org_00000009}. – \gls{CoEg_abbr_00000001} Sabatier\gls{CoEg_pers_00000040} est au Caire\gls{CoEg_place_00000010} et\\
{[rature]} peut-être pourrait-on lui adresser des instructions pour\\
qu’il ait à demander ces deux monuments à \gls{CoEg_abbr_00000003}\gls{CoEg_pers_00000016} J’ai\\
livré maintenant près de 900 objets au gouvernement égyptien\gls{CoEg_org_00000008}\\
et il me semble que le Vice-Roi\gls{CoEg_pers_00000016} doit être content.\\
\indent J’ai reçu un plan calqué et je vous en remercie. J’ai\\
l’intention d’exécuter une carte bien complète de la nécropole\\
de Memphis\gls{CoEg_place_00000005} depuis Abousyr\gls{CoEg_place_00000008} jusqu’à Dashour\gls{CoEg_place_00000028}. Je veux qu’elle\\
soit plus exacte que celle\gls{CoEg_bibl_00000002} de \gls{CoEg_abbr_00000001} Lepsius\gls{CoEg_pers_00000061}. Mais de celle-ci\\
vous ne m’avez envoyé qu’une seule feuille et je voudrais avoir\\
les deux qui sont en relations aux Pyramides d’Abousyr\gls{CoEg_place_00000008} et\\
aux pyramides de Dashour\gls{CoEg_place_00000028}\footnote{Les cartes des nécropoles memphites occupent les pl.~32 (Abousir), 33 (Saqqarah), 34 (Saqqarah-sud et Dahchour-nord) et 35 (Dahchour) des \textit{Denkmäler aus Ägypten und Äthiopien} de Karl Richard Lepsius (Berlin, Nicolaische Buchhandlung, 1849-1859, \textit{Tafelwerke} 1, t.~1).}. Je vous serais par conséquent\\
obligé si vous vouliez bien me les faire calquer et me les\\
envoyer le plus tôt possible.\\
\indent Mes 22 nouvelles caisses attendent toujours ici le moment\\
d’aller rejoindre les 50 qui sont à Alexandrie\gls{CoEg_place_00000006}. Mais je n’ai\\
pas d’argent pour fréter une barque. Les 4 nouveaux canopes
{\footnotesize\begin{center} {[2\textsuperscript{e} feuillet, 2\textsuperscript{e} page, r\textsuperscript{o}]}\end{center}}
\noindent sont emballés et j’attends une occasion pour les expédier\\
en contrebande.\\
\indent Vous avez dû recevoir la stèle de Cambyse\gls{CoEg_pers_00000030} dont je vous ai\\
parlé. En la faisant nettoyer, je me suis aperçu que ce n’est\\
ni l’an 7 ni l’an 23 qu’il faut lire, mais l’an 6. \gls{CoEg_abbr_00000001} de\\
Rougé\gls{CoEg_pers_00000032} vous dira toute l’importance de ce monument, si vilain\\
en apparence. C’est 4 ans après que mourut le bœuf qui\\
succéda à celui que Cambyse\gls{CoEg_pers_00000030} blessa de sa main, et le sarcophage\\
dans lequel furent enfermés les restes de ce jeune Apis\gls{CoEg_pers_00000011} est\\
précisément le petit sarcophage dont vous voyez la place dans\\
mon plan général en face du Rond-Point. J’ai retrouvé 8\\
fragments de la stèle dédicatoire\gls{CoEg_obj_imn} qui est, bien entendu, au nom\\
de Darius\gls{CoEg_pers_00000057}. Il me tarde vivement que tout ici arrive au\\
Louvre\gls{CoEg_org_00000002} et vous verriez alors si, au point de vue de l’art comme\\
au point de vue de la science, vous risquez quelque chose\\
à consacrer encore quelques milliers de francs au déblaiement\\
du Sérapéum\gls{CoEg_place_00000004}.\\
\indent Il y a encore dans les caisses d’Alexandrie\gls{CoEg_place_00000006} 5 statues de la\\
fournée des deux rouges\footnote{Les «~deux rouges~» peuvent se référer au «~Scribe accroupi~»\gls{CoEg_obj_00000008} et à une des statues de Sékhemka (A~102\gls{CoEg_obj_00000028} ou A~103\gls{CoEg_obj_00000029} ?)~; parmi les autres statues annoncées se trouvent peut-être les autres statues de Sékhemka (A~104\gls{CoEg_obj_00000030} ou A~105\gls{CoEg_obj_00000026}, en granit).} que vous avez. Deux de ces cinq sont\\
en granit – et l’une d’elles est d’un travail superbe.\\
\indent Je termine ce long post scriptum en vous priant de\\
nouveau d’agréer tous mes hommages. J’attends avec impatience\\
l’accusé de réception de ce que vous avez et l’avis de \gls{CoEg_abbr_00000001} de\\
Rougé\gls{CoEg_pers_00000032} sur les 39 stèles démotiques\gls{CoEg_obj_imn} du colis \gls{CoEg_abbr_00000010} 7.

\hypertarget{CoEg_Mariette_1852-12-28}{}
\section*{Le 28 décembre 1852, d’Abousir, à Persigny, ministre de l'Intérieur}
\addcontentsline{toc}{section}{Le 28 décembre 1852, d’Abousir, à Persigny, ministre de l'Intérieur} 
{\footnotesize
\noindent Institution et lieu de conservation~: Archives nationales, Pierrefitte-sur-Seine\\
Cote~: \hyperlink{CoEg_Mariette_ms_001}{20150497/118, dossier 145 «~Mariette, Auguste~»} (n. p.).\\
Support~: une feuille double.\\
Thèmes~: \gls{CoEg_keyword_00000012}~; \gls{CoEg_keyword_00000002}~; \gls{CoEg_keyword_00000007}.\\
Notes~: \begin{itemize} \item la lettre porte, d’une autre main que celle de Mariette, à l’encre et dans la marge gauche de la première page~: «~[B-A\gls{CoEg_org_00000011}. 16.~?]/7206~»~; et un tampon à l’encre noire~: «~Ministère de l’Intérieur\gls{CoEg_org_00000009}, de l’Agriculture et du Commerce/20 janvier 1853~»~; \item une copie non datée de cette lettre se trouve dans les papiers Mariette conservés au sein du fonds Maspero à la bibliothèque de l’Institut de France (ms.~4061 (2), f\textsuperscript{os}~36-38 pour cette lettre). Ces copies ne sont pas de la main de Mariette ni de Maspero, mais correspondent à une écriture ancienne (parmi elles, la lettre copiée la plus récente est de 1869). Elles mentionnent parfois que l’original se trouvait aux archives du Louvre. Cette copie n’est pas toujours très fiable, notamment pour les noms propres.\end{itemize}
\begin{center} {[1\textsuperscript{re} page, r\textsuperscript{o}]}\end{center}}
\begin{flushright} Du désert d’Abousyr\gls{CoEg_place_00000008}, le 28 Décembre 1852.\end{flushright}
\indent A Monsieur
\begin{center}Monsieur le Ministre, Secrétaire d’État au Département de\end{center}
\begin{flushright}l’Intérieur\gls{CoEg_org_00000009}.\end{flushright}

\hspace{1cm} Monsieur le Ministre\gls{CoEg_pers_00000037},\\

\indent J’ai eu souvent occasion de vous entretenir de la donation, faite par le\\
Vice-Roi\gls{CoEg_pers_00000016} d’Egypte\gls{CoEg_place_00000003} en faveur de la France\gls{CoEg_org_00000012}, de 513 des monuments découverts\\
dans l’enceinte du Sérapéum\gls{CoEg_place_00000004} de Memphis\gls{CoEg_place_00000005}. Cette donation eut lieu en février 1852,\\
ou plutôt c’est à cette époque que le \Gls{CoEg_entry_00000007}\gls{CoEg_org_00000008} en fit passer les titres officiels à\\
\gls{CoEg_abbr_00000001} l’Agent et Consul-Général\gls{CoEg_pers_00000024} de France\gls{CoEg_org_00000012}.\\
\indent Conformément aux instructions que vous m’avez transmises alors, j’ai\\
immédiatement procédé à l’emballage de ces antiquités, et j’ai l’honneur de\\
vous annoncer que 90 colis sont aujourd’hui à votre disposition.\\
\indent De ces 90 colis, 9 doivent être à Paris\gls{CoEg_place_00000002},\\
\indent \hspace{2cm}48 sont en dépôt dans les magasins du Consulat-Général\gls{CoEg_org_00000006}\\
\indent \hspace{3cm} de France\gls{CoEg_org_00000012} à Alexandrie\gls{CoEg_place_00000006},\\
\indent \hspace{2cm} 4 sont en dépôt au Caire\gls{CoEg_place_00000010},\\
\indent \hspace{2cm} 29 enfin sont encore sous ma main.\\
\indent Les 33 derniers iront sous peu se joindre à ceux qui sont à Alexandrie\gls{CoEg_place_00000006}\\
depuis le mois de Mai dernier, et vers la fin de Janvier prochain, la collection\\
de toutes les caisses que nous conservons encore en Egypte\gls{CoEg_place_00000003} sera, dans cette dernière
\begin{flushright}ville,\end{flushright}
{\footnotesize\begin{center} {[1\textsuperscript{re} page, v\textsuperscript{o}]}\end{center}}
\noindent ville, toute prête à partir pour France\gls{CoEg_place_00000016}. – Je vous prie donc, Monsieur le\\
Ministre, de vouloir bien faire donner des ordres pour qu’un bâtiment de\\
l’Etat\gls{CoEg_org_00000012} vienne les y prendre.\\
\indent Quant au contenu du colis, il est de 490 objets, – du moins pour le\\
gouvernement égyptien\gls{CoEg_org_00000008} qui les a fait vérifier par des commissions \textit{ad hoc} envoyées\\
du Caire\gls{CoEg_place_00000010}. Nous avons encore droit par conséquent à 23 objets qui sont tous de\\
fortes dimensions et dont l’expédition ne pourra être faite qu’ultérieurement.\\
Dès que ces 23 nouvelles caisses seront confectionnées, je m’empresserai de vous\\
en donner avis.\\
\indent Mais les 90 colis achevés ne contiennent pas seulement 490 objets.\\
Je joins ici, sur 90 feuilles, l’état général de tous les monuments qui\\
forment mon premier envoi, et vous y verrez que le total se monte à\\
4026. – La liste de \gls{CoEg_abbr_00000003}\gls{CoEg_pers_00000016} est donc dépassée de 3536 objets. – Ceci,\\
Monsieur le Ministre, résulte de la décision que j’ai cru devoir prendre\\
d’éluder en partie les conditions consenties au mois de février dernier entre\\
le gouvernement français\gls{CoEg_org_00000012} et le gouvernement égyptien\gls{CoEg_org_00000008}. La plus sévère de\\
ces conditions m’imposait en effet l’obligation de livrer au Vice-Roi\gls{CoEg_pers_00000016} toutes\\
celles des antiquités découvertes ou à découvrir qui ne serait pas comprises\\
dans la liste des 5153, et j’ai pensé qu’exécuter à la lettre cette condition serait\\
manquer au mandat même que vous m’avez confié. L’évènement {[\textit{sic}]} a justifié\\
mes prévisions. Forcé par les circonstances et désireux d’ailleurs de ne pas\\
donner au gouvernement égyptien\gls{CoEg_org_00000008} raison de se plaindre, j’ai effectivement\\
livré aux officiers turcs qui surveillent mes fouilles pour le compte du\\
Vice-Roi\gls{CoEg_pers_00000016} un millier environ de mauvais objets qui passent ici pour\\
l’ensemble des monuments découverts depuis février 1852 et que les agents\\
égyptiens croient d’une grande valeur précisément parce qu’ils viennent de\\
moi et qu’ils savent par les journaux l’importance que la France\gls{CoEg_org_00000012} elle-\\
même leur accorde. Or j’ai le regret de vous annoncer que tous ces monuments\\
sont aujourd’hui perdus, les uns pour nous, les autres pour tout le monde.
{\footnotesize\begin{center} {[2\textsuperscript{e} page, r\textsuperscript{o}]}\end{center}}
\noindent Les premiers ont été donnés à Fuad-\gls{CoEg_entry_00000010}\gls{CoEg_pers_00000062} à son passage au Caire\gls{CoEg_place_00000010}~; ce sont les\\
sphinx, les statues, les momies et quelques autres gros morceaux de la\\
collection. Les seconds, déposés dans un vestibule du Ministère de l’Instruction\\
Publique\gls{CoEg_org_00000017}, gisent au milieu des ustensiles de ménages des employés subalternes\\
de cette administration, et je crois pouvoir affirmer que le recensement de\\
ces antiquités n’en ferait plus reconnaître une seule dans l’état de conservation\\
où elle était lorsque je l’ai livrée aux officiers surveillants. Tous d’ailleurs,\\
transportés du Sérapéum\gls{CoEg_place_00000004} au Caire\gls{CoEg_place_00000010} sans aucune espèce de soin, abandonnés\\
le plus souvent sur la route pendant des jours et même des mois entiers,\\
ne sont arrivés au Ministère\gls{CoEg_org_00000017} que couverts de boue, mutilés ou brisés. J’ai\\
donc lieu de m’applaudir d’avoir gardé par devers moi, en contrebande,\\
ceux des monuments qui ont quelque valeur, et vous avez pu du reste,\\
Monsieur le Ministre, juger déjà par vous-même de l’opportunité de la\\
décision que j’ai prise si vous avez vu ceux des objets que j’ai réussi à\\
faire passer à Paris\gls{CoEg_place_00000002}. Aucun de ces objets ne figure sur la liste officielle\\
des 513, et je considérerais comme un malheur pour le Louvre\gls{CoEg_org_00000002}, comme un\\
malheur irréparable pour la science, qu’ils eussent été livrés au gouvernement\\
égyptien\gls{CoEg_org_00000008}, et que nous en eussions ainsi été privés à tout jamais. – Telles\\
sont les raisons pour lesquelles les 91\footnote{Le texte de la \hyperlink{CoEg_Mariette_1853-01-01}{lettre du 1\textsuperscript{er} janvier 1853} donne le chiffre de 90, qui est plus cohérent avec ce qui précède.} caisses prêtes, quoique ne contenant\\
pour tous que 490 objets, en renferment réellement 4026.\footnote{À partir de «~La liste de \gls{CoEg_abbr_00000003}\gls{CoEg_pers_00000016}~» et jusqu'à «~en renferme réellement 4026~», le texte est copié presque à l'identique dans la \hyperlink{CoEg_Mariette_1853-01-01}{lettre du 1\textsuperscript{er} janvier 1853}.}\\
\indent J’ai l’honneur d’être avec le plus profond respect,\\
\begin{center}Monsieur le Ministre,\\
\hspace{5cm} Votre très-humble\\
\hspace{5cm} et très-obéissant serviteur.\\
\hspace{5cm} \gls{CoEg_abbr_00000002} Mariette\end{center}
\indent La surveillance dont je suis ici l’objet m’engage à vous prier de ne laisser\\
donner aucune publicité à l’arrivée des caisses à Paris\gls{CoEg_place_00000002}.\\
\indent Vous remarquerez que la séries des factures ci-jointes commence à 1 et finit
\begin{flushright}à\end{flushright}
{\footnotesize\begin{center} {[2\textsuperscript{e} page, v\textsuperscript{o}]}\end{center}}
\noindent à 88~; mais les deux caisses 51 bis et 55 bis complètent les 90 colis.\\
\indent Comme les caisses doivent arriver et être ouvertes au Louvre\gls{CoEg_org_00000002}, je vous\\
serais obligé si vous vouliez bien faire passer le dossier qui accompagne\\
le présent rapport à \gls{CoEg_abbr_00000001} le Directeur Général\gls{CoEg_pers_00000002} des Musées Impériaux\gls{CoEg_org_00000001}.

\hypertarget{CoEg_Mariette_1853-01-01}{}
\section*{Le 1\textsuperscript{er} janvier 1853, d’Abousir, vraisemblablement à Nieuwerkerke, directeur général des musées impériaux}
\addcontentsline{toc}{section}{Le 1\textsuperscript{er} janvier 1853, d’Abousir, vraisemblablement à Nieuwerkerke, directeur général des musées impériaux}
{\footnotesize
\noindent Institution et lieu de conservation~: Archives nationales, Pierrefitte-sur-Seine\\
Cote~: \hyperlink{CoEg_Mariette_ms_001}{20150497/118, dossier 145 «~Mariette, Auguste~»} (n. p.).\\
Support~: une feuille double.\\
Thèmes~: \gls{CoEg_keyword_00000012}~; \gls{CoEg_keyword_00000002}~; \gls{CoEg_keyword_00000007}.\\
Notes~: \begin{itemize} \item la lettre porte, d’une autre main que celle de Mariette, à l’encre et au coin supérieur gauche~: «~Vu~», suivie de ce qui ressemble peut-être à un «~V~»~;
\item une copie non datée de cette lettre se trouve dans les papiers Mariette conservés au sein du fonds Maspero à la bibliothèque de l’Institut de France (ms.~4061 (2), f\textsuperscript{os}~39-42 pour cette lettre). Ces copies ne sont pas de la main de Mariette ni de Maspero, mais correspondent à une écriture ancienne (parmi elles, la lettre copiée la plus récente est de 1869). Elles mentionnent parfois que l’original se trouvait aux archives du Louvre. Cette copie n’est pas toujours très fiable, notamment pour les noms propres. \end{itemize}
\begin{center} {[1\textsuperscript{re} page, r\textsuperscript{o}]}\end{center}}
\begin{flushright}Du désert d’Abousyr\gls{CoEg_place_00000008}, le 1\textsuperscript{\underline{er}}  Janvier 1853.\end{flushright}
\hspace{1cm} Monsieur\gls{CoEg_pers_00000002},\\

\indent J’ai enfin terminé, il y a trois ou quatre jours seulement, ce que j’appelle\\
mon premier envoi. Il se compose de 90 caisses que je tiens dès-à-présent\\
à votre disposition. De ces 90 caisses\\
\hspace*{2cm} 9 doivent être chez vous au Louvre\gls{CoEg_org_00000002}\\
\hspace*{2cm} 48 sont en dépôt dans les magasins du Consulat -\\
\hspace*{4cm} - Général\gls{CoEg_org_00000006} de France\gls{CoEg_org_00000012} à Alexandrie\gls{CoEg_place_00000006}\\
\hspace*{2cm} 4 sont en dépôt au Caire\gls{CoEg_place_00000010}\\
\hspace*{2cm} 29 enfin sont encore sous ma main.\\
\indent Ces 33 dernières iront sous peu se joindre à celles\footnote{Mariette\gls{CoEg_pers_00000001} avait écrit «~ceux~» et a réécrit par-dessus la fin du mot.} qui sont à Alexandrie\gls{CoEg_place_00000006}\\
depuis le mois de mai dernier, et vers la fin de Janvier prochain, ou\\
plutôt de Janvier courant, la collection de toutes les caisses que vous\\
conservez encore en Egypte\gls{CoEg_place_00000003} sera, dans cette dernière ville, toute prête\\
à partir pour France\gls{CoEg_place_00000016}.\\
\indent Je viens de vous dire que j’appelais ces 90 colis mon premier envoi.\\
Je parle ainsi eu égard aux 513 monuments que nous a donnés le\\
Vice-Roi\gls{CoEg_pers_00000016}. Je ne vous envoie pas en effet la totalité de ces 513\\
objets, puisque les 90 colis ensemble sont censés n’en contenir\\
que 490 ainsi qu’il résulte de procès-verbaux dressés par les agents\\
turcs. Mon premier envoi se compose donc, officiellement, de\\
490 monuments, et mon second envoi se composera par conséquent\\
de 23 objets seulement qui épuiseront ainsi la liste de \gls{CoEg_abbr_00000003}\gls{CoEg_pers_00000016} –\\
Dès que ces 23 nouvelles caisses seront confectionnées, je vous en\\
donnerai avis, tout en vous avertissant dès aujourd’hui qu’elles\\
ne peuvent être prêtes avant quelques mois d’ici.
{\footnotesize\begin{center} {[1\textsuperscript{re} page, v\textsuperscript{o}]}\end{center}}
\indent Mais vous pensez bien que les 90 colis achevés contiennent, \textit{pour\\
nous seuls}, autre chose que 490 monuments. J’envoie en effet aujourd’hui\\
même, par l’entremise du Consul-Général\gls{CoEg_pers_00000040}, l’état du contenu de\\
chacune de ces caisses (état adressé pour vous à \gls{CoEg_abbr_00000001} le Ministre de\\
l’Intérieur\gls{CoEg_pers_00000037} et que je vous prie de réclamer) et vous y verrez que le\\
total des objets emballés se monte à 4026. En voici le détail\\
approximatif :\\
\begin{tabular}{ l c r }
  Statues de divinités & – en bronze	– & 1170 \\
 & – en d’autres matières 	– &	110.\\
  Statues de rois	&	–	&	2\\
Sphinx de rois	&	–	&	9\\
Statues de princes	& –	&	72.\\
Statues de particuliers	& – & 15\\
Statues funéraires de tout genre &	–	&	1596\\
Stèles	&	–	&	763\\
Tables à libations	& – &	11\\
Vases Canopes	&	–	&	12.\\
Médailles et monnaies	& –	&	59.\\
Vases à inscriptions	& –	&	7.\\
{[Animaux ?]} en pierre employés comme objets d’art &.&8.\\
Objets divers.&& 192\\
&&4026
\end{tabular}\\
\indent La liste officielle de \gls{CoEg_abbr_00000003}\gls{CoEg_pers_00000016} est donc dépassée de 3536 objets qui\\
sont ainsi de la contrebande. – Ceci, Monsieur, résulte de la décision\\
que j’ai crue devoir prendre d’éluder les conditions consenties au mois\\
de février dernier entre le gouvernement français\gls{CoEg_org_00000012} \& le gouvernemement\\
égyptien\gls{CoEg_org_00000008}. La plus sévère de ces conditions m’impose en effet l’obligation
{\footnotesize\begin{center} {[2\textsuperscript{e} page, r\textsuperscript{o}]}\end{center}}
\noindent de livrer aux agents du Vice-Roi\gls{CoEg_pers_00000016} toutes celles des antiquités découvertes ou\\
à découvrir qui ne seraient pas comprises dans la liste des 513, et\\
j’ai pensé qu’exécuter à la lettre cette condition serait manquer au\\
mandat même qui m’a été confié. L’évènement {[\textit{sic}]} a justifié mes\\
prévisions. Forcé par les circonstances et désireux d’ailleurs de ne\\
pas donner au gouvernement égyptien\gls{CoEg_org_00000008} raison de se plaindre, j’ai effecti-\\
-vement livré aux officiers turcs qui surveillent mes fouilles pour\\
le compte du Vice-Roi\gls{CoEg_pers_00000016} un millier environ de mauvais objets qui\\
passent ici pour l’ensemble des monuments découverts depuis février\\
1852 et que les agents égyptiens croient d’une grande valeur précisément\\
parce qu’ils viennent de moi et qu’ils savent par les journaux l’impor-\\
-tance que la France\gls{CoEg_org_00000012} elle-même leur accorde. Or j’ai le regret de vous\\
annoncer que tous ces monuments sont aujourd’hui perdus, les uns\\
pour nous, les autres pour tout le monde. Les premiers ont été\\
donnés à Fuad-\gls{CoEg_entry_00000010}\gls{CoEg_pers_00000062} à son passage au Caire\gls{CoEg_place_00000010}~; ce sont les sphinx,\\
les statues, les momies et quelques autres gros morceaux de la collection.\\
Les seconds, déposés dans un vestibule de ce qu’on appelle le\\
Ministère de l’Instruction Publique\gls{CoEg_org_00000017}, gisent au milieu des ustensiles\\
de ménages des employés subalternes de cette administration, et je\\
crois pouvoir affirmer que le recensement de ces antiquités n’en\\
ferait plus reconnaître une seule dans l’état de conservation où elle\\
était lorsqu’on l’a prise de mes mains. Tous d’ailleurs, transportés\\
du Sérapéum\gls{CoEg_place_00000004} au Caire\gls{CoEg_place_00000010} sans aucune espèce de soin, abandonnés le\\
plus souvent sur la route pendant des jours \& même des mois entiers,\\
ne sont arrivés au Ministère\gls{CoEg_org_00000017} que couverts de boue, mutilés ou brisés.\\
J’ai donc lieu de m’applaudir d’avoir gardé par devers moi,\\
en contrebande, ceux des monuments qui ont quelque valeur, et\\
vous avez pu du reste juger déjà de l’opportunité de la décision\\
que j’ai prise, en voyant ceux des objets que j’ai réussi à faire passer\\
à Paris\gls{CoEg_place_00000002}. Aucun de ces objets ne figure sur la liste officielle,
{\footnotesize\begin{center} {[2\textsuperscript{e} page, v\textsuperscript{o}]}\end{center}}
\noindent et je considérerais comme un malheur pour le Louvre\gls{CoEg_org_00000002}, comme un\\
malheur irréparable pour la science, qu’ils eussent été livrés au\\
gouvernement égyptien\gls{CoEg_org_00000008} et que nous en eussions ainsi été privés à tout\\
jamais. Demandez à \gls{CoEg_abbr_00000001} de Rougé\gls{CoEg_pers_00000032} ce qu’il aurait dit le jour où il aurait\\
su que les objets d’or, que les belles stèles d’Ouaphris\gls{CoEg_pers_00000063}\footnote{Stèle Louvre N 405\gls{CoEg_obj_00000009}.} et de Scheshonk\gls{CoEg_pers_00000064}\footnote{Stèle Louvre N 413\gls{CoEg_obj_00000010}, N~481\gls{CoEg_obj_00000016}, N~488\gls{CoEg_obj_00000017} ou IM~3736\gls{CoEg_obj_00000018}~?},\\
que les jolies statues rouges\footnote{Vraisemblablement le «~Scribe accroupi~»\gls{CoEg_obj_00000008} et une des statues de Sékhemka (A~102\gls{CoEg_obj_00000028} ou A~103\gls{CoEg_obj_00000029} ?).} qui sont maintenant à Paris\gls{CoEg_place_00000002}, ont été\\
envoyés à la Citadelle\gls{CoEg_place_00000023}, puis brisés, puis donnés à je ne sais qui.\\
Je vous répète donc que j’aurais considéré comme un malheur que\\
j’eusse suivi à la lettre les instructions de [notre ?]\footnote{Ou «~votre~» ?} gouvernement\gls{CoEg_org_00000012}, et telles\\
sont les raisons pour lesquelles les 90\footnote{Le texte de la \hyperref{CoEg_Mariette_1852-12-28}{lettre du 28 décembre 1852} donne «~91~».} caisses prêtes, quoique ne contenant\\
\textit{pour tous} que 490 objets pris sur les 513 donnés par \gls{CoEg_abbr_00000003}\gls{CoEg_pers_00000016}, en\\
renferment réellement 4026.\footnote{À partir de «~La liste de S. A.~» ~et jusqu'à «~en renferme réellement 4026~», le texte est également copié presque à l'identique à l'adresse du ministre de l'Intérieur\gls{CoEg_pers_00000037} dans la \hyperlink{CoEg_Mariette_1852-12-28}{lettre du 28 décembre 1852}.}\\
\indent Vous voyez par le chiffre auquel atteint ma contrebande la justesse\\
de la demande que je vous ai déjà faite de ne rien laisser transpirer\\
dans le public de ce que je vous envoie. J’apprends par une lettre\\
de \gls{CoEg_abbr_00000001} de Rougé\gls{CoEg_pers_00000032} que cette demande a été accueillie~; je vous en\\
remercie. Quand j’aurai les mains vides et que tout sera fini\\
ici, on pourra dire tout ce qu’on voudra. Mais jusque-là\\
je pense qu’il est prudent de faire le mort.\\
\indent Vous pensez bien, Monsieur, que je n’oublie pas le devoir que\\
m’impose la date que j’ai écrite en tête de cette lettre. Recevez, je\\
vous en prie, tous mes souhaits de nouvel an et laissez-moi en\\
même temps profiter de l’occasion pour vous exprimer la reconnaissance\\
dont je suis pénétré et que je vous dois pour les services que vous\\
m’avez rendus et l’intérêt si vif que vous voulez bien porter à\\
mes travaux. Faites agréer aussi l’expression de mon dévouement\\
à \gls{CoEg_abbr_00000001} de Longpérier\gls{CoEg_pers_00000033} et \gls{CoEg_abbr_00000001} de Viel-Castel\gls{CoEg_pers_00000034} et croyez-moi\\
\begin{center}\hspace{5cm}votre bien dévoué serviteur\\
\hspace{5cm}\gls{CoEg_abbr_00000002} Mariette\end{center}

\hypertarget{CoEg_Mariette_1853-05-06}{}
\section*{Le 6 mai 1853, d’Abousir, à Nieuwerkerke, directeur général des musées impériaux}
\addcontentsline{toc}{section}{Le 6 mai 1853, d’Abousir, à Nieuwerkerke, directeur général des musées impériaux} 
{\footnotesize
\noindent Institution et lieu de conservation~: Archives nationales, Pierrefitte-sur-Seine\\
Cote~: \hyperlink{CoEg_Mariette_ms_001}{20150497/118, dossier 145 «~Mariette, Auguste~»} (n. p.).\\
Support~: une feuille double de papier bleu de petit format.\\
Notes~: \begin{itemize} \item La lettre porte, d’une autre main que celle de Mariette, au crayon et au coin supérieur droit : «~6 mai 1853~»~; \item Cette lettre est évoquée par une note du 3 septembre 1853 de Rougé\gls{CoEg_pers_00000032} à Nieuwerkerke\gls{CoEg_pers_00000002}, glissée dans le même dossier~: Rougé\gls{CoEg_pers_00000032} lui renvoyait une lettre de Mariette\gls{CoEg_pers_00000001} (vraisemblablement \hyperlink{CoEg_Mariette_1853-08-10}{celle du 10 août 1853}) qu'il lui avait confiée et en profitait pour lui transmettre également ce mot, qu'il avait décacheté par mégarde : «~il se trouvait avec d’autres notes, dans une petite caisse, où était emballée la belle tête de basalte vert dont il vous parle. Je n’ai vu l’adresse qu’après l’avoir décacheté et je vous en demander excuse~; cela était tout chiffonné dans l’emballage et je ne m’attendais pas à trouver là une lettre pour vous.~» \end{itemize}
\begin{center} {[1\textsuperscript{re} page, r\textsuperscript{o}]}\end{center}}

\hspace{1cm} Monsieur\gls{CoEg_pers_00000002},\\

C’est pour vous que je me décide à enfermer\\
dans cette petite caisse le fragment de statue\gls{CoEg_obj_imn}\\
ci-joint. Vous en jugerez, je pense, la figure\\
digne de toute votre attention. Malgré la dureté\\
de la matière, les moindres détails des chairs y sont\\
indiqués avec une flexibilité de ciseau que, pour\\
moi ignorant des procédés de l’art, je regarde comme\\
admirable.\\
\indent Si cette jolie figure flatte vos yeux, peut-être\\
voudrez-vous la faire tailler en buste et la planter\\
sur un petit piédestal en marbre. Vous pourrez\\
ainsi la garder sur votre bureau comme un souvenir\\
de ma mission qui s’est accomplie par vous \& sous\\
votre administration, et comme un gage en même\\
temps de mon profond dévouement et de ma\\
reconnaissance. J’aimerai toujours, Monsieur,\\
à saisir toutes les occasions, si minimes qu’elles\\
soient, qui peuvent vous prouver que je sais\\
apprécier tout ce que vous avez fait pour moi.\\
\indent La figure est du temps d’Apriès\gls{CoEg_pers_00000063}~; le nom propre
{\footnotesize \begin{center} {[1\textsuperscript{re} page, v\textsuperscript{o}]}\end{center}}
\noindent du personnage n’y est pas. Mais, si mes souvenirs\\
ne me trompent pas, ce doit être le même qu’un\\
certain individu de basalte noir\gls{CoEg_obj_00000011}, agenouillé et\\
tenant devant lui une triade arrangée par les\\
restaurateurs d’antiques, lequel se nomme, je\\
crois, Ensahor\gls{CoEg_pers_00000066}. Ce dernier monument est au\\
Louvre\gls{CoEg_org_00000002}, dans la salle Henry IV.\\
\indent Veuillez, s’il vous-plaît {[\textit{sic}]}, présenter mes civilités\\
à ces Messieurs, et me croire
\begin{center}\hspace{5cm} Votre bien dévoué\\
\hspace{5cm} \gls{CoEg_abbr_00000002} Mariette\end{center}
\indent Du désert d’Abousyr\gls{CoEg_place_00000008}, le 6 Mai 1853.

\hypertarget{CoEg_Mariette_1853-07-30}{}
\section*{Le 30 juillet 1853, du Caire, à Nieuwerkerke, directeur général des musées impériaux}
\addcontentsline{toc}{section}{Le 30 juillet 1853, du Caire, à Nieuwerkerke, directeur général des musées impériaux} 
{\footnotesize \noindent Institution et lieu de conservation~: Archives nationales, Pierrefitte-sur-Seine\\
Cote~: \hyperlink{CoEg_Mariette_ms_001}{20150497/118, dossier 145 «~Mariette, Auguste~»} (n. p.).\\
Support~: une feuille double de papier fin.\\
Notes~: \begin{itemize} \item la lettre porte, au crayon et d’une autre main que celle de Mariette, au coin supérieur gauche «~a classer~», et au coin supérieur droit, de lecture difficile~: «~Rechercher/miss. scientifique/25~» (le premier mot pourrait tout aussi bien être «~Recherches~» et «~scientifique~» en fait au pluriel)~; 
\item le verso de la lettre porte l’adresse~: «~Monsieur/Monsieur le Comte E. de Niewerkerke\gls{CoEg_pers_00000002},/Directeur-Général des Musées Impériaux\gls{CoEg_org_00000001},/Intendant des Beaux-Arts\gls{CoEg_org_00000011} de la Maison/de l’Empereur\gls{CoEg_org_00000010}/au Palais du Louvre/à Paris\gls{CoEg_place_00000002}~»~; on y a aussi ajouté, sur trois lignes, une addition d’une autre main que celle de Mariette (455[+]14[=]469)~;
\item une copie non datée de cette lettre se trouve dans les papiers Mariette conservés au sein du fonds Maspero à la bibliothèque de l’Institut de France (ms.~4061 (2), f\textsuperscript{o}~43 pour cette lettre). Ces copies ne sont pas de la main de Mariette ni de Maspero, mais correspondent à une écriture ancienne (parmi elles, la lettre copiée la plus récente est de 1869). Elles mentionnent parfois que l’original se trouvait aux archives du Louvre. Cette copie n’est pas toujours très fiable, notamment pour les noms propres.\end{itemize}}
\begin{flushright}Le Caire\gls{CoEg_place_00000010}, le 30 Juillet 1853.\end{flushright}

\hspace{1cm} Monsieur\gls{CoEg_pers_00000002},\\

Cette lettre vous sera remise par \gls{CoEg_abbr_00000001} Delaporte\gls{CoEg_pers_00000067}, notre consul au Caire\gls{CoEg_place_00000010}.\\
\indent Dans tous les désagréments qu’au commencement de mes fouilles\\
m’a suscités le mauvais vouloir du gouvernement égyptien, \gls{CoEg_abbr_00000001} Delaporte\gls{CoEg_pers_00000067}\\
a été l’un de ceux qui ont le plus contribué à aplanir les difficultés,\\
et au mois de Juillet 1851 c’est même à \gls{CoEg_abbr_00000001} Delaporte\gls{CoEg_pers_00000067}, à ses démarches\\
réitérées et à son influence que j’ai dû d’obtenir la reprise de mes\\
travaux qu’un ordre exprès du Vice-Roi\gls{CoEg_pers_00000016} avait suspendus.\\
\indent Me voici au Caire\gls{CoEg_place_00000010} aujourd’hui pour faire mes adieux à \gls{CoEg_abbr_00000001}\\
Delaporte\gls{CoEg_pers_00000067} qui part pour France\gls{CoEg_place_00000016}, et je n’aurais pas voulu que\\
\gls{CoEg_abbr_00000001} Delaporte\gls{CoEg_pers_00000067} vous vît \sout{(} sans vous rappeler (car vous les connaissez\\
déjà) les services qu’il m’a rendus.\\
\indent \gls{CoEg_abbr_00000001} Delaporte\gls{CoEg_pers_00000067} rapporte du reste d’Orient une foule d’armes\\
et d’ustensiles qu’il destine à votre Musée Ethnographique\gls{CoEg_org_00000018}, et\\
à tous ces titres réunis j’espère que vous voudrez bien lui faire\\
le bon accueil qu’il mérite.\\
\indent Je retourne tout-à-l’heure à mon désert, car \gls{CoEg_abbr_00000001} Delaporte\gls{CoEg_pers_00000067}\\
vous dira le peu de temps que je reste toujours ici, et si le courrier\\
ne part pas trop tôt, je compte vous écrire un peu plus\\
longuement.\\
\indent Recevez, Monsieur, l’assurance de mon dévouement le plus\\
sincère.
\begin{center}\hspace{5cm}Votre serviteur :\\
\hspace{5cm}\gls{CoEg_abbr_00000002} Mariette\end{center}

\hypertarget{CoEg_Mariette_1853-08-10}{}
\section*{Le 10 août 1853, d’Abousir, vraisemblablement à Nieuwerkerke, directeur général des musées impériaux}
\addcontentsline{toc}{section}{Le 10 août 1853, d’Abousir, vraisemblablement à Nieuwerkerke, directeur général des musées impériaux} 
{\footnotesize \noindent Institution et lieu de conservation~: Archives nationales, Pierrefitte-sur-Seine\\
Cote~: \hyperlink{CoEg_Mariette_ms_001}{20150497/118, dossier 145 «~Mariette, Auguste~»} (n. p.).\\
Support~: une feuille double et une feuille simple, de papier fin.\\
Thèmes~: \gls{CoEg_keyword_00000012}~; \gls{CoEg_keyword_00000002}~; \gls{CoEg_keyword_00000014}.
\begin{center} {[1\textsuperscript{er} feuillet, 1\textsuperscript{re} page, r\textsuperscript{o}]}\end{center}}
\begin{flushright}Du désert d’Abousyr\gls{CoEg_place_00000008}, le 10 août 1853.\end{flushright}

\hspace{1cm} Monsieur\gls{CoEg_pers_00000002},\\

Je désire dans cette lettre, qui sera peut-être la dernière que je vous écrirai\\
d’Egypte\gls{CoEg_place_00000003}, être aussi clair et aussi franc que possible, puisqu’il s’agit (permettez-\\
-moi ce mot pour la première fois) d’intérêts grave pour moi-même et\\
peut-être aussi pour vous.\\
\indent \gls{CoEg_abbr_00000001} de Rougé\gls{CoEg_pers_00000032} a fait imprimer dans le \textit{Moniteur}\gls{CoEg_bibl_00000003}\footnote{Vraisemblablement une référence à «~Ouverture des salles égyptiennes du premier étage, au Louvre. Nouveaux monuments envoyés par M. Mariette\gls{CoEg_bibl_00000004}~», n. p., \textit{Le Moniteur}, 8 juillet 1853, p. 2 : «~L'exploration du Sérapion\gls{CoEg_place_00000004} {[\textit{sic}]} sera bientôt terminée, et M. Mariette\gls{CoEg_pers_00000001} s'empressera de communiquer au public tous les résultats de ses pénibles travaux~».} que ma mission touche à\\
sa fin, et d’un autre côté Batissier\gls{CoEg_pers_00000027} me fait savoir aujourd’hui même\\
d’Alexandrie\gls{CoEg_place_00000006} que vous lui avez écrit afin qu’il m’engageât à ne pas\\
prolonger mon séjour en Egypte\gls{CoEg_place_00000003}.\\
\indent Si j’en crois ces symptômes, je serai bientôt rappelé en France\gls{CoEg_place_00000016} et par conséquent\\
mon départ est prochain.\\
\indent Or au moment de mettre un terme à un travail que j’ai poursuivi pendant\\
trois années, j’éprouve le besoin, non pas de récapituler mon histoire pendant\\
ces trois années, mais de vous dire dans quelles circonstances particulières\\
cet ordre de rentrée m’arrive, et ceci, notez-le bien, pour que vous ne\\
puissiez pas me reprochez, à mon arrivée à Paris\gls{CoEg_place_00000002}, de ne pas vous avoir fait\\
connaître la position dans laquelle je me trouve ici.\\
\indent Je vous déclare d’abord que je suis prêt à rentrer sans vous demander\\
un sou, et quoi qu’en un pays où l’imprévu est tout il soit assez difficile\\
de compter sur {[des ?]} actions de lendemain, j’ai cependant été assez heureux\\
pour arriver juste en même temps au bout de mon argent et au bout\\
de mes travaux. Ainsi jusqu’à présent vous devez être content de moi.\\
\indent Mais si pour rentrer en France\gls{CoEg_place_00000016} je n’ai pas un sou à vous demander,\\
j’ai à vous faire connaître que des circonstances nouvelles et inattendues\\
m’obligent à laisser derrière moi en partant plus de monuments que\\
je ne l’aurais voulu. Voici ces circonstances :\\
\indent A la suite des lettres de \gls{CoEg_abbr_00000001} de Rougé\gls{CoEg_pers_00000032} qui m’engageaient à rechercher\\
un des tombeaux antiques du style de celui dont je vous ai envoyé des\\
échantillons, je me suis convaincu que ces tombeaux ne pouvaient se\\
trouver qu’aux Grandes Pyramides\gls{CoEg_place_00000007} et je me suis adressé à \gls{CoEg_abbr_00000001} Sabatier\gls{CoEg_pers_00000040}\\
pour avoir le \gls{CoEg_entry_00000005} nécessaire.
{\footnotesize\begin{center} {[1\textsuperscript{er} feuillet, 1\textsuperscript{re} page, v\textsuperscript{o}]}\end{center}}
\indent Son Altesse\gls{CoEg_pers_00000016} fut brutale. Consultée par \gls{CoEg_abbr_00000001} Sabatier\gls{CoEg_pers_00000040}, elle répondit qu’elle\\
accordait le \gls{CoEg_entry_00000005}, mais qu’elle savait que je m’appropriais tout ce que\\
je trouvais et qu’elle entendant absolument que dorénavant je n’enlevasse rien. [rature]\\
\indent Jusqu’ici rien que de très naturel. C’est un parti pris contre les Chrétiens\\
et les français en particulier et je ne suis pas consul-général pour le combattre.\\
\indent Mais voici que \gls{CoEg_abbr_00000001} Sabatier\gls{CoEg_pers_00000040} me fait écrire par Batissier\gls{CoEg_pers_00000027} que lui-même\\
tiendra désormais la main à ce que je n’enlève rien et qu’à la première\\
contravention il me fera suspendre mes travaux.\\
\indent Ici les choses s’aggravent. Vous comprenez que je me soucie peu des\\
colères et des ordres de \gls{CoEg_abbr_00000003}\gls{CoEg_pers_00000016} Je maintiens avec obstination le pavillon\\
tricolore sur ma maison et \gls{CoEg_abbr_00000003}\gls{CoEg_pers_00000016} sait qu’au besoin je me protégerais\\
moi-même. D’un autre côté comme, en cet aimable pays, tous les\\
agents de \gls{CoEg_abbr_00000003}\gls{CoEg_pers_00000016}, grands et petits, sont à vendre, je ne vois pas pourquoi\\
je me priverais de les acheter quand j’en ai besoin. Les ordres de \gls{CoEg_abbr_00000003}\gls{CoEg_pers_00000016} ne\\
m’empêcheront donc pas de faire de la contrebande, mais c’est autre\\
chose quand ces mêmes ordres me sont donnés par le consul-général\gls{CoEg_pers_00000040}.\\
\indent Voilà la position nouvelle en face de laquelle je me trouve et si vous\\
vous étonnez qu’en ces circonstances (qui ne m’effrayent pas d’ailleurs)\\
\gls{CoEg_abbr_00000001} Sabatier\gls{CoEg_pers_00000040} non seulement ait laissé faire \gls{CoEg_abbr_00000003}\gls{CoEg_pers_00000016}, mais encore l’aide à\\
faire, je vous répondrai que de mon côté je ne puis vous donner sur ce\\
sujet aucune explication parce que depuis trois ou quatre ans l’Egypte\gls{CoEg_place_00000003}\\
est devenue une mine chargée et que je ne veux pas être celui qui,\\
d’un mot, mettra le feu à la poudre.\\
\indent Vous comprenez maintenant que je sois obligé de laisser des monuments\\
en arrière. Avec du temps je les aurais eus, parce qu’ici tout est caprice\\
et que la loi d’aujourd’hui est oubliée demain. Mais du moment où\\
je suis rappelé et où je n’ai plus le temps d’agir sur {[ces Messieurs ?]}, je\\
ne je puis m’engager à vous expédier des objets ensevelis sous 50 pieds\\
de sables, qu’il faut {[par conséquent ?]} tirer de leurs trous devant tout\\
le monde et qu’au contraire il faut faire arriver ensuite à Alexandrie\gls{CoEg_place_00000006}\\
en contrebande. Si \gls{CoEg_abbr_00000003}\gls{CoEg_pers_00000016} ne le voit pas, le Consulat\gls{CoEg_org_00000006} au moins le\\
verra, et me voici un lièvre poursuivi par deux chasseurs à la fois.
{\footnotesize\begin{center} {[1\textsuperscript{er} feuillet, 2\textsuperscript{e} page, r\textsuperscript{o}]}\end{center}}
\indent Je vous répète donc que je suis prêt à rentrer, mais que je vous avertis\\
en même temps qu’il est devenu impossible de vous expédier tout ce\\
que je vous ai promis et que, bien que j’ai \textit{droit} encore à quelques\\
monuments sur les 513, il m’est impossible de n’en pas laisser derrière moi.\\
\indent Ainsi jusqu’à présent tout est clair et en supposant que j’arrive\\
demain à Paris\gls{CoEg_place_00000002}, vous {[ne m’en ?]} recevrez pas le reproche à la bouche.\\
\indent Cependant en écrivant ces lignes qui sont mon testament quant à\\
cette pauvre et vieille Egypte\gls{CoEg_place_00000003} que j’aime tant, je vous avoue que je\\
me sens involontairement le cou serré. Après tout, Monsieur, mettez\\
-vous à ma place. J’aime l’Egypte\gls{CoEg_place_00000003} parce que j’y ai eu mon premier\\
et peut-être mon dernier succès~; mais j’aime l’Egypte\gls{CoEg_place_00000003} surtout parce\\
qu’il y a des ruines et qu’en me voyant assis au milieu de ces ruines,\\
invoquant de grands noms et de grands hommes chers à mes souvenirs,\\
j’éprouve en même temps des émotions que je ne retrouverai plus.\\
Or voir la Haute-Egypte\gls{CoEg_place_00000020} est devenu mon rêve de tous les jours et\\
si vous voulez me permettre des impressions personnelles, je vous\\
avouerai que je rentrerai en France\gls{CoEg_place_00000016} bien triste parce que je n’aurai\\
pas vu, même en touriste, ces belles ruines que j’aurai pourtant\\
approchées de si près.\\
\indent Un autre chagrin se mêle à celui-là. J’avais arrangé mes petites\\
affaires ainsi : vous me donniez les 25,000 \gls{CoEg_abbr_00000013} que je vous ai demandés, je\\
vous envoyais le sarcophage\gls{CoEg_obj_imn} d’Anubis\gls{CoEg_pers_00000068}, le cercueil\gls{CoEg_obj_imn} d’Entef\gls{CoEg_pers_00000069}, un beau tombeau\\
des Pyramides\gls{CoEg_place_00000007}, trois autres sarcophages en granit, les paons et les lions symboliques\\
grecs, et tout cela expédié je rentrais en France\gls{CoEg_place_00000016} tout-à-fait content. Or\\
en partant maintenant pour France\gls{CoEg_place_00000016} il me semble que je laisse ici la\\
moitié de [rature] moi-même et c’est là ce qui fait mon chagrin.\\
\indent Mais je renonce à tous ces rêves et comme je dois à mes dépens que\\
les circonstances ne s’arrangent pas toujours au gré de mes désirs, je\\
prends mon parti et puisqu’il faut m’en aller, je m’en irai.\\
\indent Vous ai-je jamais, Monsieur, avant aujourd’hui ennuyé de moi-\\
-même, vous ai-je poursuivi, comme tant d’autres l’auraient fait, de
{\footnotesize\begin{center} {[1\textsuperscript{er} feuillet, 2\textsuperscript{e} page, v\textsuperscript{o}]}\end{center}}
\noindent mes réclamations, [rature] me suis-je fait valoir plus qu’il ne convenait pour\\
le succès même de mon entreprise ? Non, Monsieur, c’est précisément\\
ce qui fait mon embarras en ce moment, car cette fois j’ai demande {[\textit{sic}]} à\\
vous faire.\\
\indent Je voudrais que le Louvre\gls{CoEg_org_00000002}, à la fin de ma mission et en témoignage de\\
sa satisfaction, me donnât 5,000 \gls{CoEg_abbr_00000013} et voici ce que j’en ferais.\\
\indent Je consacre d’abord 3000 \gls{CoEg_abbr_00000017} à un voyage dans la Haute-Egypte\gls{CoEg_place_00000020}, et\\
quand les temps deviendront meilleurs, à l’expédition de quelques-unes\\
des caisses que je laisse derrière moi.\\
\indent Les 2000 autres francs seraient donnés, en votre nom et à titre de\\
gratification, à un français qui depuis deux ans est avec moi gratuitement,\\
qui m’aide de sa collaboration, et en se chargeant de tout ce qui est\\
soins matériels, me dit amasser de grands soucis et me permets de me\\
consacrer tout entier aux seules fouilles. Ce Français s’appelle \gls{CoEg_abbr_00000001} Bonnefoy\gls{CoEg_pers_00000070}.\\
\gls{CoEg_abbr_00000001} Bonnefoy\gls{CoEg_pers_00000070} était un ingénieur au service de Son Altesse\gls{CoEg_pers_00000016}, et quand, il\\
y a deux ans, je le recueillis chez moi, il venait d’être, avec tous les\\
employés européens {[du barrage ?]}, renvoyé de son poste sans explication.\\
Or \gls{CoEg_abbr_00000001} Bonnefoy\gls{CoEg_pers_00000070} n’a jamais touché un centime \& ses services sont\\
tous gratuits. Maintenant, au moment de me séparer de lui, je veux\\
lui faire le cadeau dont je vous parle, et s’il n’y a pas absolument\\
droit, au moins il est juste que je fasse ce que je puis pour ne pas passer\\
pour un ingrat.\\
\indent Voilà l’emploi que je voudrais faire de 5000 \gls{CoEg_abbr_00000013} que je vous demande.\\
\indent Je termine ici cette lettre, et en la fermant je vous demande la\\
permission d’être aussi franc qu’un commerçant.\\
\indent Vous savez déjà que du moment où vous me refusez les 25000 \gls{CoEg_abbr_00000013}\\
je dois rentrer le plus tôt possible en France\gls{CoEg_place_00000016}. J’espère donc que vous\\
ne verrez que le désir de bien faire dans la prière que je vais vous adresser
{\footnotesize\begin{center} {[2\textsuperscript{e} feuillet, 1\textsuperscript{re} page, r\textsuperscript{o}]}\end{center}}
\noindent Cette lettre partira du Caire\gls{CoEg_place_00000010} le 15 août et vous arrivera au commencement\\
de septembre. En confiant votre réponse à la poste avant le 18 septembre,\\
je puis avoir cette même réponse ici le 2 octobre. Comme j’ai juste\\
assez d’argent pour aller jusqu’à ce moment, je dois donc vous dire\\
\textit{que \textsuperscript{si}, au courrier du 2 octobre, je n’entends pas parler de vous, je\\
regarderai votre silence comme un refus à l’endroit des 5 000 \gls{CoEg_abbr_00000013} et\\
que je partirai immédiatement et sans attendre}, quelque pénible qu’il\\
soit ou plutôt qu’il pourra être pour ma santé de quitter le feu\\
d’un désert auquel trois ans d’existence m’ont habitué pour la\\
pluie, la neige et le froid de votre hiver de Paris\gls{CoEg_place_00000002}.\\
\indent Agréez, je vous prie, Monsieur, l’assurance de tout mon\\
respect et de tout mon dévouement.
\begin{center}\hspace{5cm} Votre très-humble serviteur\\
\hspace{5cm} \gls{CoEg_abbr_00000002} Mariette\end{center}
\indent Puisqu’il me reste de la place je ne puis m’empêcher de vous communiquer\\
une remarque que je fais à \gls{CoEg_abbr_00000001} de Rougé\gls{CoEg_pers_00000032}. C’est qu’après mon retour\\
le temps sera bientôt venu où, par nos publications comme par le\\
classement et l’exposition des objets nouveaux, Abbas-\Gls{CoEg_entry_00000002}\gls{CoEg_pers_00000016} ne\\
tardera pas à savoir que, sous la protection du consul-général\gls{CoEg_pers_00000040} et\\
avec approbation, je l’ai affreusement \textit{mis dedans} sur le nombre\\
des objets envoyés à Paris\gls{CoEg_place_00000002}. On ne manquera pas d’exploiter cette\\
circonstance et vous pouvez compter sur une de ces colères rancunières\\
qui caractérisent le Vice-Roi\gls{CoEg_pers_00000016}. Il s’ensuivra qu’il n’en sera que plus\\
sévère sur les antiquités et qu’il vous refusera tout ce qu’on pourra\\
lui demander. Maintenant comment ferez-vous pour avoir le\\
sarcophage d’Amasis\gls{CoEg_pers_00000058}, et {[mon ?]}\footnote{Ou peut-être «~un~» ?} beau tombeau des Pyramides\gls{CoEg_place_00000007} et [rature] les\\
objets que je laisse ici. Abbas-\Gls{CoEg_entry_00000002}\gls{CoEg_pers_00000016} est un barbare et soyez sûr\\
qu’il se fera une joie de vous refuser tout. Ne vaut-il pas mieux prendre\\
tout ce que nous pourrons pendant que j’y suis, et user et abuser du\\
\gls{CoEg_entry_00000005} pendant que nous l’avons. Je n’y vois réellement pas et je vous
\begin{flushright}demande pardon de mon écriture.\end{flushright}

\hypertarget{CoEg_Mariette_1853-08-28}{}
\section*{Le 28 août 1853, d’Abousir, vraisemblablement à Nieuwerkerke, directeur général des musées impériaux}
\addcontentsline{toc}{section}{Le 28 août 1853, d’Abousir, vraisemblablement à Nieuwerkerke, directeur général des musées impériaux} 
{\footnotesize \noindent Institution et lieu de conservation~: Archives nationales, Pierrefitte-sur-Seine.\\
Cote~: \hyperlink{CoEg_Mariette_ms_001}{20150497/118, dossier 145 «~Mariette, Auguste~»} (n. p.).\\
Support~: quatre feuilles doubles (après la première, elles sont numérotées par Mariette de 2 à 4 au coin supérieur gauche de la première page).\\
Thèmes~: \gls{CoEg_keyword_00000013}~; \gls{CoEg_keyword_00000012}~; \gls{CoEg_keyword_00000014}~; \gls{CoEg_keyword_00000007}~;
\gls{CoEg_keyword_00000002}~; \gls{CoEg_keyword_00000010}.\\
Note~: une copie non datée de cette lettre se trouve dans les papiers Mariette conservés au sein du fonds Maspero à la bibliothèque de l’Institut de France (ms.~4061 (2), f\textsuperscript{os}~44-47 pour cette lettre). Ces copies ne sont pas de la main de Mariette ni de Maspero, mais correspondent à une écriture ancienne (parmi elles, la lettre copiée la plus récente est de 1869). Elles mentionnent parfois que l’original se trouvait aux archives du Louvre. Cette copie n’est pas toujours très fiable, notamment pour les noms propres..
\begin{center} {[1\textsuperscript{er} feuillet, 1\textsuperscript{re} page, r\textsuperscript{o}]}\end{center}}
\begin{flushright}Du désert d’Abousyr\gls{CoEg_place_00000008}, le 28 août 1853.\end{flushright}

\hspace{1cm} Monsieur\gls{CoEg_pers_00000002},\\

Me voici depuis deux jours de retour d’un voyage à Alexandrie\gls{CoEg_place_00000006}\\
que j’ai entrepris dans des circonstances dont je dois vous rendre compte.\\
\indent Vous savez que sur les 513 objets donnés par le \Gls{CoEg_entry_00000002}\gls{CoEg_pers_00000016}, j’en ai \textit{soi-disant}\\
déjà pris 492, en sorte que nous n’avons plus droit qu’à 21.\\
\indent Ces 21 objets, parfaitement décrits dans la liste, sont tous sous le\\
sable. Ils sont de fortes dimensions et il est de toute impossibilité de les\\
faire passer en contrebande.\\
\indent Mais ces 21 objets n’épuisent pas la liste de ce que j’ai à vous envoyer\\
pour ne rien laisser ici du Sérapéum\gls{CoEg_place_00000004}.\\
\indent Outre ces 21 objets j’en ai encore une centaine, tous enfermés dans 24\\
caisses. – Pour ceux-ci je n’ai aucune espèce de droit.\\
\indent Or, il y a quinze jours encore, voici comment je comptais m’arranger\\
pour expédier tous ces monuments : – j’aurais fait vérifier officiellement\\
ceux auxquels j’ai droit, et pour les autres je les aurais fait écouler\\
peu à peu pour Alexandrie\gls{CoEg_place_00000006} en profitant des occasions qui se seraient\\
présentées.\\
\indent Dans mes calculs cette double opération m’aurait mené à la fin\\
de mon crédit, et je n’aurais rien laissé derrière moi que les sarcophages\\
et aussi les 86 \glspl{CoEg_entry_00000011} démotiques de la porte \gls{CoEg_abbr_00000011} 5\gls{CoEg_obj_00000012}.\\
\indent Mais c’est alors que je reçus de Batissier\gls{CoEg_pers_00000027} la lettre qui m’informait\\
que ma contrebande étant éventée par le \Gls{CoEg_entry_00000002}\gls{CoEg_pers_00000016}, que \gls{CoEg_abbr_00000001} Sabatier\gls{CoEg_pers_00000040}\\
\textit{tiendrait la main} désormais à ce que je me renferme dans les conditions
{\footnotesize\begin{center} {[1\textsuperscript{er} feuillet, 1\textsuperscript{re} page, v\textsuperscript{o}]}\end{center}}
\noindent de mon \gls{CoEg_entry_00000005} et qu’à la première occasion lui-même, \gls{CoEg_abbr_00000001} Sabatier\gls{CoEg_pers_00000040},\\
\textit{me ferait interdire mes travaux.}\\
\indent Je vous avoue que je fus un peu étourdi. Cependant je n’hésitais\\
pas long-temps [\textit{sic}]. J’empruntais à Hékékyan-\gls{CoEg_entry_00000003}\gls{CoEg_pers_00000048} ses outils et en deux\\
jours et deux nuits, la porte \gls{CoEg_abbr_00000010} 5\gls{CoEg_obj_00000012} fut démontée, sciée, emballée. Puis\\
les 24 caisses furent remaniées, les plus petites furent réunies en un\\
seul colis et bientôt je me trouvai à la tête de 28 caisses prêtes à\\
partir, la porte \gls{CoEg_abbr_00000010} 5\gls{CoEg_obj_00000012} comprise.\\
\indent Mais du même coup, mes plans d’argent étaient, comme vous le\\
voyez, dérangés. Le démontage de la porte \gls{CoEg_abbr_00000010} 5\gls{CoEg_obj_00000012} et l’expédition de 28\\
caisses d’une seule fois et à travers des obstacles qu’on ne renverse ici\\
que par l’argent, n’étaient pas prévus dans mon budget, et les 21\\
gros objets risquaient fort de rester en arrière. Cependant comment\\
faire ? Puisque je ne pouvais pas prendre tout faute d’argent, fallait-\\
-il laisser les 21 objets ou les 28 caisses ? Je pensais qu’en ces circonstances\\
le plus pressé était de sauver les 28 caisses auxquelles nous n’avons pas\\
droit et de laisser en place les 21 objets pour lesquels nous avons un\\
droit qui défie le consul\gls{CoEg_pers_00000040} et le \Gls{CoEg_entry_00000002}\gls{CoEg_pers_00000016}. – Si j’enlève les 28 caisses,\\
notre droit existe toujours pour les 12 objets et le départ de ceux-ci\\
n’est plus qu’une question d’argent, laquelle se vide toujours tandis\\
qu’un droit perdu ne s’acquière plus~; si au contraire j’enlève\\
ostensiblement les 21 objets – (et toujours avec la même somme\\
pour chaque opération) – je laisse derrière moi 28 caisses qui ne\\
sont pas à nous et qu’il deviendra de plus en plus impossible\\
d’emporter.\\
\indent J’ai donc cru bien faire en prenant la décision de sauver d’abord\\
ce qui est le plus susceptible d’être atteint par {[le feu ?]}, et de le
{\footnotesize\begin{center} {[1\textsuperscript{er} feuillet, 2\textsuperscript{e} page, r\textsuperscript{o}]}\end{center}}
\noindent sauver promptement, résolument, – en premier lieu parce qu’il ne\\
faut pas attendre que le nouveau système de surveillance de \gls{CoEg_abbr_00000003}\gls{CoEg_pers_00000016}\\
soit organisé~; – en second lieu parce qu’une fois débarrassé\\
de ces 28 caisses il ne me reste du Sérapéum\gls{CoEg_place_00000004} que des monuments\\
qui sont officiellement à nous et que conséquemment la nouvelle\\
surveillance ne peut atteindre.\\
\indent Voilà donc la décision qu’en présence de la position fausse dans\\
laquelle la lettre de Batissier\gls{CoEg_pers_00000027} m’a mis, j’ai cru devoir prendre, et\\
quoique cette décision ait pour résultat immédiat de me forcer à\\
retirer la promesse que je vous ai faite de vous envoyer avec mon\\
crédit les grosses statues que vous connaissez (je n’aurai pu\\
d’ailleurs vous les envoyer \textit{toutes}), je n’espère pas moins que vous\\
m’approuvez. En tous cas j’ai fait pour le mieux.\\
\indent Reste maintenant la mise à exécution de cette décision, et\\
c’est en ceci que vous allez voir que mes dépenses ont dû effectivement\\
doubler.\\
\indent Une barque ordinaire avait ses inconvénients. Les antiquités sont\\
prohibées en Egypte\gls{CoEg_place_00000003} et conséquemment ne peuvent pas voyager. Aussi,\\
à toutes mes autres expéditions, a-t-il fallu, pour la douane\\
à Boulaq\gls{CoEg_place_00000029}, celle d’Atfih\gls{CoEg_place_00000030} et celle d’Alexandrie\gls{CoEg_place_00000006}, un laissez-passer\\
spécial signé du Vice-Roi\gls{CoEg_pers_00000016}. Ici je n’avais pas de laissez-passer\\
à attendre, et comme la douane est très-curieuse, elle n’aurait\\
pas manqué de flairer du premier coup ma contrebande.\\
\indent Au contraire une \gls{CoEg_entry_00000012} de voyageur européen et surtout\\
français est {[exempte ?]}, quand elle le veut bien, des visites de la\\
\sout{dou} douane : on passe outre bravement en laissant les douaniers\\
crier, ou bien on tourne la difficulté en leur payant ce qu’ils appellent\\
un café.
{\footnotesize\begin{center} {[1\textsuperscript{er} feuillet, 2\textsuperscript{e} page, v\textsuperscript{o}]}\end{center}}
\indent Je pris donc une \gls{CoEg_entry_00000012} et j’allai porter moi-même les 28\\
caisses à Alexandrie\gls{CoEg_place_00000006}.\\
\indent Voilà comment, Monsieur, j’ai fait le voyage que je vous\\
annonçais en commençant, et comment les 28 dernières caisses sont\\
allées rejoindre les 92 qui se trouvaient déjà à Alexandrie\gls{CoEg_place_00000006}. En ceci\\
j’ai fait mon possible pour trier le meilleur parti d’une position\\
difficile et plus j’y pense plus je reste convaincu que je ne pouvais\\
faire autrement que je ne l’ai fait. Maintenant mon argent\\
est à peu près dépensé et j’ai le regret de ne pouvoir vous envoyer\\
les gros monuments que je vous avais promis. Mais enfin comment\\
faire autrement, et qu’aurait dit \gls{CoEg_abbr_00000001} de Rougé\gls{CoEg_pers_00000032} \sout{a} si j’avais laissé\\
ici, pour être emportés à la citadelle\gls{CoEg_place_00000023}, les jolies stèles royales\gls{CoEg_obj_imn} qui\\
sont contenues dans les 28 caisses ? Ne valait-il pas mieux sauver\\
ces caisses et réserver les grosses statues qu’on ne peut pas emporter\\
à la citadelle\gls{CoEg_place_00000023} et qui d’ailleurs sont officiellement à nous ? J’espère\\
donc que, dans cette affaire délicate, vous ne me blamerez {[\textit{sic}]} pas\\
de ce que j’ai fait, et que vous apprécierez au contraire la hardiesse\\
que j’ai dû déployer, surtout quand vous saurez qu’en définitive,\\
en partant de Bédréchyn\gls{CoEg_place_00000031}, je ne savais pas du tout si, après ce\\
que m’avais écrit Batissier\gls{CoEg_pers_00000027}, \gls{CoEg_abbr_00000001} Sabatier\gls{CoEg_pers_00000040} voudrait seulement\\
me recevoir, – moi et mes 28 caisses.\\
\indent Quoi qu’il en soit, c’est une affaire finie, et je vous annonce\\
que dès maintenant vous avez à Alexandrie\gls{CoEg_place_00000006} 120 caisses qui vous\\
attendant. Je vous en écrirai d’ailleurs spécialement demain.\\
\indent J’ai à vous entretenir maintenant d’une autre affaire. Comme\\
vous le pensez bien, j’ai profité des 12 heures pendant lesquelles
{\footnotesize\begin{center} {[2\textsuperscript{e} feuillet, 1\textsuperscript{re} page, r\textsuperscript{o}]}\end{center}}
\noindent j’ai vu \gls{CoEg_abbr_00000001} Sabatier\gls{CoEg_pers_00000040} à Alexandrie\gls{CoEg_place_00000006} pour causer avec lui de la\\
lettre de Batissier\gls{CoEg_pers_00000027} et de la position très-gênante dans laquelle\\
les nouveaux ordres de \gls{CoEg_abbr_00000003}\gls{CoEg_pers_00000016} me mettent.\\
\indent A mon grand étonnement, \gls{CoEg_abbr_00000001} Sabatier\gls{CoEg_pers_00000040} m’a déclaré qu’il n’avait\\
pas autorisé Batissier\gls{CoEg_pers_00000027} à m’écrire tout cela, qu’il n’avait pas dit\\
qu’il tiendrait la main à ce que je ne fasse plus de contrebande – etc –\\
qu’à la vérité \gls{CoEg_abbr_00000003}\gls{CoEg_pers_00000016} lui avait bien déclaré qu’elle savait à quoi s’en\\
tenir sur ma fidélité à remplir mes engagements vis-à-vis elle,\\
qu’elle allait me faire surveiller (il est bien temps), – mais que\\
lui, \gls{CoEg_abbr_00000001} Sabatier\gls{CoEg_pers_00000040}, ne s’était pas engagé du tout à prêter la\\
main à \gls{CoEg_abbr_00000003}\gls{CoEg_pers_00000016} – et qu’en résumé je pouvais tout aussi bien qu’avant\\
me livrer à mon métier de fraudeur, seulement que c’était à mes\\
risques et périls.\\
\indent Ainsi ma dernière lettre est, par ce fait, non avenue, et je n’en\\
suis pas fâché. C’est une distraction de Batissier\gls{CoEg_pers_00000027} qui a tout produit,\\
et comme en définitive, cela m’a donné occasion de tirer au clair ma\\
situation qui, en ce qui concerne mes rapports avec le \Gls{CoEg_entry_00000002}\gls{CoEg_pers_00000016}, me\\
semblait s’abstenir de plus en plus, je n’ai pas à me plaindre.\\
J’ai au contraire à m’en louer, car, tout compte fait, si la lettre\\
de Batissier\gls{CoEg_pers_00000027} n’était pas venue éveiller mon attention, la\\
surveillance de \gls{CoEg_abbr_00000003}\gls{CoEg_pers_00000016} se serait organisée autour de moi sans que\\
je m’en aperçusse et il serait venu un temps où le départ du\\
plus petit objet en contrebande serait devenu impossible.\\
\indent Du reste si je retire ce que j’ai dit dans ma dernière lettre sur\\
\gls{CoEg_abbr_00000001} Sabatier\gls{CoEg_pers_00000040}, je n’en persiste pas moins dans mes conclusions quant\\
à Abbas-\Gls{CoEg_entry_00000002}\gls{CoEg_pers_00000016} et à ses tendances anti-françaises. A son point\\
de vue – musulman – il a raison et je suppose qu’il n’est pas plus\\
aise de voir des chrétiens occuper les premiers postes de son pays
{\footnotesize\begin{center} {[2\textsuperscript{e} feuillet, 1\textsuperscript{re} page, v\textsuperscript{o}]}\end{center}}
\noindent que notre Empereur\gls{CoEg_pers_00000071} ne serait satisfait de voir des Anglais ou des\\
Prussiens à la tête de ses administrations, et comme ce sont les\\
français qui, sous Méhémet-Ali\gls{CoEg_pers_00000017}, avaient la haute main sur\\
tout, ce sont les Français qui, sous Abbas-\Gls{CoEg_entry_00000002}\gls{CoEg_pers_00000016}, sont les premières\\
victimes du nouvel ordre des choses. – Cela, il est vrai, n’explique\\
pas et n’excuse pas ses sympathies anglaises. Mais Abbas-\Gls{CoEg_entry_00000002}\gls{CoEg_pers_00000016}\\
n’est pas tenu à beaucoup de suite dans ses idées et on ne devient\\
pas nécessairement logique parce qu’on a en main le sceptre des\\
Sésostris et des Ramsès. C’était bon autrefois.\\
\indent En vous écrivant ma dernière lettre, j’étais sous le poids de\\
telles préoccupations \& de si grands éblouissements produits par\\
cette vilaine {[ophthalmie ?]} qui ne me quitte que pour revenir,\\
que je ne sais pas, non seulement si je vous ai dit tout ce\\
que je voulais vous dire, mais encore si j’ai bien dit le peu\\
que je vous ai dit. Dans tous les cas, pour éviter tout\\
malentendu, je vais vous résumer les parties essentielles de\\
cette lettre.\\
\indent Vous avez dû vous apercevoir, par ma correspondance de ces\\
derniers temps, que, tout en vous avouant que les fouilles du\\
Sérapéum\gls{CoEg_place_00000004} étaient à peu près terminées, je manifestais cependant\\
le désir de ne pas rentrer en France\gls{CoEg_place_00000016} immédiatement. En effet\\
pour que je rentrasse en France\gls{CoEg_place_00000016} avec le contentement de moi-même,\\
je voudrais avoir bien fini les petites choses qui me restent à\\
faire ici, vous avoir expédié quelques bons sarcophages, mes grosses\\
statues, et un bon tombeau comme celui dont vous avez des\\
échantillons. Une fois cela fait, je m’en irai faire mon tour
{\footnotesize\begin{center} {[2\textsuperscript{e} feuillet, 2\textsuperscript{e} page, r\textsuperscript{o}]}\end{center}}
\noindent dans la Haute-Egypte\gls{CoEg_place_00000020}, ce qui est un voyage qui me manquera\\
toujours si je ne le fais pas, et au mois de février prochain vous\\
me verriez bien heureux et n’amenant avec moi aucun regret\\
de ce que je laisse ici. Voilà ce que je voudrais, voilà mon rêve\\
de tous les jours et je considérerais tout cela comme une très-belle\\
fin de ma mission.\\
\indent Pour en arriver là, il suffirait du crédit de 25000 \gls{CoEg_abbr_00000013} que je\\
vous ai demandé. Je ne dis pas que ce crédit me mettrait bien à\\
mon aise~; mais enfin en me retranchant un peu d’un côté et\\
d’autres j’arriverais à mon but.\\
\indent Dans le cas où ce crédit ne pourrait m’être accordé, je sollicite\\
du Louvre\gls{CoEg_org_00000002} un cadeau de 5000 francs. Après tout ce serait bien\\
cruel pour moi de ne rien voir de la Haute-Egypte\gls{CoEg_place_00000020} et je n’y pense\\
qu’avec une vive et sincère douleur. Je voudrais donc employer 3000\\
\gls{CoEg_abbr_00000013} à ce voyage, et réserver les 2000 autres francs pour \gls{CoEg_abbr_00000001} Bonnefoy\gls{CoEg_pers_00000070}.\\
Ce n’est pas que je doive cette somme à \gls{CoEg_abbr_00000001} Bonnefoy\gls{CoEg_pers_00000070}. Je n’ai\\
aucun engagement envers lui et à la rigueur je ne lui dois rien.\\
Mais enfin, comme mes plans ont été depuis long-temps [\textit{sic}] dérangés\\
en ce qui concerne l’emploi des fonds que je pouvais me destiner\\
personnellement, il me serait désagréable de quitter \gls{CoEg_abbr_00000001} Bonnefoy\gls{CoEg_pers_00000070}\\
sans lui rien donner. Au surplus, c’est à votre disposition et\\
je me soumets d’avance à tout ce que vous voudrez bien ordonner.\\
\indent Si maintenant je me suis permis de fixer un terme à la réponse\\
que vous voudrez bien me faire, ce n’est pas que j’ose prendre sur\\
moi de vous poser des conditions. Au contraire vous me rendrez bien\\
cette justice d’avoir toujours subordonné mes désirs à vos volontés.\\
Mais en cette circonstance j’ai dû agir ainsi, parce que dans le\\
cas où vous auriez dû me répondre par un refus et où cette
{\footnotesize\begin{center} {[2\textsuperscript{e} feuillet, 2\textsuperscript{e} page, v\textsuperscript{o}]}\end{center}}
\noindent réponse se serait fait attendre, j’aurais été obligé, pour attendre\\
cette réponse, de faire des dettes qu’il [rature] vous aurait fallu payer.\\
Dans le cas où je n’aurais plus de fonds à espérer de vous, je\\
ne puis donc demeurer en Egypte\gls{CoEg_place_00000003} après le 2 octobre et voilà pourquoi\\
je me suis permis de vous dire que si votre réponse n’était pas\\
arrivée pour cette époque, je regarderais votre silence comme\\
un refus et je serais forcé de rentrer immédiatement en France\gls{CoEg_place_00000016}.\\
\indent Du reste, Monsieur, laissez-moi vous dire que j’espère bien\\
qu’il n’en sera pas ainsi. Si vous saviez que de belles choses\\
il y a encore à faire en Egypte\gls{CoEg_place_00000003} ! et les fouilles coûtent si peu\\
quand on a l’argent devant soi et qu’on peut en disposer à\\
point nommé ! Mais ce ne sont même pas des fouilles que je\\
veux faire maintenant : c’est un simple voyage d’amateur,\\
la plume à la main. Me le refuserez-vous ?
{\footnotesize\begin{flushright}29 août –\end{flushright}}
\indent J’avais laissé le bas de cette lettre en blanc pour le terminer dans la\\
soirée, quand un évènement {[\textit{sic}]} imprévu est venu déranger mes plans.\\
\indent Mon premier mouvement aurait été de n’en rien dire. Je n’aime pas\\
beaucoup à insister moi-même sur les choses qui peuvent me faire valoir\\
et je vous avoue que j’éprouve toujours un certain embarras à raconter des affaires\\
qui, parce qu’elles me sont personnelles, me paraissent ne pas devoir intéresser\\
beaucoup les autres. Cependant, comme c’est la seconde fois que pareille aventure\\
m’arrive et que, en définitive, il est bon et raisonnable que vous sachiez au juste,\\
pour vous et pour moi, à quoi vous en tenir sur ma position exacte ici, je vais\\
me risquer à vous faire le récit de ce fameux évènement {[\textit{sic}]} qui me force à\\
terminer cette lettre autrement que je n’en avais d’abord l’intention.\\
\indent J’ai l’habitude tous les soirs de [rature] monter à cheval et de faire une promenade\\
à travers le désert jusqu’au bord des terres cultivées. Hier au soir je cheminais\\
philosophiquement au milieu des buttes de sable amoncelées par les anciennes
{\footnotesize\begin{center} {[3\textsuperscript{e} feuillet, 1\textsuperscript{re} page, r\textsuperscript{o}]}\end{center}}
\noindent fouilles des Arabes, quand à 50 pas à mon côté gauche éclata un coup de fusil.\\
J’avais la tête à d’autres pensées, et bien que j’aie entendu la balle ou les plombs\\
siffler dans l’air, je ne fis attention à ce coup de fusil que pour me faire\\
remarque à moi-même du peu d’agrément que doit avoir un chasseur de\\
sanglier ou de hyène dans cette nuit obscure. Cependant, tout en marchand,\\
je me mis à réfléchir que pas un \glspl{CoEg_entry_00000009} n’est armé, qu’aucun musulman\\
n’oserait chasser la nuit, et en outre que c’était la première fois de ma vie\\
que je voyais un arabe s’aventurer seul dans l’obscurité au milieu des\\
tombeaux. J’en étais là de ces réflexions, et je commençais à m’inquiéter et\\
à m’étonner, quand tout-à-coup, à dix pas devant moi, j’aperçois un arabe\\
accroupi se dresser subitement, m’ajuster et faire feu. C’était bien et\\
dûment une tentative d’assassinat.\\
\indent L’éclair qui illumina la nuit, la détonation, les cris si singuliers dont\\
l’homme fit suivre son coup de fusil, effrayèrent mon cheval qui se\\
cabra, tourna sur lui-même, et, prenant son élan à la turque, se\\
rua en avant comme un tourbillon.\\
\indent L’Arabe criait toujours, mais je n’étais plus maître de mon cheval\\
qui avait le mors dans les dents. Il ne s’arrêta qu’au village même de Sakkarah\gls{CoEg_place_00000001}.\\
\indent Tel est, Monsieur, l’inconcevable attentat qui a failli, comme vous\\
le voyez, me coûter cher. Quel en est le but, quels en sont les auteurs ? je ne\\
saurais le dire. Ce que j’affirme, c’est [rature] que ce ne sont pas des \glspl{CoEg_entry_00000009} : les\\
\glspl{CoEg_entry_00000009} sont menteurs et voleurs, mais leur genre d’intelligence ne les porte pas\\
à attendre quelqu’un au coin d’une rue pour le tuer et d’ailleurs ils ne se\\
servent jamais de fusil. Sont-ils des Bédouins du désert qui voyant un Européen\\
tout seul, sans armes, sur un cheval, et supposant qu’en sa qualité de chercheur\\
d’or, cet Européen doit en avoir plein ses poches, se sont dit : tuons-le pour\\
avoir son cheval et son or. – La chose est possible parce qu’elle est dans les\\
mœurs de ces gens. Mais cependant les cris qu’a poussés l’homme du second coup\\
de fusil ne sont pas des cris de Bédouins. Je croirais plutôt que ces deux\\
Messieurs sont deux de ces \glspl{CoEg_entry_00000013} dont l’indiscipline est proverbiale, et\\
en effet je me rappelle parfaitement avoir entendu, dans les \glspl{CoEg_entry_00000014} et\\
au milieu des exercices du \gls{CoEg_entry_00000015}, les \Glspl{CoEg_entry_00000013} pousser ces cris étranges\\
dont j’ai encore plein les oreilles.
{\footnotesize\begin{center} {[3\textsuperscript{e} feuillet, 1\textsuperscript{re} page, v\textsuperscript{o}]}\end{center}}
\indent Quoi qu’il en soit, voilà où j’en suis et vous voyez que ce n’est pas très-agréable.\\
\indent Le soir même de l’évènement {[\textit{sic}]}, j’ai eu la visite du secrétaire principal du\\
\Gls{CoEg_entry_00000004} qui était précisément à Sakkarah\gls{CoEg_place_00000001} en tournée d’inspection. Il s’est\\
très-bien conduit. Il a passé sa nuit à faire des recherches dans la montagne\\
et ce matin il a fait arrêter deux individus que la rumeur du village a\\
désignés comme les auteurs du coup. C’est d’abord un Turc établi barbier à\\
Sakkarah\gls{CoEg_place_00000001} depuis un an environ, puis un gros vilain \Gls{CoEg_entry_00000013} qui se\\
grise de hachich et n’en est pas moins contre les chrétiens d’un fanatisme\\
outré. Les deux accusés nient, bien entendu.\\
\indent Quand {[\textit{sic}]} à moi, je vous avoue que cette affaire me laisse dans une indifférence\\
complète. Je serais assez disposé à faire une plainte officielle au \gls{CoEg_entry_00000007}\gls{CoEg_org_00000008}.\\
Mais à quoi cela m’avancera-t-il ? il est évident que le gouvernement\\
égyptien\gls{CoEg_org_00000008} aura des yeux tout paternels pour l’\gls{CoEg_entry_00000013} qui fait partie d’un\\
corps très-redouté ici, et surtout pour le Turc – qui est un Turc. Ce sont alors\\
les pauvres \Glspl{CoEg_entry_00000016} qui paieront pour les coupables qu’on déclarera\\
ne pas avoir trouvés – (c’est la loi qui le veut ainsi) et alors comment voulez-vous\\
que je m’expose à faire pendre ces pauvres diables, sans motif ? Je ne\\
bouge donc pas et si la justice égyptienne me fait demander mon témoignage\\
– ce qui est douteux – je le lui donnerai et voilà tout.\\
\indent Du reste tout dépendra de la manière dont le \Gls{CoEg_entry_00000004} prendra l’affaire.
\begin{flushright}31 août 1853.\end{flushright}
\indent J’ai appris hier matin que le secrétaire du \Gls{CoEg_entry_00000004} avait\\
reconnu mes deux individus innocents et qu’il les avait relâchés –\\
que de plus, en sortant, l’un des deux accusés avait déclaré qu’il\\
allait recommencer.\\
\indent J’avais eu jusqu’alors de la patience~; je vous avoue qu’alors\\
elle m’échappa.\\
\indent Je montai donc à cheval, et j’allai au village dans l’intention\\
de voir moi-même l’arnaoute et le Turc et de leur parler un\\
peu à ma façon.\\
\indent J’entrai dans le village à pied. J’avais à ma ceinture une paire
{\footnotesize\begin{center} {[3\textsuperscript{e} feuillet, 2\textsuperscript{e} page, r\textsuperscript{o}]}\end{center}}
\noindent de gros pistolets et je portais sur l’épaule une carabine de\\
Vincennes\gls{CoEg_place_00000032}, le sabre luisant au bout\footnote{Le «~fusil de Vincennes~» est un modèle produit entre 1759 et 1761 dans cette ville~; sa longueur importante atteignait 2,3 m avec la baïonnette. Au mécanisme complexe et d'entretien délicat, il tomba vite en désuétude et fut abandonné avant même la fin du XVIII\textsuperscript{e} siècle.}, – une vraie tournure\\
d’insurgé.\\
\indent L’\gls{CoEg_entry_00000013} n’y étais pas. Mais j’aperçus un Turc assis sur\\
un banc de pierre au milieu d’une rue, en compagnie d’une nombreuse\\
société et à côté d’un certain pèlerin à turban jaune qui m’a une\\
fois accusé d’avoir donné le mauvais œil à sa maison et d’avoir\\
fait mourir au moyen de ce mauvais œil son âne et son chameau,\\
ce qui fait que, tout sacré qu’il soit, il peut bien avoir trempé\\
dans mon affaire.\\
\indent J’avais la tête montée. Je m’approche du groupe et abaissant\\
militairement mon fusil de Vincennes\gls{CoEg_place_00000032}, je fais sonner l’arme sur\\
le pavé. Puis je m’adresse en ces propres termes à mon individu : fils\\
de Juif, est-ce toi qu’on appelle {[Aessek ?]}\footnote{La dernière lettre pourrait aussi bien être un t ou un h.} le barbier ? – Il me\\
répond : oui – et en même temps il se lève pâle et respectueux, mais\\
digne. Les femmes se mettent à crier et, découvrant leur visage, elles\\
se l’{[inondent ?]} de poussière, car il est évident pour moi que ces gens\\
craignaient que j’allais me faire justice moi-même et exécuter sur\\
place le pauvre diable. Je me contente de lui enjoindre de me suivre,\\
lui et le turban jaune et nous voilà partis pour ma maison,\\
suivis de tout le village.\\
\indent Arrivés chez moi, je dispose sur une table mes deux pistolets, je mets\\
mon fusil dans un coin, et j’entame la discussion. Ce que je leur dis\\
précisément, je n’en sais rien. Tout mon arabe y passa. Je me rappelle\\
seulement qu’à la fin, après leur avoir fait savoir que si je le voulais\\
dans huit jours ils seraient tous les deux partis pour le {[Fezaghan ?]}\footnote{La première lettre pourrait être un Z ou un J~; le h un l.}\gls{CoEg_place_imn},\\
je saisis un pistolet de chaque main, et le leur mettant sur le nez\\
de manière à leur faire sentir le froid du fer, j’ajoutai : maquereaux\\
que vous êtes, si jamais je vois encore l’un de vous dans la montagne,\\
de jour ou de nuit, vous n’aurez pas le temps de faire un pas en\\
avant que je vous aurais {[\textit{sic}]} tués comme deux chiens. –
{\footnotesize\begin{center} {[3\textsuperscript{e} feuillet, 2\textsuperscript{e} page, v\textsuperscript{o}]}\end{center}}
\indent J’étais en colère et mes gens avaient peur. Tout le monde dans la\\
chambre se taisait. Je résolus alors tout-à-coup d’en finir par une\\
scène à la mode du pays.\\
\indent Me tournant vers le Turc, je lui dis : vois-tu là-bas cette\\
porte avec une traverse au milieu ? prends ce \gls{CoEg_entry_00000017} (pièce de cinq\\
francs turque) et vas le {[\textit{sic}]} appliquer sur la porte à l’endroit de la\\
traverse. – Le Turc obéit. J’ajoute : maintenant regarde, maladroit\\
que [rature] tu es ! – Je prends un pistolet, je vise, et je passe à deux pouces\\
de la pièce. Je prends le second pistolet et cette fois la balle force\\
la pièce d’argent à passer à travers la planche sur laquelle elle était\\
appuyée. –\\
\indent Le Turc était pâle. Il comprit que, le cas échéant, il avait beaucoup\\
de chance de ne pas être manqué, et prenant ma main dans les\\
siennes, il la porta successivement à ses lèvres et à son front. L’assemblée\\
cria \textit{Allah\gls{CoEg_pers_00000072} !}. C’était la soumission du vaincu, je fis apporter le\\
café et tout fut dit. [rature]\\
\indent Pour moi, quand je fus seul, je ne sais ce que je ressentis, mais\\
je me pris à pleurer comme un enfant. Hélas ! Monsieur, pourquoi\\
Dieu\gls{CoEg_pers_00000036} a-t-il fait les hommes si méchants, alors qu’il lui était\\
plus facile encore de les faire bons ?\\
\indent Mais j’ai tort et ces détails tout personnels ne peuvent pas vous\\
intéresser. Déchirerai-je cette lettre pour la recommencer ? Vous\\
cacherai-je absolument cette aventure, comme je vous en ai caché\\
tant d’autre, parce que c’est une mission scientifique que vous m’avez\\
donnée, et non une mission de chevalier errant ? Je ne la déchirerai\\
pas. Vous saurez au contraire par ces détails dans quel milieu je\\
vis et tout ce que je souffre, Monsieur, pour mieux mériter votre\\
bienveillance et votre protection pour ma pauvre petite famille\\
qui, depuis trois ans, a bien souffert de mon absence. Les gredins\\
de Turcs ! Savez-vous que je tiendrai parole et que le premier que\\
j’attrape dans la montagne avec un fusil et des intentions équivoques,\\
je le tue comme un loup.
{\footnotesize\begin{center} {[4\textsuperscript{e} feuillet, 1\textsuperscript{re} page, r\textsuperscript{o}]}\end{center}}
\begin{flushright}1\textsuperscript{er} septembre 1853.\end{flushright}
\indent Voici deux affaires essentielles que je vous recommande tout particulièrement :\\
\indent \gls{CoEg_abbr_00000005} : – La frégate à vapeur l’\textit{Albatros}\gls{CoEg_pers_00000073} étant arrivée il y a un mois à\\
Alexandrie\gls{CoEg_place_00000006} et pensant s’en retourner immédiatement en France\gls{CoEg_place_00000016}, reçut à\\
son bord, par ordre de \gls{CoEg_abbr_00000001} Sabatier\gls{CoEg_pers_00000040}, les 82 caisses d’antiquités qui étaient\\
alors en dépôt dans les magasins du Consulat-Général\gls{CoEg_org_00000006}.\\
\indent Depuis cette époque  \gls{CoEg_abbr_00000001} Sabatier\gls{CoEg_pers_00000040} a porté lui-même à Alexandrie\gls{CoEg_place_00000006}\\
10 autres caisses qui, ajoutées aux 28 miennes, forment un total de\\
38.\\
\indent On allait embarquer ces 38 nouvelles caisses et les joindre aux 82\\
autres, quand le commandant de l’\textit{Albatros}\gls{CoEg_pers_00000073} annonça qu’il avait\\
l’ordre du Ministère\gls{CoEg_org_00000019} de stationner plusieurs mois à Alexandrie\gls{CoEg_place_00000006} et\\
qu’il n’avait l’espérance de quitter la station que pour celle de l’Archipel\gls{CoEg_place_imn},\\
en sorte que, loin d’embarquer les 38 caisses, il serait plutôt disposé\\
à débarquer les 82 autres.\\
\indent Les choses en sont là : 82 caisses sont à bord de l’\textit{Albatros}\gls{CoEg_pers_00000073} et 38\\
dans les magasins du Consulat-Général\gls{CoEg_org_00000006} – en tout 120.\\
\indent Le reste vous regarde : voulez-vous faire donner par le Ministère\\
de la Marine\gls{CoEg_org_00000019} l’ordre à l’\textit{Albatros}\gls{CoEg_pers_00000073} de s’absenter pendant 15 jours\\
d’Alexandrie\gls{CoEg_place_00000006} pour aller porter les 120 caisses à Marseilles\gls{CoEg_place_00000018}~; – ou\\
voulez-vous solliciter du même Ministère\gls{CoEg_org_00000019} l’envoi d’un navire\\
\textit{\gls{CoEg_entry_00000018}}. Dans les deux cas, faites en sorte, je vous prie, que la\\
question soit promptement résolue, car les caisses souffrent beaucoup\\
de la chaleur, les bois se fendent et je crains pour les objets qui y\\
sont contenus.\\
\indent \gls{CoEg_abbr_00000006} Vous savez que nous n’avons droit ni aux grands sarcophages
{\footnotesize\begin{center} {[4\textsuperscript{e} feuillet, 1\textsuperscript{re} page, v\textsuperscript{o}]}\end{center}}
\noindent de la tombe d’Apis\gls{CoEg_pers_00000011}, ni au tombeau que je pouvais trouver ou\\
plutôt retrouver à Gyzeh\gls{CoEg_place_00000007}, en sorte que si vous voulez avoir ces\\
objets, il faut en faire la demande à Son Altesse\gls{CoEg_pers_00000016}.\\
\indent J’ai profité de mon voyage à Alexandrie\gls{CoEg_place_00000006} pour demander à\\
\gls{CoEg_abbr_00000001} Sabatier\gls{CoEg_pers_00000040} qu’il {[\textit{sic}]} voulait faire cette demande à Son Altesse\gls{CoEg_pers_00000016} sur\\
un simple avis de moi, ou s’il fallait que le gouvernement français\gls{CoEg_org_00000012}\\
lui écrivît officiellement pour le charger de faire cette démarche\\
auprès du Vice-Roi\gls{CoEg_pers_00000016}.\\
\indent \gls{CoEg_abbr_00000001} Sabatier\gls{CoEg_pers_00000040} me répondit qu’il était prêt à faire cette démarche,\\
qu’il était même sûr qu’elle aurait du succès, mais qu’il ne pouvait\\
la faire sans avoir à montrer une lettre du Ministère\gls{CoEg_org_00000007} qui l’invite\\
à solliciter les objets d’Abbas-\Gls{CoEg_entry_00000002}\gls{CoEg_pers_00000016}.\\
\indent \gls{CoEg_abbr_00000001} Sabatier\gls{CoEg_pers_00000040} m’a donc prié de vous écrire dans ce sens.\\
\indent De votre côté faites dire, soit par le Ministère des affaires\\
Etrangères\gls{CoEg_org_00000007}, soit par le Ministère de la Maison de l’Empereur\gls{CoEg_org_00000010}, à\\
\gls{CoEg_abbr_00000001} Sabatier\gls{CoEg_pers_00000040} :\\
\indent que le mission de \gls{CoEg_abbr_00000001} Mariette\gls{CoEg_pers_00000001} touchant à sa fin, la\\
Direction Générale des Musées Impériaux\gls{CoEg_org_00000001}\footnote{Mariette\gls{CoEg_pers_00000001} avait initialement écrit «~nationaux~» et à réécrit par-dessus le mot.} désirerait posséder\\
quelques-uns des objets antiques découverts par \gls{CoEg_abbr_00000001} Mariette\gls{CoEg_pers_00000001},\\
objets qui, suivant les conventions faites en février 1852 entre le\\
gouvernement égyptien\gls{CoEg_org_00000008} et \gls{CoEg_abbr_00000001} Le Moyne\gls{CoEg_pers_00000024}, appartiennent à \gls{CoEg_abbr_00000003}\\
le Vice-Roi\gls{CoEg_pers_00000016}.\\
\indent Ces objets sont :\\
\indent quatre\gls{CoEg_obj_imn} des quarante sarcophages découverts dans la plaine\\
de Sakkarah\gls{CoEg_place_00000001}~;\\
\indent un sarcophage\gls{CoEg_obj_imn} découvert dans la plaine de Gyzeh\gls{CoEg_place_00000007}~;\\
\indent les quatre murs d’une petite chambre trouvée dans la même\\
plaine~;\\
\indent enfin cinq\gls{CoEg_obj_imn} des stèles transportées à la Citadelle\gls{CoEg_place_00000023}.
{\footnotesize\begin{center} {[4\textsuperscript{e} feuillet, 2\textsuperscript{e} page, r\textsuperscript{o}]}\end{center}}
\indent En tout onze objets.\\
\indent Avec une lettre dans ce sens \gls{CoEg_abbr_00000001} Sabatier\gls{CoEg_pers_00000040} fera la demande.\\
\indent Il est bien entendu que si vous ne m’accordez pas les 25 000 \gls{CoEg_abbr_00000013}\\
en question, cette lettre sera inutile. Mais il est bien entendu\\
en même temps que si le crédit de 25000 \gls{CoEg_abbr_00000013} m’arrivait par\\
exemple demain, je n’en mettrais pas moins la main à l’ouvrage\\
pour amener au moins les objets jusqu’au bord de l’eau. Il\\
faudrait alors, pour les embarquer, attendre que votre lettre\\
arrive de Paris\gls{CoEg_place_00000002}, et vous voyez que c’est une raison pour vous\\
presser, car le temps passe vite ici et les eaux n’attendent\\
pas.\\
\indent Il ne me reste, avant de fermer cette lettre, qu’à vous envoyer\\
de nouveaux, Monsieur, l’expression de tout mon respect et de\\
tout mon dévouement. Vos lettres sont bien rares, et si vous\\
saviez la force et la joie qu’elles me donnent quand elles\\
m’apportent quelques mots d’approbation de vous, je suis\\
sûr que vous m’écririez plus souvent.\\
\indent Présentez, s’il vous plaît, mes compliments à ces Messieurs\\
et croyez-moi
\begin{center}\hspace{5cm} Votre bien dévoué serviteur\\
\hspace{5cm} \gls{CoEg_abbr_00000002} Mariette\end{center}
\indent Je relis ma lettre et je trouve qu’en rapportant ma conversation avec\\
le Turc, ma plume a laissé échapper un gros mot. Mais je ne l’efface\\
pas parce qu’il donne à la chose la vraie couleur locale et que ce terme\\
est effectivement un de ceux dont on fait le plus d’usage en arabe.\\
\indent On me prévient du Caire que le courrier part plus tôt qu’on ne s’y attendait.\\
J’avais préparé une lettre pour \gls{CoEg_abbr_00000001} de Rougé\gls{CoEg_pers_00000032} que je ne puis par conséquent finir.\\
Je n’ai que le temps d’expédier la présente et je ne sais même pas si elle arrivera
\begin{flushright}en temps.\end{flushright}
{\footnotesize \indent \gls{CoEg_abbr_00000001} Sabatier\gls{CoEg_pers_00000040} m’a recommandé de nouveau d’être très-discret avec les journaux sur tout ce qui concerne nos\\
affaires. Il paraît qu’on ne traduit pas très-fidèlement à \gls{CoEg_abbr_00000003}\gls{CoEg_pers_00000016} ce que nous voudrions lui faire savoir.}\footnote{Ce dernier paragraphe est inscrit le long du bord vertical gauche de la feuille.}

\hypertarget{CoEg_Mariette_1855-01-26}{}

\section*{Le 26 janvier 1855, de Paris, à Fortoul, ministre de l’Instruction publique}
\addcontentsline{toc}{section}{Le 26 janvier 1855, de Paris, à Fortoul, ministre de l’Instruction publique} \label{labCoEg_Mariette_1855-01-26}
{\footnotesize
\noindent Institution et lieu de conservation~: Archives nationales, Pierrefitte-sur-Seine.\\
Cote : \hyperlink{CoEg_Mariette_ms_002}{F/17/2988/1, dossier « Mariette »} (n. p.).\\
Support : une feuille double de grand format.\\
Thème~: \gls{CoEg_keyword_00000016}.\\
Notes~: La lettre porte les annotations à l’encre : « une note pour le Ministre. Ch. F » au coin supérieur gauche et au coin supérieur droit « f\textsuperscript{o} 37. ».}

{\footnotesize\begin{center} {[1\textsuperscript{re} page, r\textsuperscript{o}]}\end{center}}
\begin{flushright}Paris\gls{CoEg_place_00000002}, le 26 Janvier 1855.\end{flushright}
\indent A Son Excellence\\
\indent\hspace{1cm} Monsieur le Ministre de l’Instruction Publique\\
\indent\hspace{8cm} à Paris\gls{CoEg_place_00000002}.\\

\hspace{1cm} Monsieur le Ministre\gls{CoEg_pers_00000240},\\

\par Au mois d’août 1850, un\gls{CoEg_pers_00000003} de vos prédécesseurs a bien voulu\\
me charger d’une mission scientifique qui a eu pour résultat la\\
découverte du \Gls{CoEg_entry_00000028}\gls{CoEg_place_00000004} de Memphis\gls{CoEg_place_00000005}. A la suite de cette découverte,\\
des travaux de déblaiement ont été ordonnés, et ce n’est qu’après\\
quatre années employées tout entières à ce travail difficile et\\
coûteux que j’ai pu, il y a quelques semaines, rentrer en France\gls{CoEg_place_00000016}.\\
\indent Mon premier soin est, tout naturellement, de publier le\\
résultat de mes recherches et l’explication des monuments\\
nombreux qui enrichissent d’une manière si imprévue le\\
domaine de l’Egyptologie.
\begin{flushright}Mais\end{flushright}
{\footnotesize\begin{center} {[1\textsuperscript{re} page, v\textsuperscript{o}]}\end{center}}

\indent Mais je me trouve, en quelque sorte, arrêté dès mes\\
premiers pas par la nécessité de connaître les monuments\\
relatifs à Apis\gls{CoEg_pers_00000011} et à Sérapis\gls{CoEg_pers_00000060} qui existent déjà dans les\\
autres Musées de l’Europe\gls{CoEg_place_00000013}, et notamment à Londres\gls{CoEg_place_00000011} et à\\
Berlin\gls{CoEg_place_00000051}.\\
\indent Dans ces circonstances, j’ai donc recours à Votre Excellence\\
pour la prier de m’accorder une indemnité de mille francs\\
qui me permette de me rendre dans ces deux villes. Une\\
absence de deux mois me mettra à même, je l’espère,\\
d’achever mon travail, et à mon retour à Paris\gls{CoEg_place_00000002} je\\
m’empresserai d’adresser à Votre Excellence mon rapport\\
sur la nouvelle mission qu’elle aura daigné m’accorder.\\
\indent J’ai l’honneur d’être avec le plus profond respect,\\
\indent de Votre Excellence,
\begin{center}Monsieur le Ministre,\end{center}
\begin{center}\hspace{5cm}Le très-humble\\
\hspace{5cm}et très-obéissant serviteur :\\
\hspace{5cm}\gls{CoEg_abbr_00000002} Mariette\end{center}

\hypertarget{CoEg_Mariette_1855-07-12}{}

\section*{Le 12 juillet 1855, de Paris, à un destinataire non désigné, au ministère de l'Instruction publique}
\addcontentsline{toc}{section}{Le 12 juillet 1855, de Paris, à un destinataire non désigné, au ministère de l'Instruction publique} \label{labCoEg_Mariette_1855-07-12}
{\footnotesize
\noindent Institution et lieu de conservation~: Archives nationales, Pierrefitte-sur-Seine.\\
Cote : \hyperlink{CoEg_Mariette_ms_002}{F/17/2988/1, dossier « Mariette »} (n. p.).\\
Support : une feuille double de petit format.\\
Thème~: \gls{CoEg_keyword_00000016}.}

\begin{flushright}Paris\gls{CoEg_place_00000002}, le 12 Juillet 1855.\end{flushright}
\hspace{1cm} Monsieur\gls{CoEg_pers_imn},\\

\indent Par un arrêté émané de \gls{CoEg_abbr_00000004} \gls{CoEg_abbr_00000001} le Ministre\gls{CoEg_pers_00000240}\\
de l’Instruction Publique, j’ai été chargé d’une\\
mission scientifique qui devait successivement me\\
conduire dans les Musées de Londres\gls{CoEg_place_00000011} et de Berlin\gls{CoEg_place_00000051}.\\
\indent Je viens de remplir la première partie de\\
cette mission, et au moment où je comptais sur\\
la présence de \gls{CoEg_abbr_00000001} le Ministre\gls{CoEg_pers_00000240} pour obtenir l’ordon-\\
-nancement des 500 derniers francs qui m’ont été\\
alloués, j’apprends que \gls{CoEg_abbr_00000001} le Ministre\gls{CoEg_pers_00000240} est\\
absent de Paris\gls{CoEg_place_00000002}.\\
\indent Dans ces circonstances, Monsieur, j’ai recours\\
à votre obligeance habituelle et vous prie de vouloir\\
bien faire mettre cette somme à ma disposition,\\
afin que je puisse, le plus tôt possible, me\\
rendre à Berlin\gls{CoEg_place_00000051}.\\
\indent A mon retour, je m’empresserai de mettre sous\\
vos yeux le résultat de cette double exploration.\\
\indent J’ai l’honneur d’être,\\
\indent \hspace{1cm}Monsieur,
\begin{center}\hspace{5cm}Votre très-humble serviteur\\
\hspace{5cm}\gls{CoEg_abbr_00000002} Mariette\end{center}

\hypertarget{CoEg_Mariette_1855-08-06}{}

\section*{Le 6 août 1855, de Paris, à Fortoul, ministre de l’Instruction publique}
\addcontentsline{toc}{section}{Le 6 août 1855, de Paris, à Fortoul, ministre de l’Instruction publique} \label{labCoEg_Mariette_1855-08-06}
{\footnotesize
\noindent Institution et lieu de conservation~: Archives nationales, Pierrefitte-sur-Seine.\\
Cote : \hyperlink{CoEg_Mariette_ms_002}{F/17/2988/1, dossier « Mariette »} (n. p.).\\
Support : une feuille double de moyen format, à en-tête « Maison de l’empereur\gls{CoEg_org_00000010}. Direction générale des musées impériaux\gls{CoEg_org_00000001} », datée du palais du Louvre.\\
Note : La lettre porte un tampon « ministère de l’Instruction publique et des Cultes\gls{CoEg_org_00000042}. Enregistré le [...?] août 1855 » et de brèves annotations à l’encre illisibles en partie supérieure (vraisemblablement de simples mentions de classement).\\
Thème~: \gls{CoEg_keyword_00000008}~; \gls{CoEg_keyword_00000014}.}

{\footnotesize\begin{center} {[1\textsuperscript{re} page, r\textsuperscript{o}]}\end{center}}
\begin{flushright}Palais du Louvre, le 6 août 1855.\end{flushright}
\indent A Son Excellence Monsieur le Ministre de l’Instruction\\
\indent\hspace{3cm} Publique et des Cultes\\

\hspace{1cm} Monsieur le Ministre\gls{CoEg_pers_00000240},\\

\indent A la suite d’une lecture que j’ai eu l’honneur de faire devant\\
l’Académie des Inscriptions et Belles-Lettres\gls{CoEg_org_00000036}, cette savante Compagnie\\
a bien voulu charger son Bureau de vous écrire à l’effet d’appeler\\
votre attention sur l’importance des monuments qu’ont produits les\\
fouilles du \Gls{CoEg_entry_00000028}\gls{CoEg_place_00000004} de Memphis\gls{CoEg_place_00000005} et l’intérêt qu’il y aurait à\\
les livrer à la publicité – Vous-même, Monsieur le Ministre,\\
dans une première audience que vous m’avez accordée, vous m’avez\\
assuré de tout votre bon vouloir et de l’empressement que vous\\
mettriez à seconder les vœux de l’Académie des Inscriptions\gls{CoEg_org_00000036}.\\
\indent Encouragé par ces assurances, je me suis donc occupé sans retard\\
du soin de réunir mes matériaux, et une première préoccupation\\
a été celle de me mettre en rapport avec des éditeurs. Mon
\begin{flushright}intention\end{flushright}
{\footnotesize\begin{center} {[1\textsuperscript{re} page, v\textsuperscript{o}]}\end{center}}
intention, dans le cas où les pourparlers auraient abouti, était de\\
me présenter devant vous avec un devis tout préparé et de vous\\
demander votre concours.\\
\indent Mais les seuls éditeurs que j’aie pu rencontrer (\gls{CoEg_abbr_00000009} Gide\gls{CoEg_pers_00000143} et\\
Baudry\gls{CoEg_pers_00000144}) ont élevé des prétentions tellement exorbitantes que j’ai\\
compris immédiatement que ces Messieurs avaient eu connaissance\\
de la démarche de l’Académie\gls{CoEg_org_00000036} et que leur but était d’exploiter\\
à leur profit une publication dont ils supposent Votre Excellence\\
disposée à faire les frais à tout prix. – Vous en jugerez par\\
les deux lettres ci-jointes. Par la première \gls{CoEg_abbr_00000009} Gide\gls{CoEg_pers_00000143} et Baudry\gls{CoEg_pers_00000144}\\
demandent à votre Ministère\gls{CoEg_org_00000042} environ cent dix mille francs.\\
Par la seconde ils déclarent que 80,000 francs leur sont nécessaires.\\
\indent Je n’ai pas cru devoir, Monsieur le Ministre, donner suite\\
à cette affaire qui devient trop visiblement une mine que\\
\gls{CoEg_abbr_00000009} Gide\gls{CoEg_pers_00000143} et Baudry\gls{CoEg_pers_00000144} se proposent d’exploiter. Mais comme,\\
tout en sauvegardant les intérêts de votre Administration\gls{CoEg_org_00000042},\\
je dois en même temps sauvegarder ceux de la science, je ne\\
pense pas qu’il faille tout-à-fait abandonner l’entreprise.\\
C’est pourquoi j’ai l’honneur de solliciter de Votre Excellence\\
une nouvelle audience dans laquelle je me propose de lui\\
faire connaître les moyens les plus certains et les plus\\
économiques d’arriver au but que nous nous proposons.\\
\indent J’ai l’honneur d’être avec le plus profond respect,
\begin{center}Monsieur le Ministre,\\
de Votre Excellence,\\
\hspace{5cm}le très-humble\\
\hspace{5cm}et très-obéissant serviteur\\
\hspace{5cm}\gls{CoEg_abbr_00000002} Mariette\end{center}

\hypertarget{CoEg_Mariette_1855-12-12}{}

\section*{Le 12 décembre 1855, de Paris, à un destinataire non désigné, au ministère de l'Instruction publique}
\addcontentsline{toc}{section}{Le 12 décembre 1855, de Paris, à un destinataire non désigné, au ministère de l'Instruction publique} \label{labCoEg_Mariette_1855-12-12}
{\footnotesize
\noindent Institution et lieu de conservation~: Archives nationales, Pierrefitte-sur-Seine.\\
Cote : \hyperlink{CoEg_Mariette_ms_002}{F/17/2988/1, dossier « Mariette »} (n. p.).\\
Support : une feuille double de moyen format, à en-tête « Maison de l’empereur\gls{CoEg_org_00000010}. Direction générale des musées impériaux\gls{CoEg_org_00000001} », datée du palais du Louvre.\\
Note : La lettre porte les annotations suivantes, à l’encre et d’une autre main : « Classer » au coin inférieur gauche ; « f\textsuperscript{o} 32 » au coin supérieur droit.\\
Thème~: \gls{CoEg_keyword_00000016}.}

\begin{flushright}Palais du Louvre, le 12 Décembre 1855.\end{flushright}
\hspace{1cm} Monsieur\gls{CoEg_pers_imn},\\

\indent Des circonstances impérieuses m’ont forcé à faire en deux fois le\\
voyage à Berlin\gls{CoEg_place_00000051} dont \gls{CoEg_abbr_00000004} \gls{CoEg_abbr_00000001} le Ministre de l’Instruction Publique\gls{CoEg_pers_00000240}\\
m’avait chargé, et c’est à mon retour seulement qu’avant-hier j’ai\\
trouvé la lettre par laquelle vous m’invitez à vous adresser mon rapport\\
sur ma visite aux collections scientifiques de l’Angleterre\gls{CoEg_place_00000052} et de la Prusse\gls{CoEg_place_00000048}.\\
\indent Je vais m’occuper sans retard du soin de rédiger mes notes et j’aurai\\
l’honneur de vous adresser mon travail aussitôt qu’il sera terminé, c’est-à-\\
-dire dans quelques jours.\\
\indent Agréez, je vous prie, Monsieur, l’assurance de ma considération\\
la plus distinguée.\\
\begin{center}\hspace{5cm}Votre très-humble serviteur :\\
\hspace{5cm}\gls{CoEg_abbr_00000002} Mariette\end{center}

\hypertarget{CoEg_Mariette_1856-02-11}{}

\section*{Le 11 février 1856, de Paris, à Rouland, ministre de l’Instruction publique}
\addcontentsline{toc}{section}{Le 11 février 1856, de Paris, à Rouland, ministre de l’Instruction publique} \label{labCoEg_Mariette_1856-02-11}
{\footnotesize
\noindent Institution et lieu de conservation~: Archives nationales, Pierrefitte-sur-Seine.\\
Cote : \hyperlink{CoEg_Mariette_ms_002}{F/17/2988/1, dossier « Mariette »} (n. p.).\\
Support : une feuille double de grand format, à en-tête « Maison de l’empereur\gls{CoEg_org_00000010}. Direction générale des musées impériaux\gls{CoEg_org_00000001} », datée du palais du Louvre.\\
Thème~: \gls{CoEg_keyword_00000014}~; \gls{CoEg_keyword_00000014}.\\
Note~: Le ministère envoya une réponse négative à Mariette le 27 février 1856 indiquant que tous les crédits de publication avait déjà été absorbés par d’autres projets (Archives nationales, F/17/2988/1, dossier « Mariette »).}

{\footnotesize\begin{center} {[1\textsuperscript{re} page, r\textsuperscript{o}]}\end{center}}
\begin{flushright}Palais du Louvre, le 11 février 1856.\end{flushright}
\indent A Monsieur\\
\indent Monsieur le Ministre de l’Instruction Publique et des Cultes
\begin{center}à Paris\gls{CoEg_place_00000002}.\end{center}

\hspace{1cm}Monsieur le Ministre,\\

\indent Au mois de septembre 1850, j’ai eu l’honneur d’être chargé par\\
\gls{CoEg_abbr_00000048} les Ministres de l’Instruction Publique\gls{CoEg_pers_00000003} et de l’Intérieur\gls{CoEg_pers_00000004} d’une\\
mission scientifique pour l’Egypte\gls{CoEg_place_00000003}.\\
\indent Comme Votre Excellence le sait déjà, cette mission a produit ses\\
fruits. Le \Gls{CoEg_entry_00000028}\gls{CoEg_place_00000004} de Memphis\gls{CoEg_place_00000005} a été découvert, et ce temple célèbre,\\
fouillé dans toutes ses parties, nous a mis entre les mains plus de\\
trois mille monuments inconnus jusqu’alors.\\
\indent Mais les travaux de déblaiement, achevés depuis dix-huit mois,\\
attendent encore aujourd’hui leur complément indispensable. Le\\
monde savant ignore en effet les résultats de cette grande entreprise\\
pour laquelle le gouvernement français\gls{CoEg_org_00000012} à déjà dépensé plus de\\
cent-ving-mille francs. Quelques-uns des monuments sont, à la\\
vérité, entrés dans les collections du Louvre\gls{CoEg_org_00000001} ; mais les plus intéressants\\
d’entre eux sont encore enfouis dans les sables de l’Egypte\gls{CoEg_place_00000003}. D’un\\
autre côté, ces matériaux si nombreux, sans les explications qui\\
les font connaître, perdent toute leur importance, et restent comme\\
autant d’énigmes. Je crois donc, Monsieur le Ministre, que la\\
publication des documents artistiques et scientifiques provenant\\
de l’exploration du \Gls{CoEg_entry_00000028}\gls{CoEg_place_00000004} est la suite nécessaire des travaux\\
qui ont été exécutés dans l’enceinte de cet édifice, et comme une publication\\
de ce genre dépasse les ressources dont je puis disposer, je viens vous demander\\
de me fournir les moyens de l’entreprendre.
{\footnotesize\begin{center} {[1\textsuperscript{re} page, v\textsuperscript{o}]}\end{center}}
Je [donnerais ?] ici à Votre Excellence divers détails, \gls{CoEg_abbr_00000005} sur la nature\\
et la composition de l’ouvrage ; \gls{CoEg_abbr_00000006} sur les dépenses que la publication\\
occasionnera.\\
\indent Le \textit{\Gls{CoEg_entry_00000028}\gls{CoEg_place_00000004}} se composera d’un fort volume \gls{CoEg_abbr_00000023} de texte, et\\
d’un atlas de cent grandes planches, accompagné d’un index\\
de vingt-deux feuilles.\\
\indent Le volume imprimé sera lui-même divisé en deux Livres, précédés\\
d’une introduction. – Dans l’introduction, je donnerai le journal\\
abrégé des fouilles ; je montrerai le \Gls{CoEg_entry_00000028}\gls{CoEg_place_00000004} tel que je l’ai retrouvé\\
je décrirai l’état des chambres inviolées, pleines de statues, de bijoux\\
et de pierres précieuses, que j’ai eu la fortune d’ouvrir ; la topographie\\
du temple, la disposition des immenses souterrains consacrés à la\\
sépulture d’Apis seront l’objet de cette introduction. – Avec le\\
premier Livre, commencera l’étude de Sérapis\gls{CoEg_pers_00000060} proprement dit. Mais\\
je désire, dans cette partie de l’ouvrage, n’étudier Sérapis\gls{CoEg_pers_00000060} que dans\\
les seuls écrivains de la tradition classique. Apis\gls{CoEg_pers_00000011}, de son côté,\\
sera l’objet d’une investigation spéciale. Au moyen des auteurs\\
grecs et latins, nous pénétrerons aussi loin que nous le pourrons\\
dans le mythe de ces deux divinités. De Memphis\gls{CoEg_place_00000005} qui fut leur\\
berceau nous les suivrons à Alexandrie\gls{CoEg_place_00000006} où elles s’établirent sous\\
les premiers Ptolémées ; de là nous les montrerons, sous les Empereurs,\\
prenant part au grand mouvement religieux des premiers siècles\\
de notre ère, et s’élançant des bouches du Nil\gls{CoEg_place_00000021} pour aller en\\
quelque sorte s’abattre sur toutes les parties du monde connu.\\
La critique des documents que nous possédons sur cette grande\\
histoire, le récit des diverses tentatives religieuses auxquelles\\
Sérapis\gls{CoEg_pers_00000060} fut mêlé, sa lutte avec le christianisme seront le\\
sujet de ce premier Livre. – Dans le second Livre, nous com-\\
-mencerons l’étude des monuments que le \Gls{CoEg_entry_00000028}\gls{CoEg_place_00000004} lui-même nous\\
en a restitués, et nous essaierons de voir dans quelles limites l’opinion\\
que nous nous étions formée d’Apis\gls{CoEg_pers_00000011} et de Sérapis\gls{CoEg_pers_00000060} d’après le seul\\
témoignage des auteurs classiques doit être modifiée. Ici nous\\
étudierons surtout le Sérapis\gls{CoEg_pers_00000060} égyptien, au moyen des textes\\
hiéroglyphiques. Le Sérapis\gls{CoEg_pers_00000060} grec ne fut après tout qu’un dieu\\
égypto-grec inventé par les Ptolémées au profit de leur religion\\
nationale. Le Sérapis\gls{CoEg_pers_00000060} égyptien, au contraire, resta sous les Lagides
{\footnotesize\begin{center} {[2\textsuperscript{e} page, r\textsuperscript{o}]}\end{center}}
\noindent tel que les \Glspl{CoEg_entry_00000029}, pendant trois mille ans, l’avaient connu et\\
adoré. Quelle influence le vieux Sérapis\gls{CoEg_pers_00000060} égyptien avait-il sur\\
le Sérapis\gls{CoEg_pers_00000060} grec ? En quelles parties les deux religions grecques et\\
égyptiennes avaient-elles assez de points de contact pour qu’un\\
dieu ait pu être, pendant un certain temps, commune à toutes\\
les deux ? quel était en définitive le vrai dogme de Sérapis\gls{CoEg_pers_00000060}, celui\\
que les prêtres enseignaient dans les sanctuaires vingt siècles avant\\
la conquête d’Alexandre\gls{CoEg_pers_00000145} ? Ce sera l’objet de notre second Livre,\\
qui se terminera par le résumé de l’histoire de Sérapis\gls{CoEg_pers_00000060} et la\\
recherche du point de vue définitif sous lequel la science doit\\
désormais envisager la mystérieuse divinité de Sinope\gls{CoEg_place_00000053}.\\
\indent Quant à l’atlas, il se composera de cent planches gravées\\
que j’ai déjà indiquées. Cette partie de l’ouvrage sera divisée en\\
deux sections. Dans la première, j’introduirai touts [\textit{sic}] les monuments\\
provenant du \Gls{CoEg_entry_00000028}\gls{CoEg_place_00000004} proprement dit. La seconde sera consacrée\\
à la publication des monuments trouvés dans les souterrains du\\
\Gls{CoEg_entry_00000028}\gls{CoEg_place_00000004}, c’est-à-dire dans la tombe d’Apis\gls{CoEg_pers_00000011}. Chacun de ces deux\\
sections sera du reste formée d’un nombre à peu près égal de\\
feuilles. Vingt de ces feuilles seront en couleur. Les plans du\\
\Gls{CoEg_entry_00000028}\gls{CoEg_place_00000004}, les dessins des quarante statues grecques découvertes\\
en avant du temple, quelques vues pittoresques destinées à donner\\
une idée générale des lieux, un assez grand nombre d’inscriptions\\
égyptiennes, grecques et phéniciennes forment cette première\\
partie. La seconde comprendra la reproduction des statues, des\\
bijoux, des amulettes précieuses, des tombeaux, et deux ou trois\\
cents des principales stèles provenant de la sépulture des Apis\gls{CoEg_pers_00000011},\\
et cette seconde partie sera, au point de vue de la science, la\\
plus importante des deux, puisque tout l’intérêt historique,\\
chronologique et religieux du \Gls{CoEg_entry_00000028}\gls{CoEg_place_00000004} est contenu dans les\\
\glspl{CoEg_entry_00000011} découverts au fond des souterrains de ce temple.\\
\indent Tel est, Monsieur le Ministre, le plan général de l’ouvrage\\
que je désire consacrer au \Gls{CoEg_entry_00000028}\gls{CoEg_place_00000004} de Memphis\gls{CoEg_place_00000005}. La publication\\
sera divisée en 25 livraisons composées de 4 planches et de 2\\
ou 3 feuilles de texte. Le prix de chaque livraison sera de 11 fr. 20,\\
soit pour l’exemplaire complet 280 francs. Mais \gls{CoEg_abbr_00000009} Gide\gls{CoEg_pers_00000143}\\
et Baudry\gls{CoEg_pers_00000144}, auxquels je me suis adressé pour établir le devis de\\
ces dépenses, déclarent qu’ils ne peuvent se charger de l’entreprise
{\footnotesize\begin{center} {[2\textsuperscript{e} page, v\textsuperscript{o}]}\end{center}}
\noindent si je ne leur assure le placement de 250 exemplaires. C’est donc,\\
au total, 70,000 francs dont il est nécessaire de faire l’avance.\\
\indent Votre Excellence comprendra qu’en présence d’une pareille somme\\
je sois obligé d’avoir recours à elle. Mais je me hâte d’ajouter\\
que ce n’est pas 70,000 francs que je viens demander. \gls{CoEg_abbr_00000001} le Ministre\\
d’Etat\gls{CoEg_pers_00000075} serait en effet disposé à accorder la moitié de cette somme\\
si Votre Excellence consentait à fournir l’autre. D’un autre côté,\\
les 35 000 francs que votre département donnerait pourraient être\\
divisés en cinq annuités, de sorte qu’en résumé c’est une somme\\
annuelle de sept mille francs pendant cinq ans que je prends la\\
liberté de solliciter.\\
\indent Si vous voulez bien, Monsieur le Ministre, vous rendre au\\
désir que j’ai l’honneur de vous exprimer, \gls{CoEg_abbr_00000009} Gide\gls{CoEg_pers_00000143} et Baudry\gls{CoEg_pers_00000144},\\
assurés pendant cinq ans du paiement de l’allocution annuelle\\
ci-dessus spécifiée, n’en mettront pas moins tout l’empressement\\
possible à faire paraître l’ouvrage qui pourra être terminé en\\
deux ans.\\
\indent Je joins à cette demande le devis détaillé dressé par \gls{CoEg_abbr_00000009} Gide\gls{CoEg_pers_00000143}\\
et Baudry\gls{CoEg_pers_00000144}.\\
\indent Dans l’attente d’une réponse favorable, je vous prie, Monsieur\\
le Ministre, de recevoir l’assurance du profond respect avec\\
lequel j’ai l’honneur d’être,\\
\begin{center}de Votre Excellence,\end{center}
\begin{center}\hspace{5cm}le très-humble\\
\hspace{5cm}et très-obéissant serviteur :\\
\hspace{5cm}\gls{CoEg_abbr_00000002} Mariette\end{center}

\hypertarget{CoEg_Mariette_1856-12-11}{}

\section*{Le 11 décembre 1856, de Paris, à Rouland, ministre de l’Instruction publique}
\addcontentsline{toc}{section}{Le 11 décembre 1856, de Paris, à Rouland, ministre de l’Instruction publique} \label{labCoEg_Mariette_1856-12-11}
{\footnotesize
\noindent Institution et lieu de conservation~: Archives nationales, Pierrefitte-sur-Seine.\\
Cote : \hyperlink{CoEg_Mariette_ms_002}{F/17/2988/1, dossier « Mariette »} (n. p.).\\
Support : une feuille double de grand format, à en-tête « Maison de l’empereur\gls{CoEg_org_00000010}. Direction générale des musées impériaux\gls{CoEg_org_00000001} », datée du palais du Louvre.\\
Note : La lettre porte trois annotations à l'encre en partie supérieure de la première page : «~M. Mariette a déjà été chargé en 1855 d’une mission en Angleterre [... ?] à Berlin pour étudier les monuments relatifs au culte d’Apis, et a reçu pour cette mission une [... ?] de 1,000 \textsuperscript{f} et aucun rapport n’est parvenu à l’Adm\textsuperscript{on}~» ; «~Précédemment (en 1850) M. Mariette a déjà reçu 4000 \textsuperscript{f} sur les fonds de l’Inst\textsuperscript{on} publique \textsuperscript{pour rechercher des manuscrits en Egypte} et quelqu’ont été les résultats de la 1\textsuperscript{re} Mission de M. Mariette ces résultats n’ont rien rapporté au Ministère de l’Instruction publique.~» ; «~Il est impossible d’accorder de nouveaux [crédits ?].~».\\
Thème~: \gls{CoEg_keyword_00000003}.}

{\footnotesize\begin{center} {[r\textsuperscript{o}]}\end{center}}
\begin{flushright}11 Décembre 1856\end{flushright}
\indent A Son Excellence,\\
\indent Monsieur le Ministre de l’Instruction Publique\\

\hspace{1cm} Monsieur le Ministre\gls{CoEg_pers_00000065},\\

\indent La découverte du \Gls{CoEg_entry_00000028}\gls{CoEg_place_00000004} de Memphis\gls{CoEg_place_00000005} et des nombreux monuments\\
que le déblaiement du temple a mis au jour m’impose le devoir de\\
rendre compte au monde savant des résultats que cette découverte nous\\
a fournis. L’histoire de l’Egypte\gls{CoEg_place_00000003} ancienne, la chronologie, la religion,\\
la philologie surtout, trouvent dans les matériaux que j’apporte un\\
secours inattendu, et peut-être ai-je le droit de me croire autorisé à\\
dire que, de toutes les découvertes archéologiques faites depuis un grand\\
nombre d’années, il n’en est pas qui ait été plus féconde que celle\\
du \Gls{CoEg_entry_00000028}\gls{CoEg_place_00000004} retrouvé sous les sables de la nécropole de Memphis\gls{CoEg_place_00000005}. Je\\
dois donc au public qui prend intérêt aux progrès de la science\\
l’ouvrage qui est la suite nécessaire de mon séjour en Egypte\gls{CoEg_place_00000003}, et c’est à\\
cet ouvrage que travaille [\textit{sic}] en ce moment.\\
\indent Mais je suis arrivé aujourd’hui à un point qu’il m’est impossible\\
de franchir, si Votre Excellence ne vient à mon aide. En 1826, des fouilles\\
faites par \gls{CoEg_abbr_00000001} Drovetti\gls{CoEg_pers_00000137} aux environs des collines de sable sous lesquelles
{\footnotesize\begin{center} {[v\textsuperscript{o}]}\end{center}}
\noindent je devais plus tard diriger mes travailleurs, ont en effet amené la\\
découverte de certains monuments, stèles, papyrus, sarcophages, qui\\
proviennent de la sépulture de divers administrateurs et employés du\\
\Gls{CoEg_entry_00000028}\gls{CoEg_place_00000004}, et qui, depuis cette époque, ont été transportés à Turin\gls{CoEg_place_00000034}.\\
Rien de plus intéressant que ces monuments qui mettent la vie intérieure\\
du temple à nu, et nous livrent sur le culte de Sérapis une foule de détails\\
intimes qu’on demanderait en vain aux autres objets recueillis dans l’enceinte\\
sacrée. Il est donc essentiel que je connaisse et que je copie ces documents dont\\
une partie seule a été publiée par le savant \gls{CoEg_abbr_00000001} Peyron\gls{CoEg_pers_00000241}, que je les étudie sur
place et me mette à même de les comparer, soit à ceux que nous possédons\\
au Louvre\gls{CoEg_org_00000002} et à la Bibliothèque Impériale\gls{CoEg_org_00000026}, soit à ceux que j’ai déjà eu\\
occasion de voir à Londres\gls{CoEg_place_00000011}. Dans ce but, Monsieur le Ministre, je viens vous\\
demander de m’allouer une somme de mille francs qui me permette d’aller\\
explorer, au profit de mes études sur le culte de Sérapis\gls{CoEg_pers_00000060}, les richesses que\\
possède le magnifique Musée\gls{CoEg_org_00000021} de Turin\gls{CoEg_place_00000034}.\\
\indent Si Votre Excellence veut bien m’accorder la faveur que je sollicite,\\
j’aurai l’honneur de lui adresser, dans les quinze jours qui suivront\\
mon retour à Paris\gls{CoEg_place_00000002}, un rapport détaillé sur ma mission.\\
\indent En attendant une réponse favorable, je vous prie \textsuperscript{d’agréer}, Monsieur le\\
Ministre, l’assurance du profond respect avec lequel j’ai l’honneur\\
d’être,
\begin{center}de Votre Excellence,\end{center}
\begin{center}\hspace{5cm}le très-humble\\
\hspace{5cm}et très-obéissant serviteur\\
\hspace{5cm}\gls{CoEg_abbr_00000002} Mariette\\
\hspace{5cm}Conservateur-adjoint du Musée Egyptien\gls{CoEg_org_00000020} du
\hspace{5cm}Louvre\gls{CoEg_org_00000002}.\end{center}

\hypertarget{CoEg_Mariette_1856-12-31}{}

\section*{Le 31 décembre 1856, de Paris, à Rouland, ministre de l’Instruction publique}
\addcontentsline{toc}{section}{Le 31 décembre 1856, de Paris, à Rouland, ministre de l’Instruction publique} \label{labCoEg_Mariette_1856-12-31}
{\footnotesize
\noindent Institution et lieu de conservation~: Archives nationales, Pierrefitte-sur-Seine.\\
Cote : \hyperlink{CoEg_Mariette_ms_002}{F/17/2988/1, dossier « Mariette »} (n. p.).\\
Support : une feuille double de grand format, à en-tête « Maison de l’empereur\gls{CoEg_org_00000010}. Direction générale des musées impériaux\gls{CoEg_org_00000001} », datée du palais du Louvre.\\
Note : La lettre porte une annotation à l'encre au coin supérieur droit : « f\textsuperscript{o} 37 »\\
Thème~: \gls{CoEg_keyword_00000008}~; \gls{CoEg_keyword_00000014}.}

{\footnotesize\begin{center} {[1\textsuperscript{re} page, r\textsuperscript{o}]}\end{center}}
\begin{flushright}Palais du Louvre, le 31 Décembre 1856.\end{flushright}
\indent A Son Excellence\\
\indent Monsieur le Ministre de l’Instruction Publique et\\
\indent \hspace{1cm}des Cultes,\hspace{2cm}à Paris\gls{CoEg_place_00000002}.\\

\hspace{1cm} Monsieur le Ministre\gls{CoEg_pers_00000065},\\

\indent Votre Excellence a bien voulu me faire demander par \gls{CoEg_abbr_00000001} Michel\\
Chevalier\gls{CoEg_pers_00000146} quelques renseignements sur l’ouvrage dans lequel je\\
désirerais consigner les résultats scientifiques de la découverte du\\
\Gls{CoEg_entry_00000028}\gls{CoEg_place_00000004} de Memphis\gls{CoEg_place_00000005}. Je m’empresse de transmettre à Votre Excellence\\
ces renseignements, que j’essaierais de rendre aussi brefs et aussi clairs\\
que possible.\\
\indent L’ouvrage dont j’ai l’honneur de vous entretenir, Monsieur le\\
Ministre, est rendu nécessaire par l’importance même et la nouveauté\\
des monuments qu’il est destiné à faire connaître. La découverte du\\
\Gls{CoEg_entry_00000028}\gls{CoEg_place_00000004} de Memphis\gls{CoEg_place_00000005} est en effet, s’il m’est permis de le dire,\\
un des grands faits archéologiques de notre temps. Je n’en veux pour\\
preuve que la lettre dont une copie est ci-jointe et qui a été adressée\\
à l’honorable prédécesseur de Votre Excellence par l’Académie des\\
Inscriptions\gls{CoEg_org_00000036} à la suite d’un vote spontané et unanime de cette savante\\
Compagnie. Vous y verrez, Monsieur le Ministre, qu’effectivement
{\footnotesize\begin{center} {[1\textsuperscript{re} page, v\textsuperscript{o}]}\end{center}}
\noindent les matériaux recueillis dans l’enceinte du \Gls{CoEg_entry_00000028}\gls{CoEg_place_00000004} ont une valeur\\
qu’il est difficile de méconnaître. L’histoire y trouve des séries\\
entières de rois ; la chronologie y remonte par des jalons sûrs\\
jusqu’à vingt siècles avant notre ère ; la religion égyptienne\\
surtout s’illumine d’un jour nouveau, et pour la première fois\\
nous voyons clair dans les mystérieuses profondeurs de cette\\
philosophie que les Platon\gls{CoEg_pers_00000106}, les Pythagore\gls{CoEg_pers_00000147}, les Solon\gls{CoEg_pers_00000148} n’avaient\\
pas dédaigné de venir apprendre en Egypte\gls{CoEg_place_00000003}. La science a donc\\
à gagner beaucoup à la publication que je désirerais faire sous\\
les auspices de Votre Excellence, et j’ose dire qu’en France\gls{CoEg_place_00000016}, en\\
Angleterre\gls{CoEg_place_00000052}, et surtout en Allemagne\gls{CoEg_place_00000070}, cette publication est attendue\\
avec la plus vive impatience.\\
\indent Votre Excellence me permettra de ne rien dire de plus sur cette partie\\
de la question, et de consacrer le reste de cette lettre aux seuls détails\\
qui concernent l’ouvrage en lui-même et les dépenses à faire pour\\
l’exécuter.\\
\indent L’ouvrage, tel que je le conçois, serait composé :\\
\indent \gls{CoEg_abbr_00000005} de deux volumes de texte \gls{CoEg_abbr_00000022}, ou d’un gros volume \gls{CoEg_abbr_00000023} ;\\
le journal abrégé des fouilles, la description et l’interprétation\\
des monuments, les résultats qu’ils fournissent à la science\\
seront réservés à ce texte ;\\
\indent \gls{CoEg_abbr_00000006} d’une suite de grandes planches gravées, d’un nombre qui\\
variera selon l’importance du crédit mis à ma disposition ; le format\\
adopté est celui de l’atlas de la publication consacrée par \gls{CoEg_abbr_00000001}\\
Lajard\gls{CoEg_pers_00000149} aux souvenirs du culte de Mithra\gls{CoEg_pers_00000150} ;\\
\indent \gls{CoEg_abbr_00000007} d’une index explicatif de 25 feuillets, donnant, au fur et\\
à mesure de la publication des livraisons, une description\\
sommaire de chacune des planches et des monuments qu’elles\\
représentes.\\
\indent Tel serait, Monsieur le Ministre, le plan général de\\
l’ouvrage. Si Votre Excellence désire le réaliser, elle a à choisir\\
entre les trois devis suivants, dressés par \gls{CoEg_abbr_00000009} Gide\gls{CoEg_pers_00000143} et Baudry\gls{CoEg_pers_00000144},\\
libraires-éditeurs à Paris\gls{CoEg_place_00000002}.
{\footnotesize\begin{center} {[2\textsuperscript{e} page, r\textsuperscript{o}]}\end{center}}
\indent Le premier devis comprend les deux volumes de texte, l’index explicatif,\\
et un gros atlas de 250 planches \gls{CoEg_abbr_00000024}, reproduisant près de 4,000\\
monuments, c’est-à-dire l’ensemble de tous les objets, sans exception,\\
recueillis pendant le déblaiement du temple ; la dépense totale se\\
monterait à 113,000 francs\\
\indent Dans le deuxième devis, le texte imprimé a le même nombre de feuilles,\\
mais un choix a déjà été fait parmi les monuments à graver, et l’atlas\\
ne contient plus que 160 planches ; la combinaison que présentent\\
le nombre et l’arrangement de ces planches est certainement la\\
meilleure ; la dépense totale serait couverte par une allocation de\\
85,000 francs.\\
\indent Le troisième devis comprend les deux volumes de texte, l’index\\
et 120 planches \gls{CoEg_abbr_00000024} gravées et tirées sur papier de Chine\gls{CoEg_place_00000039} : l’atlas\\
ne contient ici que les monuments qu’il est indispensable de publier ;\\
l’ouvrage, construit sur cette base, suffirait cependant au but\\
que nous désirons atteindre ; la dépense, réduire en quelque sorte\\
au strict nécessaire, se monte encore à 70,000 francs.\\
\indent Je laisse à Votre Excellence le choix à faire entre l’une de ces trois\\
combinaisons. Je serais très-heureux que la seconde fût adoptée ;\\
je dois dire cependant que j’ai l’assurance de faire connaître du\\
\Gls{CoEg_entry_00000028}\gls{CoEg_place_00000004} tout ce qu’il est nécessaire de publier, si la troisième\\
vient à être acceptée par Votre Excellence.\\
\indent En somme, l’ouvrage coûterait donc au minimum 70,000 francs.\\
Mais je me hâte d’ajouter, Monsieur le Ministre, que ce n’est\\
pas 70,000 fr. que je viens vous demander. Dans le cas, en effet,\\
où vous ne croiriez pas devoir faire de cette dépense l’objet d’un\\
crédit spécial, \gls{CoEg_abbr_00000025} \gls{CoEg_abbr_00000001} le Ministre d’Etat\gls{CoEg_pers_00000075} serait disposé\\
à accorder la moitié de la somme, si vous-même, Monsieur le\\
Ministre, consentiez à fournir l’autre. D’un autre côté, les 35,000\\
francs que vous m’accorderiez pourraient être divisés en cinq annuités,\\
de sorte qu’en résumé c’est une somme annuelle de 7,000 francs\\
pendant cinq ans que je prends la liberté de solliciter.
{\footnotesize\begin{center} {[2\textsuperscript{e} page, v\textsuperscript{o}]}\end{center}}
\indent Tels sont, Monsieur le Ministre, les renseignement que \gls{CoEg_abbr_00000001} Michel\\
Chevalier\gls{CoEg_pers_00000146} m’a chargé de vous transmettre. En terminant cette lettre, je\\
prends la liberté de m’autoriser de toutes les personnes honorables et savantes\\
qui prennent intérêt à mon ouvrage, pour me recommander à vous et vous\\
prier de me fournir les moyens de donner enfin son complément indispensable\\
à une découverte pour laquelle le Gouvernement Français\gls{CoEg_org_00000012} a déjà dépensé près\\
de cent vingt mille francs.\\
\indent J’ai l’honneur d’être avec le plus profond respect,\\
\indent Monsieur le Ministre,
\begin{center}de Votre Excellence,\end{center}
\begin{center}\hspace{5cm}le très-humble\\
\hspace{5cm}et très-obéissant serviteur :\\
\hspace{5cm}\gls{CoEg_abbr_00000002} Mariette\\
\hspace{5cm}Conservateur-adjoint des Musées Impériaux\gls{CoEg_org_00000001}\end{center}

\hypertarget{CoEg_Mariette_1857-01-03}{}

\section*{Le 3 janvier 1857, de Paris, à un destinataire non désigné, au ministère de l'Instruction publique}
\addcontentsline{toc}{section}{Le 3 janvier 1857, de Paris, à un destinataire non désigné, au ministère de l'Instruction publique} \label{labCoEg_Mariette_1857-01-03}
{\footnotesize
\noindent Institution et lieu de conservation~: Archives nationales, Pierrefitte-sur-Seine.\\
Cote : \hyperlink{CoEg_Mariette_ms_002}{F/17/2988/1, dossier « Mariette »} (n. p.).\\
Support : une feuille double de petit format, à en-tête « Maison de l’empereur\gls{CoEg_org_00000010}. Direction générale des musées impériaux\gls{CoEg_org_00000001} », datée du palais du Louvre.\\
Note : La lettre porte une annotation à l'encre au coin supérieur droit : « f\textsuperscript{o} 37 ».\\
Thème~: \gls{CoEg_keyword_00000008}.}

{\footnotesize\begin{center} {[1\textsuperscript{re} page, r\textsuperscript{o}]}\end{center}}
\begin{flushright}Palais du Louvre le 3 Janvier 1857.\end{flushright}

\hspace{1cm} Monsieur\gls{CoEg_pers_imn},\\

\indent J’ai l’honneur de vous adresser la note que \gls{CoEg_abbr_00000001}\\
le Ministre\gls{CoEg_pers_00000065} avait demandée à \gls{CoEg_abbr_00000001} Michel\\
Chevalier\gls{CoEg_pers_00000146}. Cette note concerne mon ouvrage sur\\
le \Gls{CoEg_entry_00000028}\gls{CoEg_place_00000004} et réunit tous les détails relatifs à\\
la publication dont je vous prie de faire les frais.\\
J’oserai, Monsieur, vous recommander cette affaire\\
avec instance.\\
\indent \gls{CoEg_abbr_00000001} le Ministre\gls{CoEg_pers_00000065} a bien voulu dire à \gls{CoEg_abbr_00000001}\\
Michel Chevalier\gls{CoEg_pers_00000146} que ma demande relative\\
à une petite mission à Turin\gls{CoEg_place_00000034} avait été accueillie\\
avec faveur et que l’arrêté serait pris dans un\\
bref délai. Si je ne craignais de vous importuner,\\
je vous demanderais, Monsieur, de hâter la\\
solution de cette affaire. Je désirerais en effet\\
faire coïncider mon voyage à Turin\gls{CoEg_place_00000034} avec un\\
congé dont je jouis en en ce moment. D’un autre côté,
{\footnotesize\begin{center} {[1\textsuperscript{re} page, v\textsuperscript{o}]}\end{center}}
\noindent les monuments que je dois étudier là-bas commencent\\
véritablement à me faire défaut. Vous me\\
rendriez donc service si vous pouviez avoir égard\\
à la petite demande que je prends la liberté\\
de vous faire en ce moment.\\
\indent J’ai l’honneur d’être,
\begin{center}Monsieur,\end{center}
\begin{center}\hspace{5cm}Votre très-humble serviteur\\
\hspace{5cm}\gls{CoEg_abbr_00000002} Mariette\end{center}

\hypertarget{CoEg_Mariette_1857-02-20}{}
\section*{Le 20 février 1857, de Paris, à Nieuwerkerke, directeur général des musées impériaux} \addcontentsline{toc}{section}{Le 20 février 1857, de Paris, à Nieuwerkerke, directeur général des musées impériaux} 
{\footnotesize \noindent Institution et lieu de conservation~: Archives nationales, Pierrefitte-sur-Seine\\
Cote~: \hyperlink{CoEg_Mariette_ms_001}{20150497/118, dossier 145 «~Mariette, Auguste~»} (n. p.)\\
Support~: une feuille double de moyen format à en-tête : «~Maison de l'Empereur\gls{CoEg_org_00000010}/Direction Générale des Musées Impériaux\gls{CoEg_org_00000001}~», datée du palais du Louvre\gls{CoEg_org_00000002}.\\
Thèmes~: \gls{CoEg_keyword_00000005}~; \gls{CoEg_keyword_00000003}.\\
Note~: la lettre porte les annotations suivantes, d’une autre main que celle de Mariette~: «~accordé~» (au crayon, partie supérieure gauche), «~fait le 21 février~» (à l’encre, partie supérieure gauche)~; «~0 28 P~» (au crayon, partie supérieure droite).}
\begin{flushright}Palais du Louvre\gls{CoEg_org_00000002}, le 20 février 1857.\footnote{Seuls «~20 février~» et ...«~7~» ont été écrits à la main pour compléter l’en-tête.}\end{flushright} 

\hspace{1cm} Monsieur le Directeur\gls{CoEg_pers_00000002},\\

Son Excellence \gls{CoEg_abbr_00000001} le Ministre de l’Instruction Publique\gls{CoEg_pers_00000065} vient de me faire\\
l’honneur de me choisir pour aller remplir à Turin\gls{CoEg_place_00000034} une mission scientifique.\\
Comme cette mission ne peut qu’agrandir mes connaissances et me rendre\\
ainsi de plus en plus digne des fonctions que je remplis au Musée du Louvre\gls{CoEg_org_00000002},\\
j’espère, Monsieur le Directeur, que vous ne vous refuserez pas à m’accorder\\
le congé de quinze jours dont j’aurais besoin pour satisfaire au vœu de\\
\gls{CoEg_abbr_00000001} le Ministre de l’Instruction Publique\gls{CoEg_pers_00000065}.\\
\indent J’ai l’honneur d’être avec un profond respect,
\begin{center}Monsieur le Directeur,\end{center}
\begin{center}\hspace{5cm}Votre très-humble serviteur\\
\hspace{5cm} \gls{CoEg_abbr_00000002} Mariette\end{center}

\hypertarget{CoEg_Mariette_1857-04-01}{}

\section*{Le 1\textsuperscript{er} avril 1857, de Paris, à Rouland, ministre de l’Instruction publique}
\addcontentsline{toc}{section}{Le 1er avril 1857, de Paris, à Rouland, ministre de l’Instruction publique} \label{labCoEg_Mariette_1857-04-01}
{\footnotesize
\noindent Institution et lieu de conservation~: Archives nationales, Pierrefitte-sur-Seine.\\
Cote : \hyperlink{CoEg_Mariette_ms_002}{F/17/2988/1, dossier « Mariette »} (n. p.).\\
Support : quarante feuilles simples de grand format, glissée dans une grande feuille formant une couverture avec le titre « Rapport de M. Mariette\gls{CoEg_pers_00000001}. Monuments d’Apis\gls{CoEg_pers_00000011} et de Sérapis\gls{CoEg_pers_00000060} ».\\
Note : La couverture porte l’annotation à l’encre : « Accuser reception. 12 juin 1857 ». Le revers arrière (troisième de couverture) porte l’annotation : « Ce rapport ne peut être publié dans les archives \gls{CoEg_abbr_00000005} à cause de la trop grande quantité de caractères à fondre – \gls{CoEg_abbr_00000006} pour les comparaisons ??? ».\\
Thème~: \gls{CoEg_keyword_00000002} ; \gls{CoEg_keyword_00000003} ; \gls{CoEg_keyword_00000006} ; \gls{CoEg_keyword_00000008} ; \gls{CoEg_keyword_00000016}.}
{\footnotesize
{\footnotesize\begin{center} {[1\textsuperscript{re} page]}\end{center}}
\begin{paracol}{2}
\switchcolumn
\begin{flushright}Paris\gls{CoEg_place_00000002}, le 1\textsuperscript{er} Avril 1857.\end{flushright}
\indent A Son Excellence Monsieur le Ministre,\\
\indent Secrétaire d’État au Département de\\
\indent l’Instruction Publique et des Cultes\gls{CoEg_org_00000042}.\\

\hspace{1cm} Monsieur le Ministre\gls{CoEg_pers_00000065},\\

\indent Au moment où, il y a deux ans, l’attention\\
se portait sur la divinité fameuse dont\\
le temple venait d’être retrouvé sous les\\
sables de la nécropole de Memphis\gls{CoEg_place_00000005}, j’ai\\
eu l’honneur de lire devant l’Académie\\
des Inscriptions et Belles-Lettres\gls{CoEg_org_00000036} un travail\\
par lequel je me suis efforcé de mettre\\
en relief un fait que je demande à Votre\\
Excellence la permission de lui rappeler.\\
Tant que l’Egypte\gls{CoEg_place_00000003}, disais-je alors, resta,\\
sous les rois prédécesseurs des Lagides, maîtresse\\
de ses destinées, il n’y eut sur les bords du\\
Nil\gls{CoEg_place_00000021} qu’un seul Sérapis\gls{CoEg_pers_00000060} et un seul \Gls{CoEg_entry_00000028}\gls{CoEg_place_00000004}.\\
Apis\gls{CoEg_pers_00000011} mort, c’est-à-dire Apis\gls{CoEg_pers_00000011} rentré à\\
sa mort dans le sein d’Osiris\gls{CoEg_pers_00000151}, tel était\\
Sérapis\gls{CoEg_pers_00000060} ; la tombe du taureau divin, tel\\
était le \Gls{CoEg_entry_00000028}\gls{CoEg_place_00000004}, et comme les lois sacrées\\
attachaient le culte de ce taureau à la\\
ville de Memphis\gls{CoEg_place_00000005}, je faisais voir que, par\\
une conséquence naturelle, il n’a jamais pu\\
y avoir d’autre Sérapis égyptien et d’autre\\
\Gls{CoEg_entry_00000028} que le \Gls{CoEg_entry_00000028}\gls{CoEg_place_00000004} et le Sérapis\gls{CoEg_pers_00000060} de Memphis\gls{CoEg_place_00000005}.
\end{paracol}

{\footnotesize\begin{center} {[2\textsuperscript{e} page]}\end{center}}

\begin{paracol}{2}
\switchcolumn
Mais quand la conquête d’Alexandre\gls{CoEg_pers_00000145} eut\\
anéanti pour toujours la puissance des \Glspl{CoEg_entry_00000029}\\
on vit tout-à-coup un nouveau Sérapis\gls{CoEg_pers_00000060}\\
apparaître. Celui-ci n’est plus Apis\gls{CoEg_pers_00000011} mort ;\\
aussi n’a-t-il plus ses autels à Memphis\gls{CoEg_place_00000005}.\\
Si le Sérapis\gls{CoEg_pers_00000060} des dynasties nationales a\\
\sout{des} \textsuperscript{quelques} points de ressemblance \sout{nombreux} avec\\
le Bacchus\gls{CoEg_pers_00000152} à tête de bœuf d’Argos\gls{CoEg_place_00000054} et d’Elée\gls{CoEg_place_00000055},\\
le Sérapis\gls{CoEg_pers_00000060} qui siège à Alexandrie\gls{CoEg_place_00000006} est plutôt\\
\sout{le Jupiter\gls{CoEg_pers_00000153}} \textsuperscript{l’Adès\gls{CoEg_pers_00000154}} des traditions helléniques. Il y eut\\
donc en Egypte\gls{CoEg_place_00000003} deux Sérapis : l’un que\\
Memphis\gls{CoEg_place_00000005} adorait depuis le règne du
\end{paracol}
\begin{paracol}{2}
\noindent (1) \foreignlanguage{greek}{Καιέχως ... ἐφ᾿ οὗ οἱ βόες Ἆπις ἐν Μέμφει\\
καὶ Μνεῦις ἐν Ἡλιουπόλει καὶ ὁ Μεν-\\
-δήσιος τράγος ἐνομίσθησαν εἶναι θεοί}.\footnote{« Kaiéchôs … sous le règne duquel les bœufs Apis à Memphis et Mnévis à Héliopolis, et le bouc de Mendès étaient considérés être des dieux. »}\\
Africain, \textit{\gls{CoEg_abbr_00000026} Syncell., Chronogr.}\gls{CoEg_bibl_00000011} \gls{CoEg_abbr_00000032} 54,\\
55, Dindort.
\switchcolumn
\noindent Céchoüs\gls{CoEg_pers_00000155} de Manéthon\gls{CoEg_pers_00000116} (1), et que, sous les\\
\Glspl{CoEg_entry_00000029}, sous les Ethiopiens, sous les Perses,\\
sous les Grecs et même sous les Romains,\\
reste sans mélange Apis\gls{CoEg_pers_00000011} mort ; l’autre\\
qui, sous l’un des premiers Ptolémées, arriva,\\
dit-on, de Sinope\gls{CoEg_place_00000053} à Alexandrie\gls{CoEg_place_00000006} et\\
inaugura dans cette capitale de l’Egypte\gls{CoEg_place_00000003}\\
grecque [rature] le culte du dieu cosmopolite auquel\\
le monde connu des anciens devait bientôt\\
rendre des hommages.\\
\indent Cette situation, Monsieur le Ministre, est\\
le point de départ de la mission dont je vais\\
avoir l’honneur de vous rendre compte. Le Sérapis\gls{CoEg_pers_00000060}\\
égyptien, qui n’existait que par les momies\\
conservées dans la tombe d’Apis\gls{CoEg_pers_00000011} n’a pas en\\
effet habité un autre temple que le \Gls{CoEg_entry_00000028}\gls{CoEg_place_00000004}\\
de Memphis\gls{CoEg_place_00000005}, et conséquemment nous\\
n’avons pas à demander à d’autres Musées\\
que le Musée du Louvre\gls{CoEg_org_00000002} des souvenirs de son\\
culte. Mais il n’en est pas ainsi du\\
Sérapis\gls{CoEg_pers_00000060} d’Alexandrie\gls{CoEg_place_00000006}. Celui-ci a eu des\\
autels dans toutes les parties de l’Egypte\gls{CoEg_place_00000003},\\
en Syrie\gls{CoEg_place_00000056}, en Grèce\gls{CoEg_place_00000057}, en Sicile\gls{CoEg_place_00000058}, en Italie\gls{CoEg_place_00000059},\\
dans les Gaules\gls{CoEg_place_00000060}, et les monuments qui nous\\
parlent de lui peuvent ainsi se rencontrer\\
dans toutes les collections archéologiques de\\
l’Europe\gls{CoEg_place_00000013}. D’un autre côté, si le déblaiement
\end{paracol}

{\footnotesize\begin{center} {[3\textsuperscript{e} page]}\end{center}}

\begin{paracol}{2}
\switchcolumn
\noindent du \Gls{CoEg_entry_00000028}\gls{CoEg_place_00000004} nous a fait connaître Apis\gls{CoEg_pers_00000011}\\
mort sous ses véritables traits, nous avons\\
encore bien des choses à apprendre sur Apis\gls{CoEg_pers_00000011}\\
vivant. Là est un nouveau sujet d’étude\\
dont les matériaux doivent être cherchés et\\
recueillis. Ainsi, interroger les Musées sur les\\
monuments du Sérapis\gls{CoEg_pers_00000060} grec ; comparer ces\\
monument à ceux que le \Gls{CoEg_entry_00000028}\gls{CoEg_place_00000004} de\\
Memphis\gls{CoEg_place_00000005} nous a mis entre les mains ; de-\\
-mander à ces mêmes Musées l’explication du\\
culte rendu, non seulement à l’Apis\gls{CoEg_pers_00000011} des tombeaux,\\
mais aussi à l’Apis\gls{CoEg_pers_00000011} vivant et nourri dans\\
le temple célèbre si vanté par les Grecs,\\
telle est, au moment où je rassemble et\\
coordonne les éléments d’une histoire générale\\
de Sérapis\gls{CoEg_pers_00000060}, la tâche que je me sens obligé\\
d’accomplir, et tel est en même temps,\\
Monsieur le Ministre, l’objet, restreint aux\\
seules collections de l’Angleterre\gls{CoEg_place_00000052}, de la Prusse\gls{CoEg_place_00000048}\\
et du Piémont\gls{CoEg_place_00000061}, de la mission dont j’ai été\\
honoré et sur laquelle je vais fournir à Votre\\
Excellence quelques explications.\\
\indent Je ferais tort au résultat lui-même que\\
je me suis proposé d’atteindre en demandant\\
cette mission si j’entrais ici dans tous les\\
détails du problème difficile dont les\\
Musées de Berlin\gls{CoEg_org_00000040}, de Londres\gls{CoEg_org_00000005} et de Turin\gls{CoEg_org_00000021}\\
m’ont livré la solution. Autant les notes\\
dont je me suis enrichi gagneront à prendre\\
leur place naturelle dans mon travail sur\\
le \Gls{CoEg_entry_00000028}\gls{CoEg_place_00000004}, autant elles perdraient à\\
être détachées de l’ensemble auquel elles\\
appartiennent et se trouveraient ici\\
dépaysées. J’ajoute donc soigneusement\\
toutes ces notes à celles que je possédais\\
déjà, et, en attendant que des circonstances\\
favorables me permettent de les publier\\
avec l’ouvrage qui est la conséquence\\
nécessaire de la découverte du \Gls{CoEg_entry_00000028}\gls{CoEg_place_00000004}, je\\
vais, si Votre Excellence le veut bien,\\
réserver l’état de mes connaissances actuelles
\end{paracol}

{\footnotesize\begin{center} {[4\textsuperscript{e} page]}\end{center}}

\begin{paracol}{2}
\noindent \\
\\
\\
\\
\\
(1) Hérodote\gls{CoEg_pers_00000108}, II\gls{CoEg_bibl_00000021}, 183 ; Diodore\gls{CoEg_pers_00000031} de Sicile\gls{CoEg_place_00000058}, I\gls{CoEg_bibl_00000022}, 85 ;\\
Strabon\gls{CoEg_bibl_00000023}, \gls{CoEg_abbr_00000027} XXII\gls{CoEg_pers_00000159}, \gls{CoEg_abbr_00000028} 4 § 14.\\
\\
(2) Voyez le grand recueil de planches que\\
le Roi\gls{CoEg_pers_00000134} de Prusse\gls{CoEg_place_00000048} a fait publier par \gls{CoEg_abbr_00000001}\\
Lepsius\gls{CoEg_pers_00000061} sous le titre de \textit{Denkmaler} [\textit{sic}]\\
\textit{aus Aegypten und Aethiopien … nach\\
dasen landern gesendeten und in den\\
jahren 1842-1845 aus gefuhrten wissen-\\
-schaflichen expedition}\gls{CoEg_bibl_00000002}, Berlin\gls{CoEg_place_00000051}, 1849. Un\\
des fils du roi Snéfrou\gls{CoEg_pers_00000162}, de la V\textsuperscript{\underline{e}} dynastie,\\
était \includegraphics[height=6pt]{CoEg_Mariette_hiero_1857-04-01_4_1.png} \footnote{\foreignlanguage{translit}{\Gls{CoEg_aeg_00000005} \gls{CoEg_aeg_00000002}} « servant (prêtre ?) d’Apis\gls{CoEg_pers_00000011} ».} \textit{gardien} ? \textit{d’Apis\gls{CoEg_pers_00000011}} (\textit{Denkm.\gls{CoEg_bibl_00000002}\\
\gls{CoEg_abbr_00000029}} III, \textit{\gls{CoEg_abbr_00000030}} 16, 17). Ce titre correspond peut-\\
-être à celui de βουκόλος τοῦ Ὀσοράπι\footnote{« Bouvier d’Osiris-Apis/Osirapis ».}\\
si connu par les papyrus grecs. \Gls{CoEg_abbr_00000031}\\
Brunet de Presles\gls{CoEg_pers_00000051}, \textit{Sur le \Gls{CoEg_entry_00000028}\gls{CoEg_place_00000004} de\\
Memphis\gls{CoEg_place_00000005}}\gls{CoEg_bibl_00000001}, \gls{CoEg_abbr_00000032} 15 ; [rature] \textit{Description of the\\
Greek papyri in the British Museum\gls{CoEg_org_00000005}},\\
\gls{CoEg_abbr_00000033} 1\gls{CoEg_bibl_00000013}, \gls{CoEg_abbr_00000032} 33 ; Letronne\gls{CoEg_pers_00000097}, \textit{Inscr. gr. et lat. de l’Eg.}\gls{CoEg_bibl_00000014} \textsubscript{\gls{CoEg_abbr_00000034} 1, \gls{CoEg_abbr_00000032} 297.}\\
(3) \Gls{CoEg_entry_00000030} S. Sharpe\gls{CoEg_pers_00000164}, \textit{Egyptian inscriptions\\
from the British Museum\gls{CoEg_org_00000005} and other\\
sources}\gls{CoEg_bibl_00000015}, Londres\gls{CoEg_place_00000011}, 1840, 1\textsuperscript{\underline{ère}} série, \gls{CoEg_abbr_00000035} 19,\\
20, 86, et Lepsius\gls{CoEg_pers_00000061}, \textit{Denkmaler}\gls{CoEg_bibl_00000002} [\textit{sic}], \textit{\gls{CoEg_abbr_00000029}} II,\\
\textit{\gls{CoEg_abbr_00000030bis}} 23, 32, 112, 123, 138.\\
(4) \textit{Renseignements sur les soixante-quatre\\
Apis\gls{CoEg_pers_00000011} du \Gls{CoEg_entry_00000028}\gls{CoEg_place_00000004}} publiés dans le\\
\textit{Bulletin Archéologique}, 1\textsuperscript{\underline{ère}} année,\\
\gls{CoEg_abbr_00000032} 45, 53, 66, 85, 93 et 2\textsuperscript{\underline{e}} année \gls{CoEg_abbr_00000032}\\
58, 74.\gls{CoEg_bibl_00000016}\\
(5) Pline\gls{CoEg_pers_00000165}, VIII\gls{CoEg_bibl_00000012}, 71 ; Ammien Marcellin\gls{CoEg_pers_00000166},\\
XXII\gls{CoEg_bibl_00000024}, 14 ; Tacite\gls{CoEg_pers_00000111}, \textit{\gls{CoEg_abbr_00000036}}\gls{CoEg_bibl_00000025} II, 59.\\
(6) Suétone\gls{CoEg_pers_00000167}, [Tit. ?]\gls{CoEg_bibl_00000026}, \gls{CoEg_abbr_00000037} 5.\\
(7) Spartianus\gls{CoEg_pers_00000169}, \textit{in Adrian.}\gls{CoEg_bibl_00000027}, \gls{CoEg_abbr_00000037} 12.
\switchcolumn
\noindent sur les trois divinités dont je faisais plus\\
haut la distinctions à savoir Apis\gls{CoEg_pers_00000011}\\
vivant, Apis\gls{CoEg_pers_00000011} mort et Sérapis\gls{CoEg_pers_00000060}.
\begin{center}§ 1. D’Apis\gls{CoEg_pers_00000011} vivant\end{center}
C’est à Memphis\gls{CoEg_place_00000005} même, dans une\\
partie réservée du grand temple de\\
Phtah\gls{CoEg_pers_00000047}, qu’Apis\gls{CoEg_pers_00000011} était nourri (1). Si\\
nous en croyons Manéthon\gls{CoEg_pers_00000116}, le culte de\\
cette divinité fut inauguré sous Céchoüs\gls{CoEg_pers_00000155},\\
l’un des rois de la II\textsuperscript{\underline{e}} dynastie. Une\\
inscription hiéroglyphique du temps de\\
Mycérinus\gls{CoEg_pers_00000161} (2) vient à l’appui de cette\\
assertion de l’historien national, en nous\\
montrant dans une phrase ainsi conçue\\
\includegraphics[height=6pt]{CoEg_Mariette_hiero_1857-04-01_4_2.png}\footnote{\foreignlanguage{translit}{\Gls{CoEg_aeg_00000001} \gls{CoEg_aeg_00000002} \gls{CoEg_aeg_00000006} \gls{CoEg_aeg_00000004}} « fête d’Apis\gls{CoEg_pers_00000011} dans le palais ».} \textit{panégyrie\\
d’Apis\gls{CoEg_pers_00000011} dans le sanctuaire}, que déjà, sous\\
la II\textsuperscript{\underline{e}} dynastie, les autels du dieu étaient\\
debout. La trace d’Apis\gls{CoEg_pers_00000011} ne se perd pas\\
dans les dynasties qui suivent. Des stèles\\
du Musée Britannique\gls{CoEg_org_00000005} nous font\\
connaître quelques personnages, hommes \&\\
femmes, qui, de la II\textsuperscript{\underline{e}} dynastie à la XIX\textsuperscript{\underline{e}} ,\\
s’appelaient Hapi\gls{CoEg_pers_00000163} comme le dieu\gls{CoEg_pers_00000188} (3). De la\\
XIX\textsuperscript{\underline{e}}  dynastie aux derniers Ptolémées, la\\
persistance du culte d’Apis\gls{CoEg_pers_00000011} est assurée\\
par les nombreux \glspl{CoEg_entry_00000011} découverts\\
au milieu des ruines du \gls{CoEg_entry_00000028}\gls{CoEg_place_00000004} et\\
aujourd’hui conservés au Louvre\gls{CoEg_org_00000001} (4). Sous\\
les Romaines même certitude. Apis\gls{CoEg_pers_00000011} refusa,\\
dit-on, de manger de la main de Germa-\\
-nicus\gls{CoEg_pers_00000168} (5), et, à la vue de \textsuperscript{ce triste} présage, les\\
prêtres osèrent prédire au prince sa fin\\
prématurée. Titus\gls{CoEg_pers_00000170} alla aussi visiter le dieu\\
à Memphis\gls{CoEg_place_00000005} et lui rendre un hommage (6).\\
Sous l’empereur Adrien\gls{CoEg_pers_00000171} qui lui-même\\
vint s’incliner devant l’étable sacrée \textsuperscript{(7)}, on\\
frappa à Alexandrie\gls{CoEg_place_00000006} \textsuperscript{et à Memphis\gls{CoEg_place_00000005}} des médailles au
\end{paracol}

{\footnotesize\begin{center} {[5\textsuperscript{e} page]}\end{center}}
\begin{paracol}{2}
\noindent (1) Zoega\gls{CoEg_pers_00000174}, \textit{Numi Aegypt.}\gls{CoEg_bibl_00000017}, \gls{CoEg_abbr_00000032} 139-148,\\
\gls{CoEg_abbr_00000038} VII, \gls{CoEg_abbr_00000039} Rome\gls{CoEg_place_00000065} 1787 ; Tochon\gls{CoEg_pers_00000175}\\
d’Annecy\gls{CoEg_place_00000071}, \textit{Méd. des nomes}\gls{CoEg_bibl_00000018}, \gls{CoEg_abbr_00000032} 139 [rature].
\switchcolumn
\noindent type d’Apis\gls{CoEg_pers_00000011} (1). Sous l’empereur Julien\gls{CoEg_pers_00000172}\\
un Apis\gls{CoEg_pers_00000011} se manifesta, et l’édit seul de\\
Théodose\gls{CoEg_pers_00000173} dispersa les adorateurs du\\
taureau divin et mit fin pour toujours\\
au culte qui lui était rendu. Ainsi\\
l’antique autel élevé sous Céchoüs\gls{CoEg_pers_00000155} par\\
des générations presque contemporaines du\\
déluge, résista, chose incroyable, à\\
l’effort de quarante siècles, et ne tomba\\
que sous les coups du christianisme.\\
\noindent Apis\gls{CoEg_pers_00000011} était un taureau dont certaines\\
marques révélaient l’origine céleste. Ces\\
marques étaient au nombre de vingt-neuf
\end{paracol}
\begin{paracol}{2}
\noindent (2) \textit{De nat. anim.}\gls{CoEg_bibl_00000038} XI, 10.\\
(3) III\gls{CoEg_bibl_00000021}, 28.\\
(4) XXII\gls{CoEg_bibl_00000024}, 14.\\
(5) \textit{Polyhist}.\gls{CoEg_bibl_00000028}, XXXII.\\
(6) \textit{De Is. et Osir.}\gls{CoEg_bibl_00000029} XXXVII.\\
(7) Ap. Euseb.\gls{CoEg_pers_00000114}, \textit{Prep. Evangel.}\gls{CoEg_bibl_00000030}, III, 13.\\
(8) Strabon\gls{CoEg_pers_00000159}, \textit{Géogr.}\gls{CoEg_bibl_00000023} \gls{CoEg_bibl_00000023} XVII, \gls{CoEg_abbr_00000028} 1, § 14 ;\\
Pomponius Mela\gls{CoEg_pers_00000178}, \textit{de situ orbis}\gls{CoEg_bibl_00000031}, I, 9 ;\\
Pline\gls{CoEg_bibl_00000012}, VIII\gls{CoEg_bibl_00000012}, 46.
\switchcolumn
\noindent selon Elien\gls{CoEg_pers_00000180} (2). Hérodote\gls{CoEg_pers_00000108} (3), Ammien\\
Marcellin\gls{CoEg_pers_00000166} (4), Solin\gls{CoEg_pers_00000176} (5) les ont en partie\\
décrites. Plutarque\gls{CoEg_pers_00000109} (6), Porphyre\gls{CoEg_pers_00000177} (7) et\\
d’autres auteurs (8) en font des empreintes du\\
soleil \& de la lune. Quant aux monuments,\\
ils nous montrent le plus souvent Apis\gls{CoEg_pers_00000011}\\
sous la forme d’un taureau couvert de\\
tâches [\textit{sic}] blanches et noires. Les \glspl{CoEg_entry_00000011} du\\
\Gls{CoEg_entry_00000028}\gls{CoEg_place_00000004} offrent de très-nombreux\\
\scriptsize{exemples de ces \sout{représentations} tâches [\textit{sic}] que j’ai retrouvées}\\
sur une stèle\gls{CoEg_obj_imn} du Musée\gls{CoEg_org_00000040} de Berlin\gls{CoEg_place_00000051} et qu’on\\
distingue encore \sout{jusque} sur l’Apis\gls{CoEg_pers_00000011} de la\\
fameuse Table Isiaque\gls{CoEg_obj_00000025} à Turin\gls{CoEg_place_00000034}. Le front\\
du dieu est orné d’un triangle blanchâtre
\end{paracol}
\begin{paracol}{2}
\noindent (9) \textit{\Gls{CoEg_abbr_00000042}}\\
(10) \textit{\Gls{CoEg_abbr_00000042}}\gls{CoEg_bibl_00000012}
\switchcolumn
\noindent dont parlent Hérodote\gls{CoEg_pers_00000108} et Strabon\gls{CoEg_pers_00000159} (9).\\
Sur le poitrail paraît le croissant lunaire
de Pline\gls{CoEg_pers_00000165} (10). Un autre croissant se dessine\\
sur le flanc, et enfin les poils de la queue\\
sont \textit{doubles} [rature], c’est-à-dire qu’ils\\
sont alternativement blancs \& noirs.
\end{paracol}
\begin{paracol}{2}
\noindent (11) Pour des représentations en couleur d’Apis\gls{CoEg_pers_00000011},\\
voyez la Table Isiaque\gls{CoEg_obj_00000025} et la plupart\\
des stèles du \Gls{CoEg_entry_00000028}\gls{CoEg_place_00000004} au Louvre\gls{CoEg_org_00000001}. J’ai\\
publié une de ces images dans le\\
\textit{Bulletin archéologique}, 1\textsuperscript{\underline{ère}} année,\\
\gls{CoEg_abbr_00000032} 54\gls{CoEg_bibl_00000016}. [ratures] \Gls{CoEg_abbr_00000040} Birch\gls{CoEg_pers_00000179}, \textit{Observation\\
on a bronze figure of a bull, found\\
in Cornwall\gls{CoEg_place_00000076}}\gls{CoEg_bibl_00000039}, \gls{CoEg_abbr_00000032} 10.\\
(12) Birch\gls{CoEg_pers_00000179}, \textit{Gallery of antiquities selected}
\switchcolumn
\noindent (11). Un bronze du Musée Britannique\gls{CoEg_org_00000005} \textsuperscript{(12)}\\
et les nombreuses figurines de toutes\\
matières que nous possédons \sout{aujourd’h}\\
au Louvre\gls{CoEg_org_00000001} depuis la découverte du\\
\Gls{CoEg_entry_00000028}\gls{CoEg_place_00000004} nous font voir certaines autres\\
marques que les images peintes ne\\
nous montrent pas. C’est ainsi que
\end{paracol}
{\footnotesize\begin{center} {[6\textsuperscript{e} page]}\end{center}}
\begin{paracol}{2}
\noindent \textit{from the British Museum\gls{CoEg_org_00000005}}, \gls{CoEg_abbr_00000035} 26\gls{CoEg_bibl_00000042}.\\
(1) Ou plutôt le vautour.
\switchcolumn
\noindent l’aigle (1) d’Hérodote\gls{CoEg_pers_00000108}, les ailes éployées,\\
est parfaitement reconnaissable sur les\\
statues en ronde-bosse d’Apis\gls{CoEg_pers_00000011}. La\\
présence de la divinité dans le corps du\\
taureau était donc révélée aux prêtres\\
par les marques extérieures que l’animal\\
portait. Les uns était produits par\\
la couleur de la robe ; les autres consis-\\
-taient en \textit{épis}. Par un usage qui\\
remonte aux temps les plus reculés,\\
les Arabes ont encore aujourd’hui les\\
même croyances et attachent des\\
propriétés heureuses ou néfastes à certaines\\
combinaisons des épis de leurs chevaux\\
qui leurs paraissent former une \textit{lance},\\
un \textit{luth} ou une \textit{tente}. L’\textit{aigle},
\end{paracol}
\begin{paracol}{2}
\noindent (2) Hérodote\gls{CoEg_pers_00000108} et Pline\gls{CoEg_bibl_00000012}, \textit{loc. cit.}
\switchcolumn
\noindent l’\textit{escabot} (κάνθαρος) d’Apis\gls{CoEg_pers_00000011} (2) n’étaient\\
sans doute que des \textit{épis} dans lesquels les\\
prêtres initiés savaient voir les symboles\\
exigés de l’animal divin.\\
\indent Je crois que la manifestation d’Apis\gls{CoEg_pers_00000011},\\
ce que les Grecs appelaient la θεοφανία,\\
s’entendait du premier veau qui, pourvu\\
de vingt-neuf marques, venait au\\
monde après la mort d’un Apis\gls{CoEg_pers_00000011}. Les\\
fêtes par lesquelles cet évènement [\textit{sic}] était\\
célébré dans toute l’Egypte\gls{CoEg_place_00000003} ont été
\end{paracol}
\begin{paracol}{2}
\noindent (3) \textit{\Gls{CoEg_abbr_00000042}\gls{CoEg_bibl_00000022}}\\
(4) \textit{\Gls{CoEg_abbr_00000042}\gls{CoEg_bibl_00000024}}
\switchcolumn
\noindent 
décrites par Diodore\gls{CoEg_pers_00000031} (3) et Elien\gls{CoEg_pers_00000180} (4). Mais\\
les contradictions \sout{qu’on} \textsuperscript{que l’on} remarque dans\\
les récits de ces deux auteurs ne sont pas\\
écartées par le témoignage des textes\\
égyptiens recueillis dans le \Gls{CoEg_entry_00000028}\gls{CoEg_place_00000004}.\\
La question est, à mon avis, une de\\
celles qui ne sont pas encore résolues.\\
Tout ce qu’on peut jusqu’à présent\\
affirmer, c’est que Diodore\gls{CoEg_pers_00000031} ne s’est\\
pas trompé en disant qu’à la mort\\
d’un Apis\gls{CoEg_pers_00000011} les prêtres se mettaient \sout{à la}\\
immédiatement à la recherche d’un
\end{paracol}

{\footnotesize\begin{center} {[7\textsuperscript{e} page]}\end{center}}

\begin{paracol}{2}
\noindent \\
\\
\\
\\
(1) Je ne saurais fixer la position de la ville\\
ainsi nommée. Un certain \textit{Ahmès\gls{CoEg_pers_00000181}},\\
dont j’ai retrouvé le sarcophage\gls{CoEg_obj_imn} dans\\
la riche collection\gls{CoEg_org_00000040} de Berlin\gls{CoEg_place_00000051}, était\\
prêtre du temple de ce lieu.\\
(2) Stèle\gls{CoEg_obj_imn} de la XXII\textsuperscript{\underline{e}} dynastie au nom de\\
Pétisis\gls{CoEg_pers_00000183}, salle d’Apis\gls{CoEg_pers_00000011}, au Louvre\gls{CoEg_org_00000001}.\\
\\
\\
(3) C’est par cette expression, dont les stèles du\\
\Gls{CoEg_entry_00000028}\gls{CoEg_place_00000004} offrent quelques autres exemples,\\
que l’on désignait le lieu natal d’Apis\gls{CoEg_pers_00000011}.\\
Selon Elien\gls{CoEg_pers_00000180} (\textit{\gls{CoEg_abbr_00000042}\gls{CoEg_bibl_00000039}}) on bâtissait au\\
dieu, sur l’endroit même qu’il avait choisi\\
pour se manifester, un édifice tourné vers\\
le soleil levant, et on l’y nourrissait\\
de lait pendant quatre mois. Le terme\\
accompli, les prêtres se rendaient en pompes\\
à la demeure provisoire du dieu, et\\
le saluaient du nom d’Apis\gls{CoEg_pers_00000011}. Il était\\
de là emmené dans le temple de Vulcain\gls{CoEg_pers_00000186}.\\
\Gls{CoEg_abbr_00000031} Jablonski\gls{CoEg_pers_00000184}, \textit{Pantheon\gls{CoEg_bibl_00000032}}, 2\textsuperscript{\underline{e}} \gls{CoEg_abbr_00000033} \gls{CoEg_abbr_00000032} 185.\\
(4) Ou \textit{il accomplissait ses transformations}. C’est\\
ainsi qu’Ahmès\gls{CoEg_pers_00000185}, chef des nautoniers,\\
désigne le temps qui s’écoule immédiatement\\
après sa naissance. \Gls{CoEg_abbr_00000040} de Rougé\gls{CoEg_pers_00000032},\\
\textit{Mémoire sur le tombeau d’Ahmès\gls{CoEg_bibl_00000043}}, \gls{CoEg_abbr_00000032} 108,\\
110.\\
(5) La \foreignlanguage{hebrew}{אוֹן} de la Bible\gls{CoEg_bibl_00000019}, la \foreignlanguage{coptic}{ⲱⲙ} des coptes,\\
traduit Ἥλιοῦ πόλις par la Septante\gls{CoEg_bibl_00000020}.\\
(6) Nom égyptien du Nil\gls{CoEg_place_00000021}, ce qui aura\\
donné lieu à l’erreur de Diodore\gls{CoEg_pers_00000031}\\
qui substitue Nilopolis\gls{CoEg_place_00000062} à Héliopolis\gls{CoEg_place_00000024}.\\
(7) Diodore\gls{CoEg_pers_00000031} a traduit presque littéralement :\\
θεὸν [rature] ἀνάγουσιν εἰς Μέμφιν, εἰς τὸ\\
τοῦ Ἡφαίστου τέμενος\footnote{« Ils le conduisent ainsi à Memphis, et le font entrer comme une divinité dans le temple de Vulcain », trad. Jean-Chrétien-Ferdinand Hœfer, t. 1, Paris, Charpentier, 1846, p. \href{https://gallica.bnf.fr/ark:/12148/bpt6k54537384/f127.item}{96}.} (I 85).
\switchcolumn
\noindent nouveau, recherche qui fut souvent longue\\
puisque, comme le dit une stèle du\\
\Gls{CoEg_entry_00000028}\gls{CoEg_place_00000004} : \includegraphics[height=6pt]{CoEg_Mariette_hiero_1857-04-01_7_1_1.png}\\
\includegraphics[height=6pt]{CoEg_Mariette_hiero_1857-04-01_7_1_2.png}\\
\includegraphics[height=6pt]{CoEg_Mariette_hiero_1857-04-01_7_1_3.png}\footnote{\foreignlanguage{translit}{\Gls{CoEg_aeg_00000022}·n·\gls{CoEg_aeg_00000061}\gls{CoEg_aeg_00000058} \gls{CoEg_aeg_00000032} \gls{CoEg_aeg_00000033} \gls{CoEg_aeg_00000034} \gls{CoEg_aeg_00000026} 3 \gls{CoEg_aeg_00000035}\gls{CoEg_aeg_00000059} \gls{CoEg_aeg_00000041}wt \gls{CoEg_aeg_00000038} \gls{CoEg_aeg_00000037}w \gls{CoEg_aeg_00000040}w \gls{CoEg_aeg_00000039}} « on le trouva à Hout-ched-abed\gls{CoEg_place_00000050} après qu'ils eurent parcouru pendant trois mois les lagunes du Delta\gls{CoEg_place_00000072} et toutes les îles de la Basse-Égypte\gls{CoEg_place_00000072} ».} \textit{il fut trouvé à Hat-schat-}[\textit{avat} ?]\gls{CoEg_place_00000050}\\
(1) \textit{après que, pendant trois mois, on eût parcouru\\
toutes les vallées de la Haute-Egypte\gls{CoEg_place_00000020}\footnote{La traduction de Mariette\gls{CoEg_pers_00000001} offre un parallélisme séduisant qui permet d’englober l’Égypte\gls{CoEg_place_00000003} tout entière, mais \foreignlanguage{translit}{\gls{CoEg_aeg_00000038}} ne peut désigner que la Basse-Égypte\gls{CoEg_place_00000072} ; peut-être a-t-il été induit en erreur en lisant le signe du poisson \foreignlanguage{translit}{\gls{CoEg_aeg_00000036}} (avec un tilapia ; d’où sa traduction « vallée »), mais il est plus vraisemblable d’admettre qu’il s’agit d’un oxyrhynque, \foreignlanguage{translit}{\gls{CoEg_aeg_00000041} \gls{CoEg_aeg_00000038}} étant une expression attestée pour désigner les marais du Delta\gls{CoEg_place_00000072}.} et les îles\\
de la Basse-Egypte\gls{CoEg_place_00000072}} (2). Le voyage du dieu\\
à Héliopolis\gls{CoEg_place_00000024} (et \textit{non pas} à Nilopolis\gls{CoEg_place_00000062})\\
est une autre assertion de Diodore\gls{CoEg_pers_00000031} que les\\
monuments sont venus confirmer. Enfin\\
les fêtes de la proclamation\\
et de l’installation définitive d’Apis\gls{CoEg_pers_00000011} dans\\
le temple de Vulcain\gls{CoEg_pers_00000186} trouvent dans les\\
témoignages réunis des écrivains de la\\
tradition classique et des monuments une\\
confirmation satisfaisante. L’épitaphe\gls{CoEg_obj_00000031}\\
de l’Apis\gls{CoEg_pers_00000011} de l’an 28 de Ptolémée Evergète\\
II\gls{CoEg_pers_00000187} énumère ces diverses circonstances dans\\
les termes suivants : \includegraphics[height=6pt]{CoEg_Mariette_hiero_1857-04-01_7_2_1.png}\\
\includegraphics[height=6pt]{CoEg_Mariette_hiero_1857-04-01_7_2_2.png}\\
\includegraphics[height=6pt]{CoEg_Mariette_hiero_1857-04-01_7_2_3.png}\\
\includegraphics[height=6pt]{CoEg_Mariette_hiero_1857-04-01_7_2_4.png}\\
\includegraphics[height=6pt]{CoEg_Mariette_hiero_1857-04-01_7_2_5.png}\footnote{\foreignlanguage{translit}{\Gls{CoEg_aeg_00000042} \gls{CoEg_aeg_00000005} \gls{CoEg_aeg_00000014} \gls{CoEg_aeg_00000015} \gls{CoEg_aeg_00000043} \gls{CoEg_aeg_00000044} \gls{CoEg_aeg_00000045} \gls{CoEg_aeg_00000010} \gls{CoEg_aeg_00000003}-\gls{CoEg_aeg_00000014} \gls{CoEg_aeg_00000010} (\gls{CoEg_aeg_00000016}) 28 \gls{CoEg_aeg_00000026} (1) \gls{CoEg_aeg_00000063} \gls{CoEg_aeg_00000052} 24 \gls{CoEg_aeg_00000010} \gls{CoEg_aeg_00000046}} … \foreignlanguage{translit}{\gls{CoEg_aeg_00000028}\gls{CoEg_aeg_00000058} [?] \gls{CoEg_aeg_00000010} \gls{CoEg_aeg_00000003}-\gls{CoEg_aeg_00000014} \gls{CoEg_aeg_00000044} \gls{CoEg_aeg_00000010} \gls{CoEg_aeg_00000016} 28 \gls{CoEg_aeg_00000032} \gls{CoEg_aeg_00000016} 31 \gls{CoEg_aeg_00000064}-\gls{CoEg_aeg_00000060}} … \foreignlanguage{translit}{\gls{CoEg_aeg_00000016} 31 \gls{CoEg_aeg_00000064}-\gls{CoEg_aeg_00000060} \gls{CoEg_aeg_00000052} 20 \gls{CoEg_aeg_00000053}\gls{CoEg_aeg_00000058} \gls{CoEg_aeg_00000032} \gls{CoEg_aeg_00000047} \gls{CoEg_aeg_00000028} \gls{CoEg_aeg_00000006} \gls{CoEg_aeg_00000003} \gls{CoEg_aeg_00000048} \gls{CoEg_aeg_00000049} \gls{CoEg_aeg_00000062} \gls{CoEg_aeg_00000010} (\gls{CoEg_aeg_00000064}) \gls{CoEg_aeg_00000060} \gls{CoEg_aeg_00000052} 20 \gls{CoEg_aeg_00000054}·\gls{CoEg_aeg_00000061}\gls{CoEg_aeg_00000058} [ꞽw ?] \gls{CoEg_aeg_00000003} \gls{CoEg_aeg_00000011} 23 \gls{CoEg_aeg_00000010} \gls{CoEg_aeg_00000046}} « La majesté de ce dieu auguste naquit (à) Memphis à l’intérieur du temple, le 24\textsuperscript{e} jour du premier mois de l'inondation de (l'an) XXVIII du roi … il se manifesta au temple de Memphis de l’an XXVIII à l’an XXXI, premier mois de la germination … le 20 (du premier mois de la germination de l'an XXXI, il alla à Héliopolis apparaître dans le temple de Hâpy qui s’y trouve, le 21\textsuperscript{e} jour du (premier) mois de la germination ; on l’intronisa (dans ?) le temple de Ptah, le 23\textsuperscript{e} jour [du premier mois de la germination de l'an XXXI] du roi ... ». La graphie de \foreignlanguage{translit}{\gls{CoEg_aeg_00000003}} « temple », \includegraphics[height=6pt]{CoEg_Mariette_hiero_1857-04-01_7_2_note_2.png} à la 4\textsuperscript{e} ligne de cet extrait, est en fait \includegraphics[height=6pt]{CoEg_Mariette_hiero_1857-04-01_7_2_note_1.png} (1\textsuperscript{er} signe de la 9\textsuperscript{e} ligne conservée de la stèle\gls{CoEg_obj_00000031}).}\\
etc., \textit{la naissance de ce dieu auguste (eut lieu)\\
à Memphis\gls{CoEg_place_00000005}, dans le temple} (3)\textit{, en l’an 28,\\
le 24 de \gls{CoEg_entry_00000023}, du roi (Ptolémée Evergète II\gls{CoEg_pers_00000187}).\\
Il resta} (4) \textit{dans le temple de Memphis de\\
l’an 28 à l’an 31 et le premier \gls{CoEg_entry_00000024} … L’an\\
31, le 20 de \gls{CoEg_entry_00000024}, il alla à On\gls{CoEg_place_00000024}} (5) \textit{dans\\
le temple d’Hapi\gls{CoEg_pers_00000188}} (6) \textit{jusqu’au 21} (du même\\
mois)\textit{. Il fut introduit dans le temple de\\
Phtah\gls{CoEg_pers_00000047}} (7) \textit{le 23 du roi} etc. Ainsi les\\
monuments, comme Elien\gls{CoEg_pers_00000180} et Diodore\gls{CoEg_pers_00000031}, placent\\
un certain intervalle entre la manifestation\\
proprement dite et l’arrivée du dieu dans\\
le temple de Vulcain\gls{CoEg_pers_00000186}. Quatre mois\\
selon Elien\gls{CoEg_pers_00000180}, quarante jours selon Eusèbe\gls{CoEg_pers_00000114},\\
suffisent aux cérémonies qui séparent\\
le premier du second de ces évènements [\textit{sic}],\\
tandis que nous venons de voir la stèle
\end{paracol}

{\footnotesize\begin{center} {[8\textsuperscript{e} page]}\end{center}}
\begin{paracol}{2}
\noindent \\
\\
\\
\\
\\
\\
\\
\\
\\
\\
\\
\\
\\
\\
\\
\\
\\
\\
\\
\\
\\
\\
\\
\\
\\
\\
\\
\\
\\
\\
\\
\\
\\
\\
\\
\\
\\
\\
\\
\\
(1) \Gls{CoEg_abbr_00000031} Strabon\gls{CoEg_pers_00000159}, \textit{Géogr.}\gls{CoEg_bibl_00000023}, \Gls{CoEg_abbr_00000041} XVII\textsuperscript{\underline{e}}, \gls{CoEg_abbr_00000028} 1,\\
\S [rature] 10.
\switchcolumn
\noindent d’Evergète II\gls{CoEg_pers_00000187} fixe à deux ans et demi\\
le temps qui s’était écoulé depuis le jour\\
où le taureau naquit jusqu’à celui où\\
il vint occuper, sous le nom d’Apis\gls{CoEg_pers_00000011},\\
l’étable \sout{inoccupée depuis} \textsuperscript{laissée vacante par} la mort de\\
son prédécesseur. Ces contradictions laissent\\
debout les difficultés que je signalais\\
en commençant ce paragraphe, et l’on\\
voit que ce côté de la question, encore\\
imparfaitement étudié, aurait besoin\\
de preuves nouvelles pour constituer un\\
ensemble capable de prendre sa place\\
au milieu de faits définitivement\\
acquis à l’histoire.\\
\noindent Une fois installé dans l’étable sacrée, le\\
jeune veau était regardé comme un dieu.\\
La vâche [\textit{sic}] à Memphis\gls{CoEg_place_00000005}, le bélier à Thèbes\gls{CoEg_place_00000019},\\
le crocodile à Ombos\gls{CoEg_place_00000063}, l’épervier à Héliopolis\gls{CoEg_place_00000024}\\
recevaient l’hommage des Egyptiens à\\
titre d’animaux sacrés, symboles d’Hathor\gls{CoEg_pers_00000189},\\
d’Ammon\gls{CoEg_pers_00000190}, de Sébek\gls{CoEg_pers_00000191} et de \sout{Pht} Phré\gls{CoEg_pers_00000192} ; ils\\
n’étaient pas dieux. S’il m’était permis\\
de faire une comparaison que n’autorise\\
peut-être pas la nature opposée des\\
choses, j’essaierais de \sout{faire voir} \textsuperscript{montrer} que\\
l’Egypte\gls{CoEg_place_00000003} ancienne, en admettant les\\
animaux dans les temples où \textsuperscript{elle adorait ses dieux,}, a, dans\\
une mesure différente, obéi à l’idée qui,\\
\sout{dans les} aux premier siècles de notre ère,\\
introduisait la colombe, le poisson,\\
l’agneau au sein des basiliques chrétiennes.\\
Apis\gls{CoEg_pers_00000011}, au contraire, seul avec Mnévis\gls{CoEg_pers_00000156}\\
de tous les animaux qu’on adorait en\\
Egypte\gls{CoEg_place_00000003}, était vénéré pour lui-même\\
et prenait rang parmi les divinités. On\\
appellera donc Apis\gls{CoEg_pers_00000011} un animal\\
divin, plutôt qu’un animal sacré,\\
et c’est là, je crois, une distinction\\
d’autant plus légitime que\footnote{Mariette avait écrit « qu’on/qu’en/qu’au » puis a barré l’apostrophe et complété « qu... ».} l’antiquité\\
classique semble l’avoir \sout{ad} déjà connue\\
\noindent et admise (1). Quant aux attributs\\
qui caractérisent Apis\gls{CoEg_pers_00000011}, on les trouve\\
énoncés dans divers titres donnés au
\end{paracol}
{\footnotesize\begin{center} {[9\textsuperscript{e} page]}\end{center}}

\begin{paracol}{2}
\noindent \\
\\
\\
\\
\\
\\
\\
\\
\\
\\
\\
\\
\\
\\
\\
\\
\\
\\
\\
(1) III\gls{CoEg_bibl_00000021}, 28.\\
\\
\\
\\
\\
(2) I\gls{CoEg_bibl_00000031}, 9. Sur la conception surnaturelle\\
d’Apis, voyez encore Plutarque\gls{CoEg_pers_00000109}, \textit{de Is.\\
et Osir.}\gls{CoEg_bibl_00000029}, XLIII ; \textit{Sympos.}\gls{CoEg_bibl_00000033}, \gls{CoEg_abbr_00000041} VIII,\\
\textit{quest.} I ; Elien\gls{CoEg_pers_00000180}, \gls{CoEg_abbr_00000042}\gls{CoEg_bibl_00000038} ; [Suindos ?]\gls{CoEg_pers_imn}\\
in voce Ἅπιδες et [Ἅπις ?] ; Porphyre\gls{CoEg_pers_00000177},\\
\gls{CoEg_abbr_00000026} Euseb.\gls{CoEg_pers_00000114} \textit{Prepar. Evangel.} III\gls{CoEg_bibl_00000030}, 13, \gls{CoEg_abbr_00000052}
\switchcolumn
\noindent dieu par des monuments qui appartiennent\\
aux Musées de Londres\gls{CoEg_org_00000005}, de Berlin\gls{CoEg_org_00000040} et\\
surtout de Paris\gls{CoEg_org_00000002}. Le titre principal,\\
inséparable en quelque sorte du nom\\
d’Apis\gls{CoEg_pers_00000011}, est celui que les \glspl{CoEg_entry_00000011}\\
du \Gls{CoEg_entry_00000028}\gls{CoEg_place_00000004} répètent à satiété en\\
cette forme \includegraphics[height=6pt]{CoEg_Mariette_hiero_1857-04-01_9_1.png} \footnote{\foreignlanguage{translit}{\Gls{CoEg_aeg_00000002} \gls{CoEg_aeg_00000008} \gls{CoEg_aeg_00000009} \gls{CoEg_aeg_00000010} \gls{CoEg_aeg_00000011}} « Apis\gls{CoEg_pers_00000011}, renouvelé de vie de Ptah\gls{CoEg_pers_00000047} ».}, et qu’on\\
traduit \sout{par} \textsuperscript{soit par} \textit{Apis\gls{CoEg_pers_00000011}, le revivifié par Phtah\gls{CoEg_pers_00000047}},\\
soit, moins nettement quant au sens\\
\sout{naturel} \textsuperscript{philosophique} de cette dénomination, par Apis\gls{CoEg_pers_00000011},\\
la seconde vie de Phtah\gls{CoEg_pers_00000047}. Apis\gls{CoEg_pers_00000011}, dans\\
son caractère essentiel, passait donc pour\\
une émanation de Phtah\gls{CoEg_pers_00000047} ; il est quelquefois\\
même \includegraphics[height=6pt]{CoEg_Mariette_hiero_1857-04-01_9_2.png} \footnote{\foreignlanguage{translit}{\Gls{CoEg_aeg_00000007} \gls{CoEg_aeg_00000010} \gls{CoEg_aeg_00000011}} « fils de Ptah\gls{CoEg_pers_00000047} ».} le propre \textit{fils de Phtah\gls{CoEg_pers_00000047}}. C’est\\
à Phtah\gls{CoEg_pers_00000047} qu’il doit le jour ; c’est le Vulcain\gls{CoEg_pers_00000186}\\
de l’Egypte\gls{CoEg_place_00000003} qui, prenant la forme d’un\\
feu céleste, féconde la vâche [\textit{sic}] devenue\\
mère sans le contact du mâle et par\\
conséquent restée vierge. Γίνεται ὁ Ἄπις\\
ἐκ βοός, dit Hérodote\gls{CoEg_pers_00000108} (1), ἥτις οὐκετι\\
οἵη τε γίνεται ἐς γαστέρα ἄλλον βάλλεσθαι\\
γόνον. Ἀιγύπτιοι [\textit{sic}] δὲ λέγουσι, σέλας ἐπὶ\\
τὴν βοῦν ἐκ τοῦ οὐρανοῦ κατίσχειν, καὶ\\
μιν ἐν τούτου τίκτειν τὸν Ἄπιν.\footnote{Le texte habituel est ὁ δὲ Ἆπις οὗτος ὁ Ἔπαφος γίνεται μόσχος ἐκ βοός ἥτις οὐκέτι οἵη τε γίνεται ἐς γαστέρα ἄλλον βάλλεσθαι γόνον. Αἰγύπτιοι δὲ λέγουσι, σέλας ἐπὶ τὴν βοῦν ἐκ τοῦ οὐρανοῦ κατίσχειν, καί μιν ἐκ τούτου τίκτειν τὸν Ἆπιν~: « Cet Apis, appelé aussi Épaphus, est un jeune bœuf, dont la mère ne peut en porter d’autre. Les Égyptiens disent qu’un éclair descend du ciel sur elle, et que de cet éclair elle conçoit le dieu Apis. » (trad. Larcher, t. 1, Paris, 1850, p. \href{https://gallica.bnf.fr/ark:/12148/bpt6k203223n/f246.item}{247}).} \textit{Rarò\\
nascitur}, dit Pomponius Méla\gls{CoEg_pers_00000178} (2), \textit{nec\\
coitu pecoris, ut aiunt, sed divinitùs et\\
cœlesti igne conceptus.}\footnote{« Sa naissance est un prodige rare ; on assure même dans le pays, qu’il n’est point le fruit d’un accouplement ordinaire, mais que sa mère le conçoit surnaturellement d’un rayon de feu céleste. » (trad. C. P. Fradin, t. 1, Paris, Ch. Pougens – Poitiers, E. P. J. Catineau, 1804, \gls{CoEg_abbr_00000032} \href{https://gallica.bnf.fr/ark:/12148/bpt6k6529663b/f128.image}{87}.} – Une autre\\
appellation tout aussi fréquente est\\
celle d’\textit{Apis-Osiris}, ou d’\textit{Osiris-Apis}.\\
Au dessus de trois taureaux noirs et\\
blancs comme Apis\gls{CoEg_pers_00000011}, j’ai trouvé sur\\
un papyrus\gls{CoEg_obj_imn} de Berlin\gls{CoEg_place_00000051} des légendes qui\\
confirment cette identité du taureau\\
divin et du président de l’\gls{CoEg_entry_00000025}.\\
On lit en effet au dessus du premier\\
taureau \includegraphics[height=6pt]{CoEg_Mariette_hiero_1857-04-01_9_3.png}\footnote{Plutôt [\foreignlanguage{translit}{\Gls{CoEg_aeg_00000065}}?] \foreignlanguage{translit}{\gls{CoEg_aeg_00000024} \gls{CoEg_aeg_00000050} \gls{CoEg_aeg_00000067}wy} « [image ?] d'Osiris aux cornes pointues » ? Les signes dessinés par Mariette pour le premier mot ne semblent pas correspondre. Il est possible que le premier mot soit en fait une graphie plurielle, et que les trois extraits soient à lire comme une séquence continue, au pluriel, qui s'applique aux trois taureaux. La traduction de cette expression et des deux citations hiéroglyphiques qui suivent reste de toute façon hasardeuse sans référence plus précise au papyrus permettant de retrouver le contexte de ces extraits.} \textit{figure cachée\\
d’Osiris\gls{CoEg_pers_00000151} qui s’est orné de cornes} ; au dessus\\
du second \includegraphics[height=6pt]{CoEg_Mariette_hiero_1857-04-01_9_4.png} \footnote{Plutôt \foreignlanguage{translit}{\gls{CoEg_aeg_00000050}w \gls{CoEg_aeg_00000051} \gls{CoEg_aeg_00000024}} « celui qui a fourni (?) l'oreille d'Osiris » ?}\textit{Osiris\gls{CoEg_pers_00000151} qui s’est\\
orné de l’oreille du taureau}, et au dessus\\
du troisième \includegraphics[height=6pt]{CoEg_Mariette_hiero_1857-04-01_9_5.png} \footnote{Plutôt \foreignlanguage{translit}{\gls{CoEg_aeg_00000069} \gls{CoEg_aeg_00000068} \gls{CoEg_aeg_00000024}} « celui qui a dissimulé le visage d'Osiris » ?} \textit{Osiris\gls{CoEg_pers_00000151} qui}
\end{paracol}

{\footnotesize\begin{center} {[10\textsuperscript{e} page]}\end{center}}

\begin{paracol}{2}
\noindent \\
(1) Lepsius\gls{CoEg_pers_00000061}, \textit{Auswahl des wichtigsten\\
urkunden des Aegyptischen alter-\\
thums}\gls{CoEg_bibl_00000034} [rature], \gls{CoEg_abbr_00000030bis} XVI.\\
\\
\\
\\
(2) \textit{De Is. et Osir.}\gls{CoEg_bibl_00000029}, XX.\\
\\
\\
\\
(3) idem, XXIX.\\
\\
\\
\\
(4) \Gls{CoEg_abbr_00000042}\gls{CoEg_bibl_00000023}.\\
\\
(5) I\gls{CoEg_bibl_00000022}, 85.
\switchcolumn
\noindent \textit{change de face}. Une grande stèle\gls{CoEg_obj_imn} du\\
Musée Britannique\gls{CoEg_org_00000005}\footnote{La stèle \href{https://www.britishmuseum.org/collection/object/Y_EA886}{EA 886} correspondrait tout à fait à cette description, si elle n'avait été acquise qu'en 1875 lors de la succession Harris.} (1) donne à une\\
figure d’Apis le nom de \includegraphics[height=6pt]{CoEg_Mariette_hiero_1857-04-01_10_1.png} \footnote{\foreignlanguage{translit}{\gls{CoEg_aeg_00000002}-\gls{CoEg_aeg_00000024}} « Apis-Osiris ».} \textit{Apis-\\
-Osiris}. La tradition classique tout entière\\
confirme du reste l’identité déjà certifiée\\
par les monuments hiéroglyphiques. « On\\
« entretenait à Memphis\gls{CoEg_place_00000005}, dit Plutarque\gls{CoEg_pers_00000109}\\
« (2), le bœuf Apis\gls{CoEg_pers_00000011} qu’on regarde comme\\
« l’image d’Osiris\gls{CoEg_pers_00000151}, et qui, à ce titre, doit\\
« être au même endroit que son corps » – « la\\
« plupart des prêtres, dit le même auteur\\
« (3), veulent que le nom de Sérapis\gls{CoEg_pers_00000060} soit\\
« formé de ceux d’Apis\gls{CoEg_pers_00000011} et d’Osiris\gls{CoEg_pers_00000151}, fondé\\
« sur ce point de doctrine qu’ils enseignent\\
« qu’Apis\gls{CoEg_pers_00000011} et l’image d’Osiris\gls{CoEg_pers_00000151} » Strabon\gls{CoEg_pers_00000159}\\
\sout{«} (4) et Diodore\gls{CoEg_pers_00000031} ne sont pas plus\\
explicites. « Quelques-uns, dit le\\
« second de ces écrivains (5), expliquent\\
« le culte d’Apis\gls{CoEg_pers_00000011} par la tradition que\\
« l’âme d’Osiris\gls{CoEg_pers_00000151} passe dans un\\
« taureau, et que depuis ce moment\\
« jusqu’à ce jour elle se manifeste aux\\
« hommes sous cette forme [qu’elle change ?]\\
« successivement ». Ainsi Apis\gls{CoEg_pers_00000011} est\\
l’animal d’Osiris\gls{CoEg_pers_00000151}, ou plutôt il est Osiris\gls{CoEg_pers_00000151}\\
lui-même. C’est l’âme d’Osiris\gls{CoEg_pers_00000151} qui\\
l’anime. Sa naissance est célébrée comme\\
la théophanie d’Osiris\gls{CoEg_pers_00000151} ; à sa mort\\
on le pleure comme si Osiris\gls{CoEg_pers_00000151} était\\
mort. Apis\gls{CoEg_pers_00000011} est par conséquent Osiris\gls{CoEg_pers_00000151} des-\\
-cendu sur la terre, et l’on voit par là que\\
je n’ai pas eu tort de le regarder autre part\\
comme une incarnation du grand juge\\
de l’enfer égyptien. – En résumé, le\\
double \sout{d’A} rôle d’Apis\gls{CoEg_pers_00000011} est celui-ci ; Apis\gls{CoEg_pers_00000011}\\
est le taureau revivifié par Phtah\gls{CoEg_pers_00000047} ; il est\\
le fils de Phtah\gls{CoEg_pers_00000047}, tandis que, selon\\
une tradition conservée par Hérodote\gls{CoEg_pers_00000108}
\end{paracol}

{\footnotesize\begin{center} {[11\textsuperscript{e} page]}\end{center}}

\begin{paracol}{2}
\noindent \\
\\
\\
\\
\\
\\
\\
\\
\\
\\
\\
\\
\\
\\
\\
\\
\\
\\
\\
\\
\\
\\
\\
\\
\\
\\
\\
\\
\\
\\
\\
\noindent (1) Clément\gls{CoEg_pers_00000112} d’Alexandrie\gls{CoEg_place_00000006}, \textit{Pædagogos}\gls{CoEg_bibl_00000008},\\
lib. III, \gls{CoEg_abbr_00000028} 2, \gls{CoEg_abbr_00000032} 216.
\switchcolumn
\noindent et plusieurs autres écrivains, il a été conçu\\
dans le sein de sa mère par l’opération\\
d’un feu céleste. D’un autre côté, les\\
monuments hiéroglyphiques, d’accord\\
avec la plupart des auteurs de la Grèce\gls{CoEg_place_00000057}\\
et de Rome\gls{CoEg_place_00000065}, nous font voir donc Apis\gls{CoEg_pers_00000011}\\
en représentant d’Osiris\gls{CoEg_pers_00000151}, ou plutôt Osiris\gls{CoEg_pers_00000151}\\
\sout{lui-même} descendu \sout{(} au milieu des hommes.\\
Apis\gls{CoEg_pers_00000011} sera donc l’invocation d’Osiris\gls{CoEg_pers_00000151}\\
par l’opération de Phtah\gls{CoEg_pers_00000047} ; c’est à Osiris\gls{CoEg_pers_00000151}\\
qu’il devra son âme ; mais c’est Phtah\gls{CoEg_pers_00000047}\\
qui [rature] aura déposé dans le sein de la vâche [\textit{sic}]\\
la semence d’où est sorti le corps du\\
fils divin. Tel est Apis\gls{CoEg_pers_00000011} dans son rôle\\
principal ; telle est la pensée philosophique\\
qui a créé et soutenu pendant quarante\\
siècles le culte de ce dieu étrange auquel\\
\sout{j’appliquerai} \textsuperscript{je serais tenté d’appliquer} ici l’exclamation célèbre\\
\scriptsize{de [rature] Clément\gls{CoEg_pers_00000112} d’Alexandrie\gls{CoEg_place_00000006} [rature] : « Les sanctuaires}\\
\footnotesize{« sont ombragés par des voiles \sout{d} tissus d’or ;}\\
« mais si vous \sout{avancez} pénétrez dans le fond du\\
« temple et que vous cherchiez la \textit{statue},\\
« un employé du temple s’avance d’un air\\
« grave en chantant un hymne en langue\\
« égyptienne et soulève un peu le voile, comme\\
« pour vous montrer le dieu. Que voyez-vous\\
« alors ? un chat, un crocodile, un serpent\\
« indigène, ou quelque autre animales\\
« dangereux ! Le dieu des Egyptiens paraît !\\
« c’est une bête sauvage se vautrant sur\\
« un tapis de pourpre ! (1)\\
\indent La mort d’Apis\gls{CoEg_pers_00000011} donne lieu à de\\
graves problèmes dont je vais essayer de\\
bien poser les termes. C’est \sout{, je crois,}\\
Pline\gls{CoEg_bibl_00000012} qui, le premier, a mentionné un fait\\
sur lequel l’attention s’est, avec raison,\\
depuis long-temps portée. \textit{Non est fas},
\end{paracol}
\begin{paracol}{2}
\noindent (2) VIII\gls{CoEg_bibl_00000012}, 46.
\switchcolumn
\noindent dit Pline\gls{CoEg_pers_00000165} (2) : \textit{eum} (Apidum) \textit{certos vitæ\\
excedere annos, mersumque in sacerdotum}\end{paracol}
\begin{paracol}{2}
\noindent (3) [rature]. XXII\gls{CoEg_bibl_00000024}, 14, 7.
\switchcolumn
\noindent \textit{fonte necant}.\footnote{« Des lois sacrées ne permettent pas qu’il vive au-delà d’un nombre d’années déterminé », trad. Ajasson de Grandsagne, t. 6, Paris, C. L. F. Panckoucke, 1829, p.~\href{https://gallica.bnf.fr/ark:/12148/bpt6k5802657d/f378.item}{369}.} On lit aussi dans Ammien \textsuperscript{Marcelin}\gls{CoEg_pers_00000166} (3) :
\textit{Apis\gls{CoEg_pers_00000011}, quum post vivendi spatium præsti-\\
tutum sacro fonte è vita abierit, nic enim}
\end{paracol}

{\footnotesize\begin{center} {[12\textsuperscript{e} page]}\end{center}}

\begin{paracol}{2}
\noindent \\
\\
\\
\\
(1) \gls{CoEg_abbr_00000028} 32.\gls{CoEg_bibl_00000028}\\
\\
\\
\\
(2) \textit{De Is. et Osir.}\gls{CoEg_bibl_00000029}, \sout{\gls{CoEg_abbr_00000028} 56.} LVI.\\
\\
\\
\\
\\
\\
\\
\\
\\
\\
\\
\\
\\
\\
\\
\\
\\
\\
\\
\\
(3) \textit{De Is. et Osir.}\gls{CoEg_bibl_00000029}, XXXVII.\\
\\
\\
\\
\\
(4) \Gls{CoEg_abbr_00000041} XXII\gls{CoEg_bibl_00000024}.
\switchcolumn
\noindent \textit{ultra eum trahere licet ætatem quam\\
secreta librorum præscribit auctoritas mys-\\
ticorum alter cum publico quaeritur\\
luctu.}\footnote{« [Apis,] après qu’il a vécu le temps prescrit, et que [absent de la citation : \textit{immersus} « plongé »] dans une fontaine, il disparaît (car il n’est permis, ni de le conserver au-delà du terme fixé par l’autorité des livres mystiques [absent de la citation : \textit{necatur choragio pari, bos femina, quae ei inventa cum notis certis offertur, quo perempto} « ni de lui donner plus d’une fois l’année une génisse sur laquelle se rencontrent certains signes »]), on en cherche un nouveau avec un deuil universel. » trad. Guillaume de Moulines, t. 2, Lyon, Jean-Marie Bruyset père et fils, 1778, p.~\href{https://gallica.bnf.fr/ark:/12148/bpt6k6472037r/f225.item}{202-203}.} Ce même usage a été connu de\\
Solin\gls{CoEg_pers_00000176} \textsuperscript{(1)} : \textit{statum ævi spatium est, quod\\
ut affuit, profundo sacri fontis immersus\\
necatur, ne diem longius trahat, quam\\
licebit.}\footnote{« Le nombre de ses années est déterminé : quand le temps en est venu, on le fait mourir en le noyant dans la fontaine sacrée, car il ne peut vivre au-delà de l’époque fixée. » trad. Agnant, Paris, C. L. F. Panckoucke, 1847, p.~\href{https://gallica.bnf.fr/ark:/12148/bpt6k23660m/f251}{249}.} Enfin nous devons à Plutarque\gls{CoEg_pers_00000109}\\
(2) le renseignement qu’on trouvera contenu\\
dans le passage suivant de son utile Traité\\
sur Osiris : ποιεῖ δὲ τετράγωνον ἡ πεντὰς\\
ἀφ´ ἑαυτῆς, ὅσον τῶν γραμμάτων παρ´ Αἰγυπ-\\
-τίοις τὸ πλῆθός ἐστι, καὶ ὅσων ἐνιαυτῶν\\
ἔζη χρόνον ὁ Ἆπις. \textit{Multiplié par lui-\\
-même, le nombre cinq produit un carré\\
égal au nombre de lettres égyptiennes et\\
à celui des années que vit Apis\gls{CoEg_pers_00000011}.} Ainsi,\\
par un usage bien extraordinaire, ce dieu\\
dont on célébrait la naissance avec de si\\
grandes manifestations de joie et dont on\\
pleurait la mort avec tant de marques\\
de deuil ne pouvait vivre au-delà d’un\\
certain nombre d’années \sout{dont Plutarque\gls{CoEg_pers_00000109}\\
fixe le chiffre à vingt-cinq}, et on le\\
noyait dans une fontaine sacrée quand\\
la vieillesse le conduisait à l’âge qu’il\\
lui était défendu de franchir. – On voit\\
\scriptsize{déjà d’ici \sout{où} \textsuperscript{à quelle conclusion} nous mène cette fin inattendue}\\
\footnotesize {du dieu. « Apis\gls{CoEg_pers_00000011}, dit en effet Plutarque\gls{CoEg_pers_00000109} (3),}\\
« a plusieurs traits de ressemblance avec\\
« les [formes/parures ?] de la lune par le mélange des\\
« marques claires et obscures qu’il a sur\\
« le corps » C’est à la lune elle-même\\
qu’Apis\gls{CoEg_pers_00000011}, selon Ammien Marcellin\gls{CoEg_pers_00000166} (4)
était consacré, comme Mnévis\gls{CoEg_pers_00000156} au soleil.\\
C’est encore à la lune qu’il doit en quelque\\
sorte la naissance, puisque, selon quelques\\
écrivains à la tête desquels se place\\
Hérodote\gls{CoEg_pers_00000108}, le feu céleste qui féconde\\
la vâche [\textit{sic}]-mère est une [vapeur/partie ?] de\\
la Lune. En s’arrêtant aux seuls\\
témoignages classiques, Apis\gls{CoEg_pers_00000011} \sout{peut} \textsuperscript{pouvait} donc,
\end{paracol}

{\footnotesize\begin{center} {[13\textsuperscript{e} page]}\end{center}}

\begin{paracol}{2}
\noindent \\
\\
\\
\\
\\
\\
\\
\\
\\
\\
\\
\\
\\
\\
\\
\\
\\
\\
\\
\\
\\
\\
\\
\\
\\
\\
\\
\\
(1) \textit{Die Chronologie des Aegypter – Ein-\\
-leitung und erstes Hieil Kritik der\\
quellen}\gls{CoEg_bibl_00000010}, \gls{CoEg_abbr_00000032} 160, Berlin\gls{CoEg_place_00000051}, 1849. \Gls{CoEg_abbr_00000040}\\
Dodwell\gls{CoEg_pers_00000193}, \textit{Append. ad dissert. Cyprian.}\gls{CoEg_bibl_00000044}\\
§ 14 ; Marsham\gls{CoEg_pers_00000194}, \textit{Can. Chronic.}\gls{CoEg_bibl_00000045}, \gls{CoEg_abbr_00000032} 9 ;\\
Vignoles\gls{CoEg_pers_00000196}, \textit{Ann. Aegypt. in Miscell.\\
Berolin.}\gls{CoEg_bibl_00000046}, \gls{CoEg_abbr_00000034} IV, \gls{CoEg_abbr_00000032} 11.
\switchcolumn
\noindent à la rigueur, revêtir dans une des parties\\
de son dogme des attributs qui le\\
rapprochent de la lune à laquelle il\\
serait plus spécialement consacré.\\
Or les vingt-cinq ans de vie accordés\\
au dieu ne trouvent-ils pas dans ces\\
rapprochements une confirmation régulière ?\\
Apis\gls{CoEg_pers_00000011}, divinité luni-solaire mise à mort\\
à vingt-cinq ans, ne représenterait-il\\
pas ce cycle également luni-solaire qui,\\
tous les vingt-cinq ans, ramenant en\\
conjonction (ἀποκατάστασις) le soleil et\\
la lune aux mêmes points du ciel,\\
se serait en quelque sorte personnifié dans\\
Apis\gls{CoEg_pers_00000011} ? Apis\gls{CoEg_pers_00000011} ne serait donc, en définitive,\\
que le symbole vivant d’un cycle astro-\\
-nomique, et il n’est pas besoin d’appuyer\\
long-temps [\textit{sic}] sur cette conclusion pour faire\\
voir quel secour inespéré l’histoire et la\\
chronologie trouveraient dans la série\\
des Apis\gls{CoEg_pers_00000011} révélée par le \Gls{CoEg_entry_00000028}\gls{CoEg_place_00000004}, s’il\\
était bien prouvé que ces animaux se\\
suivaient de quart de siècle en quart\\
de siècle dans les souterrains du temple.\\
Malheureusement, j’ai le regret de dire\\
que, malgré les doctes investigations\\
de \gls{CoEg_abbr_00000001} Lepsius\gls{CoEg_pers_00000061}\textsuperscript{(1)}, la tombe d’Apis\gls{CoEg_pers_00000011} s’est\\
toujours refusée à nous livrer la moindre\\
trace de la période si désirée, et \textsuperscript{par conséquent} \sout{sous}\\
\textsuperscript{du caractère astronomique attribué à Apis\gls{CoEg_pers_00000011}. Sous}\\
Ramsès II\gls{CoEg_pers_00000026}, quatre Apis\gls{CoEg_pers_00000011} sont morts en\\
quatorze ans, et rien ne prouve qu’on\\
ait songé à compléter, par les années\\
de l’un, ce qui manquait à l’autre\\
pour atteindre vingt cinq ans. Il y a\\
plus : à la dernière ligne d’un grand\\
\gls{CoEg_entry_00000011}\gls{CoEg_obj_imn} rédigé, sous la XXII\textsuperscript{e} dynastie\\
au nom d’un certain Pétisis\gls{CoEg_pers_00000183}, petit-fils\\
du roi Osorkon II\gls{CoEg_pers_00000195}, on lit \sout{cette phrase} :\\
\includegraphics[height=6pt]{CoEg_Mariette_hiero_1857-04-01_13.png} \footnote{\foreignlanguage{translit}{\Gls{CoEg_aeg_00000012} \gls{CoEg_aeg_00000013} \gls{CoEg_aeg_00000010} \gls{CoEg_aeg_00000014} \gls{CoEg_aeg_00000015} \gls{CoEg_aeg_00000016} 26}, « la durée de vie entière de ce dieu fut de vingt-six ans ».} \textit{la durée humaine de\\
ce dieu} (fut) \textit{de 26 ans}, \sout{qui, à mon avis} \textsuperscript{et cette phrase ne laisse,}\\
\sout{ne laisse} \textsuperscript{à mon avis,} aucune prise au doute et me\\
paraît devoir nous \sout{engager} \textsuperscript{forcer} à renoncer
\end{paracol}

{\footnotesize\begin{center} {[14\textsuperscript{e} page]}\end{center}}

\begin{paracol}{2}
\switchcolumn
\noindent sans retour à la [période ?]. Ainsi, malgré\\
toutes les apparences qui nous engageraient\\
à rapprocher Apis\gls{CoEg_pers_00000011} de la lune et les\\
vingt-cinq années d’Apis\gls{CoEg_pers_00000011} du cycle\\
lunaire qui s’accomplit en ce même\\
nombre d’années, il faut se rendre à\\
l’évidence des faits et reconnaître qu’en\\
ce point l’érudition moderne, égarée\\
par des lueurs trompeuses, avait fait\\
fausse route. Ce qu’on avait appelé la\\
période d’Apis\gls{CoEg_pers_00000011} n’existe par conséquent\\
pas. – Je demanderai à dire sur ce\\
sujet un dernier mot. La mort excep-\\
-tionnellement imposée à Apis\gls{CoEg_pers_00000011} est une\\
tradition qui \sout{peut-être avait} \textsuperscript{a} été trop\\
répandue dans l’antiquité pour être de\\
tous points contournée. Par sa parenté\\
avec Sérapis\gls{CoEg_pers_00000060}, le fameux taureau de\\
Memphis\gls{CoEg_place_00000005} avait presque pris sa place\\
dans le panthéon grec et romain, et\\
il me paraît difficile qu’en pareil cas\\
Plutarque\gls{CoEg_pers_00000109} et Pline\gls{CoEg_pers_00000165} aient été les\\
inventeurs naïfs d’un fait inexact\\
dont chacun pourrait, de leur temps\\
même, vérifier l’authenticité. On\\
doit dont croire que tout, dans la\\
tradition rapportée par ces écrivains,\\
n’est pas faux, et que peut-être la\\
science de Marshaw\gls{CoEg_pers_00000194} et de Vignoles\gls{CoEg_pers_00000196} [virgule barrée]\\
s’est fourvoyée sur les traces du seul\\
Plutarque\gls{CoEg_pers_00000109} pour n’avoir pas suffisamment\\
distingué le cycle lunaire qui n’a\\
rien de commun avec Apis\gls{CoEg_pers_00000011}, et le point\\
de dogme qui forçait les prêtres à\\
donner volontairement la mort au dieu,\\
une fois que celui-ci avait atteint\\
un âge déterminé. Envisagé de\\
cette manière, la question, ce me semble,\\
est ramené à son véritable point de\\
vue. Les rapprochements tentés entre
\end{paracol}

{\footnotesize\begin{center} {[15\textsuperscript{e} page]}\end{center}}

\begin{paracol}{2}
\switchcolumn
\noindent Apis\gls{CoEg_pers_00000011} et la période luni-solaire sont\\
d’évidence faux, puisque les monuments\\
du \Gls{CoEg_entry_00000028}\gls{CoEg_place_00000004}, avec leur autorité souveraine,\\
nous prouvent qu’Apis\gls{CoEg_pers_00000011} pouvait dépasser\\
vingt-cinq ans ; mais en devons-nous conclure\\
que le fait lui-même de la mort du dieu\\
soit dû à la seule imagination et à la\\
crédulité des écrivains qui nous font\\
connaître cet usage ? Je ne le crois pas.\\
A mon avis, la fontaine dans laquelle\\
les prêtres noyaient le taureau existait à\\
Memphis\gls{CoEg_place_00000005} ; seulement ce n’est pas à vingt-\\
-cinq ans qu’on l’y menait, mais à\\
vingt-huit. Apis\gls{CoEg_pers_00000011} est en effet, comme nous\\
le savons déjà, l’image la plus parfaite\\
d’Osiris\gls{CoEg_pers_00000151} ; bien plus, il est Osiris\gls{CoEg_pers_00000151} lui-même\\
naissant, vivant et mourant sur la terre.\\
Or Osiris\gls{CoEg_pers_00000151} fut violemment mis à mort à\\
vingt-huit ans. Dès lors pourquoi Apis\gls{CoEg_pers_00000011}\\
ne serait-il pas mort comme lui, c’est-à-\\
dire à vingt-huit ans, et pourquoi serait-\\
-il mort autrement que lui à vingt-cinq ?\\
d’un autre côté pourquoi aurait-il été\\
permis à Apis\gls{CoEg_pers_00000011} de dépasser un âge qu’Osiris\gls{CoEg_pers_00000151}\\
ne dépassa point ? [rature], un\\
Apis\gls{CoEg_pers_00000011} de vingt-neuf ans [rature] aurait-il\\
pu encore être Osiris\gls{CoEg_pers_00000151}, qui n’a jamais eu\\
vingt-neuf ans ? Je crois donc qu’effectivement\\
Apis\gls{CoEg_pers_00000011} terminait par une mort violente\\
une vie qu’il ne devait pas prolonger au\\
delà d’un certain temps ; mais je crois en\\
même temps que ce terme doit être reculé\\
jusqu’à vingt-huit ans, non parce qu’Apis\gls{CoEg_pers_00000011}\\
aurait été le type vivant d’une période\\
avec laquelle il n’avait absolument rien\\
à faire, mais parce que c’était un point\\
de ressemblance avec Osiris\gls{CoEg_pers_00000151}. Telle est, à\\
mon sens, la solution du fameux problème\\
de la période d’Apis\gls{CoEg_pers_00000011}. Que nos Apis\gls{CoEg_pers_00000011} vivent\\
maintenant huit ans comme celui de\\
Darius I\gls{CoEg_pers_00000057}, seize ans comme celui d’Ouaphris\gls{CoEg_pers_00000063},\\
vingt-deux ans comme celui d’Evergète II\gls{CoEg_pers_00000187}, ou\\
vingt-six ans comme l’Apis\gls{CoEg_pers_00000011} de Scheschonk IV\gls{CoEg_pers_00000198},\\
nous n’avons plus à nous en inquiéter ;
\end{paracol}

{\footnotesize\begin{center} {[16\textsuperscript{e} page]}\end{center}}

\begin{paracol}{2}
\noindent \\
\\
\\
\\
\\
\\
\\
\\
\\
\\
\\
\\
\\
\\
\\
\\
\\
\\
(1) Ce qu’une inscription hiéroglyphique du\\
\Gls{CoEg_entry_00000028}\gls{CoEg_place_00000004} exprime par \includegraphics[height=6pt]{CoEg_Mariette_hiero_1857-04-01_16.png} \footnote{Peut-être \foreignlanguage{translit}{\gls{CoEg_aeg_00000017} \gls{CoEg_aeg_00000018}w} «~complet de forme/à la manifestation achevée~» (?)  plutôt que \foreignlanguage{translit}{\gls{CoEg_aeg_00000017} \gls{CoEg_aeg_00000028}} «~qui n'est pas advenu~» (?). Cette citation est sans doute à rapprocher de \hyperlink{CoEg_Mariette_1857-04-01_31}{celle de la 31\textsuperscript{e} page} de ce rapport, que Mariette traduit par «~incréé~».} (stèle\gls{CoEg_obj_imn}\\
du règne de Ramsès II\gls{CoEg_pers_00000026}).\\
\\
\\
\\
\\
\\
\\
(2) Ou peut-être vingt-huit, le chiffre des\\
années que vivaient Osiris\gls{CoEg_pers_00000151} et Apis\gls{CoEg_pers_00000011}.
\switchcolumn
\noindent nos Apis\gls{CoEg_pers_00000011} vivent ce qu’ils peuvent sans qu’on\\
songe à compléter par l’un le cycle commencé\\
par l’autre, et le plus glorieux d’entre\\
eux sans doute est celui qui, image\\
accomplie d’Osiris\gls{CoEg_pers_00000151}, prolonge sa vie jusqu’aux\\
vingt-huit ans après lesquels, à l’exemple\\
de la victime des embûches de Typhon\gls{CoEg_pers_00000012},\\
il termine son existence dans les eaux\\
du Nil\gls{CoEg_place_00000021}.\\
\indent Si, arrivés au point où nous en\\
sommes, nous jetons un regard en arrière\\
sur la route que nous venons de parcourir,\\
il me semble que les traits principaux de\\
la figure d’Apis\gls{CoEg_pers_00000011}, tels que nous les avons\\
recueillis chemin faisant, peuvent se\\
résumer ainsi qu’il suit :\\
\indent \gls{CoEg_abbr_00000005} Apis\gls{CoEg_pers_00000011} occupait à Memphis\gls{CoEg_place_00000005} même une\\
partie réservée du grand temple de Vulcain\gls{CoEg_pers_00000186} ;\\
\indent \gls{CoEg_abbr_00000006} Apis\gls{CoEg_pers_00000011} n’avait pas de père (1), \sout{; sa mère}\\
dans le sens charnel du mot ; sa mère était\\
fécondée par le dieu Phtah\gls{CoEg_pers_00000047} qui prenait la\\
forme d’un feu céleste ; elle produisait\\
Apis\gls{CoEg_pers_00000011} sans le contact du mâle ; de là le\\
culte de la mère d’Apis\gls{CoEg_pers_00000011} qui, amené à\\
Memphis\gls{CoEg_place_00000005}, était adorée comme une vâche [\textit{sic}]\\
divine ;\\
\indent \gls{CoEg_abbr_00000007} Apis\gls{CoEg_pers_00000011} se reconnaissait à vingt-neuf\\
marques (2) qu’il devait porter sur le corps ;\\
sa manifestation s’entendant du premier\\
veau qui, pourvu de ces marques, venait\\
au monde après la mort d’un Apis\gls{CoEg_pers_00000011}, une\\
fois ce jeune veau signalé et reconnu,\\
il était amené à Memphis\gls{CoEg_place_00000005} et installé\\
\sout{dans} en grande pompe dans l’étable sacrée ;\\
\indent \gls{CoEg_abbr_00000049} Apis\gls{CoEg_pers_00000011} était regardé comme Osiris\gls{CoEg_pers_00000151} fait\\
chair et vivant au milieu des hommes ;\\
il était l’incarnation d’Osiris\gls{CoEg_pers_00000151} par le\\
secours de Phtah\gls{CoEg_pers_00000047} ; de là la nécessité pour\\
Apis\gls{CoEg_pers_00000011} d’avoir son temple à Memphis\gls{CoEg_place_00000005}, ville\\
spécialement consacrée à Phtah\gls{CoEg_pers_00000047} ;\\
\indent \gls{CoEg_abbr_00000050} les Apis\gls{CoEg_pers_00000011} mourraient à tous les âges et\\
à l’instant même de leur mort, les prêtres\\
se mettaient en quête d’un successeur sans
\end{paracol}

{\footnotesize\begin{center} {[17\textsuperscript{e} page]}\end{center}}

\begin{paracol}{2}
\noindent \\
\\
\\
\\
\\
\\
\\
\\
\\
\\
\\
\\
\\
\\
\\
\\
\\
\\
\\
\\
\sout{(1)} [rature].\\
\\
\\
\\
\\
\\
\\
\\
\\
(2) J’ai le regret d’être obligé de me\\
séparer sur ce point des conclusions aux-\\
-quelles est arrivé \gls{CoEg_abbr_00000001} François Lenormant\gls{CoEg_pers_00000200}\\
dans \sout{son} le \sout{très-bon} travail \sout{sur le} \textsuperscript{bien fait qu’il}\\
\textsuperscript{a consacré au} \textit{Rituel} et qu’il a inséré au \textit{Correspondant}\\
sous ce titre : [rature] \textit{Les Livres chez les Egyptiens\gls{CoEg_bibl_00000047}}.
\switchcolumn
\noindent s’embarrasser du nombre d’années qu’avait\\
atteint le premier Apis\gls{CoEg_pers_00000011} ; mais les Apis\gls{CoEg_pers_00000011}\\
n’avaient pas le droit de dépasser vingt-\\
-huit ans, et quand la vieillesse les\\
conduisait à cet âge, les prêtres les\\
noyaient, parce qu’Osiris\gls{CoEg_pers_00000151}, le prototype\\
d’Apis\gls{CoEg_pers_00000011}, était mort lui-même à vingt-\\
huit ans.\\
\indent De l’Apis\gls{CoEg_pers_00000011} vivant adoré à Memphis\gls{CoEg_place_00000005}\\
dans le grand temple de Phtah\gls{CoEg_pers_00000047}, je passe\\
maintenant à l’Apis\gls{CoEg_pers_00000011} mort \sout{conservé dans}\\
inhumé dans les souterrains du \Gls{CoEg_entry_00000028}\gls{CoEg_place_00000004}.
\begin{center}\S . 2.\\
\noindent D’Apis\gls{CoEg_pers_00000011} mort, ou du Sérapis\gls{CoEg_pers_00000060} égyptien.\end{center}
\noindent Dans le système psychologique de l’ancienne\\
Egypte\gls{CoEg_place_00000003}, l’âme humaine, à sa séparation du\\
corps, s’identifie avec Osiris\gls{CoEg_pers_00000151}. Le mort n’est\\
plus alors un propre tel ou tel individu,\\
prêtre ou roi, soldat ou scribe ; il devient\\
Osiris\gls{CoEg_pers_00000151} \sout{(1)} Dans le grand rituel\gls{CoEg_obj_00000022} de Turin\gls{CoEg_place_00000034},\\
\textit{Aufankh\gls{CoEg_pers_00000201}} n’est jamais Aufankh\gls{CoEg_pers_00000201} tout\\
court, mais toujours Osiris\gls{CoEg_pers_00000151} Aufankh\gls{CoEg_pers_00000201},\\
et ainsi de tous les autres rituels, sans\\
aucune exception. Le dieu des morts,\\
chargé de guider l’âme à la lumière\\
divine, force celle-ci à pénétrer et à\\
s’absorber en lui-même, sans que pour\\
cela l’individualité de l’âme soit\\
anéantie (2). – Appliqué à Apis\gls{CoEg_pers_00000011}, cette\\
doctrine nous révèle la vraie signification\\
du culte d’Apis\gls{CoEg_pers_00000011} mort, en même temps\\
qu’il nous fait connaître l’origine du\\
nom du dieu principal d’Alexandrie\gls{CoEg_place_00000006}.\\
Apis, à sa mort, entre en effet comme\\
tous les défunts dans le sein d’Osiris\gls{CoEg_pers_00000151}, et\\
devient \textit{Osiris\gls{CoEg_pers_00000151}-Apis\gls{CoEg_pers_00000011}}. Or Osiris\gls{CoEg_pers_00000151}-Apis\gls{CoEg_pers_00000011},\\
c’est l’Ὁσόραπις des papyrus, d’où\\
évidemment les Grecs ont tiré le nom de
\end{paracol}

{\footnotesize\begin{center} {[18\textsuperscript{e} page]}\end{center}}

\begin{paracol}{2}
\noindent
\\ (1). \Gls{CoEg_abbr_00000031} Champollion\gls{CoEg_pers_00000094}, \textit{Dict. Eg.}\gls{CoEg_bibl_00000048}, \gls{CoEg_abbr_00000032} 64 ;\\
Brunet de Presles\gls{CoEg_pers_00000051}, \textit{Mémoire sur le\\
Sérapéum de Memphis}\gls{CoEg_bibl_00000001}, \gls{CoEg_abbr_00000032} 9 ; extrait\\
du tome II de la première série des\\
Mémoires présentés par divers savants\\
à l’Académie des Inscriptions ;\\
1852.\\
\\
\\
\\
\\
\\
\\
\\
\\
\\
\\
\\
\\
\\
\\
\\
\\
\\
\\
\\
\\
\\
\\
\\
\\
\\
\\
(2) « Osiris\gls{CoEg_pers_00000151}, dit Plutarque\gls{CoEg_pers_00000109}, aime à faire du\\
« bien (ἀγαθοποὶος) . . . le second nom qu’on donne\\
« à ce dieu et qui est celui d’Onuphris (\includegraphics[height=6pt]{CoEg_Mariette_hiero_1857-04-01_18_2.png} \footnote{\foreignlanguage{translit}{\gls{CoEg_aeg_00000019}} « Ounennéfer » (littéralement « celui qui est continuellement bon »).}
« l’être \textit{bon}) signifie \textit{bienfaisant} (εὐεργέτης).\\
\textit{De Is. et Osir.}\gls{CoEg_bibl_00000029}, \gls{CoEg_abbr_00000051} III, et suiv.
\switchcolumn
\noindent la divinité qu’ils ont révélée au monde\\
sous le nom de Sérapis\gls{CoEg_pers_00000060} (1). Le berceau de\\
ce dieu que les Pères appellent le\\
transfuge de Sinope\gls{CoEg_place_00000053} ne doit donc pas\\
être cherché autre part que dans la\\
tombe d’Apis\gls{CoEg_pers_00000011} ; Sérapis\gls{CoEg_pers_00000060} n’est originai-\\
-rement qu’Apis\gls{CoEg_pers_00000011} mort, et c’est là un\\
fait qui me semble irrévocablement\\
acquis à l’histoire des religions de\\
l’antiquité. – Mais maintenant faut-il\\
s’arrêter là et ne voir dans Apis\gls{CoEg_pers_00000011} qu’un\\
défunt vulgaire qui, à l’exemple de\\
tout le monde, prend à sa mort le nom\\
d’Osiris\gls{CoEg_pers_00000151} ? l’identification d’Osiris\gls{CoEg_pers_00000151} et\\
d’Apis\gls{CoEg_pers_00000011}, déjà sûre de son vivant, n’est-\\
-elle pas, si je puis m’exprimer ainsi,\\
d’un degré supérieure ? La réponse à cette\\
question n’est pas douteuse. Je disais\\
tout-à-l’heure que la fusion d’Osiris\gls{CoEg_pers_00000151}\\
et d’Aufankh\gls{CoEg_pers_00000201} n’est pas si complète que\\
l’individualité de l’âme disparaisse.\\
Il est remarquable en effet que jamais\\
le défunt ne s’identifie avec le dieu\\
au point de prendre les titres caractéris-\\
-tiques de celui-ci ; jamais, par\\
exemple, vous ne trouverez \includegraphics[height=6pt]{CoEg_Mariette_hiero_1857-04-01_18_1_1.png}\\
\includegraphics[height=6pt]{CoEg_Mariette_hiero_1857-04-01_18_1_2.png} \footnote{\foreignlanguage{translit}{\gls{CoEg_aeg_00000024} \gls{CoEg_aeg_00000020} \gls{CoEg_aeg_00000021} \gls{CoEg_aeg_00000031} \gls{CoEg_aeg_00000023}} « l'Osiris Ioufânkh, juste de voix, qui préside à l'Occident ».} l’Osiris Aufankh\gls{CoEg_pers_00000201},\\
le justifié, qui réside dans l’\gls{CoEg_entry_00000025}.\\
Le défunt ne devient donc Osiris\gls{CoEg_pers_00000151} que\\
jusqu’à une certaine limite dans laquelle\\
ne sont point compris les attributs\\
propres à l’essence du dieu. Mais\\
Apis\gls{CoEg_pers_00000011} est-il dans ces conditions ? nullement.\\
Si Apis\gls{CoEg_pers_00000011} vivant est Osiris\gls{CoEg_pers_00000151} fait chair ;\\
s’il est le divin \textit{bienfaiteur} par excellence (2)\\
\xout{\includegraphics[height=6pt]{CoEg_Mariette_hiero_1857-04-01_18_2.png}}\footnote{\foreignlanguage{translit}{\gls{CoEg_aeg_00000019}} « Ounennéfer ».} {[rature]}\\
{[rature]}) descendu sur la terre, Apis\gls{CoEg_pers_00000011} mort\\
sera Apis\gls{CoEg_pers_00000011}, incarnation d’Osiris\gls{CoEg_pers_00000151}, rentré\\
à sa mort dans le sein du dieu qu’il\\
avait représenté ici-bas ; il sera Osiris\gls{CoEg_pers_00000151}
\end{paracol}

{\footnotesize\begin{center} {[19\textsuperscript{e} page]}\end{center}}

\begin{paracol}{2}
\noindent \\
\\
\\
\\
\\
\\
\\
\\
\\
\\
\\
\\
(1) \textit{Notice sommaire des Monuments égyptiens\\
du Louvre}\gls{CoEg_bibl_00000049}, \gls{CoEg_abbr_00000032} 110.\\
\\
\\
\\
\\
\\
\\
\\
\\
\\
\\
\\
\\
\\
\\
\\
\\
\\
\\
\\
\\
\\
\\
(2) \textit{Saturn.}\gls{CoEg_bibl_00000050}, \gls{CoEg_abbr_00000051} I, \gls{CoEg_abbr_00000037} 7.\\
\switchcolumn
\noindent revenu de son pélérinage [\textit{sic}] sur la terre.\\
Tel sera Apis\gls{CoEg_pers_00000011} mort. L’égyptien fidèle\\
aux antiques traditions, séparant\\
Osiris-Aufankh\gls{CoEg_pers_00000201} d’Osiris-Apis\gls{CoEg_pers_00000011}, pouvait\\
donc sans s’étonner lire sur des milliers\\
de statuettes funéraires \includegraphics[height=6pt]{CoEg_Mariette_hiero_1857-04-01_19_1.png}\\
\includegraphics[height=6pt]{CoEg_Mariette_hiero_1857-04-01_19_2.png} \footnote{\foreignlanguage{translit}{\gls{CoEg_aeg_00000024}-\gls{CoEg_aeg_00000002}, \gls{CoEg_aeg_00000014} \gls{CoEg_aeg_00000025} \gls{CoEg_aeg_00000031} \gls{CoEg_aeg_00000023}} « l'Osiris Apis, grand dieu, qui préside à l'Occident ».} \textit{Osiris-Apis, dieu grand,\\
qui réside dans l’\gls{CoEg_entry_00000025}} ; il voyait\\
en définitive dans Apis\gls{CoEg_pers_00000011} mort, non plus\\
un être absorbé en Osiris\gls{CoEg_pers_00000151}, mais Osiris\gls{CoEg_pers_00000151}\\
lui-même dans la personnification de\\
l’un de ses rôles les plus glorieux : celui\\
d’un dieu, type de l’homme, comme l’a\\
appelé \gls{CoEg_abbr_00000001} de Rougé\gls{CoEg_pers_00000032} (1), mort au\\
milieu des hommes. – Ainsi se révèlent,\\
et le nom véritable de Sérapis\gls{CoEg_pers_00000060}, et la\\
raison d’être du culte de ce dieu qui,\\
loin d’être un produit de l’esprit\\
hellénique à l’un des premiers contacts\\
des deux religions grecques et égyptiennes,\\
\sout{et} procède au contraire \sout{tout entier} de\\
cette source tout entière égyptienne qui\\
remonte dans la nuit des temps jusqu’à\\
près de trois mille ans avant la fondation\\
d’Alexandrie\gls{CoEg_place_00000006}.\\
\indent Le temple consacré à Apis\gls{CoEg_pers_00000011} mort était\\
le \Gls{CoEg_entry_00000028}\gls{CoEg_place_00000004} de Memphis\gls{CoEg_place_00000005}.\\
\indent Le \Gls{CoEg_entry_00000028}\gls{CoEg_place_00000004} était situé à quelques\\
kilomètres de Memphis\gls{CoEg_place_00000005} et au milieu de\\
l’un des cimetières de cette ville. En 1850,\\
j’ai eu la fortune d’en retrouver les premiers\\
vestiges entre la grande Pyramide de\\
Sakkarah\gls{CoEg_place_00000001} et les buttes ruinées d’Abousyr\gls{CoEg_place_00000008}.\\
Tandis que Memphis\gls{CoEg_place_00000005} elle-même abritait\\
l’\Gls{CoEg_entry_00000035} proprement dit, le \Gls{CoEg_entry_00000028}\gls{CoEg_place_00000004}\\
occupait donc sa place au milieu des\\
tombeaux. Macrobe\gls{CoEg_pers_00000202} (2) avait déjà\\
fait remarquer que les temples de\\
Sérapis\gls{CoEg_pers_00000060} étaient toujours exclus de\\
l’enceinte des villes égyptiennes.\\
\indent Le \Gls{CoEg_entry_00000028}\gls{CoEg_place_00000004} se composait de deux\\
temples, d’époque et d’origines différentes,
\end{paracol}

{\footnotesize\begin{center} {[20\textsuperscript{e} page]}\end{center}}

\begin{paracol}{2}
\noindent \\
\\
\\
\\
\\
\\
\\
\\
\\
(1) Je les ai retrouvés dans le même\\
état ; seulement en quelques parties\\
de l’allée la couche de sable\\
n’avait pas moins de quatre-vingts\\
pieds d’épaisseur.\\
(2) Geogr.\gls{CoEg_bibl_00000023}, \gls{CoEg_abbr_00000051} XVII, \gls{CoEg_abbr_00000028} 1, § 14.\\
\\
\\
\\
\\
\\
\\
\\
\\
\\
\\
(3) I\gls{CoEg_bibl_00000051}, 18.
\switchcolumn
\noindent réunis par une allée de sphinx qui\\
n’avait pas moins de neuf cents mètres\\
de longueur. Strabon\gls{CoEg_pers_00000159} a mentionné cette\\
allée de sphinx dans un passage\\
célèbre : « On trouve à Memphis\gls{CoEg_place_00000005},\\
« dit le géographie, un temple de Sérapis\gls{CoEg_pers_00000060} [rature]\\
« dans un endroit tellement sablonneux\\
« que les vents y accumulent des amas de\\
« sable sous lesquels nous vîmes des sphinx\\
« enterrés (1), les uns à moitié, les autres\\
« jusqu’à la tête : d’où l’on peut conjecturer\\
« que la route vers le temple ne serait\\
« pas sans danger, si l’on était surpris par\\
« un coup de vent (2) » Strabon\gls{CoEg_pers_00000159} n’aurait\\
pas écrit ces lignes que, vraisemblablement,\\
le \Gls{CoEg_entry_00000028}\gls{CoEg_place_00000004} serait encore aujourd’hui sous\\
les sables qui l’ont recouvert pendant\\
tant de siècles.\\
\indent Le principal des deux temples qui for-\\
-maient le \Gls{CoEg_entry_00000028}\gls{CoEg_place_00000004} de Memphis\gls{CoEg_place_00000005} était\\
situé à l’extrémité occidentale de l’allée\\
de sphinx. J’ai la certitude qu’il existait\\
déjà sous Aménophis III\gls{CoEg_pers_00000091}, l’un des rois\\
de la XVIII\textsuperscript{\underline{e}} dynastie, et qu’on y venait\\
encore adorer Sérapis\gls{CoEg_pers_00000060} sous Ptolémée\\
Césarion\gls{CoEg_pers_00000203}. Pausanias\gls{CoEg_pers_00000204} (3) a donc pu dire\\
avec raison : « Le plus ancien des temples\\
de \sout{Mem} Sérapis\gls{CoEg_pers_00000060} est à Memphis\gls{CoEg_place_00000005} ». Si\\
l’on jette les yeux sur le plan de cet\\
édifice, on s’aperçoit bien vite d’un\\
fait sur lequel il est important\\
d’insister ; c’est que le \Gls{CoEg_entry_00000028}\gls{CoEg_place_00000004} égyptien\\
a été bâti tout entier pour la tombe\\
d’Apis\gls{CoEg_pers_00000011} et les souterrains ouverts aujour-\\
-d’hui à la curiosité des voyageurs.\\
Le dieu adoré dans le sanctuaire du\\
\Gls{CoEg_entry_00000028}\gls{CoEg_place_00000004}, c’est-à-dire Sérapis\gls{CoEg_pers_00000060}, est\\
donc bien, comme je viens de l’indiquer,\\
Apis\gls{CoEg_pers_00000011} mort.\\
\indent Une autre remarque également digne\\
d’attention se tire de l’état actuel
\end{paracol}

{\footnotesize\begin{center} {[21\textsuperscript{e} page]}\end{center}}

\begin{paracol}{2}
\noindent \\
\switchcolumn
\noindent des lieux et de l’impossibilité complète\\
où j’ai été de trouver un seul mot grec\\
dans l’enceinte du \Gls{CoEg_entry_00000028}\gls{CoEg_place_00000004}. En vain\\
d’Alexandre\gls{CoEg_pers_00000145} au fils de César\gls{CoEg_pers_00000205} et de\\
Cléopâtre\gls{CoEg_pers_00000206}, les Ptolémées vinrent-ils à\\
l’envi accomplir leurs actes de dévotion\\
dans les temples ; en vain, en souvenir\\
soit de ces visites, soit des Apis qui\\
moururent sous le règne de ces princes,\\
le \Gls{CoEg_entry_00000028}\gls{CoEg_place_00000004} se couvrit-il de textes\\
égyptiens rédigés au nom des rois grecs\\
de l’Egypte\gls{CoEg_place_00000003} ; en vain tout autour de\\
cet édifice, la langue grecque \& le\\
style grec dominaient-ils dans les restes\\
que j’ai retrouvés ; une fois le pylône\\
d’entrée franchi, le grec disparaît\\
totalement, au point que, dans les deux\\
cents \glspl{CoEg_entry_00000011} ptolémaïques recueillis\\
en diverses parties du temple, on ne\\
trouve pas une seule lettre grecque.\\
La conclusion nécessaire de cet état de\\
choses est celle-ci : c’est que le dieu\\
adoré dans le \Gls{CoEg_entry_00000028}\gls{CoEg_place_00000004} de Memphis\gls{CoEg_place_00000005} se\\
refusa toujours, même pendant la\\
domination grecque, à être grec, et\\
qu’il persista à rester sous les Lagides\\
ce qu’il avait été sous les \Glspl{CoEg_entry_00000029},\\
c’est-à-dire un dieu purement égyptien.\\
– Ainsi déjà se distinguent deux Sérapis\gls{CoEg_pers_00000060} :\\
l’un dont Memphis\gls{CoEg_place_00000005}, en vertu des\\
lois sacrées, gardait le temple et qui\\
fut le Sérapis\gls{CoEg_pers_00000060} égyptien, ou Apis\gls{CoEg_pers_00000011}\\
mort, sous les \Glspl{CoEg_entry_00000029} comme sous\\
les Ptolémées ; l’autre que nous ren-\\
-controns à Alexandrie\gls{CoEg_place_00000006}, et qui, par\\
là seul [virgule barrée] n’étant plus Apis\gls{CoEg_pers_00000011} mort, revêt\\
un caractère nouveau qui nous forcera\\
tout-à-l’heure à voir en lui un\\
Sérapis\gls{CoEg_pers_00000060} que le panthéon égyptien\\
ne compte point parmi ses dieux.\\
\indent A l’extrémité orientale de l’allée\\
de sphinx se trouvait le second des
\end{paracol}

{\footnotesize\begin{center} {[22\textsuperscript{e} page]}\end{center}}

\begin{paracol}{2}
\noindent \\
\\
\\
\\
\\
\\
\\
\\
\\
\\
(1) Duc de Luynes\gls{CoEg_pers_00000207}, \textit{Inscription phénicienne\\
sur une pierre à libation du Sérapéum\\
de Memphis}\gls{CoEg_bibl_00000052}, dans le \textit{Bulletin Archéo-\\
logique de l’Athenæum Français},\\
\gls{CoEg_abbr_00000034} 1, \gls{CoEg_abbr_00000032} 77, 78.\\
\\
\\
\\
\\
\\
\\
\\
\\
\\
\\
\\
\\
\\
\\
\\
\\
\\
\\
\\
(2) Voyez aussi Bernard. Peyron\gls{CoEg_pers_00000211}, \textit{Papyri\\
greci del Museo britannico di Londra\\
e della bibliotheca Vaticana}\gls{CoEg_bibl_00000053}, Turin\gls{CoEg_place_00000034},\\
1841 ; Reuvens\gls{CoEg_pers_00000212}, \textit{Lettre à \gls{CoEg_abbr_00000001} Letronne}\gls{CoEg_bibl_00000054},\\
\gls{CoEg_abbr_00000034} III, \gls{CoEg_abbr_00000032} 84 et suiv. ; Letronne\gls{CoEg_pers_00000097},\\
\textit{Inscriptions grecques et latines de\\
l’Egypte}\gls{CoEg_bibl_00000014}, \gls{CoEg_abbr_00000034} 1, \gls{CoEg_abbr_00000032} 208, \gls{CoEg_abbr_00000034} II, \gls{CoEg_abbr_00000032} 482\\
\gls{CoEg_abbr_00000052}
\switchcolumn
\noindent deux temples dont se composent le\\
\Gls{CoEg_entry_00000028}\gls{CoEg_place_00000004}. Celui-ci ne remonte plus\\
à Aménophis III\gls{CoEg_pers_00000091} et n’a pas à l’endroit\\
du grec, le parti pris du \Gls{CoEg_entry_00000028}\gls{CoEg_place_00000004}\\
égyptien. Au contraire, architecture,\\
art, écriture, tout y est grec. A son\\
tour l’égyptien est exclu de ces lieux\\
qu’il semble ne point connaître. Evi-\\
-demment ce temple servait aux Grecs\\
ainsi qu’aux étrangers établis en assez\\
grand nombre à Memphis\gls{CoEg_place_00000005} (1) et l’on y\\
sacrifiait au dieu mixte dont les\\
Alexandrins avaient inauguré la\\
statue dans leurs murs. Quoique je\\
n’en aie pas trouvé la preuve directe,\\
j’ai la conviction que le \Gls{CoEg_entry_00000028}\gls{CoEg_place_00000004} dans\\
lequel s’accomplirent les faits rapportés\\
par les papyrus grecs\gls{CoEg_obj_imn} du Musée\gls{CoEg_org_00000005} de\\
Londres\gls{CoEg_place_00000011} et de Paris\gls{CoEg_place_00000002} est le \Gls{CoEg_entry_00000028}\gls{CoEg_place_00000004} que\\
nous avons maintenant sous les yeux. Là,\\
à côté d’une chapelle consacrée à Anubis\gls{CoEg_pers_00000068}\\
se trouvait la chapelle dédiée à l’Astarté\gls{CoEg_pers_00000208}\\
des Phéniciens ; là, dans les mêmes\\
bâtiments qui logeaient les \glspl{CoEg_entry_00000036}\\
du temple, vivaient les deux \glspl{CoEg_entry_00000037}\\
sœurs toujours jumelles chargées de\\
représenter Isis\gls{CoEg_pers_00000209} et Nephthys\gls{CoEg_pers_00000210} dans les\\
cérémonies funèbres de Sérapis\gls{CoEg_pers_00000060} ; là\\
se voyaient aussi les κάτοχοι, cénobites\\
païens qui, voués à une prison volontaire,\\
prédisaient l’avenir ou guérissaient les\\
malades par des songes ; là se tenaient\\
des marchés et se vendaient des denrées\\
de toute nature ; là enfin se rencontrait\\
tout le vaste ensemble d’administrateurs,\\
de soldats, de prêtres, de marchands,\\
d’illuminés, qui donnaient au \Gls{CoEg_entry_00000028}\gls{CoEg_place_00000004}\\
grec de Memphis\gls{CoEg_place_00000005} le caractère si bien\\
résumé dans le beau \textit{Mémoire}\gls{CoEg_bibl_00000001} de\\
\gls{CoEg_abbr_00000001} Brunet de Presles\gls{CoEg_pers_00000051} (2).\\
\indent Ces quelques mots suffisent pour montrer\\
qu’il existe entre le Sérapis\gls{CoEg_pers_00000060} d’origine
\end{paracol}

{\footnotesize\begin{center} {[23\textsuperscript{e} page]}\end{center}}

\begin{paracol}{2}
\noindent \\
\\
\\
\\
\\
\\
\\
\\
\\
\\
\\
\\
\\
\\
\\
\\
\\
\\
\\
\\
\\
\\
\\
\\
\\
\\
\\
\\
\\
\\
\\
\\
\\
\\
\\
\\
\\
\\
\\
\\
(1) \textit{Hist.}\gls{CoEg_bibl_00000040} \gls{CoEg_abbr_00000051} IV, \gls{CoEg_abbr_00000028} 83, 84.
\switchcolumn
\noindent égyptienne et le Sérapis\gls{CoEg_pers_00000060} d’importation\\
grecque une différence radicale que\\
la nécessité où les Egyptiens se sont\\
trouvés de conserver à chacun d’entre\\
eux un temple spécial fait mieux\\
ressortir encore. Le véritable Sérapis\gls{CoEg_pers_00000060},\\
le Sérapis\gls{CoEg_pers_00000060} national et antique est, je\\
le répète encore une fois, Apis\gls{CoEg_pers_00000011} mort, et\\
il n’est pas autre chose, \textsuperscript{même} pendant la\\
domination grecque. Le Sérapis\gls{CoEg_pers_00000060} grec,\\
au contraire, quoique vivant à côté du\\
premier, possède un dogme et des attributs\\
qui l’éloignent de lui et nous forcent à\\
le regarder en quelque sorte comme un\\
dieu nouveau. C’est ce que nous allons\\
voir dans le paragraphe suivant.
\begin{center}
§ III.\\
Du Sérapis\gls{CoEg_pers_00000060} grec.\end{center}
L’origine du Sérapis\gls{CoEg_pers_00000060} grec ne se perd\\
pas, comme l’origine du Sérapis\gls{CoEg_pers_00000060} égyptien,\\
dans la nuit des temps. Le premier des\\
Lagides, Ptolémée Sôter\gls{CoEg_pers_00000213}, vers l’an 300\\
avant Jésus-Christ\gls{CoEg_pers_00000235}, eut un songe.\\
Il vit un jeune homme d’une beauté\\
merveilleuse qui lui ordonnait d’envoyer\\
dans le Pont\gls{CoEg_place_00000077} le plus sûr de ses amis\\
y chercher sa statue. La statue du\\
jeune homme fut trouvée à Sinope\gls{CoEg_place_00000053} et\\
amenée à Alexandrie\gls{CoEg_place_00000006}. Dès que\\
Timothée\gls{CoEg_pers_00000214} l’interprète et Manéthon\gls{CoEg_pers_00000116} le\\
Sébennyte l’eurent vu, ils conjecturèrent\\
par un cerbère et un [dragon ?] qui y\\
étaient représentés que c’était une statue\\
de Pluton\gls{CoEg_pers_00000215}, et ils persuadèrent à\\
Ptolémée\gls{CoEg_pers_00000213} que cette statue de Pluton\gls{CoEg_pers_00000215} ne\\
pouvait être que celle du dieu égyptien\\
Sérapis\gls{CoEg_pers_00000060}. Telle est, en résumé, l’origine\\
du Sérapis\gls{CoEg_pers_00000060} d’Alexandrie\gls{CoEg_place_00000006}, comme nous\\
la trouvons racontée dans les [\sout{récits} ?] \textsuperscript{ouvrages} de\\
Tacite\gls{CoEg_pers_00000111} (1) et de quelques autres écrivains\\
\end{paracol}

{\footnotesize\begin{center} {[24\textsuperscript{e} page]}\end{center}}

\begin{paracol}{2}
\noindent (1) \textit{De Is. et Osir.}\gls{CoEg_bibl_00000029}, [rature] XXVI, XXVII.\\
(2) \textit{Protrept.}\gls{CoEg_bibl_00000055}, \gls{CoEg_abbr_00000032} 13.\\
(3) \textit{Saturn.}\gls{CoEg_bibl_00000050}, \gls{CoEg_abbr_00000051} I, \gls{CoEg_abbr_00000028} 7. \Gls{CoEg_abbr_00000031} aussi\\
Denys le Périégète\gls{CoEg_pers_00000216}, \textit{in descript. Orb.}\gls{CoEg_bibl_00000056}\\
\gls{CoEg_abbr_00000053} 255 ; Théophile\gls{CoEg_pers_00000242} d’Antioche\gls{CoEg_place_00000078}, \textit{ad\\
Autolyc.}\gls{CoEg_bibl_00000057} \gls{CoEg_abbr_00000051} 1, \gls{CoEg_abbr_00000028} 14 ; Cyrille\gls{CoEg_pers_00000217} d’Ale-\\
xandrie\gls{CoEg_place_00000006}, \textit{advers. Julian}\gls{CoEg_bibl_00000058}, \gls{CoEg_abbr_00000028} 1, \gls{CoEg_abbr_00000032} 13,\\
etc.\\
\\
\\
\\
\\
\\
(4) \Gls{CoEg_abbr_00000042}\gls{CoEg_bibl_00000029}
\switchcolumn \noindent parmi lesquels on peut citer Plutarque\gls{CoEg_pers_00000109} (1),\\
Clément\gls{CoEg_pers_00000112} d’Alexandrie\gls{CoEg_place_00000006} (2) et Macrobe\gls{CoEg_pers_00000202}\\
(3). Si nous en croyons ces auteurs, le dieu\\
qui plus tard emplit le monde de son\\
nom était donc un dieu emprunté à\\
la religion grecque par les Grecs d’Egypte\gls{CoEg_place_00000003},\\
quelques années seulement après la conquête\\
macédonienne ; il était Pluton\gls{CoEg_pers_00000215} lui-même\\
qu’une assimilation plus ou moins juste\\
de Timothée\gls{CoEg_pers_00000214} l’interprète et de Manéthon\gls{CoEg_pers_00000116}\\
le Sébennyte identifiaient avec le Sérapis\gls{CoEg_pers_00000060}\\
égyptien, « car ce n’est pas Sérapis\gls{CoEg_pers_00000060},\\
« dit Plutarque\gls{CoEg_pers_00000109} (4), qu’on appelait ce dieu\\
« à Sinope\gls{CoEg_place_00000053}, mais arrivé à Alexandrie\gls{CoEg_place_00000006} il\\
« y reçut ce nom, qui est celui que les\\
« Egyptiens donnent à Pluton\gls{CoEg_pers_00000215} . . . . »\\
\indent La connaissance que nous possédons\\
maintenant de l’antique et véritable\\
Sérapis\gls{CoEg_pers_00000060}, de son origine, de l’idée philo-\\
-sophique dont il est le symbole, nous\\
permet-elle d’accepter comme vraie\\
la tradition dont Tacite\gls{CoEg_pers_00000111} s’est fait le\\
principal écho ? C’est ici que, tout en\\
reconnaissant l’importance du problème,\\
je dois avouer que je n’en aperçois que\\
confusément encore la solution. Jusqu’à\\
ce que des matériaux mieux étudiés ou\\
plus abondants nous apportent les\\
éléments d’une conviction plus arrêtée,\\
je crois cependant que la tradition dont\\
nous nous occupons ne doit être accueillie\\
qu’avec une grande réserve. Il me paraît\\
en effet difficile d’admettre, en premier\\
lieu que l’élévation soudaine \textsuperscript{et brillante} de\\
Sérapis\gls{CoEg_pers_00000060} se soit accomplie à une époque\\
aussi reculée que celle de Sôter\gls{CoEg_pers_00000213}, en\\
second lieu que Sérapis\gls{CoEg_pers_00000060} ne soit que le\\
Pluton\gls{CoEg_pers_00000215} des traditions helléniques. Des\\
deux parts certains arguments nous amènent\\
à des conclusions contraires. Un mot\\
d’explication le prouvera.
\end{paracol}

{\footnotesize\begin{center} {[25\textsuperscript{e} page]}\end{center}}

\begin{paracol}{2}
\noindent \\
\\
\\
\\
\\
\\
\\
\\
\\
\\
\\
\\
\\
(1) En l’an 2 de Jésus-Christ\gls{CoEg_pers_00000235}. \Gls{CoEg_abbr_00000031} Letronne\gls{CoEg_pers_00000097},\\
\textit{Inscript. gr. et lat. de l’Egypte}\gls{CoEg_bibl_00000014}, \gls{CoEg_abbr_00000034} II,\\
\gls{CoEg_abbr_00000032} 161 et 167.\\
\\
(1) Letronne\gls{CoEg_pers_00000097}, \sout{\textit{Inscr. gr. et lat.}} \textsuperscript{\Gls{CoEg_abbr_00000054}}\gls{CoEg_bibl_00000014}, \gls{CoEg_abbr_00000034} 1, \gls{CoEg_abbr_00000032} 121,\\
temple de Cysis\gls{CoEg_place_00000069}, et \gls{CoEg_abbr_00000034} I, \gls{CoEg_abbr_00000032} 427, Mont\\
Claudien.\\
(2) \Gls{CoEg_abbr_00000054}\gls{CoEg_bibl_00000014}, temple du Mont Claudien, \gls{CoEg_abbr_00000034} I,\\
\gls{CoEg_abbr_00000032} 153.\\
(3) \Gls{CoEg_abbr_00000054}\gls{CoEg_bibl_00000014} Alexandrie, \gls{CoEg_abbr_00000034} I, \gls{CoEg_abbr_00000032} 445.\\
(4) \Gls{CoEg_abbr_00000054}\gls{CoEg_bibl_00000014}, \gls{CoEg_abbr_00000034} II, \gls{CoEg_abbr_00000032} 228.\\
\\
\\
\\
\\
\\
\\
\\
\\
\\
\\
\\
\\
\\
\rotatebox{90}{‡} (\includegraphics[height=6pt]{CoEg_Mariette_hiero_1857-04-01_25.png} \footnote{\foreignlanguage{translit}{\gls{CoEg_aeg_00000024}-\gls{CoEg_aeg_00000002}, \gls{CoEg_aeg_00000014} \gls{CoEg_aeg_00000025} \gls{CoEg_aeg_00000031} \gls{CoEg_aeg_00000023}} « l'Osiris Apis, grand dieu, qui préside à l'Occident ».}),
\switchcolumn
\noindent Les inscriptions grecques et latines recueillies\\
dans les diverses parties de l’Egypte\gls{CoEg_place_00000003}\\
ne \sout{d} nous donnent pas à penser que la\\
grande faveur dont a joui Sérapis\gls{CoEg_pers_00000060} date\\
du règne de Sôter\gls{CoEg_pers_00000213}. En effet le nom de\\
Sérapis\gls{CoEg_pers_00000060} n’apparaît pas une seule fois [virgule barrée]\\
sur les monuments, hors de Memphis\gls{CoEg_place_00000005},\\
avant le règne d’Auguste\gls{CoEg_pers_00000218}. Jusqu’alors,\\
toutes les fois qu’un papyrus nous livre\\
le nom de ce dieu célèbre, c’est le Sérapis\gls{CoEg_pers_00000060}\\
égyptien de Memphis\gls{CoEg_place_00000005} qui est\\
mentionné, et jamais le Sérapis\gls{CoEg_pers_00000060} égypto-\\
grec d’Alexandrie\gls{CoEg_place_00000006}. A partir d’Auguste\gls{CoEg_pers_00000218},\\
(1) les \glspl{CoEg_entry_00000011} à Sérapis\gls{CoEg_pers_00000060} deviennent\\
plus fréquents, et on en trouve d’assez\\
nombreux commençant par la\\
formule si connue Σαράπιδι καὶ\\
Ἴσιδι, θεοῖς μεγίστοις\footnote{« À Sérapis et Isis, les très grands dieux ».} sous Trajan\gls{CoEg_pers_00000219} (1),\\
sous Adrien\gls{CoEg_pers_00000171} (2), sous Commode\gls{CoEg_pers_00000220} (3) et\\
jusques sous Gallien\gls{CoEg_pers_00000221} (4). Ainsi les\\
traces de Sérapis\gls{CoEg_pers_00000060} grec ne se rencontrent\\
pas sur les monuments avant notre ère,\\
et si le culte de ce dieu (comme il n’en\\
faut pas douter puisque nous voyons le\\
\Gls{CoEg_entry_00000028}\gls{CoEg_place_00000004} de Memphis\gls{CoEg_place_00000005} accepter dès\\
Philométor\gls{CoEg_pers_00000222} des Grecs et des Phéniciens\\
dans son enceinte) [rature] \textsuperscript{fut pratiqué par des étrangers}\\
\sout{Sôter\gls{CoEg_pers_00000213}} \textsuperscript{avant Auguste\gls{CoEg_pers_00000218}}, il ne fut pas, [rature] \textsuperscript{sous les Ptolémées},\\
aussi universellement établi que\\
voudrait nous le faire croire les Grecs. –\\
D’un autre côté ce même résultat est\\
celui auquel nous fait arriver l’étude\\
du caractère propre de Sérapis\gls{CoEg_pers_00000060}. Que\\
Sérapis\gls{CoEg_pers_00000060} soit Pluton\gls{CoEg_pers_00000215}, selon la conjecture\\
des deux personnages que Plutarque\gls{CoEg_pers_00000109}\\
appelle Timothée\gls{CoEg_pers_00000214} l’interprète et\\
Manéthon\gls{CoEg_pers_00000116} le Sébennyte, \sout{ce que} c’est\\
ce qui n’est pas prouvé. Le rôle de\\
Pluton\gls{CoEg_pers_00000215} est sans doute compris dans\\
celui d’Osorapis\gls{CoEg_pers_00000223} considéré comme\\
maître de l’enfer égyptien\rotatebox{90}{‡}, mais celui
\end{paracol}

{\footnotesize\begin{center} {[26\textsuperscript{e} page]}\end{center}}

\begin{paracol}{2}
\noindent \\
\rotatebox{90}{‡} (\includegraphics[height=6pt]{CoEg_Mariette_hiero_1857-04-01_26.png}\footnote{\foreignlanguage{translit}{\gls{CoEg_aeg_00000019}} « Ounennéfer » (littéralement « celui qui est continuellement bon »).}),\\
\\
\\
\\
\\
\\
\\
\\
(1) L’érudition moderne doit à \gls{CoEg_abbr_00000001} Alfred\\
Maury\gls{CoEg_pers_00000243} un ouvrage très-remarquable\\
que l’on consultera avec beaucoup de\\
fruit sur ce rôle du Dionysos\gls{CoEg_pers_00000224} des\\
traditions grecques. Voyez en effet\\
\textit{Histoire des religions de la Grèce\\
antique depuis les origines jusqu’à\\
leur complète constitution}\gls{CoEg_bibl_00000059}, \gls{CoEg_abbr_00000034} 1,\\
\gls{CoEg_abbr_00000032} 121.
\switchcolumn
\noindent d’Osorapis\gls{CoEg_pers_00000223} dans son type principal\\
de dieu bon\rotatebox{90}{‡}, mort au milieu des hommes,\\
est bien loin d’être compris dans le rôle\\
de Pluton\gls{CoEg_pers_00000215}. En assimilant Pluton\gls{CoEg_pers_00000215} à\\
Sérapis\gls{CoEg_pers_00000060}, les Grecs ont donc pris le moindre\\
côté de la ressemblance qui existe entre\\
ces deux divinités, et ils ont négligé\\
l’essentiel. Sérapis\gls{CoEg_pers_00000060} sera par conséquent\\
Pluton\gls{CoEg_pers_00000215}, mais il sera surtout Dionysos\gls{CoEg_pers_00000224}\\
sous sa forme de médiateur (1) et c’est en\\
définitive le dogme d’Apis\gls{CoEg_pers_00000011} mort que\\
les Grecs se seront en quelque sorte\\
approprié à l’époque où le culte du\\
grand Sérapis\gls{CoEg_pers_00000060} devient florissant à\\
Alexandrie\gls{CoEg_place_00000006}. – Maintenant ce dogme\\
avait-il quelque raison de séduire les\\
contemporains de Sôter\gls{CoEg_pers_00000213}, trois cents ans\\
avant Jésus-Christ\gls{CoEg_pers_00000235} ? Considérer comme\\
démiurge Phtah\gls{CoEg_pers_00000047}, qui effectivement\\
remplit dans la cosmogonie égyptienne\\
la fonction d’organisateur, et en même\\
temps retrouve dans son titre habituel\\
de \textit{Seigneur de la Sagesse} le type du\\
λόγος θεῖος\footnote{« Verbe divin. »} ; avec d’un autre côté dans\\
Osiris\gls{CoEg_pers_00000151}-Ounnofré\gls{CoEg_pers_00000225} le dieu bon par\\
essence, c’est à la vérité une\\
ressemblance qui rapproche la\\
théologie égyptienne des idées philoso-\\
-phiques qui avaient cours parmi\\
les Grecs du temps de Sôter\gls{CoEg_pers_00000213}, et cette\\
ressemblance est assez remarquable\\
pour que, quarante ans à peine après\\
que la grande voix de Platon\gls{CoEg_pers_00000106} avait\\
cessé de se faire entendre, les Grecs\\
venus en Egypte\gls{CoEg_place_00000003} et pénétrant pour\\
la première fois dans les \sout{mystères des}\\
sanctuaires égyptiens en aient été\\
frappés. A la rigueur le Platonisme\\
dans tout son éclat servirait donc\\
à nous faire trouver le motif de\\
l’empressement des Alexandrins,\\
rencontrant, à leur premier pas sur
\end{paracol}

{\footnotesize\begin{center} {[27\textsuperscript{e} page]}\end{center}}

\begin{paracol}{2}
\noindent \\
\\
\\
\\
\\
\\
\\
\\
\\
\\
\\
\\
\\
\\
\\
\\
\\
\\
\\
\\
\\
\\
\\
\\
\\
\\
\\
\\
\\
\\
\\
(1) quoi qu’on eût pu tout aussi bien\\
opposer à l’Adès\gls{CoEg_pers_00000154} des Grec l’Osiris\gls{CoEg_pers_00000151} égyptien.
\switchcolumn
\noindent la terre d’Egypte\gls{CoEg_place_00000003}, une divinité qui\\
devait à l’apparence ne leur être point\\
inconnue. – Je concevrais mieux cependant\\
que les récits de Tacite\gls{CoEg_pers_00000111}, de Plutarque\gls{CoEg_pers_00000109}, de\\
Clément\gls{CoEg_pers_00000112} d’Alexandrie\gls{CoEg_place_00000006} et de Macrobe\gls{CoEg_pers_00000202}\\
s’appliquassent, trois cents ans plus tard,\\
aux premiers temps de notre ère. Alors les\\
philosophes et les théologiens, à la\\
lueur du Néoplatonisme, pouvaient\\
discerner au loin le vrai dogme d’Apis\gls{CoEg_pers_00000011}\\
mort et discuter sur Osiris\gls{CoEg_pers_00000151} qui s’incarne\\
dans un vulgaire quadrupède, sur Phtah\gls{CoEg_pers_00000047}\\
qui féconde la mère du taureau, sur la\\
vâche-mère [\textit{sic}] que n’a point touché le\\
mâle, enfin sur Sérapis\gls{CoEg_pers_00000060}, forme sensible\\
du dieu descendu parmi les hommes et\\
mort au milieu d’eux. Ce qui ne\\
s’explique qu’avec une certaine difficulté\\
sous Ptolémée Sôter\gls{CoEg_pers_00000213} trouve donc mieux\\
sa raison d’être à une époque postérieure,\\
tout entière empreinte des idées mêmes\\
dont le fameux taureau de Memphis\gls{CoEg_place_00000005}\\
est le représentant. – Je croirais donc\\
en définitive que le culte de Sérapis\gls{CoEg_pers_00000060},\\
établi peut-être \sout{sans pompe et} sans\\
éclat au milieu de la nouvelle ville que\\
venait de fonder Alexandre\gls{CoEg_pers_00000145}, ne prit\\
son essor qu’à l’époque des grandes luttes\\
philosophiques dont Alexandrie\gls{CoEg_place_00000006} fut\\
un brillant théâtre. Sans Sôter\gls{CoEg_pers_00000213} on s’en\\
tient à de vagues points de contact entre\\
Pluton\gls{CoEg_pers_00000215} et Osorapis\gls{CoEg_pers_00000223} (1), et les contem-\\
porains de ce prince, satisfaits de\\
rencontrer dans l’antique théologie\\
égyptienne des dogmes philosophiques\\
de loin en loin semblables à ceux qu’ils\\
apportaient eux-mêmes des écoles\\
d’Athènes\gls{CoEg_place_00000066}, imaginèrent le culte du\\
dieu mixte, amalgamé de grec et\\
d’égyptien, qu’ils appelèrent Sérapis\gls{CoEg_pers_00000060}.\\
Plus tard, les disputes du Néoplatonisme,\\
l’éclat du Christianisme naissant,\\
donnèrent au dogme d’Apis\gls{CoEg_pers_00000011} mort
\end{paracol}

{\footnotesize\begin{center} {[28\textsuperscript{e} page]}\end{center}}

\begin{paracol}{2}
\noindent \\
\\
\\
\\
\\
\\
\\
\\
\\
\\
\\
\\
\\
\\
\\
(1) \textit{Le dieu Sérapis\gls{CoEg_pers_00000060} et son origine, ses rapports,\\
ses attributs et son histoire}, dissertation\\
jointe aux notes du tome V des œuvres\\
complètes\gls{CoEg_bibl_00000041} de Tacite\gls{CoEg_pers_00000111}, par J. L. Burnouf\gls{CoEg_pers_00000227},\\
Paris\gls{CoEg_place_00000002}, 1828.\\
\\
\\
\\
\\
\\
\\
\\
\\
\\
\\
\\
\\
(2) Macrobe\gls{CoEg_pers_00000202}, \textit{Saturn.}\gls{CoEg_bibl_00000050} I, 20.
\switchcolumn
\noindent un à-propos qui servit à la renommée\\
de Sérapis\gls{CoEg_pers_00000060}. Le dieu de Sôter\gls{CoEg_pers_00000213}, humble\\
symbole de la fusion des deux religions\\
grecque et égyptienne, s’envole alors\\
des rivages d’Alexandrie\gls{CoEg_place_00000006}, s’arrête à\\
Athènes\gls{CoEg_place_00000066}, à Rome\gls{CoEg_place_00000065}, dans toutes les\\
frontières du monde connu, et ne\\
succombe après trois siècles de [latin ?] que\\
sous les coups du christianisme\\
triomphant. Tel fut Sérapis\gls{CoEg_pers_00000060}.\\
\noindent Je n’entrerai pas dans plus de détails\\
sur l’histoire de ce dieu. Je suis dispensé\\
de cette tâche, au profit même de la\\
science, par un \textit{excursus} sur la matière\\
que nous devons à un savant illustre,\\
\gls{CoEg_abbr_00000001} Guigniaut\gls{CoEg_pers_00000226} (1). D’ailleurs, quand les\\
inscriptions commencent à ne plus\\
{[\sout{nommer} ?]} \textsuperscript{adresser à} Sérapis\gls{CoEg_pers_00000060} que l’invocation\\
Διῒ Ἡλίῳ μεγάλῳ Σαράπιδι\footnote{« À Zeus Hélios le grand Sérapis ».} ; quand\\
les monuments nous montrent ce dieu\\
sous la forme d’un homme aux yeux sévères,\\
à la barbe épaisse, à la tête surmontée\\
du \textit{\gls{CoEg_entry_00000026}}, \sout{qu’ils nous apprennent à\\
nommer Jupiter\gls{CoEg_pers_00000153}-Sérapis\gls{CoEg_pers_00000060}} ; quand nous\\
entendons un oracle, interrogé par\\
Nicocréon\gls{CoEg_pers_00000228}, roi de Cypre\gls{CoEg_place_00000067}, décrire ainsi\\
Sérapis\gls{CoEg_pers_00000060} : « je vais te faire connaître\\
« la nature de ma divinité : le cercle\\
« élevé des cieux couronne ma tête ; mes\\
« oreilles sont dans l’air ; le bassin des mers\\
« est mon ventre ; la terre forme mes\\
« pieds ; mes yeux sont dans le disque\\
« brillant du soleil (2) », on croit\\
que le taureau auquel Céchoüs\gls{CoEg_pers_00000155} rendit\\
le premier ses hommages avait, trois\\
ou quatre mille ans plus tard, tellement\\
dévié de sa route qu’il n’est plus le\\
dieu qui nous appartient et auquel\\
nous consacrons en ce moment notre\\
attention. Je m’arrêterai donc là,\\
et en terminant ces courtes remarques,\\
je résumerai en quelques lignes les notions
\end{paracol}

{\footnotesize\begin{center} {[29\textsuperscript{e} page]}\end{center}}

\begin{paracol}{2}
\noindent \\
\switchcolumn
\noindent que nous possédons maintenant sur Apis\gls{CoEg_pers_00000011}\\
mort, ou Sérapis\gls{CoEg_pers_00000060}, comme j’ai résumé plus\\
haut celles que la critiques des textes et\\
des monuments nous avait mises entre\\
les mains sur Apis\gls{CoEg_pers_00000011} vivant :\\
\indent \gls{CoEg_abbr_00000005} Apis\gls{CoEg_pers_00000011}, incarnation d’Osiris\gls{CoEg_pers_00000151}, retourne\\
à sa mort dans le sein du dieu qu’il\\
avait représenté sur la terre ; il\\
devient Osiris\gls{CoEg_pers_00000151}-Apis\gls{CoEg_pers_00000011}, Osorapis\gls{CoEg_pers_00000223} ou\\
Sérapis\gls{CoEg_pers_00000060} ; aux yeux des Egyptiens, le\\
taureau dans sa tombe est la forme\\
sensible du dieu qui est venu vivre et\\
mourir au milieu des hommes ; c’est là\\
le véritable Sérapis\gls{CoEg_pers_00000060} des traditions égyp-\\
-tiennes ;\\
\indent \gls{CoEg_abbr_00000006} ce dogme doit être aussi ancien\\
qu’Apis\gls{CoEg_pers_00000011} lui-même, c’est-à-dire remonter\\
à la II\textsuperscript{\underline{e}} dynastie ; il persiste jusques\\
sous les Ptolémées qui, même en présence\\
du Sérapis\gls{CoEg_pers_00000060} d’Alexandrie\gls{CoEg_place_00000006}, tinrent à\\
garder pur de tout mélange le Sérapis\gls{CoEg_pers_00000060}\\
national de l’Egypte\gls{CoEg_place_00000003} ; sous les rois des\\
dynasties pharaoniques comme sous les\\
rois successeurs d’Alexandre\gls{CoEg_pers_00000145}, le Sérapis\gls{CoEg_pers_00000060}\\
de Memphis\gls{CoEg_place_00000005} fut donc toujours le\\
dieu fait chair ;\\
\indent \gls{CoEg_abbr_00000007} l’histoire nous apprend qu’un autre\\
Sérapis\gls{CoEg_pers_00000060} existe à Alexandrie\gls{CoEg_place_00000006} ; si, comme\\
le prétendant Tacite\gls{CoEg_pers_00000111} et quelques autres\\
écrivains, ce dieu fut amené de Sinope\gls{CoEg_place_00000053}\\
sous Ptolémée Sôter\gls{CoEg_pers_00000213}, ce qui est douteux\\
et pourrait être l’objet de discussions\\
plus approfondies que celles auxquelles\\
nous pouvons nous livrer en ce moment,\\
il ne fut pas tout-à-fait Apis\gls{CoEg_pers_00000011} mort\\
et ne dut son élévation qu’à certains\\
points de ressemblance que les Grecs crurent\\
remarquer entre Osiris\gls{CoEg_pers_00000151} rapproché d’Adès\gls{CoEg_pers_00000154},\\
Osorapis\gls{CoEg_pers_00000223} rapproché de Dionysos\gls{CoEg_pers_00000224}, et les\\
idées philosophiques que Platon\gls{CoEg_pers_00000106} venait\\
alors d’émettre ; quant à l’éclat dont\\
Sérapis\gls{CoEg_pers_00000060} brille, on ne doit le voir
\end{paracol}

{\footnotesize\begin{center} {[30\textsuperscript{e} page]}\end{center}}

\begin{paracol}{2}
\noindent \\
\switchcolumn
\noindent commencer qu’après l’ère chrétienne ;\\
au milieu des docteurs de l’école\\
d’Alexandrie\gls{CoEg_place_00000006} ; Sérapis\gls{CoEg_pers_00000060} était alors plus\\
véritablement Apis\gls{CoEg_pers_00000011} mort, tandis que\\
sous Sôter\gls{CoEg_pers_00000213} il n’a dû être qu’un dieu\\
amalgamé d’Osiris\gls{CoEg_pers_00000151} et d’Apis\gls{CoEg_pers_00000011}, de\\
Pluton\gls{CoEg_pers_00000215} \& de Bacchus\gls{CoEg_pers_00000152} ; ce dieu\\
cosmopolite eut des autels jusqu’à\\
Memphis\gls{CoEg_place_00000005}, mais l’entrée du \Gls{CoEg_entry_00000028}\gls{CoEg_place_00000004}\\
de cette ville lui fut toujours défendue ;\\
\indent \gls{CoEg_abbr_00000049} Le Jupiter\gls{CoEg_pers_00000153}-Sérapis\gls{CoEg_pers_00000060} que l’on\\
rencontre après Adrien\gls{CoEg_pers_00000171} n’a presque rien\\
conservé de Sérapis\gls{CoEg_pers_00000060} ; le culte se\\
maintient pourtant à Alexandrie\gls{CoEg_place_00000006}\\
jusqu’à l’édit de Théodose\gls{CoEg_pers_00000173} qui\\
étouffa, sur le lieu même de sa\\
naissance, le dieu dégénéré.\\
\indent Tels sont dans leur ensemble les traits\\
généraux qui caractérisent Sérapis\gls{CoEg_pers_00000060}.\\
Les présenter sous une forme moins\\
confuse était difficile sans faire un\\
livre tout entier ; les réunir dans un\\
aperçu et quelques pages sans laisser\\
échapper de regrettables \gls{CoEg_entry_00000031}\\
était également impossible. C’est\\
dire que ce résumé est loin d’être\\
définitif, et que je regarde comme\\
plus importants que les résultats\\
acquis les résultats qui [rature] restent\\
à acquérir. On pardonnera donc,\\
et les fautes inséparables de tout\\
travail plus large que le cadre dans\\
lequel on est obligé de le faire entrer,\\
et le manque de preuves dont quelques-\\
-unes des propositions les plus [essentielles ?]\\
auraient besoin d’être appuyées. Néan-\\
-moins j’espère que les brèves explications\\
dans lesquelles je suis entré auront
\end{paracol}

{\footnotesize\begin{center} {[31\textsuperscript{e} page]}\end{center}}

\begin{paracol}{2}
\noindent \\
\\
\\
\\
\\
\\
\\
\\
\\
\\
\\
\\
\\
(1) François Lenormant\gls{CoEg_pers_00000200}, \textit{Les livres Egyptiens}\gls{CoEg_bibl_00000047},\\
\gls{CoEg_abbr_00000032} 17.\\
\\
\\
\\
\\
\\
\\
\\
\\
\begin{flushright}chose remarquable,\end{flushright}
\noindent \\
\\
\noindent (2) Jamblique\gls{CoEg_pers_00000229}, \textit{de Mysteriis}\gls{CoEg_bibl_00000060}, Sect. VIII,\\
\gls{CoEg_abbr_00000037} 2.\\
\\
\sout{(3) Comme Ammon\gls{CoEg_pers_00000190} \textit{le mari de sa mère},\\
c’est-à-dire le dieu qui se donne\\
la naissance à lui-même. \gls{CoEg_abbr_00000001}\\
François Lenormant\gls{CoEg_pers_00000200} (\gls{CoEg_abbr_00000032} 19) s’est\\
mépris sur cette appellation toute\\
symbolique, qui n’a conséquemment\\
rien d’obscène.}\\
(\sout{4}3) Mémoire, encore inédit, lu à\\
l’Académie des inscriptions et belles-lettres\gls{CoEg_org_00000036}.
\switchcolumn
\noindent laissé une impression générale assez\\
claire sur Apis\gls{CoEg_pers_00000011} vivant et sur Apis\gls{CoEg_pers_00000011}\\
mort. J’espère surtout qu’on n’aura\\
pas vu sans satisfaction la descriptions\\
entraîner vers les régions pures de\\
la métaphysiques cette religion\\
égyptienne que jusqu’ici l’on a\\
presque toujours considérée comme un\\
grossier tissu de fables ridicules. La\\
religion égyptienne (j’en demande\\
pardon au jeune savant dont j’ai eu\\
le plaisir de citer tout-à-l’heure le\\
nom) ne fut pas en effet aussi\\
\textit{impure} et aussi \textit{dégradante} (1) qu’on\\
le dit. Si, à l’exemple de toutes les\\
autres formes de paganisme, elle\\
ne sut pas ou ne voulut pas maintenir\\
son culte à la hauteur de dogme,\\
elle eut du moins, presque autant\\
que le Mosaïsme, la perception\\
nette, lumineuse, infaillible de la\\
divinité. Au delà se ses symboles si\\
capricieusement choisis, au-delà\\
du dieu « qui se vautre sur un tapis\\
« de pourpre », elle vit et adora,\\
un Dieu unique (θεός εἷς), antérieur\\
au premier Dieu (πρῶτος καὶ τοῦ\\
πρώτου θεοῦ), immortel, incréé,\\
invisible et caché dans les profondeurs\\
inaccessibles de son essence (2). Le\\
Dieu \textit{un} (\includegraphics[height=6pt]{CoEg_Mariette_hiero_1857-04-01_31_1.png}\footnote{\foreignlanguage{translit}{\Gls{CoEg_aeg_00000027}} « unique ».}), le Dieu \textit{seul} (\includegraphics[height=6pt]{CoEg_Mariette_hiero_1857-04-01_31_2.png}\footnote{\foreignlanguage{translit}{\Gls{CoEg_aeg_00000055}} « seul ».}),\\
le dieu \textit{incréé} (\includegraphics[height=6pt]{CoEg_Mariette_hiero_1857-04-01_31_3.png}\hypertarget{CoEg_Mariette_1857-04-01_31}{}\footnote{Cette citation réunit la particule de négation et le verbe \foreignlanguage{translit}{\gls{CoEg_aeg_00000028}} «~advenir »~; il ne s'agit cependant pas d'un participe (on attendrait l'auxiliaire négatif \foreignlanguage{translit}{tm}), et il manque un sujet au verbe, mais le contexte manque pour pouvoir analyser cet extrait.}) et \textit{inengendré}\\
(\includegraphics[height=6pt]{CoEg_Mariette_hiero_1857-04-01_31_4.png}\footnote{\foreignlanguage{translit}{\Gls{CoEg_aeg_00000028} \gls{CoEg_aeg_00000057}\gls{CoEg_aeg_00000058}} « apparu de lui-même ».}) \sout{(3)}, le Dieu éternel (\includegraphics[height=6pt]{CoEg_Mariette_hiero_1857-04-01_31_5.png}\footnote{\foreignlanguage{translit}{\Gls{CoEg_aeg_00000028} \gls{CoEg_aeg_00000006} \gls{CoEg_aeg_00000029}} « apparu en tête ».})\\
n’apparaît pas seulement dans un\\
chapitre célèbre de Jamblique\gls{CoEg_pers_00000229}. \gls{CoEg_abbr_00000001}\\
de Rougé\gls{CoEg_pers_00000032} l’a retrouvé dans les textes\\
hiéroglyphique (\sout{4}3), et j’ai cru moi-même\\
l’apercevoir dans l’expression par\\
laquelle, à la manière du Jéhovah-
\end{paracol}

{\footnotesize\begin{center} {[32\textsuperscript{e} page]}\end{center}}

\begin{paracol}{2}
\noindent \\
\\
(1) \includegraphics[height=6pt]{CoEg_Mariette_hiero_1857-04-01_32.png}\footnote{\foreignlanguage{translit}{\Gls{CoEg_aeg_00000030} \gls{CoEg_aeg_00000014}w} « Ennéade des dieux », plutôt que \foreignlanguage{translit}{\gls{CoEg_aeg_00000056} \gls{CoEg_aeg_00000014}w} « origine des dieux » ?} : \textit{Paouat Neterou}, le\\
Seigneur des Dieux.\\
\\
\\
\\
\\
\\
\\
\\
(1) \Gls{CoEg_abbr_00000054}\gls{CoEg_bibl_00000060} \gls{CoEg_abbr_00000037} 3.
\switchcolumn
\noindent Elohim\gls{CoEg_pers_00000230} de la \sout{Bible\gls{CoEg_bibl_00000019}} Genèse\gls{CoEg_bibl_00000061}, les Egyptiens\\
ont le plus communément désigné la\\
divinité (1). Ainsi au sommet du\\
panthéon égyptien plane un Dieu\\
digne de l’être, et c’est au dessous de\\
lui seulement qu’apparaissent ces\\
divinités inférieures qu’on trouve à\\
l’état latent dans quelques livres\\
de la Bible\gls{CoEg_bibl_00000019} et que Plotin\gls{CoEg_pers_00000231} devait\\
appeler plus tard les \textit{puissances} de\\
Dieu, δυνάμεις. « Le Dieu égyptien,\\
« dit Jamblique\gls{CoEg_pers_00000229} (1), quand il est considéré\\
« comme cette force active qui amène\\
« les choses à la lumière s’appelle Ammon\gls{CoEg_pers_00000190},\\
« quand il est l’esprit intelligent qui\\
« résume toutes les intelligences, il est\\
« Esneph (Chneph, Chnouphis)\gls{CoEg_pers_00000232}, quand\\
« il est celui qui accomplit toutes choses\\
« avec art et vérité, il s’appelle Phtah\gls{CoEg_pers_00000047},\\
« et enfin quand il est le dieu bon\\
« et bienfaisant, on le nomme Osiris\gls{CoEg_pers_00000151} ».\\
Osiris\gls{CoEg_pers_00000151}, Phtah\gls{CoEg_pers_00000047}, Ammon\gls{CoEg_pers_00000190}, Sébek\gls{CoEg_pers_00000191},\\
Phré\gls{CoEg_pers_00000192} et tous les dieux qui peuplent\\
le ciel égyptien ne sont donc que\\
des divinités partielles, représentant\\
le Dieu ineffable et incompréhensible :\\
ils sont les puissances du Dieu rendues\\
visibles. La notion judicieuse, raisonnée,\\
philosophique de la divinité n’a donc\\
point manqué à l’Egypte\gls{CoEg_place_00000003}, et\\
si l’Egypte\gls{CoEg_place_00000003} s’en était tenue là, elle\\
eût presque égalé le Mosaïsme\\
dans la connaissance de Dieu\gls{CoEg_pers_00000036}. En\\
tous cas le polythéisme grec, avec\\
ses dogmes mal définis, son culte à\\
la merci de chacun, n’est pas à\\
comparer pour la grandeur et\\
l’immutabilité de principes avec\\
cette religion égyptienne qui peut tout\\
au moins invoquer, à l’honneur de sa
\end{paracol}

{\footnotesize\begin{center} {[33\textsuperscript{e} page]}\end{center}}

\begin{paracol}{2}
\noindent \\
\switchcolumn
\noindent bonne constitution intérieure, une\\
durée de quatre mille ans. Quand on\\
prend la religion égyptienne à son\\
origine et qu’on voit dans quel sol\\
généreux elle plonge ses racines, il\\
est donc sage de mesurer ses accusations.\\
– D’ailleurs je puiserais au besoin dans\\
le travail\gls{CoEg_bibl_00000047} que \gls{CoEg_abbr_00000001} François Lenormant\gls{CoEg_pers_00000200}\\
a consacré au \textit{Rituel} la réfutation\\
de l’opinion \textsuperscript{elle-même} que le jeune écrivain soutient.\\
Que le \textit{Rituel} existât déjà au XVI\textsuperscript{\underline{e}}\\
siècle avant notre ère, c’est ce qui ne\\
fait pas de doute, et il est probable\\
que des générations bien antérieurs\\
l’ont possédé. D’un bout de l’Egypte\gls{CoEg_place_00000003}\\
à l’autre, le \textit{Rituel} était dès cette\\
époque le livre de tout le monde. Le\\
pauvre et le riche tenaient à en voir\\
une copie plus ou moins complète avec\\
leur tombeau. Nul écrit sur les matières\\
religieuses n’était plus populaire. Evi-\\
-demment, si un livre de ce genre peut\\
saisir et conserver l’empreinte du peuple\\
pour lequel il a été écrit, nous devons\\
trouver dans le \textit{Rituel} le reflet de\\
l’Egypte\gls{CoEg_place_00000003} et de ses croyances, bonnes ou\\
mauvaises ; l’impureté et la dégradation\\
y seront, ou elles ne seront nulle part.\\
Or que lisons-nous dans le \textit{Rituel} ? Je\\
ne veux pas prolonger ce débat outre\\
mesure ; mais je ne puis m’empêcher\\
de faire remarquer que la pensée\\
dominante du \textit{Rituel}, celle qui\\
plane sur tout le livre et lui donne\\
le souffle et la vie, est précisément\\
la croyance la plus élevée, la plus\\
morale, la plus divine qui ait\\
jamais été révélée à la conscience de\\
l’homme : celle de l’immortalité de\\
l’âme. Les peuples qui, dès le temps\\
d’Abraham\gls{CoEg_pers_00000100}, faisaient de cette
\end{paracol}

{\footnotesize\begin{center} {[34\textsuperscript{e} page]}\end{center}}

\begin{paracol}{2}
\noindent \\
\\
\\
\\
\\
\\
\\
\\
\\
\\
\\
\\
\\
\\
\\
(1) Voyez l’édition du \textit{Rituel} publiée par\\
\gls{CoEg_abbr_00000001} Lepsius\gls{CoEg_pers_00000061} sous le titre de \textit{Todtenbuch\\
des Aegypter}\gls{CoEg_bibl_00000035}, \Gls{CoEg_abbr_00000037} 125, lignes 1, 2, 3, 4,\\
et suivantes.\\
(2) Cette qui frappe et celle qui récompense.
\switchcolumn
\noindent croyance un dogme national sont-\\
-ils nombreux ? D’un autre côté abordons\\
sans plus de détours un chapitre\\
fameux ; celui où l’âme du mort,\\
présente devant le \sout{D}dieu qui va\\
le juger, rend en quelques sorte\\
le compte moral de ses actions sur\\
la terre. Là se développe l’esprit\\
lui-même qui préside à la vie\\
de l’ancienne société égyptienne ;\\
là se rencontrent les vertus exaltées\\
et les vices flétris. Que va nous dire\\
le \textit{Rituel} ? L’âme pénètre dans la\\
grande salle de jugement ; elle aperçoit\\
son juge et les quarante-deux assesseurs\\
auxquels elle tient ce langage que je\\
traduis directement de l’original (1) : « O\\
« Dieux Seigneurs de la double Justice (2), soyez\\
« (moi) favorables ; sois (moi) favorable, ô\\
« là, grand Dieu, Seigneur de la double\\
« Justice ! Je suis venu vers toi, et c’est toi\\
« qui m’as conduit pour que (je puisse)\\
« contempler tes beautés ! Connaissant ton\\
« nom, je le prononcerai moi-même, et je\\
« prononcerai moi-même le nom de tes qua-\\
« rante-deux dieux qui sont avec toi dans\\
« la salle de la double Justice . . . . . Je vous\\
« connais aussi, ô Dieux Seigneurs de la\\
« double Justice ! Je vous ai apporté la\\
« vérité, et j’ai éloigné de vous les mensonges !\\
« Je n’ai pas commis de fraudes envers mon\\
« prochain ! . . . . Je n’ai pas été hypocrite\\
« devant un tribunal ! Je n’ai pas proféré\\
« de mensonges ! Je n’ai pas fait de mal !\\
« Je ne me suis pas fait le chef de tous les\\
« hommes pour les forcer à travailler toute\\
« la journée ! . . . . . Je n’ai pas fait avoir\\
« faim ! Je n’ai pas fait avoir soif ! Je\\
« n’ai pas fait pleurer ! Je n’ai pas assassiné !\\
« Je n’ai pas donné l’ordre de tuer furtivement !\\
« . . . . . Je n’ai pas augmenté le poids du\\
\end{paracol}

{\footnotesize\begin{center} {[35\textsuperscript{e} page]}\end{center}}

\begin{paracol}{2}
\noindent \\
\\
(1) \textit{Todtenbuch}\gls{CoEg_bibl_00000035}, \gls{CoEg_abbr_00000037} 125, lig. 38.\\
\\
\\
\\
\\
\\
\\
\\
\\
\\
(2) \gls{CoEg_abbr_00000032} 18.\\
\\
\\
\\
\\
\\
\\
\\
\\
\\
\\
\\
\\
\\
\\
\\
\\
\\
(3) \gls{CoEg_abbr_00000032} 19.
\switchcolumn
\noindent « plateau (de la balance) ! . . . . \textit{Je n’ai pas}\\
« \textit{ôté le lait de la bouche des petits enfants !}\\
Et plus loin (1) le mort ajoute ces phrases\\
empreintes d’une charité si naïve :\\
« J’ai donné à manger à celui qui avait\\
« faim ! J’ai donné à boire à celui qui\\
« avait soif ! J’ai fourni des vêtements\\
« à celui qui était nu ! » . . . « Aucun\\
« orphelin n’a été maltraité par moi »,\\
dit à Béni-Hassann\gls{CoEg_place_00000040} une légende\\
dont j’emprunte la traduction au\\
travail\gls{CoEg_bibl_00000047} de \gls{CoEg_abbr_00000001} François Lenormant\gls{CoEg_pers_00000200} (2),\\
« aucune veuve n’a été violentée par\\
« moi ; aucun mendiant n’a été bâtonné\\
« par mes ordres ; aucun pâtre n’a été\\
« frappé par moi ; aucun chef de\\
« famille n’a été opprimé par moi ».\\
Rien n’est plus clair que ce beau\\
langage. Consultez tout le \textit{Rituel}\\
et les milliers d’inscriptions qui\\
couvrent l’Egypte\gls{CoEg_place_00000003}, et vous n’y trouve-\\
-rez pas un mot qui dégrade la\\
conscience en l’avilissant. Au\\
contraire les hommages rendus à la\\
plus saine morale se rencontrent\\
à chaque pas. En vain \gls{CoEg_abbr_00000001} François\\
Lenormant\gls{CoEg_pers_00000200} invoque-t-il la fameuse\\
légende d’Ammon\gls{CoEg_pers_00000190} qui se dit \textit{le\\
mari de sa mère}, équivalent chaste,\\
dit le jeune savant (3), qui voile\\
la brutalité de l’expression égyptienne.\\
Il n’y a ici ni chasteté, ni\\
brutalité. Ammon\gls{CoEg_pers_00000190}, \textit{le mari de sa\\
mère}, est le dieu qui s’engendre lui-\\
même, qui se donne la naissance\\
à lui-même ; c’est le dieu incréé\\
et rien de plus. Je répète donc que\\
la civilisation égyptienne a laissé\\
dans les nombreux vestiges que le
\end{paracol}

{\footnotesize\begin{center} {[36\textsuperscript{e} page]}\end{center}}

\begin{paracol}{2}
\noindent \\
\\
\\
(1) \gls{CoEg_abbr_00000032} 18.\\
\\
\\
\\
\\
\\
\\
\\
\\
\\
\\
\\
\\
\begin{flushright} de ses temples, figures \end{flushright} 
\noindent (1) Les représentations d’Ammon\gls{CoEg_pers_00000190} ithyphallique\\
sont toutes symboliques et n’ont\\
absolument rien d’obscène. La différence\\
des civilisations nous les fait seule\\
trouver telles. Des figures réellement\\
obscènes par l’intention ne se rencontrent\\
que sur un papyrus\gls{CoEg_obj_00000027} du Musée\gls{CoEg_org_00000021} de\\
Turin\gls{CoEg_place_00000034}, [\sout{On n’en trouve} ?] et quelques figurines\\
de nos collections, d’époque grecque.
\switchcolumn
\noindent temps a respectés des reflets qui\\
n’accusent pas du tout une religions\\
dégradante. Je n’oserais pas dire,\\
comme \gls{CoEg_abbr_00000001} François Lenormant\gls{CoEg_pers_00000200} (1),\\
qu’on y rencontre « des aspirations\\
« qui s’élèvent presque à la hauteur\\
« de l’Evangile\gls{CoEg_bibl_00000062} » ; mais je pense que\\
la société égyptienne, dans sa raideur\\
\sout{si} peu sympathique aux étrangers,\\
laisse loin derrière elle, sous le\\
rapport des idées morales et religieuses,\\
la société fleurie des Grecs ; je pense\\
que jamais, par exemple, les amours\\
des dieux et des déesses, si fréquemment\\
représentés sur les édifices publics et\\
privés de la Grèce\gls{CoEg_place_00000057}, n’ont blessé les yeux\\
d’un \sout{ho} égyptien habitué aux figures\\
froides, mais toujours chastes \textsuperscript{d’intention (1)}, [\sout{de ses} ?]\\
\sout{temples} ; je maintiens surtout (et\\
c’est là ce que je voulais prouver)\\
que la religion égyptienne, par\\
l’élévation de ses principes, par la\\
pensée fermement conçue qui présida\\
à son organisation, par la fixité\\
de ses dogmes et la pureté de sa\\
morale, n’est pas une religion\\
indigne de ce nom. – Maintenant\\
que cette religion ait dévié de la\\
route dans laquelle nous la voyons\\
s’engager à son point de départ,\\
je ne le nie pas. Son malheur est\\
d’avoir, comme toutes les religions\\
dont le culte est compliqué, enfanté\\
bien des superstitions qui, empiétant\\
sur le dogme, durent souvent le faire\\
oublier. Le vulgaire, mis en présence\\
d’un Dieu qu’il n’apercevait qu’à\\
travers les abstractions derrières\\
lesquelles on le cachait, ne demandait\\
point aux parties retirées du temple\\
l’explication des mystères qui y
\end{paracol}

{\footnotesize\begin{center} {[37\textsuperscript{e} page]}\end{center}}

\begin{paracol}{2}
\noindent \\
\\
\\
\\
\\
\\
\\
\\
\\
\\
\begin{flushright}\textsuperscript{à première vue}\end{flushright}
\switchcolumn
\noindent étaient enseignés ; il lui était plus\\
commode de sacrifier aux symboles de\\
la divinité toujours présente à ses yeux,\\
et c’est en songeant moins au créateur\\
qu’aux pratiques propres à l’honorer\\
qu’il satisfaisait à ce besoin consolant\\
d’aimer et d’adorer Dieu\gls{CoEg_pers_00000036} qui est\\
dans le cœur de tous les hommes. De\\
là ces apparences singulières qui, de\\
tout temps, ont flotté à la surface\\
de la religion égyptienne, et qu’on est\\
tenté de prendre pour le fond même\\
du dogme. Plus que toute autre parce\\
que l’unité et la simplicité étaient\\
chez elle moins rigoureuses, la religion\\
égyptienne, vue de loin au milieu de\\
ceux qui la cultivaient, peut donc\\
passer pour une religion sans solidité\\
et sans profondeur. Mais les prêtres\\
et les esprits éclairés qui ne manquèrent\\
point au pays où Moïse\gls{CoEg_pers_00000099} trouva \sout{son}\\
ses instituteurs, ne se sont pas conten-\\
-tés de cette nourriture grossière : ils\\
n’ont pas confondu, comme on le [rature] fait\\
si souvent, les pratiques de la piété\\
avec la piété elle-même. C’est pour\\
eux que le Dieu unique, le Dieu\\
sans commencement ni fin, le Dieu\\
créateur de toutes choses, planant dans\\
la partie invisible du sanctuaire ;\\
c’est pour le vulgaire que de Dieu\gls{CoEg_pers_00000036} et\\
de sa puissance l’Egypte\gls{CoEg_place_00000003} descendit\\
aux symboles qui à leur tour per-\\
-sonnifient les émanations divines,\\
que Thoth\gls{CoEg_pers_00000233} fut retrouvé dans l’ibis,\\
Horus\gls{CoEg_pers_00000234} dans l’épervier, et que le\\
bélier passe pour représenter Chnouphis\gls{CoEg_pers_00000232}.\\
Ainsi s’expliquent les [... ?] superstitions\\
et ces pratiques étranges dont je parlais
\end{paracol}

{\footnotesize\begin{center} {[38\textsuperscript{e} page]}\end{center}}

\begin{paracol}{2}
\noindent \\
\\
\\
\\
\\
\\
\\
\begin{flushright}\textsuperscript{et rend moins coupable jusqu’}\end{flushright}
\switchcolumn
\noindent tout-à-l’heure. – En résumé, de quelque\\
point de vue qu’on la considère, la\\
religion égyptienne [\sout{est si grande} ?] \textsuperscript{mérite} notre\\
attention, parce qu’au plus haut\\
sommet où elle repose, on rencontre\\
à côté d’elle une preuve vivante\\
de respect que nous lui devons, c’est-\\
-à-dire un Dieu \sout{digne d’être honoré} \textsuperscript{vraiment [divin ?]},\\
un Dieu dont la seule présence [... ?] \textsuperscript{épure}\\
aux plus lointaines erreurs dans lesquels\\
ses adorateurs se sont plongés. Si une\\
étude mieux réglée [\sout{nous rend plus} ?] \textsuperscript{montre ces assertions sous un}\\
\textsuperscript{jour plus certain ; si elle donne plus d’évidence}\\
\sout{évidentes} à ces erreurs, qui d’ailleurs\\
{[rature]} \textsuperscript{ne dénoncent} pas plus la dégradation\\
de la religion originelle que le limon\\
apporté à l’embouchure du fleuve\\
par ses affluents \sout{ne} [rature] \textsuperscript{n’accuse} la\\
pureté de sa source, l’Egypte\gls{CoEg_place_00000003} avec\\
ses hautes aspirations vers la vérité\\
religieuse, avec son culte public\\
réglé sur les besoins d’un peuple\\
ignorant, me paraîtrait ainsi\\
semblable au colosse du songe de\\
Nabuchodonosor\gls{CoEg_pers_00000236} : tête d’or et pieds\\
d’argile. Mais on voit que ce\\
n’est pas en vain que la Bible\gls{CoEg_bibl_00000019}\\
elle-même aura vanté \textit{la Sagesse\\
des Egyptiens}.\\
\indent Pour en revenir une dernière fois à\\
Apis\gls{CoEg_pers_00000011}, objet principal de ce débat,\\
on remarquera qu’Apis\gls{CoEg_pers_00000011} occupe au\\
milieu des divinités qui peuplèrent\\
les bords du Nil\gls{CoEg_place_00000021} une place à part.\\
La théologie égyptienne est un\\
système que j’arrivais à diviser\\
en trois couches superposées à la\\
manière de terrains géologiques : dans\\
la couche la plus profonde, \textsuperscript{la plus lointaine,} celle\\
qui tient aux \sout{origines mêmes} \textsuperscript{âges primitifs}de\\
dogme, se dérobe aux regards humains
\end{paracol}

{\footnotesize\begin{center} {[39\textsuperscript{e} page]}\end{center}}

\begin{paracol}{2}
\noindent \\
\switchcolumn
\noindent le Dieu unique, universel et incréé,\\
le Dieu de la métaphysique ; au dessus\\
de lui et dans un contact immédiat\\
se rencontrent ses puissances divinisées,\\
conception déjà \sout{p} moins pure de\\
l’idée divine ; à la surface du\\
sol, apparaissent enfin aux yeux de\\
tous ces mêmes puissances dans les\\
symboles qu’on leur a si curieusement\\
choisis. Quel rang occupe Apis\gls{CoEg_pers_00000011} dans\\
cette hiérarchie ? Tous les béliers, de\\
quelque propriété qu’il fussent doués,\\
en quelque partie de l’Egypte\gls{CoEg_place_00000003} qu’ils\\
vinssent, étaient respectés comme les\\
symboles animés de Chnouphis\gls{CoEg_pers_00000232}, tous\\
les éperviers étaient également sacrés parce\\
qu’on les regardait comme des symboles\\
d’Horus\gls{CoEg_pers_00000234} ; mais je me hâte de rappeler\\
que tous les taureaux, sans distinction\\
de forme, de couleur, de lieux, n’étaient\\
pas des symboles d’Osiris\gls{CoEg_pers_00000151}. Apis\gls{CoEg_pers_00000011} était\\
un dieu lui-même, choisi parmi tous\\
les autres animaux de son espèce pour\\
ses qualités propres et individuelles ;\\
il était l’animal dans le\sout{quel} corps\\
duquel Osiris\gls{CoEg_pers_00000151} passait pour habiter ;\\
il était en un mot, non pas un\\
symbole, mais une incarnation d’Osiris\gls{CoEg_pers_00000151},\\
comme Mnévis\gls{CoEg_pers_00000156} était une incarnation\\
de Phré\gls{CoEg_pers_00000192}. Sans être une émanation\\
directe de la divinité et sans représenter,\\
comme Osiris\gls{CoEg_pers_00000151}, Phtah\gls{CoEg_pers_00000047}, Ammon\gls{CoEg_pers_00000190} et les\\
autres dieux, l’une des puissance de\\
l’Etre suprême, Apis\gls{CoEg_pers_00000011} était donc plus\\
qu’un animal sacré. Comme je\\
l’ai dit \& comme je le répète en\\
terminant ces trop longues digressions,\\
il était un animal divin, occupant\\
par une exception que le seul Mnévis\gls{CoEg_pers_00000156}\\
partage avec lui, une place intermédiaire\\
entre les dieux et leurs symboles. Tel\\
était Apis\gls{CoEg_pers_00000011}.
\end{paracol}

{\footnotesize\begin{center} {[40\textsuperscript{e} page]}\end{center}}

\begin{paracol}{2}
\noindent \\
\switchcolumn
J’espère, Monsieur le Ministre, que\\
Votre Excellence aura trouvé les\\
explications qui précèdent la preuve des\\
soins que j’ai eus à remplir la\\
mission qui m’a été confiée. J’espère\\
aussi qu’en présence des résultats importants\\
dont je viens d’exposer la substance, Votre\\
Excellence ne regrettera pas de m’avoir\\
fourni les moyens de compléter mes études\\
sur un sujet si digne de toute notre\\
attention. – J’ajouterai qu’à Berlin\gls{CoEg_place_00000051}, à\\
Londres\gls{CoEg_place_00000011} et à Turin\gls{CoEg_place_00000034}, j’ai trouvé dans les\\
honorables et savants conservateurs des\\
beaux établissements scientifiques que\\
possèdent ces villes, une complaisance et\\
un dévouement que je signale avec un\\
véritable plaisir à Votre Excellence.\\
\indent J’ai l’honneur d’être avec le plus\\
profond respect,
\begin{center}Monsieur le Ministre,\end{center}
\begin{center}\hspace{3cm}de Votre Excellence,\\
\hspace{3cm}le très-humble\\
\hspace{3cm}et très-obéissant serviteur\\
\hspace{3cm}\gls{CoEg_abbr_00000002} Mariette\end{center}
\end{paracol}

\hypertarget{CoEg_Mariette_1857-08-26}{}

\section*{Le 26 août 1857, de Paris, à Servaux, chef du bureau des travaux historiques}
\addcontentsline{toc}{section}{Le 26 août 1857, de Paris, à Servaux, chef du bureau des travaux historiques au ministère de l'Instruction publique} \label{labCoEg_Mariette_1857-08-26}
{\footnotesize
\noindent Institution et lieu de conservation~: Archives nationales, Pierrefitte-sur-Seine.\\
Cote : \hyperlink{CoEg_Mariette_ms_002}{F/17/2988/1, dossier « Mariette »} (n. p.).\\
Support : une feuille double de moyen format, à en-tête « Maison de l’empereur\gls{CoEg_org_00000010}. Direction générale des musées impériaux\gls{CoEg_org_00000001}. Cabinet ».\\
Note : La lettre porte en partie supérieure les trois annotations suivantes à l’encre : « Mariette », « 37 », « Le Ministère\gls{CoEg_org_00000041} disait faire les frais de publications du \gls{CoEg_entry_00000028}\gls{CoEg_place_00000004} ».\\
Thème~: \gls{CoEg_keyword_00000008}.}

{\footnotesize\begin{center} {[1\textsuperscript{re} page, r\textsuperscript{o}]}\end{center}}
\begin{flushright}Paris\gls{CoEg_place_00000002}, le 26 août 1857.\end{flushright}

\hspace{1cm} Mon cher Monsieur Servaux\gls{CoEg_pers_00000237},\\

\indent Vous apprendrez avec satisfaction que le Ministère d’Etat\gls{CoEg_org_00000041} vient enfin de prendre une\\
décision favorable au sujet de ma publication du \Gls{CoEg_entry_00000028}\gls{CoEg_place_00000004}. Voici à quelles\\
conditions : \gls{CoEg_abbr_00000001} Fould\gls{CoEg_pers_00000075} ne fait que strictement les frais de l’ouvrage, c’est-à-dire\\
qu’il accorde 54 mille francs. Il n’y a pas un centime pour moi dans cette somme.\\
Mais comme les 54 mille francs paient le tirage de 300 exemplaires, il ne\\
m’en demande pour sa part que 200, et m’abandonne les 100 à titre de\\
rémunération pour un droit d’auteur en même temps qu’à titre de récompense pour\\
ma découverte du \Gls{CoEg_entry_00000028}\gls{CoEg_place_00000004}. C’est à moi de placer ces 100 exemplaires comme je\\
l’entendrai, et vous voyez d’ici de quel secours vous [rature] allez m’être bientôt.\\
\indent En attendant je prépare activement les deux premières livraisons que je\\
désire avoir terminées pour l’époque à laquelle se réunit le Conseil Supérieur\\
de l’Instruction Publique\gls{CoEg_org_00000037}. C’est vous dire dans quels embarras je suis plongé. Les\\
dessinateurs, les graveurs, les photographes m’[entourent ?], et organiser une\\
grande affaire comme celle-là où tout est à \sout{faire} créer à nouveau n’est pas une\\
petite chose. Aussi mes journées ne suffisent-elles pas.\\
\indent Je n’ai cependant que de très-bonnes nouvelles à vous donner de mon rapport\\
auquel j’emploie toutes mes soirées. J’y travaille sans relâche, autant que\\
me le permet le surcroît exceptionnel de besogne qui vient de m’arriver,\\
et je fais tout ce que je puis pour l’avancer. J’espère que d’ici à huit ou dix\\
jours, je pourrai avoir le plaisir de vous l’aller porter moi-même.\footnote{Si ce rapport a bien été envoyé, il n'a pas été conservé parmi les dossiers de mission de Mariette\gls{CoEg_pers_00000001}.} Je\\
tenais à vous donner cette assurance pour que vous ne pensiez pas que,\\
cette fois-ci encore, j’élude la difficulté au lieu de la résoudre. En tous\\
cas, attendez-moi bientôt au Ministère\gls{CoEg_org_00000042}.
{\footnotesize\begin{center} {[1\textsuperscript{re} page, v\textsuperscript{o}]}\end{center}}
\indent Je profite de l’occasion, mon cher Monsieur Servaux\gls{CoEg_pers_00000237}, pour vous exprimer\\
tout le plaisir que j’éprouve chaque fois que je vous vois et qu’il m’est\\
permis de vous serrer bien affectueusement la main –
\begin{center}\hspace{5cm}Votre tout dévoué\\
\hspace{5cm}\gls{CoEg_abbr_00000002} Mariette\end{center}

\hypertarget{CoEg_Mariette_1857-10-04}{}

\section*{Le 4 octobre 1857, de Paris, à Rouland, ministre de l’Instruction publique}
\addcontentsline{toc}{section}{Le 4 octobre 1857, de Paris, à Rouland, ministre de l’Instruction publique} \label{labCoEg_Mariette_1857-10-04}
{\footnotesize
\noindent Institution et lieu de conservation~: Archives nationales, Pierrefitte-sur-Seine.\\
Cote : \hyperlink{CoEg_Mariette_ms_002}{F/17/2988/1, dossier « Mariette »} (n. p.).\\
Support : une feuille double de grand format, à en-tête « Maison de l’empereur\gls{CoEg_org_00000010}. Direction générale des musées impériaux\gls{CoEg_org_00000001} » datée du palais du Louvre.\\
Note : La lettre porte en partie supérieure l’annotation à l’encre : « Parvenu au \gls{CoEg_abbr_00000043} \gls{CoEg_abbr_00000044} le 9 Octobre. [signature] » ; plusieurs passages ont été largement soulignés par l’administration lors du traitement de cette lettre, et ce marquage n’a pas été repris.\\
Thème~: \gls{CoEg_keyword_00000004}.}

\begin{flushright}Palais du Louvre, le 4 octobre 1857.\end{flushright}
\indent A Son Excellence\\
\indent Monsieur le Ministre de l’Instruction Publique et des Cultes\\
\indent \hspace{5cm} à Paris\gls{CoEg_place_00000002}.\\

\hspace{1cm} Monsieur le Ministre\gls{CoEg_pers_00000065},\\

\indent Son Altesse le Vice-Roi\gls{CoEg_pers_00000080} d’Egypte\gls{CoEg_place_00000003}, ayant appris que je devais\\
accompagner le Prince Napoléon\gls{CoEg_pers_00000074} dans le voyage que Son Altesse Impériale\\
doit faire en Orient\gls{CoEg_place_00000074}, m’a prié de me mettre à sa disposition pendant le\\
temps qui précèderait [\textit{sic}] le départ du Prince\gls{CoEg_pers_00000074} ; Son Altesse le Vice-Roi\gls{CoEg_pers_00000080}\\
désirerait que je préparasse les monuments antiques de l’Egypte\gls{CoEg_place_00000003} à\\
recevoir l’auguste visiteur qui les attend. Je pars en conséquence\\
pour Alexandrie\gls{CoEg_place_00000006} le 10 de ce mois.\\
\indent Mais pendant le séjour que je vais faire sur les bords du Nil\gls{CoEg_place_00000021}, je\\
compte ne pas oublier les études archéologiques auxquelles je suis voué.\\
D’un autre côté, il me serait très agréable de mettre les résultats\\
que ces études pourront produire sous le patronage de Votre Excellence.\\
Je viens donc, Monsieur le Ministre, vous prier de m’accorder une\\
mission gratuite pour l'Egypte\gls{CoEg_place_00000003}, au retour de laquelle je promets de\\
vous adresser un rapport détaillé qui pourra être inséré, si vous le\\
jugez convenable, aux Archives des Missions scientifiques\gls{CoEg_bibl_00000036}.\footnote{La fin de ce paragraphe a été soulignée, avec l’annotation « Arrêté et lettre d’avis » en marge gauche. Le rapport en question ne semble pas avoir été écrit.}\\
\indent J’ose espérer, Monsieur le Ministre, que Votre Excellence daignera\\
se rendre au vœu que je prends la liberté de lui exprimer. En attendant,\\
je la prie d’accepter l’assurance du profond respect avec lequel\\
j’ai l’honneur d’être,
\begin{center}Monsieur le Ministre,\\
de Votre Excellence,\end{center}
\begin{center}\hspace{5cm}le très-humble\\
\hspace{5cm}et très-obéissant serviteur :\\
\hspace{5cm}\gls{CoEg_abbr_00000002} Mariette\\
\hspace{5cm}Conservateur-adjoint au Musée du Louvre\gls{CoEg_org_00000002}.\end{center}

\hypertarget{CoEg_Mariette_1857-10-05}{}

\section*{Le 5 octobre 1857, de Paris, à un destinataire non désigné, au ministère de l'Instruction publique}
\addcontentsline{toc}{section}{Le 5 octobre 1857, de Paris, à un destinataire non désigné, au ministère de l'Instruction publique} \label{labCoEg_Mariette_1857-10-05}
{\footnotesize
\noindent Institution et lieu de conservation~: Archives nationales, Pierrefitte-sur-Seine.\\
Cote : \hyperlink{CoEg_Mariette_ms_002}{F/17/2988/1, dossier « Mariette »} (n. p.).\\
Support : une feuille double de petit format, à en-tête « Maison de l’empereur\gls{CoEg_org_00000010}. Direction générale des musées impériaux\gls{CoEg_org_00000001} » datée du palais du Louvre.\\
Note : La lettre porte un tampon : « Ministère de l’Instruction publique et des Cultes\gls{CoEg_org_00000042}. Cabinet. 9 octobre 1857 », et en partie supérieure l’annotation à l’encre : « accorder/faire signer/d’urgence/[V ?] ».\\
Thème~: \gls{CoEg_keyword_00000004}.}

\begin{flushright}5 octobre 1857\end{flushright}

\hspace{1cm} Monsieur\gls{CoEg_pers_imn},\\

\indent Je prends la liberté de vous adresser et de vous\\
recommander une lettre destinée à \gls{CoEg_abbr_00000004} \gls{CoEg_abbr_00000001} le Ministre\\
de l’Instruction Publique\gls{CoEg_pers_00000065}.\\
\indent Un ordre subit de \gls{CoEg_abbr_00000015} le Prince Napoléon\gls{CoEg_pers_00000074}\\
m’oblige à partir dans le courant de cette semaine.\\
Je vous serais donc particulièrement obligé si vous\\
vouliez bien m’adresser la réponse de \gls{CoEg_abbr_00000004}\gls{CoEg_pers_00000065} le plus tôt\\
possible.\\
\indent J’ai l’honneur d’être,
\begin{center}Monsieur,\\
\hspace{5cm}Votre très-humble serviteur\\
\hspace{5cm}\gls{CoEg_abbr_00000002} Mariette\end{center}

\hypertarget{CoEg_Mariette_1857-10-26}{}
\section*{Le 26 octobre 1857, d’Alexandrie, à Nieuwerkerke, directeur général des musées impériaux} \addcontentsline{toc}{section}{Le 26 octobre 1857, d’Alexandrie, à Nieuwerkerke, directeur général des musées impériaux} 
{\footnotesize \noindent Institution et lieu de conservation~: Archives nationales, Pierrefitte-sur-Seine\\
Cote~: \hyperlink{CoEg_Mariette_ms_001}{20150497/118, dossier 145 «~Mariette, Auguste~»} (n. p.).\\
Support~: une feuille double de petit format à en-tête : «~Maison de l'Empereur\gls{CoEg_org_00000010}/Direction Générale des Musées Impériaux\gls{CoEg_org_00000001}~», dont la date a été laissée vierge.\\
Thèmes~: \gls{CoEg_keyword_00000005}~; \gls{CoEg_keyword_00000004}.\\
Note~: la lettre porte au coin supérieur gauche les annotations suivantes, à l’encre et d’une autre main que celle de Mariette~: «~\sout{rép. 16.}~» et «~fait 16. \gls{CoEg_abbr_00000021}~».

\begin{center} {[1\textsuperscript{re} page, r\textsuperscript{o}]}\end{center}}
\begin{flushright}Alexandrie\gls{CoEg_place_00000006}, le 26 octobre 1857\end{flushright} 

\hspace{1cm} Monsieur le Comte\gls{CoEg_pers_00000002},\\

Mon plus vif désir, avant de quitter Paris\gls{CoEg_place_00000002},\\
eût été de vous faire mes adieux et de vous\\
serrer la main. Je n’ai pas oublié qu'il y a\\
sept ans, dans ce moment et dans une\\
circonstance pareille, je vous quittais en\\
recevant de vous de bonnes paroles d’encou-\\
-ragement, et je tenais cette fois encore\\
à emporter avec moi ces consolations de\\
voyage. Mais, occupé comme vous l’avez\\
été, je n’ai pas réussi à vous rencontrer,\\
et j’ai dû, malgré mes démarches réitérées,\\
partir sans vous avoir vu. Je suis donc\\
obligé, Monsieur le Comte, de confier à\\
cette lettre mes adieux et mes remerciements\\
pour la permission que vous m’avez accordée\\
d’entreprendre un voyage qui doit avoir, je\\
l’espère, une heureuse influence sur le reste de ma vie.
{\footnotesize\begin{center} {[1\textsuperscript{re} page, v\textsuperscript{o}]}\end{center}}
\indent D’après vos instructions, j’ai entretenu \gls{CoEg_abbr_00000015}\\
le Prince Napoléon\gls{CoEg_pers_00000074} de mon congé, et le Prince\gls{CoEg_pers_00000074}\\
a bien voulu me promettre que, de son côté,\\
il dirait deux mots de cette question à\\
\gls{CoEg_abbr_00000001} Fould\gls{CoEg_pers_00000075}. Voici, je pense, comment l’affaire\\
peut s’arranger :\\
\indent Jusqu’au moment du départ du prince\gls{CoEg_pers_00000074}, c’est-\\
-à-dire jusqu’au commencement de janvier, je\\
voyage incognito et sans qu’un journal\\
parle de moi. Vous pourriez donc, jusques\\
-là, m’accorder la faveur que vous avez\\
faite à quelques uns {[\textit{sic}]} de mes collègues et\\
me permettre de votre propre gré de m’absenter\\
du Louvre\gls{CoEg_org_00000002} pendant deux mois. – Mais une\\
fois le Prince\gls{CoEg_pers_00000074} décidé à partir, vous pourriez\\
exposer à \gls{CoEg_abbr_00000001} Fould\gls{CoEg_pers_00000075} que j’ai été désigné\\
pour faire partie de l’expédition et que\\
vous demandez pour moi un congé de trois\\
mois à partir du 1\textsuperscript{er} Janvier. A mon retour\\
en France\gls{CoEg_place_00000016} avec le Prince\gls{CoEg_pers_00000074}, je reprendrai\\
mes fonctions et tout serait dit. Comme\\
j’ai une femme\gls{CoEg_pers_00000005} et cinq enfants\footnote{Marguerite Louise\gls{CoEg_pers_00000006} (1846-1861), Joséphine Cornélie\gls{CoEg_pers_00000007} (1847-1873), Sophie Éléonore\gls{CoEg_pers_00000008} (1849-1885), Émilie Marie\gls{CoEg_pers_00000083} (1855-1871), Alphonse Paulin\gls{CoEg_pers_00000084} (1856-1879).} auxquels\\
je laisse mon seul traitement pour vivre, je\\
compte donc, Monsieur le Comte, sur votre\\
complaisance pour moi et sur l’intérêt que\\
vous m’avez toujours montré.
{\footnotesize\begin{center} {[2\textsuperscript{e} page, r\textsuperscript{o}]}\end{center}}
\indent J’ai maintenant une demande à vous faire,\\
en mon nom, mais au bénéfice du Consulat-\\
-Général\gls{CoEg_org_00000006} de France\gls{CoEg_org_00000012} à Alexandrie\gls{CoEg_place_00000006}. Vous savez\\
que le Consulat-Général\gls{CoEg_org_00000006} est ici le Palais de la\\
Nation Française, offert jadis par Méhémet-\\
Ali\gls{CoEg_pers_00000017} au Gouvernement Français\gls{CoEg_org_00000012}. Mais les deux\\
grands panneaux du Salon de réception que\\
couvrait {[\textit{sic}]} autrefois Louis-Philippe\gls{CoEg_pers_00000077} et sa\\
famille sont vides, et il serait très désirable,\\
surtout en vue du voyage du Prince Napoléon\gls{CoEg_pers_00000074}\\
qui doit recevoir tous les nationaux français,\\
qu’on pût y voir les portraits de \gls{CoEg_abbr_00000016}\\
l’Empereur\gls{CoEg_pers_00000071} et l’Impératrice\gls{CoEg_pers_00000076}. Ne pourriez-\\
-vous pas faire ce cadeau au Consulat-Général\gls{CoEg_org_00000006}~?\\
\indent Lors du passage et de l’embarquement de mes\\
énormes caisses du Sérapéum\gls{CoEg_place_00000004}, le Consulat\gls{CoEg_org_00000006}\\
s’est donné beaucoup de mal et a dépensé\\
assez d’argent pour le Louvre\gls{CoEg_org_00000002}, et le Consulat\gls{CoEg_org_00000006}\\
verrait avec beaucoup de plaisir que vous\\
consentiez à lui prouver votre reconnaissance\\
en le mettant à même d’orner officiellement\\
son salon de réception des tableaux les plus\\
indispensables\footnote{Mariette\gls{CoEg_pers_00000001} avait d'abord écrit «~du tableau le plus indispensable~» et a ensuite ajouté les terminaisons plurielles.}.\footnote{Tout ce paragraphe est signalé au crayon avec l'annotation «~en [?]/au ministre\gls{CoEg_pers_00000078}/et a M. de Morny\gls{CoEg_pers_00000079}~»}\\
\indent Je suis en Egypte\gls{CoEg_place_00000003} pour préparer le voyage\\
archéologique du Prince\gls{CoEg_pers_00000074}~; mais, vû {[\textit{sic}]} le peu\\
de temps que j’ai encore passé jusqu’ici, je n’ai
{\footnotesize\begin{center} {[2\textsuperscript{e} page, v\textsuperscript{o}]}\end{center}}
\noindent pu rien faire. Soyez sûr cependant que je\\
n’oublie pas le Louvre\gls{CoEg_org_00000002}, et que si les fonctions\\
de conservateur consistent à soigner des collections,\\
je soigne les vôtres bien efficacement puisque\\
je les augmente. Aussi au retour du Prince\gls{CoEg_pers_00000074},\\
c’est-à-dire à la fin de février, aurai-je\\
à mettre à votre disposition une quarantaine\\
de caisses nouvelles.\\
\indent J’espère, Monsieur le Comte, que vous\\
daignerez me continuer la faveur dont vous\\
voulez bien m’honorer. En attendant je vous\\
reste toujours aussi personnellement dévoué qu’on\\
peut l’être et je n’oublierai jamais que c’est\\
à vous que je dois tout ce que je suis en\\
ce monde.\\
\indent J’ai l’honneur d’être,
\begin{center}Monsieur le Comte,\end{center}
\begin{center}\hspace{5cm}votre très-humble serviteur\\
\hspace{5cm} \gls{CoEg_abbr_00000002} Mariette\end{center}
\hypertarget{CoEg_Mariette_1857-11-29}{}
\section*{Le 29 novembre 1857, d’Assiout, à Nieuwerkerke, directeur général des musées impériaux} \addcontentsline{toc}{section}{Le 29 novembre 1857, d’Assiout, à Nieuwerkerke, directeur général des musées impériaux} 
{\footnotesize
\noindent Institution et lieu de conservation~: Archives nationales, Pierrefitte-sur-Seine\\
Cote~: \hyperlink{CoEg_Mariette_ms_001}{20150497/118, dossier 145 «~Mariette, Auguste~»} (n. p.).\\
Support~: une feuille double de moyen format à en-tête : «~Maison de l'Empereur\gls{CoEg_org_00000010}/Direction Générale des Musées Impériaux\gls{CoEg_org_00000001}~», dont la date a été laissée vierge.\\
Thèmes~: \gls{CoEg_keyword_00000005}~; \gls{CoEg_keyword_00000004}.\\
Note~: la lettre porte au coin supérieur gauche, les annotations suivantes, d’une autre main que celle de Mariette~: «~Son congé est en règle./L’en prévenir~» (au crayon) et «~Remis la lettre d’avis/et le congé datés du 15 \gls{CoEg_abbr_00000018}/à son beau frère\gls{CoEg_pers_imn}/31 \gls{CoEg_abbr_00000018} 1857/{[signature illisible]}~» (à l'encre rouge).

\begin{center} [1\textsuperscript{re} page, r\textsuperscript{o}]\end{center}}

\begin{flushright} Syout\gls{CoEg_place_00000033}, le 29 Novembre 1857
\end{flushright}

\hspace{1cm} Monsieur le Comte\gls{CoEg_pers_00000002}\\

Comme cette lettre ne vous arrivera sans doute qu’à la fin de Décembre,\\
je prends la liberté de vous écrire pour vous recommander d’une manière\\
toute spéciale l’affaire de mon congé.\\
\indent Vous me connaissez assez, Monsieur le Comte, pour savoir qu’en vous\\
entretenant de ce sujet, je pense moins à moi qu’à ceux\footnote{La famille Mariette est alors composée de sa femme Éléonore (née Millon)\gls{CoEg_pers_00000005} et de leurs cinq premiers enfants Marguerite Louise\gls{CoEg_pers_00000006} (1846-1861), Joséphine Cornélie\gls{CoEg_pers_00000007} (1847-1873), Sophie Éléonore\gls{CoEg_pers_00000008} (1849-1885), Émilie Marie\gls{CoEg_pers_00000083} (1855-1871), et Alphonse Paulin\gls{CoEg_pers_00000084} (1856-1879).} que j’ai laissés\\
à Paris\gls{CoEg_place_00000002} et qui comptent sur moi pour vivre. Aussi est-ce en même temps\\
un appel à votre générosité comme homme et à votre justice comme\\
chef que je viens vous faire. Je vous en prie donc, Monsieur le Comte,\\
faites que mon congé me soit accordé et que ma famille ne manque\\
de rien. Dans la position particulière que la fortune me fait, c’est là le\\
plus ardent de mes souhaits, et vous me rendrez au moins cette justice\\
qu’en vous écrivant cette lettre j’accomplis le plus sacré et le plus\\
naturel de mes devoirs.\\
\indent J’ai du reste fait savoir cet état de choses à \gls{CoEg_abbr_00000015} le Prince\\
Napoléon\gls{CoEg_pers_00000074}, et je ne doute pas que, de son côté, \gls{CoEg_abbr_00000003} ne soit disposée\\
à dire quelques mots en ma faveur à \gls{CoEg_abbr_00000001} Fould\gls{CoEg_pers_00000075}.\\
\indent Mon voyage ne sera certes pas perdu pour le Louvre\gls{CoEg_org_00000002}. J’ai déjà quelques\\
stèles pour vous, sans compter une quarantaine de caisses du Sérapéum\gls{CoEg_place_00000004}.\\
Je profiterai, pour vous expédier le tout gratis, du moyen de\\
transport que le Vice-Roi\gls{CoEg_pers_00000080} met à la disposition du Prince Napoléon\gls{CoEg_pers_00000074}.\\
Vous voyez que je sers aussi le Louvre\gls{CoEg_org_00000002}, et que certainement le Louvre\gls{CoEg_org_00000002} gagnera\\
bien plus à me voir éloigné de lui que près de lui. En cela, je crois\\
fermement, Monsieur le Comte, bien mériter de vous. Dans ma première\\
absence, j’ai réussi à procurer à votre Musée Egyptien\gls{CoEg_org_00000020} les plus belles
{\footnotesize \begin{center} {[1\textsuperscript{re} page, v\textsuperscript{o}]}\end{center}}
\noindent stèles, les plus beaux bijoux, les plus belles statues, qu’aucun Musée Egyptien\\
possède. Je n’espère pas être aussi heureux cette fois-ci, mais\\
au moins, encore une fois, mon absence n’aura pas été inutile au\\
Louvre\gls{CoEg_org_00000002}.\\
\indent Je vous prie, Monsieur le Comte, de me permettre de profiter de\\
l’occasion pour vous remercier de toutes vos bontés pour moi\\
et vous prier d’accepter l’expression de la profonde reconnaissance
\begin{center}\hspace{5cm}de votre très-humble\\
\hspace{5cm}et très-obéissant serviteur\\
\hspace{5cm} \gls{CoEg_abbr_00000002} Mariette\end{center}
\hypertarget{CoEg_Mariette_1858-01-23}{}
\section*{Le 23 janvier 1858, du Caire, à Nieuwerkerke, directeur général des musées impériaux} \addcontentsline{toc}{section}{Le 23 janvier 1858, du Caire, à Nieuwerkerke, directeur général des musées impériaux} 
{\footnotesize
\noindent Institution et lieu de conservation~: Archives nationales, Pierrefitte-sur-Seine.\\
Cote~: \hyperlink{CoEg_Mariette_ms_001}{20150497/118, dossier 145 «~Mariette, Auguste~»} (n. p.).\\
Support~: une feuille double de petit format à en-tête : «~Maison de l'Empereur\gls{CoEg_org_00000010}/Direction Générale des Musées Impériaux\gls{CoEg_org_00000001}~», dont la date a été laissée vierge.\\
Thèmes~: \gls{CoEg_keyword_00000005}~; \gls{CoEg_keyword_00000004}.\\
Note~: la lettre porte au coin supérieur gauche, les annotations suivantes, d’une autre main que celle de Mariette~: «~qu’il revienne/au plus tot~» (au crayon) et «~rép. 8 février~» (à l’encre).
\begin{center} {[1\textsuperscript{re} page, r\textsuperscript{o}]}\end{center}}

\begin{flushright} Du Caire\gls{CoEg_place_00000010}, le 23 Janvier 1858\end{flushright}

\indent A Monsieur le Comte de Nieuwerkerke\gls{CoEg_pers_00000002},
\begin{center}Directeur-Général des Musées Impériaux\gls{CoEg_org_00000001}\end{center}
\begin{flushright}à Paris\gls{CoEg_place_00000002}.\end{flushright}

\hspace{1cm} Monsieur le Comte\gls{CoEg_pers_00000002},\\

Dans ma dernière lettre, tout en vous remerciant\\
de l’obligeance que vous aviez mise à m’accorder\\
un congé jusqu’au 1\textsuperscript{er} Janvier, je vous faisais\\
observer que, devant rester en voyage avec \gls{CoEg_abbr_00000015}\\
le Prince Napoléon\gls{CoEg_pers_00000074} pendant les mois de Janvier\\
et de Février, il était important pour moi\\
d’obtenir pour ces deux mois un congé de \gls{CoEg_abbr_00000004}\\
\gls{CoEg_abbr_00000001} Fould\gls{CoEg_pers_00000075}. Je vous priais en même temps\\
de faire au Ministre d’Etat\gls{CoEg_pers_00000075} la demande de\\
ce congé, que \gls{CoEg_abbr_00000015}\gls{CoEg_pers_00000074} devait appuyer de son\\
côté.\\
\indent Aujourd’hui j’apprends pas une lettre de\\
\gls{CoEg_abbr_00000001} Ferri-Pisani\gls{CoEg_pers_00000081} que, grâce à vous \& au
{\footnotesize \begin{center} [1\textsuperscript{re} page, v\textsuperscript{o}]\end{center}}
\noindent Prince Napoléon\gls{CoEg_pers_00000074}, mon congé est accordé, non\\
pas pour deux mois comme je l’avais demandé,\\
\textit{mais pour six mois}.\\
\indent Si, Monsieur le Comte, cette prolongation\\
de congé m’a été accordée sur votre instance et\\
avec votre autorisation, je n’ai rien à dire.\\
Si, au contraire, vous n’avez pas participé\\
à cette solution, je vous prie de croire que\\
je n’ai fait aucune demande au Ministère\gls{CoEg_org_00000010},\\
qu’on m’a accordé six mois malgré moi,\\
et que la faveur de \gls{CoEg_abbr_00000004} \gls{CoEg_abbr_00000001} Fould\gls{CoEg_pers_00000075} m’a\\
complètement pris au dépourvu. Mon\\
intention formelle est de rentrer au Louvre\gls{CoEg_org_00000002}\\
le plus tôt possible. Si le Prince Napoléon\gls{CoEg_pers_00000074}\\
vient en Egypte\gls{CoEg_place_00000003} (ce que nous ignorons encore\\
ici), mon désir est de rentrer avec lui en\\
France\gls{CoEg_place_00000016}, et j’espère que ce sera au commencement\\
de Mars. S’il ne vient pas, mon retour\\
sera encore plus prompt, car aussitôt la\\
nouvelle arrivée, je ferai mes préparatifs de\\
départ. Dans tous les cas, Monsieur le Comte,\\
croyez que je tiens assez à mes fonctions du\\
Louvre\gls{CoEg_org_00000002} pour avoir hâte à les reprendre, et que,\\
si je jouis en ce moment d’un congé de six
{\footnotesize \begin{center} [2\textsuperscript{e} page, r\textsuperscript{o}]\end{center}}
\noindent mois, ce n’est pas moi qui l’ai demandé.\\
\indent J’ai, Monsieur le Comte, une autre prière\\
à vous faire. Il s’agit de mes appointements\\
pendant les deux mois de Janvier et de Février.\\
Vous savez mieux que personnes dans quelles\\
conditions je vis. Je mange mon traitement\\
à mesure qu’il m’est servi, et si mon traitement\\
ne m’était pas servi, je ne mangerais pas du\\
tout, ni moi, ni les miens\footnote{La famille Mariette est alors composée de sa femme Êléonore (née Millon)\gls{CoEg_pers_00000005} et de leurs cinq premiers enfants Marguerite Louise\gls{CoEg_pers_00000006} (1846-1861), Joséphine Cornélie\gls{CoEg_pers_00000007} (1847-1873), Sophie Éléonore\gls{CoEg_pers_00000008} (1849-1885), Émilie Marie\gls{CoEg_pers_00000083} (1855-1871) et Alphonse Paulin\gls{CoEg_pers_00000084} (1856-1879).}. Or c’est là un\\
malheur contre lequel il est de mon devoir\\
de me [garder ?]. Je vous supplie donc de faire\\
tout votre possible pour que mes honoraires des\\
deux mois de Janvier \& de Février soient mis\\
à la disposition de ma femme\gls{CoEg_pers_00000005}. C’est là une\\
prière que je vous fais et que, je l’espère,\\
vous daignerez écouter. Dans la triste vie que\\
je mène ici, isolé de tout le monde, sans\\
plaisir et même sans distraction, il m’est\\
pénible de voir ma tristesse augmentée par\\
l’idée que ma famille souffre de mon absence\\
et manque des choses les plus nécessaires à\\
la vie. Encore une fois, Monsieur le Comte,\\
j’ai recours à votre bonté, à votre bienveillance\\
pour moi. Je n’ai pas besoin d’appuyer plus
{\footnotesize \begin{center} [2\textsuperscript{e} page, v\textsuperscript{o}]\end{center}}
\noindent sur ce sujet que vous connaissez aussi bien que\\
moi.\\
\indent Du reste vous apprendrez avec satisfaction que,\\
quel que soit l’état de mes petites affaires\\
particulières, mes affaires scientifiques vont\\
au mieux. Si le Prince Napoléon\gls{CoEg_pers_00000074} vient, il\\
trouvera à son arrivée toute une collection\\
d’antiquités qui l’attend. Les petits objets,\\
je crois, seront perdus pour vous, et le Prince\gls{CoEg_pers_00000074}\\
voudra sans doute les garder. Mais il est\\
quelques gros monuments qui prendront le\\
chemin du Louvre\gls{CoEg_org_00000002}. Au milieu d’eux,\\
vous remarquerez comme artiste un beau\\
fragment\gls{CoEg_obj_imn} de la XII\textsuperscript{e} dynastie, et une statue\gls{CoEg_obj_imn}\\
entière de cet art de la XVIII\textsuperscript{e} qui a donné\\
de si splendides spécimens au Musée\gls{CoEg_org_00000021} de\\
Turin\gls{CoEg_place_00000034}.\\
\indent Je suis revenu de la Haute-Egypte\gls{CoEg_place_00000020} il y a\\
une quinzaine de jours. Le Vice-Roi\gls{CoEg_pers_00000080} m’a\\
traité comme un fonctionnaire de la Maison\\
de l’Empereur\gls{CoEg_org_00000010}, et ce ne sont pas les honneurs\\
qui m’ont manqué ici. Malheureusement\\
je suis atteint de la plus cruelle des maladies :
je m’ennuie.\\
\indent Veuillez croire, Monsieur le Comte, au dévouement\\
et au respect
\begin{center}\hspace{5cm} de votre très-humble serviteur\\
\hspace{5cm} \gls{CoEg_abbr_00000002} Mariette\end{center}
\hypertarget{CoEg_Mariette_1860-12-20}{}
\section*{Le 20 décembre 1860, de Boulaq, à Nieuwerkerke, directeur général des musées impériaux} \addcontentsline{toc}{section}{Le 20 décembre 1860, de Boulaq, à Nieuwerkerke, directeur général des musées impériaux} 
{\footnotesize
\noindent Institution et lieu de conservation : Archives nationales, Pierrefitte-sur-Seine\\
Cote : \hyperlink{CoEg_Mariette_ms_001}{20150497/118, dossier 145 «~Mariette, Auguste~»} (n. p.).\\
Support : une feuille double.\\
Thème~: \gls{CoEg_keyword_00000005}.\\
Note~: la lettre porte au coin supérieur gauche l’annotation suivante au crayon, d’une autre main que celle de Mariette~: «~a classer~» (au crayon)~; la page est tamponnée «~Maison de l’Empereur\gls{CoEg_org_00000010}. Musées impériaux\gls{CoEg_org_00000001}. 10 janvier 1861~».
\begin{center} {[1\textsuperscript{re} page, r\textsuperscript{o}]}\end{center}}
 
\begin{flushright}Boulaq\gls{CoEg_place_00000029}, le 20 décembre 1860.\end{flushright}

\hspace{1cm} Monsieur le Comte\gls{CoEg_pers_00000002},\\

\indent J’ai l’honneur de vous accuser réception de la lettre par\\
laquelle vous me faites part de la décision que vous avez\\
prise en ce qui regarde ma position au Musée du Louvre\gls{CoEg_org_00000002}.\\
\indent J’ai, au sujet de cette lettre, à vous remercier de deux\\
choses. Pour la première, c’est de m’avoir conservé, bien\\
qu’à titre honoraire, dans des fonctions qu’en réalité je ne\\
remplis pas. Il est vrai qu’un hasard heureux m’a mis\\
autrefois entre les mains une assez bonne découverte, et que la\\
collection du Sérapéum\gls{CoEg_place_00000004} me fera toujours vivement et ardemment\\
souhaiter de ne pas quitter l’établissement scientifique où\\
cette collection est conservée~; mais je reconnais moi-même que\\
mes absences deviennent trop longues, et je suis le premier à\\
dire que vous auriez pu sans injustice me rayer du nombre\\
de vos fonctionnaires. J’ai donc à vous remercier de ne l’avoir\\
point fait, et de m’avoir au contraire, bien qu’absent,\\
conservé une place auprès de vous. – La seconde chose qui\\
m’oblige à vous exprimer ma reconnaissance, c’est de\\
m’avoir transmis votre décision dans des termes qui m’ont\\
convaincu que votre bienveillance envers moi est toujours la même.\footnote{La minute de la lettre de Nieuwerkerke\gls{CoEg_pers_00000002}, datée du 29 novembre 1860, est conservée avec cette lettre~:\\
\indent «~Mon cher Mariette,\\
\indent Vous comprendrez facilement que malgré la bonne volonté dont vous êtes à juste titre l’objet, l’irrégularité de votre position dans l’administration ne peut durer plus longtemps. \gls{CoEg_abbr_00000003} le Vice Roi\gls{CoEg_pers_00000080} d’Egypte\gls{CoEg_place_00000003}, Connaissant tout votre mérite et toute l’étendue des services que vous pouviez lui rendre, vous a offert des avantages dont vous ne pourriez pas trouver l’équivalent en France\gls{CoEg_place_00000016}, et je conçois que vous les ayez acceptés comme aurait fait tout autre à votre place, mais vos fonctions de Directeur des monuments historiques de l’Egypte\gls{CoEg_place_00000003} et de Conservateur du Musée\gls{CoEg_org_00000022} du Caire\gls{CoEg_place_00000010}, me paraissent définitives, et par suite – incompatibles avec celles de conservateur adjoint au Louvre\gls{CoEg_org_00000002}. En qualité de Chef d’administration, je ne puis m’empecher [\textit{sic}] de la regretter puisqu’en somme cela prive le Musée\gls{CoEg_org_00000002} de vos services.\\
\indent Vous le savez, \gls{CoEg_abbr_00000001} de Rougé\gls{CoEg_pers_00000032} qui veut bien remplir gratuitement les fonctions de conservateur est presque entièrement absorbé par les travaux de Conseiller d’Etat, il a donc peu de temps à consacrer au Musée\gls{CoEg_org_00000002} et depuis sa nomination de professeur au collège\gls{CoEg_org_00000023} de France\gls{CoEg_place_00000016}, sa presence [\textit{sic}] au Louvre\gls{CoEg_org_00000002} est naturellement encore devenue plus rare (bien qu’il fasse tout ce qui lui est possible de faire pour suppléer à votre absence) en sorte que le Musée Egyptien\gls{CoEg_org_00000020} se trouve presque toujours sans conservateur ni conservateur adjoint.\\
\indent Vous devez comprendre qu’une organisation aussi insolite à {[\textit{sic}]} bien des inconvénients. Or comme j’ignore combien de temps pourrait durer cet état de choses vous trouverez naturel que poussé par \sout{un} les nécessités administratives je prenne un peu malgre [\textit{sic}] \textsuperscript{moi}, une \textsubscript{mesure} de regularité [\textit{sic}] puisqu’il nous faut au moins un conservateur Adjoint au Musée Egyptien\gls{CoEg_org_00000020}. J’ai donc proposé à \gls{CoEg_abbr_00000001} le Ministre d’etat et de la Maison de l’Empereur\gls{CoEg_pers_00000075}, de vous nommer Conservateur Adjoint honoraire, et de nommer \gls{CoEg_abbr_00000001} Dévéria\gls{CoEg_pers_00000082} {[\textit{sic}]}, qui est en mesure de faire son service, Conservateur Adjoint~; par ce moyen, vous conserverez votre titre, ce qui doit être pour vous maintenant la seule chose à laquelle vous puissiez attacher quelque importance.\\
\indent C’est à mon grand deplaisir [\textit{sic}], cependant que cette mesure, ajournée par moi autant qu’il m’a été possible de le faire, est devenue necessaire [\textit{sic}] et, par suite, vous privera de votre traitement~; mais je n’ai pu trouver aucun autre moyen d’obvier aux inconvénients dont je viens de vous parler. J’ajouterai que si plus tard par une raison quelconque, la place de conservateur devenait vacante, la mesure que je prends aujourd’hui ne vous ferait pas perdre les droits que vos travaux et vos et vos [\textit{sic}] découvertes vous donnent à l’occuper.\\
\indent Croyez bien, mon Cher Mariette\gls{CoEg_pers_00000001}, qu’il n’y a dans tout ceci rien de personnel, et n’y \textsuperscript{voyez} que l’obligation dans laquelle je suis de veiller au bon ordre et à la regularité [\textit{sic}] du service dans l’Administration que je dirige. Je desire [\textit{sic}] vivement que nos rapports restent les mêmes que par le passé.\\
\indent Veuillez agréer, mon Cher Mariette\gls{CoEg_pers_00000001}, l’assurance de mes sentiments distingués.~»}
{\footnotesize \begin{center} [1\textsuperscript{re} page, v\textsuperscript{o}]\end{center}}
\indent Si vous vouliez me permettre un souvenir personnel, je vous\\
rappellerais en effet, Monsieur le Comte, qu’il y a\\
dix ans, au moment où je venais en Egypte\gls{CoEg_place_00000003} pour la\\
première fois, vous avez accompagné mon départ d’encouragements\\
qui semblent m’avoir porté bonheur. Depuis lors, à\\
diverses reprises, j’ai eu des preuves de l’intérêt que vous\\
daignez me montrer, et cette fois encore, votre bonne\\
lettre vient me trouver jusqu’au milieu de travaux qui\\
font maintenant l’occupation de ma vie. Je vous\\
remercie donc bien sincèrement et du fond de mon\\
cœur, Monsieur le Comte, non seulement de m’avoir\\
permis de rester conservateur-adjoint du Louvre\gls{CoEg_org_00000002}, mais\\
encore de m’avoir prouvé que vous êtes toujours pour\\
moi celui qui, en 1850, encouragea de ses souhaits mes\\
premiers pas.\\
\indent Veuillez, Monsieur le Comte, agréer l’assurance du\\
profond respect avec lequel,\\
\begin{center} j’ai l’honneur d’être,\end{center}
\begin{center}\hspace{5cm} Votre tout dévoué serviteur,\\
\hspace{5cm} \gls{CoEg_abbr_00000002} Mariette\end{center}
\gls{CoEg_abbr_00000008} J’irai passer cet été en France\gls{CoEg_place_00000016}, et serai à Paris\gls{CoEg_place_00000002} vers la fin\\
d’Avril.

\hypertarget{CoEg_Mariette_1867-04-13}{}
\section*{Le 13 avril 1867, de Paris, à Nieuwerkerke} \addcontentsline{toc}{section}{Le 13 avril 1867, de Paris, à Nieuwerkerke} 
{\footnotesize
\noindent Institution et lieu de conservation~: Archives nationales, Pierrefitte-sur-Seine\\
Cote~: \hyperlink{CoEg_Mariette_ms_001}{20150497/118, dossier 145 «~Mariette, Auguste~»} (n. p.).\\
Support~: une feuille double de petit format.}

\begin{flushright}Paris\gls{CoEg_place_00000002}-Auteuil, le 13 Avril 1867.\end{flushright}

\hspace{1cm} Monsieur le Comte\gls{CoEg_pers_00000002},\\

\indent L’invitation que vous avez bien voulu m’adresser\\
pour le Vendredi 5 Avril et les Vendredis suivants\\
a été mise à une adresse qui n’est plus la mienne\\
depuis trois mois, et ne me parvient qu’aujourd’hui\\
Samedi.\\
\indent Je m’empresse de vous écrire afin que, comprenant\\
mon absence, vous ayez la bonté de l’excuser.\\
\indent J’ai l’honneur d’être\\
\begin{center} Monsieur le Comte,\end{center}
\begin{center}\hspace{5cm} Votre très-dévoué serviteur\\
\hspace{5cm} \gls{CoEg_abbr_00000002} Mariette\end{center}

\hypertarget{CoEg_Mariette_1879-11-06}{}

\section*{Le 6 novembre 1879, de Paris, à Ferry, président de la commission des missions scientifiques}
\addcontentsline{toc}{section}{Le 6 novembre 1879, de Paris, à Ferry, président de la commission des missions scientifiques} \label{labCoEg_Mariette_1879-11-06}
{\footnotesize
\noindent Institution et lieu de conservation~: Archives nationales, Pierrefitte-sur-Seine.\\
Cote : \hyperlink{CoEg_Mariette_ms_002}{F/17/2988/1, dossier « Mariette »} (n. p.).\\
Support : une feuille double de grand format, de papier épais et vergeté.\\
Note : La lettre porte les annotations suivantes : « Mariette » au coin supérieur gauche, au crayon vert ; « oui » en partie supérieure gauche, au crayon ; « N\textsuperscript{o} 1 » au centre, au crayon .\\
Thème~: \gls{CoEg_keyword_00000008}~; \gls{CoEg_keyword_00000014}~; \gls{CoEg_keyword_00000015}.}

{\footnotesize \begin{center} [1\textsuperscript{re} page, r\textsuperscript{o}]\end{center}}
\begin{flushright}Paris\gls{CoEg_place_00000002}, 5, rue Le Peletier.\\
6 novembre 1879\end{flushright}
\indent A Monsieur le Président de la Commission des Missions\\
\indent Scientifiques\gls{CoEg_org_00000039}\footnote{La commission des travaux historiques était présidée en 1879 par le ministre de l’Instruction publique (arrêté du 1\textsuperscript{er} février 1879 : \textit{Bulletin administratif de l’Instruction publique} 438, 1879, p. \href{https://education.persee.fr/doc/baip_1254-0714_1879_num_22_438_59593}{123-124})}.\\

\hspace{1cm} Monsieur le Président\gls{CoEg_pers_00000238},\\

\indent Il existe en Egypte\gls{CoEg_place_00000003}, particulièrement dans les nécropoles\\
de Memphis\gls{CoEg_place_00000005}, des tombes de style uniforme, aussi remarquables\\
par la masse extraordinaire des matériaux employés dans\\
leur construction que par la variété des représentations qui\\
en décorent les chambres ; nous les appelons des \textit{\glspl{CoEg_entry_00000027}}.\\
Aucun monument ne dépasse les \glspl{CoEg_entry_00000027} en antiquité.\\
Avec les \glspl{CoEg_entry_00000027}, la science touche à ce qu’on peut\\
appeler justement la nuit des siècle, et pénètre aussi\\
loin qu’il est possible d’aller aujourd’hui dans l’histoire\\
de l’homme civilisé. A ce titre, les \glspl{CoEg_entry_00000027} méritent toute\\
notre attention, et j’y ai vivement insisté dans le Mémoire\\
que j’ai eu l’honneur de lire il y a quelques jours\footnote{Mariette semble avoir fait sa communication au cours de la séance du 10 octobre 1879 (Comptes rendus des séances de l’Académie des inscriptions et belles-lettres, 1879, p. \href{https://www.persee.fr/doc/crai_0065-0536_1879_num_23_4_88961}{258}).} devant\\
l’Académie des Inscriptions\gls{CoEg_org_00000036}, Mémoire qui a eu pour\\
résultat la démarche que le bureau de la savante\\
Compagnie a faite auprès de \gls{CoEg_abbr_00000009} les Ministres de\\
l’Instruction Publique\gls{CoEg_pers_00000238} et des Affaires Etrangères\gls{CoEg_pers_00000239}.
{\footnotesize \begin{center} [1\textsuperscript{re} page, v\textsuperscript{o}]\end{center}}
Malheureusement, comme tous les monuments situés sur\\
les bords du Nil\gls{CoEg_place_00000021}, les \glspl{CoEg_entry_00000027} sont exposés à mille causes\\
de détérioration. Des bas-reliefs s’effacent, des inscriptions\\
disparaissent ; ou bien les sables du désert arrivent, et les\\
\glspl{CoEg_entry_00000027}, engloutis et noyés dans cette marée montante,\\
sont bientôt comme s’ils n’existaient pas.\\
\indent Il est donc important de recueillir tous les renseignements\\
que les \glspl{CoEg_entry_00000027} peuvent nous fournir, de copier les\\
textes qui s’y trouvent, de prendre un calque des représentations\\
si intéressantes qu’on y rencontre, et c’est à cet utile travail\\
que je voudrais occuper mon temps pendant l’hiver et\\
le printemps prochain.\\
\indent Mais il me faut engager un ou deux dessinateurs,\\
un photographe, un architecte, des mouleurs. Il me faut\\
faire des frais de toute sorte en outils, en appareils de\\
photographie, de moulages en plâtres, d’estampages en\\
papier.\\
\indent C’est sur ces motifs que je me base, Monsieur le\\
Président, pour solliciter une mission en Egypte\gls{CoEg_place_00000003} qui\\
me permettrait de réunir les matériaux d’une publication\\
que l’on pourrait consacrer ultérieurement à la monographie\\
des \glspl{CoEg_entry_00000027}. Une somme de dix mille francs me serait\\
nécessaire, et l’importance de la tâche que je voudrais\\
remplir me fait penser que vous voudrez bien me l’accorder.\footnote{Mariette se vit effectivement attribuer, par décision du 3 février 1880, une mission « pour réunir les matériaux nécessaires à la publication d'une monographie de Martabas » ([\textit{sic}] : \textit{Archives des missions scientifiques et littéraires} (3\textsuperscript{e} série) 15 bis \textit{Table générale}, Paris, Ernest Leroux, 1890, p. \href{https://gallica.bnf.fr/ark:/12148/bpt6k30470107/f66.item}{44}).}
{\footnotesize \begin{center} [2\textsuperscript{e} page, r\textsuperscript{o}]\end{center}}
Le sable et le désert se présentent en Egypte\gls{CoEg_place_00000003} dans des conditions\\
telles qu’il n’est possible d’y travailler avec quelque\\
fruit que pendant la saison d’hiver. Notre but serait donc\\
d’autant plus vite et d’autant mieux atteint que vous\\
mettriez plus rapidement à ma disposition le crédit que\\
je prends la liberté de vous demander.\\
\indent Je vous prie, Monsieur le Président, d’agréer l’assurance\\
de mon profond respect et de me croire
\begin{flushright}\hspace{5cm}Votre très-dévoué serviteur\\
\hspace{5cm}\gls{CoEg_abbr_00000002} Mariette\\
\hspace{5cm}Membre de l’Institut\gls{CoEg_org_00000004}.\end{flushright}

\cleardoublepage
\part*{Annexes}
\phantomsection
\addcontentsline{toc}{part}{Annexes}

\cleardoublepage
\section*{Destinataires des lettres}
\phantomsection
\addcontentsline{toc}{section}{Destinataires des lettres}

\begin{center} \textsc{Destinataires non dénommés}\end{center}
\begin{itemize}
\item \hyperlink{CoEg_Mariette_1855-12-12}{Le 12 décembre 1855, de Paris, à un fonctionnaire de l’Instruction publique} (\hyperlink{CoEg_Mariette_ms_002}{Archives nationales, F/17/2988/1})~;
\item \hyperlink{CoEg_Mariette_1857-01-03}{Le 3 janvier 1857, de Paris, à un fonctionnaire de l’Instruction publique} (\hyperlink{CoEg_Mariette_ms_002}{Archives nationales, F/17/2988/1})~;
\item \hyperlink{CoEg_Mariette_1855-07-12}{Le 12 juillet 1855, de Paris, à un fonctionnaire de l’Instruction publique} (\hyperlink{CoEg_Mariette_ms_002}{Archives nationales, F/17/2988/1})~;
\item \hyperlink{CoEg_Mariette_1857-10-05}{Le 5 octobre 1857, de Paris, à un fonctionnaire de l’Instruction publique} (\hyperlink{CoEg_Mariette_ms_002}{Archives nationales, F/17/2988/1}).
\end{itemize}

\begin{center} \textsc{Louis Camaret\gls{CoEg_pers_00000093},\\
recteur de l’académie de Douai}\end{center}
\begin{itemize}
\item \hyperlink{CoEg_Mariette_1846-05-24}{Le 24 mai 1846, de Boulogne-sur-Mer} (\hyperlink{CoEg_Mariette_ms_002}{Archives nationales, F/17/2988/1}).
\end{itemize}

\begin{center} \textsc{Marie Dombibau de Crouseilhes\gls{CoEg_pers_00000086},\\
ministre de l'Instruction publique (1851)}\end{center}
\begin{itemize}
\item \hyperlink{CoEg_Mariette_1851-10-14b}{Le 14 septembre 1851, de Saqqarah} (copies~: \hyperlink{CoEg_Mariette_ms_001}{Archives nationales, 20150497/118} et \hyperlink{CoEg_Mariette_ms_002}{Archives nationales, F/17/2988/1}).
\end{itemize}

\begin{center} \textsc{Félix Esquirou de Parieu\gls{CoEg_pers_00000003},\\
ministre de l'Instruction publique (1849-1851)}\end{center}
\begin{itemize}
\item \hyperlink{CoEg_Mariette_1850-05-20}{Le 20 mai 1850, de Paris} (\hyperlink{CoEg_Mariette_ms_002}{Archives nationales, F/17/2988/1})~;
\item \hyperlink{CoEg_Mariette_1850-07-06}{Le 6 juillet 1850, de Paris} (\hyperlink{CoEg_Mariette_ms_002}{Archives nationales, F/17/2988/1})~;
\item \hyperlink{CoEg_Mariette_1850-08-27}{Le 27 août 1850, de Paris} (\hyperlink{CoEg_Mariette_ms_002}{Archives nationales, F/17/2988/1}).
\end{itemize}

\begin{center} \textsc{Léon Faucher\gls{CoEg_pers_00000085},\\
ministre de l'Intérieur (1851)}\end{center}
\begin{itemize}
\item \hyperlink{CoEg_Mariette_1851-10-14b}{Le 14 septembre 1851, de Saqqarah} (copies~: \hyperlink{CoEg_Mariette_ms_001}{Archives nationales, 20150497/118} et \hyperlink{CoEg_Mariette_ms_002}{Archives nationales, F/17/2988/1}).
\end{itemize}

\begin{center} \textsc{Jules Ferry\gls{CoEg_pers_00000238},\\
ministre de l'Instruction publique (1879-1881, 1882, 1883)}\end{center}
\begin{itemize}
\item \hyperlink{CoEg_Mariette_1879-11-06}{Le 6 novembre 1879, de Paris} (\hyperlink{CoEg_Mariette_ms_002}{Archives nationales, F/17/2988/1}).
\end{itemize}

\begin{center} \textsc{Hippolyte Fortoul\gls{CoEg_pers_00000240},\\
ministre de l'Instruction publique (1851-1856)}\end{center}
\begin{itemize}
\item \hyperlink{CoEg_Mariette_1855-01-26}{Le 26 janvier 1855, de Paris} (\hyperlink{CoEg_Mariette_ms_002}{Archives nationales, F/17/2988/1})~;
\item \hyperlink{CoEg_Mariette_1855-08-06}{Le 6 août 1855, de Paris} (\hyperlink{CoEg_Mariette_ms_002}{Archives nationales, F/17/2988/1}).
\end{itemize}

\begin{center} \textsc{Arnaud Le Moyne\gls{CoEg_pers_00000024},\\
consul général et agent de France en Égypte (...-1852)}\end{center}
\begin{itemize}
\item \hyperlink{CoEg_Mariette_1851-09-14a}{Le 14 septembre 1851, de Saqqarah} (copies~: \hyperlink{CoEg_Mariette_ms_001}{Archives nationales, 20150497/118} et \hyperlink{CoEg_Mariette_ms_002}{Archives nationales, F/17/2988/1}).
\end{itemize}

\begin{center} \textsc{Adrien de Longpérier\gls{CoEg_pers_00000033},\\
conservateur des antiques et sculptures au musée du Louvre}\end{center}
\begin{itemize}
\item \hyperlink{CoEg_Mariette_1849-10-20}{Le 20 octobre 1849, de Paris} (\hyperlink{CoEg_Mariette_ms_002}{Archives nationales, F/17/2988/1}).
\end{itemize}

\begin{center} \textsc{Émilien de Nieuwerkerke\gls{CoEg_pers_00000002},\\
directeur général des musées nationaux puis intendant des beaux-arts\\
et surintendant des musées impériaux}\end{center}
\begin{itemize}
\item \hyperlink{CoEg_Mariette_1850-07-08}{Le 8 juillet 1850, de Paris} (\hyperlink{CoEg_Mariette_ms_001}{Archives nationales, 20150497/118})~;
\item \hyperlink{CoEg_Mariette_1851-02-28}{Le 28 février 1851, de Saqqarah} (\hyperlink{CoEg_Mariette_ms_001}{Archives nationales, 20150497/118})~;
\item \hyperlink{CoEg_Mariette_1851-08-31}{Le 31 août 1851, de Saqqarah} (\hyperlink{CoEg_Mariette_ms_001}{Archives nationales, 20150497/118})~;
\item \hyperlink{CoEg_Mariette_1852-01-16}{Le 16 janvier 1852, d’Abousir} (\hyperlink{CoEg_Mariette_ms_001}{Archives nationales, 20150497/118}) [vraisemblablement]~;
\item \hyperlink{CoEg_Mariette_1852-08-04}{Le 4 août 1852, d’Abousir} (\hyperlink{CoEg_Mariette_ms_001}{Archives nationales, 20150497/118}) [vraisemblablement]~;
\item \hyperlink{CoEg_Mariette_1852-09-04}{Le 4 septembre 1852, d’Abousir} (\hyperlink{CoEg_Mariette_ms_001}{Archives nationales, 20150497/118}) [vraisemblablement]~;
\item \hyperlink{CoEg_Mariette_1852-11-12}{Le 12 novembre 1852, d’Abousir} (\hyperlink{CoEg_Mariette_ms_001}{Archives nationales, 20150497/118}) [vraisemblablement]~;
\item \hyperlink{CoEg_Mariette_1853-01-01}{Le 1\textsuperscript{er} janvier 1853, d’Abousir} (\hyperlink{CoEg_Mariette_ms_001}{Archives nationales, 20150497/118}) [vraisemblablement]~;
\item \hyperlink{CoEg_Mariette_1853-05-06}{Le 6 mai 1853, d’Abousir} (\hyperlink{CoEg_Mariette_ms_001}{Archives nationales, 20150497/118})~;
\item \hyperlink{CoEg_Mariette_1853-07-30}{Le 30 juillet 1853, du Caire} (\hyperlink{CoEg_Mariette_ms_001}{Archives nationales, 20150497/118})~;
\item \hyperlink{CoEg_Mariette_1853-08-10}{Le 10 août 1853, d’Abousir} (\hyperlink{CoEg_Mariette_ms_001}{Archives nationales, 20150497/118}) [vraisemblablement]~;
\item \hyperlink{CoEg_Mariette_1853-08-28}{Le 28 août 1853, d’Abousir} (\hyperlink{CoEg_Mariette_ms_001}{Archives nationales, 20150497/118}) [vraisemblablement]~;
\item \hyperlink{CoEg_Mariette_1857-02-20}{Le 20 février 1857, de Paris} (\hyperlink{CoEg_Mariette_ms_001}{Archives nationales, 20150497/118})~;
\item \hyperlink{CoEg_Mariette_1857-10-26}{Le 26 octobre 1857, d’Alexandrie} (\hyperlink{CoEg_Mariette_ms_001}{Archives nationales, 20150497/118})~;
\item \hyperlink{CoEg_Mariette_1857-11-29}{Le 29 novembre 1857, d’Assiout} (\hyperlink{CoEg_Mariette_ms_001}{Archives nationales, 20150497/118})~;
\item \hyperlink{CoEg_Mariette_1858-01-23}{Le 23 janvier 1858, du Caire} (\hyperlink{CoEg_Mariette_ms_001}{Archives nationales, 20150497/118})~;
\item \hyperlink{CoEg_Mariette_1860-12-20}{Le 20 décembre 1860, de Boulaq} (\hyperlink{CoEg_Mariette_ms_001}{Archives nationales, 20150497/118})~;
\item \hyperlink{CoEg_Mariette_1867-04-13}{Le 13 avril 1867, de Paris} (\hyperlink{CoEg_Mariette_ms_001}{Archives nationales, 20150497/118}).
\end{itemize}

\begin{center} \textsc{Victor Fialin de Persigny\gls{CoEg_pers_00000037},\\
ministre de l'Intérieur (1852-1854, 1860-1863)}\end{center}
\begin{itemize}
\item \hyperlink{CoEg_Mariette_1852-08-20}{Le 20 août 1852, d’Abousir} (\hyperlink{CoEg_Mariette_ms_001}{Archives nationales, 20150497/118})~;
\item \hyperlink{CoEg_Mariette_1852-09-03}{Le 3 septembre 1852, d’Abousir} (\hyperlink{CoEg_Mariette_ms_001}{Archives nationales, 20150497/118})~;
\item \hyperlink{CoEg_Mariette_1852-12-28}{Le 28 décembre 1852, d’Abousir} (\hyperlink{CoEg_Mariette_ms_001}{Archives nationales, 20150497/118}).
\end{itemize}

\begin{center} \textsc{Gustave Rouland\gls{CoEg_pers_00000065},\\
ministre de l'Instruction publique (1856-1863)}\end{center}
\begin{itemize}
\item \hyperlink{CoEg_Mariette_1856-02-11}{Le 11 février 1856, de Paris} (\hyperlink{CoEg_Mariette_ms_002}{Archives nationales, F/17/2988/1})~;
\item \hyperlink{CoEg_Mariette_1856-12-11}{Le 11 décembre 1856, de Paris} (\hyperlink{CoEg_Mariette_ms_002}{Archives nationales, F/17/2988/1})~;
\item \hyperlink{CoEg_Mariette_1856-12-31}{Le 31 décembre 1856, de Paris} (\hyperlink{CoEg_Mariette_ms_002}{Archives nationales, F/17/2988/1})~;
\item \hyperlink{CoEg_Mariette_1857-04-01}{Le 1er avril 1857, de Paris} (\hyperlink{CoEg_Mariette_ms_002}{Archives nationales, F/17/2988/1})~;
\item \hyperlink{CoEg_Mariette_1857-10-04}{Le 4 octobre 1857, de Paris} (\hyperlink{CoEg_Mariette_ms_002}{Archives nationales, F/17/2988/1}).
\end{itemize}

\begin{center} \textsc{Narcisse-Achille de Salvandy\gls{CoEg_pers_00000090},\\
ministre de l'Instruction publique (1837-1839, 1845-1848)}\end{center}
\begin{itemize}
\item \hyperlink{CoEg_Mariette_1846-04-13}{Le 13 avril 1846, de Boulogne-sur-Mer} (\hyperlink{CoEg_Mariette_ms_002}{Archives nationales, F/17/2988/1})~;
\item \hyperlink{CoEg_Mariette_1846-05-25}{Le 25 mai 1846, de Boulogne-sur-Mer} (\hyperlink{CoEg_Mariette_ms_002}{Archives nationales, F/17/2988/1})~;
\item \hyperlink{CoEg_Mariette_1846-09-29}{Le 29 septembre 1846, de Boulogne-sur-Mer} (\hyperlink{CoEg_Mariette_ms_002}{Archives nationales, F/17/2988/1}).
\end{itemize}

\begin{center} \textsc{Eugène Servaux\gls{CoEg_pers_00000237},\\
chef du bureau des travaux historiques au ministère de l'Instruction publique}\end{center}
\begin{itemize}
\item \hyperlink{CoEg_Mariette_1857-08-26}{Le 26 août 1857, de Paris} (\hyperlink{CoEg_Mariette_ms_002}{Archives nationales, F/17/2988/1}).
\end{itemize}

\cleardoublepage
\phantomsection
\addcontentsline{toc}{section}{Personnes}

\cleardoublepage
\phantomsection
\addcontentsline{toc}{subsection}{Contemporains de Mariette}
\glossarystyle{index}
\printnoidxglossary[type={contemp}]

\cleardoublepage
\subsection*{Fonctions occupées par des contemporains}
\phantomsection
\addcontentsline{toc}{subsection}{Fonctions occupées par des contemporains}
Ces listes répertorient les personnes mentionnées dans l’index précédent et qui ont successivement occupé une fonction commune.\\

\begin{center} \textsc{Consuls généraux et agents de France en Égypte}\end{center} \begin{itemize}
\item ...-1852~: Arnaud Le Moyne\gls{CoEg_pers_00000024}~;
\item 1852-...~: Sabatier\gls{CoEg_pers_00000040}. \end{itemize}

\begin{center} \textsc{Conservateurs au département\\
égyptien du musée du Louvre}\end{center} \begin{itemize}
\item 1826-1832~: Jean-François Champollion le Jeune\gls{CoEg_pers_00000094}~;
\item 1849-1872~: Emmanuel de Rougé\gls{CoEg_pers_00000032}. \end{itemize}

\begin{center} \textsc{Conservateurs adjoints au\\
département égyptien du musée du Louvre}\end{center} \begin{itemize}
\item 1855-1861~: Auguste Mariette\gls{CoEg_pers_00000001}\footnote{En 1861, Mariette fut nommé conservateur adjoint honoraire~; voir sa \hyperlink{CoEg_Mariette_1860-12-20}{lettre à Nieuwerkerke du 20 décembre 1860}.}~;
\item 1861-1871~:Théodule Devéria\gls{CoEg_pers_00000082}. \end{itemize}

\begin{center} \textsc{Directeurs de la Bibliothèque royale} \end{center} \begin{itemize}
\item 1838-1839~: Jean Antoine Letronne\gls{CoEg_pers_00000097}~;
\item 1838-1839~: Edme-François Jomard\gls{CoEg_pers_00000141}~;
\item 1839-1840~: Jean Antoine Letronne\gls{CoEg_pers_00000097}.\end{itemize}

\begin{center} \textsc{Ministres d’État français}\end{center} \begin{itemize}
\item 1852-1860~: Achille Fould\gls{CoEg_pers_00000075}~;
\item 1863~: Adolphe Billault\gls{CoEg_pers_00000078}. \end{itemize}

\begin{center} \textsc{Ministres de l’Instruction publique français}\end{center} \begin{itemize}
\item 1837-1839, 1845-1848~: Narcisse-Achille de Salvandy\gls{CoEg_pers_00000090}~;
\item 1849-1851~: Félix Esquirou de Parieu\gls{CoEg_pers_00000003}~;
\item 1851~: Marie Dombidau de Crouseilhes\gls{CoEg_pers_00000086}~;
\item 1851-1856~: Hippolyte Fortoul\gls{CoEg_pers_00000240}~;
\item 1856-1863~: Gustave Rouland\gls{CoEg_pers_00000065}~;
\item 1873, 1876-1877~: William Henry Waddington\gls{CoEg_pers_00000239}~;
\item 1879-1881, 1882, 1883~: Jules Ferry\gls{CoEg_pers_00000238}. \end{itemize}

\begin{center} \textsc{Ministres de l’Intérieur français}\end{center} \begin{itemize}
\item 1850-1851~: Jules Baroche\gls{CoEg_pers_00000004}~;
\item 1851~: Léon Faucher\gls{CoEg_pers_00000085}~;
\item 1851-1852~: Charles de Morny\gls{CoEg_pers_00000079}~;
\item 1852-1854~: Victor Fialin de Persigny\gls{CoEg_pers_00000037}~;
\item 1854-1858, 1859-1860~: Adolphe Billault\gls{CoEg_pers_00000078}~;
\item 1860-1863~: Victor Fialin de Persigny\gls{CoEg_pers_00000037}. \end{itemize}

\begin{center} \textsc{Présidents de la Société de géographie} \end{center} \begin{itemize}
\item 1838~: Narcisse-Achille de Salvandy\gls{CoEg_pers_00000090}~;
\item 1848~: Edme-François Jomard\gls{CoEg_pers_00000141}~;
\item 1854~: Hippolyte Fortoul\gls{CoEg_pers_00000240}~;
\item 1856~: Joseph-Daniel Guigniaut\gls{CoEg_pers_00000226}~;
\item 1860~: Gustave Rouland\gls{CoEg_pers_00000065}~;
\item 1860~: Victor Fialin de Persigny\gls{CoEg_pers_00000037}.\end{itemize}

\begin{center} \textsc{Vice-rois d’Égypte} \end{center} \begin{itemize}
\item 1805-1848~: Méhémét Ali\gls{CoEg_pers_00000017}~;
\item 1848~: Ibrahim Pacha\gls{CoEg_pers_00000087}~;
\item 1848-1854~: Abbas I\textsuperscript{er} Hilmi\gls{CoEg_pers_00000016}~;
\item 1854-1863~: Saïd Pacha\gls{CoEg_pers_00000080}.\end{itemize}

\cleardoublepage
\phantomsection
\addcontentsline{toc}{subsection}{Personnages historiques}
\printnoidxglossary[type={hist}]

\cleardoublepage
\phantomsection
\addcontentsline{toc}{subsection}{Figures mythiques et religieuses}
\printnoidxglossary[type={myth}]

\cleardoublepage
\phantomsection
\addcontentsline{toc}{section}{Bateaux}
\printnoidxglossary[type={boat}]

\cleardoublepage
\phantomsection
\addcontentsline{toc}{section}{Institutions}
\printnoidxglossary[type={org}]

\cleardoublepage
\phantomsection
\addcontentsline{toc}{section}{Lieux}
\printnoidxglossary[type={place}]

\cleardoublepage
\phantomsection
\addcontentsline{toc}{section}{Objets}
\printnoidxglossary[type={obj}]

\cleardoublepage
\phantomsection
\addcontentsline{toc}{section}{Publications}
\printnoidxglossary[type={bibl}]

\cleardoublepage
\phantomsection
\addcontentsline{toc}{section}{Thèmes}
\printnoidxglossary[type={keyword}]

\cleardoublepage
\phantomsection
\addcontentsline{toc}{section}{Glossaire}
\printnoidxglossary[type={entry}]

\cleardoublepage
\phantomsection
\addcontentsline{toc}{section}{Lexique égyptien}
\printnoidxglossary[type={aeg}]

\cleardoublepage
\phantomsection
\addcontentsline{toc}{section}{Abréviations}
\printnoidxglossary[type={abbr}]

\tableofcontents
\addcontentsline{toc}{part}{Table des matières}

\end{document}